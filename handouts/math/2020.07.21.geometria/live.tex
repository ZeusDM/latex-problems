\documentclass[10pt, a4paper]{article}
\usepackage[utf8]{inputenc}
\usepackage[brazilian]{babel}
\usepackage{lmodern}
\usepackage[left=2cm, right=2cm, top=2cm, bottom=2.5cm]{geometry}
\usepackage{indentfirst}
\usepackage[inline]{enumitem}

\usepackage{pgf,tikz,pgfplots}
\pgfplotsset{compat=1.15}
\usepackage{mathrsfs}
\usetikzlibrary{arrows}

\usepackage[pensi,
			problem-list
			]{zeus}

\title{Geometria -- Crux Mathematicorum -- Live}
\author{Guilherme Zeus Moura}
\mail{zeusdanmou@gmail.com}
\titlel{Turma Olímpica}
\titler{{\footnotesize v. 1} -- 21 de Julho de 2020}

\renewcommand{\playerA}[1]{Guilherme}
\renewcommand{\playerB}[1]{Zeus}

\begin{document}	
	\zeustitle

	\problem{math/crux/problems/4560}

	\noindent\textit{Solução.}

	\noindent\textbf{Dica principal.} Inverta por $A$ (com qualquer valor pro raio $r$). Procure mais sobre o que é inversão na internet (ou peça pra eu te mandar alguns artigos sobre isso). Esse é um ótimo problema pra \textit{aprender} sobre inversão.

	\begin{rem}
		Se você quiser, você pode submeter a sua solução para esse problema na revista Crux Mathematicorum. Procure no Google sobre ela.
	\end{rem}

	\newpage
	\problem{math/crux/problems/4509}

	\noindent\textit{Solução 1, calculando $AM$ com Stweart.}

	Se $A$ estiver na circunferência de diâmetro $BC$, funciona! $R = AM$ e $AB^2 + AC^2 = BC^2 = 4 R^2$.

	Se $A$ estiver na mediatriz de $BC$, funciona! (ver arquivo do Miguel)
	
	Vamos calcular $AM$ (com duas leis de cossenos em $AMB$ e $AMC$, ou com Stweart).
	\begin{align*}
	AB^2 = AM^2 + BM^2 - 2 AM \cdot BM \cdot \cos(\angle AMB)\\
	AC^2 = AM^2 + CM^2 - 2 AM \cdot CM \cdot \cos(\angle AMC)
	\end{align*}

	Somando, temos \[2 \cdot AM^2 = AB^2 + AC^2 - 2 BM^2.\]
	\[4 \cdot AM^2 = 2(AB^2 + AC^2) - BC^2.\]

	Logo, $4R \cdot AM = AB^2 + AC^2 \iff$
	\[4 \cdot AM^2 - 8R \cdot AM + BC^2 = 0.\]

	\[ AM = R \pm \sqrt{R^2 - (BC/2)^2}\]	
	\[ AM = R \pm \sqrt{R^2 - BM^2}\]
	\[ AM = AO \pm OM\]

	Usando Desigualdade Triangular, $A, O, M$ são colineares. Portanto:

	\begin{itemize}
		\item Se $O \neq M$, $OM$ é mediatriz de $BC$ e, portanto, $A$ está na mediatriz.
		\item Se $O = M$, então $AM = AO \implies A$ está no círculo de diâmetro $BC$. 
	\end{itemize}

	\noindent\textit{Solução 2, calculando $AM$ com (uma) Lei dos Cossenos.}

	Seja $D$ o ponto tal que $ABDC$ é um paralogramo. $M$ é ponto médio de $AD$. Portanto, usando Lei dos Cossenos em $ADC$, temos \[ 4 AM^2 = AD^2 = AC^2 + AB^2 + 2 \cdot AB \cdot AC \cdot \cos A\]

	Desse modo, as seguintes linhas são equivalentes (i.e, $\iff$).
	\[ 4R \cdot AM = AB^2 + BC^2\]
	\[ 4R^2 \cdot 4 AM^2 = (b^2 + c^2)^2\]
	\[ (2R)^2 \cdot (b^2 + c^2 + 2bc\cos A) = (b^2 + c^2)^2\]
	\[ a^2 \cdot (b^2 + c^2 + 2bc\cos A) = (b^2 + c^2)^2 \sin^2 A\]
	\[ (b^2 + c^2 - 2bc\cos A) \cdot (b^2 + c^2 + 2bc\cos A) = (b^2 + c^2)^2 \sin^2 A\]
	\[ (b^2 + c^2)^2 - (2bc\cos A)^2 = (b^2 + c^2)^2 \sin^2 A\]
	\[ (b^2 + c^2)^2  (1 - \sin^2 A) = (2bc\cos A)^2\]
	\[ (b^2 + c^2)^2  \cos^2 A = (2bc\cos A)^2\]
	\[ \cos^2 A ( (b^2 + c^2)^2 - 4b^2c^2 ) = 0\]
	\[ \cos^2 A (b^2 - c^2)^2 = 0\]
	\[ \cos^2 A = 0  \text{ ou } b^2 - c^2 = 0\]
	\[ \angle A = 90^\circ  \text{ ou } b = c\]


	\newpage
	\problem{math/crux/problems/4494}

\definecolor{ffqqqq}{rgb}{0.9,0,0.2}
\definecolor{xdxdff}{rgb}{0.2,0.3,0.8}
\definecolor{qqwuqq}{rgb}{0.,0.39215686274509803,0.}
\definecolor{yqyqyq}{rgb}{0.5019607843137255,0.5019607843137255,0.5019607843137255}
\definecolor{uququq}{rgb}{0.25098039215686274,0.25098039215686274,0.25098039215686274}

	Seja $D$ a segunda interseção de $AP$ com o cicuncírculo de $ABC$.

	$AODQ$ é cíclico, pois $AP \cdot PD = BP \cdot PC = OP \cdot PQ$. Podemos marcar os ângulos como no diagrama abaixo, usando esse quadrilátero cíclico e usando os ângulos isósceles que aparecem naturalmente quando traçamos raios.

	Desse modo:
	\begin{align*}
		     & \text{$AP$ é bissetriz de $\angle BAC$}\\
		\iff & \text{$D$ é ponto médio do arco $BC$}\\
		\iff & \text{$O, D, X$ são colineares}\\
		\iff & \angle QOX = \angle QOD\\
		\iff & {\color{xdxdff} \angle \mathrm{azul}} = {\color{qqwuqq} \angle \mathrm{verde}}\\
		\iff & {\color{xdxdff} \angle \mathrm{azul}} + {\color{ffqqqq} \angle \mathrm{vermelho}}
			   = {\color{qqwuqq} \angle \mathrm{verde}} + {\color{ffqqqq} \angle \mathrm{vermelho}}\\
		\iff & \angle QAM = \angle AQM\\
		\iff & MA = MQ
	\end{align*}

\begin{center}
\begin{tikzpicture}[line cap=round,line join=round,>=triangle 45,x=0.6cm,y=0.6cm]
\clip(-7.7,-15.4) rectangle (11.,3.3);
\draw [shift={(4.072151213421018,-1.9061742331499145)},line width=.4pt,color=qqwuqq,fill=qqwuqq,fill opacity=0.25000000178813934] (0,0) -- (-133.2591508421268:1.3533466320123562) arc (-133.2591508421268:-101.95704497690572:1.3533466320123562) -- cycle;
\draw [shift={(1.123089173771369,1.9513317669607155)},line width=.4pt,color=qqwuqq,fill=qqwuqq,fill opacity=0.25000000178813934] (0,0) -- (-108.58167821260392:1.3533466320123562) arc (-108.58167821260392:-77.27957234738284:1.3533466320123562) -- cycle;
\draw [shift={(-2.533726979967753,-8.92618267271153)},line width=.4pt,color=xdxdff,fill=xdxdff,fill opacity=0.1400000715255737] (0,0) -- (3.228178821974422:1.6916832900154453) arc (3.228178821974422:46.74084915787322:1.6916832900154453) -- cycle;
\draw [shift={(3.0661662311397055,-6.656470399708009)},line width=.4pt,color=ffqqqq,fill=ffqqqq,fill opacity=0.10000000149011612] (0,0) -- (78.04295502309428:1.804462176016475) arc (78.04295502309428:102.72042765261718:1.804462176016475) -- cycle;
\draw [shift={(1.123089173771369,1.9513317669607155)},line width=.4pt,color=ffqqqq,fill=ffqqqq,fill opacity=0.10000000149011612] (0,0) -- (-77.27957234738284:1.804462176016475) arc (-77.27957234738284:-52.60209971785996:1.804462176016475) -- cycle;
\draw [shift={(-2.533726979967753,-8.92618267271153)},line width=.4pt,color=ffqqqq,fill=ffqqqq,fill opacity=0.10000000149011612] (0,0) -- (46.74084915787322:1.804462176016475) arc (46.74084915787322:71.41832178739611:1.804462176016475) -- cycle;
\draw [shift={(4.072151213421018,-1.9061742331499145)},line width=.4pt,color=xdxdff,fill=xdxdff,fill opacity=0.14000000715255737] (0,0) -- (-133.2591508421268:1.6916832900154453) arc (-133.2591508421268:-89.74648050622798:1.6916832900154453) -- cycle;
\draw [line width=.4pt,color=yqyqyq] (4.072151213421018,-1.9061742331499145) circle (2.9133889207678982cm);
\draw [line width=.4pt,color=yqyqyq] (4.1015572525658754,-8.551939079888069) circle (3.987497942417069cm);
\draw [line width=.4pt] (4.072151213421018,-1.9061742331499145)-- (4.1015572525658754,-8.551939079888069);
\draw [line width=.4pt,dash pattern=on 2pt off 2pt,color=yqyqyq] (-1.6001362833266117,-3.1866046478271906) circle (3.4890063194744196cm);
\draw [line width=.4pt] (-0.44,-3.7)-- (8.6,-3.66);
\draw [line width=.4pt] (4.072151213421018,-1.9061742331499145)-- (-2.533726979967753,-8.92618267271153);
\draw [line width=.4pt] (1.123089173771369,1.9513317669607155)-- (-2.533726979967753,-8.92618267271153);
\draw [line width=.4pt] (-2.533726979967753,-8.92618267271153)-- (3.0661662311397055,-6.656470399708009);
\draw [line width=.4pt] (-2.533726979967753,-8.92618267271153)-- (8.94403187624301,-8.278813613395526);
\draw [line width=.4pt] (8.94403187624301,-8.278813613395526)-- (1.123089173771369,1.9513317669607155);
\draw [shift={(4.072151213421018,-1.9061742331499145)},line width=.4pt,color=qqwuqq] (-133.2591508421268:1.3533466320123562) arc (-133.2591508421268:-101.95704497690572:1.3533466320123562);
\draw [shift={(4.072151213421018,-1.9061742331499145)},line width=.4pt,color=qqwuqq] (-133.2591508421268:1.2067340802110176) arc (-133.2591508421268:-101.95704497690572:1.2067340802110176);
\draw [shift={(1.123089173771369,1.9513317669607155)},line width=.4pt,color=qqwuqq] (-108.58167821260392:1.3533466320123562) arc (-108.58167821260392:-77.27957234738284:1.3533466320123562);
\draw [shift={(1.123089173771369,1.9513317669607155)},line width=.4pt,color=qqwuqq] (-108.58167821260392:1.2067340802110176) arc (-108.58167821260392:-77.27957234738284:1.2067340802110176);
\draw [shift={(-2.533726979967753,-8.92618267271153)},line width=.4pt,color=xdxdff] (3.228178821974422:1.6916832900154453) arc (3.228178821974422:46.74084915787322:1.6916832900154453);
\draw[line width=.4pt,color=xdxdff] (-1.0770170085901134,-8.247386931329055) -- (-0.9236791168661516,-8.175934748025636);
\draw [shift={(3.0661662311397055,-6.656470399708009)},line width=.4pt,color=ffqqqq] (78.04295502309428:1.804462176016475) arc (78.04295502309428:102.72042765261718:1.804462176016475);
\draw[line width=.4pt,color=ffqqqq] (2.980668364500555,-4.938718820085577) -- (2.9722587382737538,-4.769759648319436);
\draw[line width=.4pt,color=ffqqqq] (3.128770652511292,-4.937732182026231) -- (3.1349284644494784,-4.768675963893597);
\draw [shift={(1.123089173771369,1.9513317669607155)},line width=.4pt,color=ffqqqq] (-77.27957234738284:1.804462176016475) arc (-77.27957234738284:-52.60209971785996:1.804462176016475);
\draw[line width=.4pt,color=ffqqqq] (1.9179569080298524,0.42615451184248043) -- (1.9961406195962625,0.2761370769128186);
\draw[line width=.4pt,color=ffqqqq] (1.7837923771986952,0.3634238189120643) -- (1.8487795775358098,0.20723615189088568);
\draw [shift={(-2.533726979967753,-8.92618267271153)},line width=.4pt,color=ffqqqq] (46.74084915787322:1.804462176016475) arc (46.74084915787322:71.41832178739611:1.804462176016475);
\draw[line width=.4pt,color=ffqqqq] (-1.7143210339863488,-7.414047007440214) -- (-1.6337237278242427,-7.265312351839757);
\draw[line width=.4pt,color=ffqqqq] (-1.5872639459516071,-7.490150604916329) -- (-1.4941692212942812,-7.34890154906762);
\draw [line width=.4pt] (3.0661662311397055,-6.656470399708009)-- (4.072151213421018,-1.9061742331499145);
\draw [shift={(4.072151213421018,-1.9061742331499145)},line width=.4pt,color=xdxdff] (-133.2591508421268:1.6916832900154453) arc (-133.2591508421268:-89.74648050622798:1.6916832900154453);
\draw[line width=.4pt,color=xdxdff] (3.4830739312711856,-3.401418545924584) -- (3.4210657963080466,-3.5588126841113907);
\draw [line width=.4pt] (3.0661662311397055,-6.656470399708009)-- (1.123089173771369,1.9513317669607155);
\begin{scriptsize}
\draw [fill=uququq] (1.123089173771369,1.9513317669607155) circle (1pt);
\draw[color=uququq] (1.172036180404428,2.280270607369959) node {$A$};
\draw [fill=uququq] (-0.44,-3.7) circle (1pt);
\draw[color=uququq] (-0.0813104516079282,-3.950123454864655) node {$B$};
\draw [fill=uququq] (8.6,-3.66) circle (1pt);
\draw[color=uququq] (8.91844465127424,-3.4263410242821415) node {$C$};
\draw [fill=uququq] (2.3959565356994568,-3.6874515197535422) circle (1pt);
\draw[color=uququq] (2.2496019316139014,-4.193237449089143) node {$P$};
\draw [fill=uququq] (4.072151213421018,-1.9061742331499145) circle (1pt);
\draw[color=uququq] (4.294510325232024,-1.554211516665049) node {$O$};
\draw [fill=uququq] (-2.533726979967753,-8.92618267271153) circle (1pt);
\draw[color=uququq] (-2.742892161232229,-9.313398873535888) node {$Q$};
\draw [fill=uququq] (4.1015572525658754,-8.551939079888069) circle (1pt);
\draw[color=uququq] (3.843394781227905,-8.907394883932183) node {$X$};
\draw [fill=uququq] (3.0661662311397055,-6.656470399708009) circle (1pt);
\draw[color=uququq] (3.2343887968223446,-7.036036283108705) node {$D$};
\draw [fill=uququq] (8.94403187624301,-8.278813613395526) circle (1pt);
\draw[color=uququq] (8.828221542473417,-8.681837111930122) node {$M$};
\end{scriptsize}
\end{tikzpicture}
\end{center}

	%\problem{math/crux/problems/4503}
	%\problem{math/crux/problems/4440}

\end{document}
