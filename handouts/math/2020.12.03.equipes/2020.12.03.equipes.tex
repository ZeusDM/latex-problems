\documentclass[10pt,a4paper]{article}
\usepackage[utf8]{inputenc}
\usepackage[brazilian]{babel}
\usepackage{lmodern}
\usepackage[left=2.5cm, right=2.5cm, top=2.5cm, bottom=2.5cm]{geometry}

\usepackage[]{zeus}

\title{Competição em Equipes}
\author{Guilherme Zeus Moura}
\mail{zeusdanmou@gmail.com}
\titlel{Turma Olímpica}
\titler{3 de dezembro de 2020}

\begin{document}	
	%\twocolumn[\zeustitle]
	\zeustitle

	\section{Instruções}

	\begin{enumerate}[label* = \textbf{\Roman*.}]
		\item Os problemas tem dificuldades variadas e a ordem foi aleatorizada usando \url{random.org}.
		\item Essa competição será feita com três equipes: Alvin (A), Simon (S) e Theodore (T).
		\item Essa competição terá duração de $3$ horas e $30$ minutos. Começaremos às 15:15 e terminaremos às 18:45. (Se a minha internet voltar, vou conversar com vocês às 18:45.)
		\item Vocês enviarão as suas submissões para os problemas através desse formulário:
			
			\begin{center}
				\url{https://forms.gle/aeeKao7s7HecVTXT9}
			\end{center}

		\item Vocês poderão acessar o placar de pontuações e submissões através dessa planilha: (se tudo der certo, ele será atualizado automaticamnte.)

			\begin{center}
				\footnotesize\url{https://docs.google.com/spreadsheets/d/1lwVlXJKQ8NdgT22gWyoIn-OPHg_YJTcC3qe2CAPJDbU/edit?usp=sharing}
			\end{center}

		\item Para cada problema, uma equipe pode submeter quantas soluções quiser: apenas a última submissão será contabilizada.
			\begin{enumerate}[label*=\textbf{\arabic*.}]
				\item a equipe que enviar a primeira solução correta ganha $10$ pontos, menos a quantidade de submissões que enviou anteriormente;
				\item a equipe que enviar a segunda solução correta ganha $7$ pontos, menos a quantidade de submissões que enviou anteriormente;
				\item a equipe que enviar a terceira solução correta ganha $4$ pontos, menos a quantidade de submissões que enviou anteriormente.
			\end{enumerate}
		\item No item anterior, uma equipe nunca ganha menos que $3$ pontos numa questão, caso a última submissão esteja correta.
		\item Poderão ser disponibilizadas pontuações parciais, dependendo dos conteúdos de todas as submissões.
			\begin{enumerate}[label* = \textbf{\arabic*.}]
				\item Caso nenhuma equipe tenha enviado uma solução correta, cada equipe pode ganhar no máximo $7$ (de $10$) pontos em pontuações parciais.
				\item Caso uma equipe tenha enviado uma solução correta, cada equipe pode ganhar no máximo $5$ (de $7$) pontos em pontuações parciais.
				\item Caso duas equipes tenham enviado soluções corretas, a terceira equipe pode ganhar no máximo $3$ (de $4$) pontos em pontuações parciais.
			\end{enumerate}
	\end{enumerate}

	\vfill

	\newpage
	\zeustitle

	\problem{math/usa/mo/1998/3}
	\problem{math/putnam/2018/a1}
	\problem{math/china/girls/2018/2}
	\problem{math/british/2017/round1/5}
	\problem{math/rmm/2015/1}
	\problem{math/apmo/2006/2}
	\problem{math/rmm/2011/1}

\end{document}
