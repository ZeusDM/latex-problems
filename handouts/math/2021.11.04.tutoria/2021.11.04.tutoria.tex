\documentclass[10pt,a4paper]{scrartcl}
\usepackage[utf8]{inputenc}
\usepackage[brazilian]{babel}
\usepackage{lmodern}
\usepackage{euler}
\usepackage[left=2.5cm, right=2.5cm, top=2.5cm, bottom=2.5cm]{geometry}

\usepackage[hide, problem-list]{zeus}

\setlength{\columnsep}{4em}

\title{Banco de Problemas para a Tutoria}
\author{Guilherme Zeus Dantas e Moura}
\mail{\href{https://guilhermezeus.com}{\texttt{guilhermezeus.com}}}
\titlel{\includegraphics[width=3cm]{MM_logo_c}}
\titler{4 de Novembro de 2021}

\begin{document}	
	%\twocolumn[\zeustitle]
	\zeustitle

	%\sloppy

	\problem{math/usa/tstst/2014/3}

	\begin{sk}
		A gente acha que a resposta são todos os polinômios da forma 
		\[
			P(T_8(\tfrac{x}{\sqrt{2}})),
		\] onde \(T_8\) é o \(8\)\textsuperscript{o} polinômio de Chebyshev e \(P\) é um polinômio qualquer.
		\[
			\cos(8x) = T_8(\cos x)
		\]
	\end{sk}

	\problem{math/usa/mo/2013/1}
	\problem{math/iran/geo/2018/a/1}
	\problem{math/iran/geo/2017/a/1}

	\newpage
	\problem{math/putnam/2014/a3}
	\begin{prob}
		Find the $2000$\textsuperscript{th} digit in the square root of $N = 11\dots1$, where $N$ contains $1998$ digits, all of them $1$'s.
	\end{prob}
\end{document}
