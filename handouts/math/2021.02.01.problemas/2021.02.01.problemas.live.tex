\documentclass[10pt,a4paper]{article}
\usepackage[utf8]{inputenc}
\usepackage[brazilian]{babel}
\usepackage[left=2.5cm, right=2.5cm, top=2.5cm, bottom=2.5cm]{geometry}
\usepackage[prob-boxed, dem-boxed]{zeus}
\usepackage{parskip}
\usepackage{transparent}

%\usepackage{lmodern}
%\usepackage{eulervm}
%\usepackage[scaled]{helvet}
%\usepackage[T1]{fontenc}
%\usepackage{mdsymbol}
%\usepackage[italic]{mathastext}

%\renewcommand{\familydefault}{\sfdefault}
%\usepackage[eulergreek, EULERGREEK]{sansmath}
%\sansmath


\usepackage[printwatermark]{xwatermark}
\newwatermark[firstpage, angle=0,scale=3,xpos=-69,ypos=118]{{\transparent{0.3}\includegraphics[scale=0.55]{pensi_pdf}}}

\title{Problemas Sortidos IV}
\author{Guilherme Zeus Dantas e Moura}
\mail{zeusdanmou@gmail.com}
\titlel{Turma Olímpica}
\titler{01 de Fevereiro de 2021}

\begin{document}	
	\zeustitle
	
	\problem{math/imosl/1998/N2}
	\begin{sol}
		As soluções são: $(0, x)$, $(x, 0)$, $(x, x)$ e $(m, m')$ para $x$ real e $m, m'$ inteiros. É fácil checar que elas funcionam.

		Suponha que $(a, b)$ não satisfaz nenhum dos casos acima. Suponha também, sem perda de generalidade, que $|a| > |b|$.

		Temos que, para todos os inteiros $n$, \[
			a\{bn\} = b\{an\}
		\]

		Como $\frac{a}{b} = \frac{\floor{an}}{\floor{bn}} \in \QQ$, podemos reescrever $\frac{a}{b} = \frac{p}{q}$ com $p, q$ inteiros primos entre si.
		Ficamos com, para todo $n$ inteiro, \[
			\frac{p}{q}\{bn\} = \left\{\frac{p}{q}bn\right\}.
		\]

		Repetindo o argumento acima, temos, para todo $m, k$ inteiros, \[
			\left(\frac{p}{q}\right)^k\{bmq^k\} = \left\{bmp^k\right\}.
		\]

		Escolha $k$ grande o suficiente tal que $\left(\frac{p}{q}\right)^k > 10^{10}$. Temos, então, para todo $m$ inteiro, \[
			10^{10} \{bmq^k\} < 1.
		\]

		Defina $\varepsilon = \{bq^k\} < 10^{-10}$. Se $\varepsilon \neq 0$, tomando $m_0 = \floor{\frac{1}{2\epsilon}}$, temos  $\{bm_0q^k\} \approx \frac{1}{2} \gg 10^{-10}$, um absurdo. Se $\varepsilon = 0$, temos $\{bq^k\} = \{bp^k\} = 0$, isto é, $bq^k, bp^k$ inteiros. Como $q, p$ são primos entre si, temos $b$ inteiro. Jogando $n = 1$ na equação orignial, temos $a$ também inteiro; um caso já analisado.


	\end{sol}

	\problem{math/memo/2019/2}
	\begin{sk}
		Vamos conjecturar casos pequenos:
		\begin{itemize}
			\item $n = 3$: $0$ pontos boêmios.
			\item $n = 4$: $1$ ponto boêmio.
		\end{itemize}
		\begin{conj}
			A resposta é $n-3$.
		\end{conj}

		O seguinte lema é a parte crucial da solução.

		\begin{lem}
			Se $A_i$ é boêmio em $A_1A_2A_3A_4$, então $A_i$ é boêmio em $A_1A_2A_3A_4A_5$.
		\end{lem}
		\begin{sk}
			Usar definição algébrica de feixe convexo.
		\end{sk}
	\end{sk}

	\problem{math/hmic/2016/2}
	\begin{sol}
		O raio do circuncírculo de $HMN$, $\omega$, é metade do raio do circuncírculo de $HAB$, por homotetia de centro $H$ e razão $2$.

		O raio do circuncírculo de $HAB$ é igual ao raio do circuncírculo de $ABC$, $\Omega$, por reflexão pela reta $AB$.

		Como $\omega$ passa pelo centro de $\Omega$, e tem metade de seu raio; $\omega$ e $\Omega$ são tangentes internas.
	\end{sol}

	\problem{math/egmo/2012/7}
	\begin{sol}
		Defina $H'$ como reflexão de $H$ por $BC$.


		Note que a composição das reflexões por $AB$ e por $BC$ é uma rotação por  $B$ com ângulo $2\angle B$. Como essas reflexões levam $L \to K \to M$, o ângulo $\angle LBM = 2\angle B$. Como $\triangle LBM$ é isósceles, $90^\circ - \angle B = \angle BLM = \angle BEM$.

		Defina $X$ como a segunda intersecção de $EM$ com o $\Gamma$. O ângulo $\angle BEX = 90^\circ - \angle B \implies \angle BAX = 90^\circ - \angle B \implies X = H'$.

		$E, M, H'$ são colineares $\iff$ $KH$, $EM$, $BC$ são concorrentes.
	\end{sol}

\end{document}
