\documentclass[aspectratio=169, handout]{beamer}
\usepackage{lmodern}
\usepackage[euler-digits]{eulervm}
\usefonttheme{serif}
\renewcommand*{\thefootnote}{\fnsymbol{footnote}}
\setbeamertemplate{navigation symbols}{}

\hypersetup{colorlinks = true}

\mode<presentation>
{
  %\usetheme{Warsaw}
  % or ...

  %\setbeamercovered{transparent}
  % or whatever (possibly just delete it)
}


\usepackage[brazil]{babel}

\usepackage[utf8]{inputenc}

\usepackage[T1]{fontenc}

\title[Aproximando potências]{Aproximando $\left(1+\sqrt{2}\right)^n$}

\subtitle{\vspace{1em}{\tiny Adaptado de Dmitry Fuchs, UC Davis, em Berkeley Math Circle 2015-2016\footnote{O material original está disponível \href{https://mathcircle.berkeley.edu/circle-archives/handouts/2015-2016}{aqui}.}.}}

\author
{Guilherme Zeus Dantas e Moura\\\tiny \href{mailto:zeusdanmou@gmail.com}{\texttt{zeusdanmou@gmail.com}}}

\institute{Matematicamente}

\date
{19 de abril de 2021}

%\pgfdeclareimage[height=0.5cm]{mm_2}{mm_2.png}
%\logo{\pgfuseimage{mm_2}}

% If you wish to uncover everything in a step-wise fashion, uncomment
% the following command: 

%\beamerdefaultoverlayspecification{<+->}


\begin{document}

\begin{frame}
  \titlepage
\end{frame}

\section{Esquentando}

\begin{frame}{Esquentando}{}
	Sem usar calculadora, tente aproximar:
	\begin{columns}
		\begin{column}{5cm}
			\begin{itemize}
				\item $\left(1+\sqrt{2}\right)^0$ \pause $= 1$
				\item $\left(1+\sqrt{2}\right)^1$ \pause $\approx  2{,}4142\dots$
				\item $\left(1+\sqrt{2}\right)^2$ \pause $\approx  5{,}8284\dots$
				\item $\left(1+\sqrt{2}\right)^3$ \pause $\approx 14{,}0711\dots$
				\item $\left(1+\sqrt{2}\right)^4$ \pause $\approx 33{,}9706\dots$
			\end{itemize}
		\end{column}
		\begin{column}{6cm}
			\begin{itemize}
				\item $\left(1+\sqrt{2}\right)^5$ \pause $\approx 82{,}0122\dots$
				\item $\left(1+\sqrt{2}\right)^6$ \pause $\approx 197{,}9949\dots$
				\item $\left(1+\sqrt{2}\right)^7$ \pause $\approx 478{,}0021\dots$
				\item $\left(1+\sqrt{2}\right)^8$ \pause $\approx 1153{,}9991\dots$
				\item $\left(1+\sqrt{2}\right)^9$ \pause $\approx 2786{,}0004\dots$
			\end{itemize}
		\end{column}
	\end{columns}

	\vspace{1em}
	\pause O número $\begin{pmatrix}1+\sqrt{2}\end{pmatrix}^n$ fica cada vez mais perto de um inteiro!
\end{frame}

\begin{frame}{Esquentando}{}
	\begin{columns}
		\begin{column}{5cm}
			\begin{itemize}
				\item $\frac{\left(1+\sqrt{2}\right)^0}{\sqrt{2}}$ \pause $\approx  0{,}7071$
				\item $\frac{\left(1+\sqrt{2}\right)^1}{\sqrt{2}}$ \pause $\approx  1{,}7071\dots$
				\item $\frac{\left(1+\sqrt{2}\right)^2}{\sqrt{2}}$ \pause $\approx  4{,}1213\dots$
				\item $\frac{\left(1+\sqrt{2}\right)^3}{\sqrt{2}}$ \pause $\approx  9{,}9497\dots$
				\item $\frac{\left(1+\sqrt{2}\right)^4}{\sqrt{2}}$ \pause $\approx 24{,}0208\dots$
			\end{itemize}
		\end{column}
		\begin{column}{6cm}
			\begin{itemize}
				\item $\frac{\left(1+\sqrt{2}\right)^5}{\sqrt{2}}$ \pause $\approx 57{,}9914\dots$
				\item $\frac{\left(1+\sqrt{2}\right)^6}{\sqrt{2}}$ \pause $\approx 140{,}0036\dots$
				\item $\frac{\left(1+\sqrt{2}\right)^7}{\sqrt{2}}$ \pause $\approx 337{,}9985\dots$
				\item $\frac{\left(1+\sqrt{2}\right)^8}{\sqrt{2}}$ \pause $\approx 816{,}0006\dots$
				\item $\frac{\left(1+\sqrt{2}\right)^9}{\sqrt{2}}$ \pause $\approx 1969{,}9997\dots$
			\end{itemize}
		\end{column}
	\end{columns}
	
	\vspace{1em}
	\pause O número $\frac{\left(1+\sqrt{2}\right)^n}{\sqrt{2}}$ fica cada vez mais perto de um inteiro!
\end{frame}

\begin{frame}{Por que esses números são quase inteiros?}{}
	\pause Vamos fazer casos pequenos!

	\pause Por que $\begin{pmatrix}1+\sqrt{2}\end{pmatrix}^5$ é tão próximo de $82$?

	\pause Sabemos que
	\[
		\left(1+x\right)^5 = 1 + 5x + 10x^2 + 10x^3 + 5x^4 + x^5.
	\]
	\pause Jogando $x = \sqrt{2}$,\[
		\begin{pmatrix}1+\sqrt{2}\end{pmatrix}^5 = 1 + 5\sqrt{2} + 20 + 20\sqrt{2} + 20 + 4\sqrt{2}.
	\]
	\pause Jogando $x = -\sqrt{2}$ \[
		\begin{pmatrix}1-\sqrt{2}\end{pmatrix}^5 = 1 - 5\sqrt{2} + 20 - 20\sqrt{2} + 20 - 4\sqrt{2}.
	\]
	\pause Somando as equações,\[
		\begin{pmatrix}1+\sqrt{2}\end{pmatrix}^5 + \begin{pmatrix}1-\sqrt{2}\end{pmatrix}^5 = 82.
	\]
\end{frame}

\begin{frame}{Por que esses números são quase inteiros?}{}
	Sabemos que
	\[
		\left(1+\sqrt{2}\right)^5 + \left(1-\sqrt{2}\right)^5 = 82.
	\]

	\pause Como $\left| 1-\sqrt{2} \right| < 1$, o número $\begin{pmatrix}1-\sqrt{2}\end{pmatrix}^5$ é bem pequeno.

	\[
		\left(1+\sqrt{2}\right)^5 \approx 82.
	\]
\end{frame}

\begin{frame}[t]{Por que esses números são quase inteiros?}
	$n = 5$ não é especial! Isso funciona para todo $n$.

	\alt<2>{
		\begin{itemize}
			\item $(1+\sqrt{2})^n + (1-\sqrt{2})^n$ é inteiro.
			\item $(1-\sqrt{2})^n$ é pequeno.
		\end{itemize}
	}{{\ \\[.6em]Alguém quer explicar?}}
\end{frame}

\begin{frame}{Por que esses números são quase inteiros?}
	E na divisão por $\sqrt{2}$?

	\pause
	\[
		\frac{\begin{pmatrix}1+\sqrt{2}\end{pmatrix}^5}{\sqrt{2}} = \frac{1}{\sqrt{2}} + 5 + 10\sqrt{2} + 20 + 10\sqrt{2} + 4
	\]
	\[
		\frac{\begin{pmatrix}1-\sqrt{2}\end{pmatrix}^5}{\sqrt{2}} = \frac{1}{\sqrt{2}} - 5 + 10\sqrt{2} - 20 + 10\sqrt{2} - 4
	\]

	\pause Podemos subtrair as equações \[
		\frac{\begin{pmatrix} 1+\sqrt{2} \end{pmatrix}^5}{\sqrt{2}} - \frac{\begin{pmatrix} 1 - \sqrt{2} \end{pmatrix}^5}{\sqrt{2}} = 58.
	\]
\end{frame}

\begin{frame}{Por que esses números são quase inteiros?}
	\[
		\frac{\begin{pmatrix} 1+\sqrt{2} \end{pmatrix}^5}{\sqrt{2}} - \frac{\begin{pmatrix} 1 - \sqrt{2} \end{pmatrix}^5}{\sqrt{2}} = 58.
	\]

	\pause \[
		\frac{\begin{pmatrix} 1+\sqrt{2} \end{pmatrix}^5}{\sqrt{2}} \approx 58.
	\]


	\pause De novo, $n = 5$ não é especial! Isso funciona para todo $n$.

	\begin{itemize}
		\item $\frac{(1+\sqrt{2})^n}{\sqrt{2}} - \frac{(1-\sqrt{2})^n}{\sqrt{2}}$ é inteiro.
		\item $\frac{(1-\sqrt{2})^n}{\sqrt{2}}$ é pequeno.
	\end{itemize}
\end{frame}

\begin{frame}{Isso só funciona com $1+\sqrt{2}$?}
	\pause Não!

	\pause	
	
	\begin{itemize}
		\item Também funciona com $\begin{pmatrix}1+\sqrt{3}\end{pmatrix}^n$ \pause e com $\frac{\left(1+\sqrt{3}\right)^n}{\sqrt{3}}$.

		\pause \item  $\begin{pmatrix}1+\sqrt{4}\end{pmatrix}^n$ também é próximo de inteiro, mas esse é menos interessante\dots

		\pause \item \alt<6>{Funciona com $\begin{pmatrix}1+\sqrt{5}\end{pmatrix}^n$?}{Funciona com $\begin{pmatrix}2+\sqrt{5}\end{pmatrix}^n$, pois $| 2 - \sqrt{5} | < 1$.}\pause
	\end{itemize}

	\pause Por exemplo, $\begin{pmatrix}1+\sqrt{3}\end{pmatrix}^{10} \approx 23167{,}956\dots$ e  $\begin{pmatrix}2+\sqrt{5}\end{pmatrix}^4 \approx 321{,}997\dots$.

	\vspace{3em}
	\pause Agora que desvendamos o porquê dessas calculações serem bem próximas de inteiros, \underline{\textbf{que perguntas podemos fazer?}}

\end{frame}

\begin{frame}{Que inteiros são esses?}	
	Agora que a gente sabe o motivo, vamos olhar para \textbf{quais} inteiros são esses. \pause

	\begin{table}\renewcommand{\arraystretch}{2}
		\begin{tabular}{r|cccccccccc}
			$n$ & $1$ & $2$ & $3$ & $4$ & $5$ & $6$ & $7$ & $8$ & $9$ & $10$ \\
			$\big(1+\sqrt{2}\big)^n$ (aprox.) & $2$ & $6$ & $14$ &  $34$ & $82$ & $198$ &  $478$ & $1154$ & $2786$ &  $6726$ \\
			$\frac{\left(1+\sqrt{2}\right)^n}{\sqrt{2}}$ (aprox.) & $2$ & $4$ & $10$ &  $24$ & $58$ & $140$ &  $338$ & $186$ & $1970$ &  $4756$ \\
		\end{tabular}
	\end{table}

	\pause Magicamente, observamos que, nas duas sequências da tabela acima, \[
		a_{n+2} = 2a_{n+1} + a_{n}.
	\]

	Você consegue provar esse fato?

	\pause \emph{\footnotesize Dica: Olhe para a sequência $b_n = (1+\sqrt{2})^n$, sem aproximar.}
\end{frame}

\begin{frame}{Que inteiros são esses?}
	Se definirmos $b_n = (1+\sqrt{2})^n$, sem aproximar\pause, também temos \[
		b_{n+2} = 2b_{n+1} + b_n,
	\]
	que é verdade pois \pause \[
		(1+\sqrt{2})^2 = 1 + 2\sqrt{2} + 2 = 2 (1+\sqrt{2}) + 1
	\]
	\pause e então, multiplicando por $(1+\sqrt{2})^n$ \[
		(1+\sqrt{2})^{n+2} = 2(1+\sqrt{2})^{n+1} + (1+\sqrt{2})^n.
	\]
\end{frame}

\begin{frame}{Que inteiros são esses?}
	Sabemos que \[
		b_{n+2} = 2b_{n+1} + b_n.
	\]
	
	\pause Também sabemos que:

	\begin{itemize}
		\item $a_{n+2}$ e $2a_{n+1} + a_n$ são inteiros. \pause
		\item $a_{n+2} \approx b_{n+2}$.
		\item $2a_{n+1} + a_n \approx 2b_{n+1} + b_n$.
	\end{itemize}

	\pause Logo, provamos que \[
		a_{n+2} = 2a_{n+1} + a_n.
	\]
\end{frame}

\begin{frame}{Resumo da brincadeira}
	A sequência $a_n$ é descrita das seguintes maneiras: \pause

	\begin{itemize}
		\item Motivação inicial: $a_n$ é o inteiro mais próximo de $(1+\sqrt{2})^n$, {\small para $n$ grande}. \pause
		\item Fórmula exata: $a_n = (1+\sqrt{2})^n + (1-\sqrt{2})^n$ \pause
		\item Recorrência: $\begin{cases} a_1 = 2, \\ a_2 = 6, \\ a_{n+2} = 2a_{n+1} + a_n\end{cases}$
	\end{itemize}

	\vspace{1em}
	\pause {\small \textit{Observação:} Dá pra fazer a mesma coisa com as sequências $\frac{\left(1+\sqrt{2}\right)^n}{\sqrt{2}}$, $(1+\sqrt{3})^n$, \dots}

	\vspace{1em}
	\pause Agora que desvendamos varios mistérios sobre a sequência $a_n$, \underline{\textbf{que novos caminhos podemos seguir?}}
\end{frame}

\begin{frame}{Mudando de perspectiva}
	O que fizemos até agora foi
	\begin{center}
		Motivação $\implies$ Fórmula Exata $\implies$ Recorrência.
	\end{center}

	\pause Será que conseguimos usar o nosso aprendizado para conseguir fazer
	\begin{center}
		Recorrência $\implies$ Fórmula Exata.
	\end{center}

	\pause Por exemplo, você deve conhecer a sequência de Fibonacci, definida por $F_0 = 0$, $F_1 = 1$ e  \[
		F_{n+2} = F_{n+1} + F_n.
	\]
	Como podemos achar uma fórmula exata?
\end{frame}

\begin{frame}{Uma nova quest!}
	Considere uma sequência $a_0, a_1, a_2, \dots$ que satisfaz, para todo $n \ge 0$,  \[
		a_{n+2} = K \cdot a_{n+1} + L \cdot a_n
	\]
	para $K$ e $L$ fixos. \pause Conseguimos descobrir uma fórmula direta para $a_n$?

	\pause Perceba que essa condição é bem simples. Ela diz que: \pause
	\[
		\begin{cases}
			a_2 = K a_1 + L a_0 \\
			a_3 = K a_2 + L a_1 \\
			a_4 = K a_3 + L a_2 \\
			\hspace{.5em} \vdots
		\end{cases}
	\]

\end{frame}

\begin{frame}{Uma nova quest!}
	\[
		\begin{cases}
			a_2 = K a_1 + L a_0 \\
			a_3 = K a_2 + L a_1 \\
			a_4 = K a_3 + L a_2 \\
			\hspace{.5em} \vdots
		\end{cases}
	\]
	
	Existem infinitas sequências. \pause Basta escolher $a_0$ e $a_1$ arbitrariamente.

	\vspace{1em}

	\pause Já que existem infinitas soluções, uma boa ideia é achar soluções bonitas!

	\vspace{1em}

	\pause Alguém tem alguma sugestão? \pause \emph{Dica:} Lembrem-se do que a gente já fez.
\end{frame}

\begin{frame}{Solução conveniente para a nova quest}
	Vamos considerar $a_n = \alpha^n$, com $\alpha$ constante. \pause A nossa condição vira
	
	\[
		\begin{cases}
			\alpha^2 = K \alpha + L \\
			\alpha^3 = K \alpha^2 + L \alpha \\
			\alpha^4 = K \alpha^3 + L \alpha^2 \\
			\hspace{.5em} \vdots
		\end{cases}
	\]

	\pause Todas essas equações são equivalentes a \[
		\alpha^2 = K \alpha + L.
	\]

	\pause Portanto, se pegarmos $\alpha$ como raiz do polinômio $x^2 - Kx - L$, a sequência $a_n = \alpha^n$ funciona!
\end{frame}

\begin{frame}{Achando novas soluções\dots}
	De modo geral, sabemos que $x^2 - Kx - L$ possui duas raízes, $\alpha$ e  $\beta$.

	\vspace{.5em}
	As sequências $\left(\alpha^n\right)$ e  $\left(\beta^n\right)$ funcionam!

	\vspace{.5em}
	\pause Quaisquer sequências $\left(A\alpha^n\right)$ e  $\left(B\beta^n\right)$ funcionam!

	\vspace{.5em}
	\pause A \emph{soma} das sequências, $\left(A\alpha^n + B\beta^n\right)$, também funciona!

	\vspace{.5em}
	\pause Isso é um ótimo progresso! \pause Achamos \textbf{várias} sequências que funcionam, mas será que achamos todas?

	\vspace{1em}
	\pause \emph{Dica:} o que define uma sequência que funciona?
\end{frame}

\begin{frame}{Achamos todas as soluções?}
	Achamos as soluções $\left(A\alpha^n + B\beta^n\right)$.

	\pause Uma solução é definida por $a_0$ e $a_1$. Se for da forma acima, terá que valer \[
		\begin{cases}
			a_0 = A + B\\
			a_1 = A\alpha + B\beta
		\end{cases}
	\]

	\pause Conseguimos reescrever como \[
		\begin{cases}
			A = \frac{a_0 \beta - a_1}{\beta - \alpha}\\
			B = \frac{a_0 \alpha - a_1}{\alpha - \beta}
		\end{cases}
	\]
\end{frame}

\begin{frame}{Juntando os pedaços\dots}
	Considere uma sequência $a_0, a_1, a_2, \dots$ que satisfaz $a_{n+2} = K a_{n+1} + L a_n$.

	\pause Sejam $\alpha$ e  $\beta$ as raízes de $x^2 - Kx - L = 0$.

	A sequência \[
		\left( \frac{a_0\beta - a_1}{\beta - \alpha} \cdot \alpha^n + \frac{a_0\alpha - a_1}{\alpha - \beta} \cdot \beta^n \right) 
	\]
	também satizfaz a recorrência, e também tem os mesmos termos iniciais.

	\vspace{.5em}
	\pause Portanto, como $a_0$ e $a_1$ definem unicamente uma solução, as duas sequências citadas são iguais!
	\pause Ou seja, \textbf{qualquer} sequência que satisfaz a recorrência é da forma $(A\alpha^n + B\beta^n)$. 

	\vspace{.5em}
	\pause {\tiny \emph{Observação:} Isso só funciona quando $\alpha \neq \beta$.}
\end{frame}
%
\begin{frame}{Hora de arregaçar as mangas}
	\begin{itemize}
		\item Calcule uma fórmula direta para o $n$-ésimo termo da sequência de Fibonacci, definida por $F_0 = 0$, $F_1 = 1$ e  \[
				F_{n+2} = F_{n+1} + F_n.
			\]
			
		\item Calcule uma fórmula direta para o $n$-ésimo termo da sequência definida por $a_0 = 1$, $a_1 = 5$ e  \[
				a_{n+2} = 4a_{n+1} - 3a_n.
			\]
	\end{itemize}
\end{frame}

\begin{frame}[t]{Hora de arregaçar as mangas}
	\begin{itemize}
		\item Calcule uma fórmula direta para o $n$-ésimo termo da sequência de Fibonacci, definida por $F_0 = 0$, $F_1 = 1$ e  \[
				F_{n+2} = F_{n+1} + F_n.
			\]
	\end{itemize}
	
\end{frame}

\begin{frame}[t]{Hora de arregaçar as mangas}
	\begin{itemize}
		\item Calcule uma fórmula direta para o $n$-ésimo termo da sequência definida por $a_0 = 1$, $a_1 = 5$ e  \[
				a_{n+2} = 4a_{n+1} - 3a_n.
			\]
	\end{itemize}
	
\end{frame}

\begin{frame}{Próximos passos?}

	Que novas perguntas podemos fazer? Que novos caminhos podemos seguir?\pause

	\vspace{1em}

	\begin{itemize}
		\item O que acontece quando o polinômio tem raíz dupla? Ou seja, $\alpha = \beta$? \pause
		\item Será que conseguimos resolver recorrências da forma $a_{n+3} = K \cdot a_{n+2} + L \cdot a_{n+1} + M \cdot a_n$? \pause
		\item É assim que a gente ensina um computador a resolver recorrências?
	\end{itemize}

	\vspace{1em}

	\pause \textbf{Agora vocês \emph{escolhem!}} Que pergunta vocês querem tentar responder?

	\pause {\tiny PS: Estou escondendo o fato de que eu não preparei essa última parte.}

\end{frame}

\begin{frame}{O que acontece quando o polinômio tem raíz dupla? Ou seja, $\alpha = \beta$?}
	Vamos fazer casos pequenos. Pegaremos $\alpha = \beta = 1$. O polinômio é $(x-1)^2 = x^2 - 2x + 1$.

	A sequência $a_0, a_1, a_2$ segue a recorrência \[
		a_{n+2} = 2a_{n+1} - a_n \iff \frac{a_{n+2} + a_{n}}{2} = a_{n+1}
	\]

	A sequência $(\alpha^n) = (1)$ funciona. Em geral, a sequência $(A\alpha^n) = (A)$ funciona.
	
	A sequência $(n)$ funciona. Em geral, a sequência  $(Bn)$ funciona.

	Finalmente, a sequência  $(A + Bn)$ funciona. \[
		\begin{cases}
			a_0 = A\\
			a_1 = A + B
		\end{cases}
		\iff 
		\begin{cases}
			A = a_0\\
			B = a_1 - a_0
		\end{cases}
	\]

\end{frame}

\begin{frame}{O que acontece quando o polinômio tem raíz dupla? Ou seja, $\alpha = \beta$?}
	Pegaremos $\alpha = \beta$. O polinômio é $(x-\alpha)^2 = x^2 - 2\alpha x + \alpha^2$.

	A sequência $a_0, a_1, a_2$ segue a recorrência \[
		a_{n+2} = \underbrace{(2\alpha)}_K a_{n+1} + \underbrace{(-\alpha^2)}_L a_n
	\]

	A sequência $(\alpha^n)$ funciona. Em geral, a sequência $(A\alpha^n)$ funciona.
	
	A sequência $(n\alpha^n)$ funciona. Em geral, a sequência  $(Bn\alpha^n)$ funciona.

	Finalmente, a sequência  $((A + Bn)\alpha^n)$ funciona. \[
		\begin{cases}
			a_0 = A\\
			a_1 = (A + B)\alpha
		\end{cases}
		\iff 
		\begin{cases}
			A = a_0\\
			B = \frac{a_1}{\alpha} - a_0
		\end{cases}
	\]

\end{frame}

\begin{frame}{É assim que a gente ensina um computador a resolver recorrências?}

	O que é um computador? \textbf{Modelo 1:}
	\begin{itemize}
		\item Uma moeda pra fazer operações: soma, multiplicação de reais.
		\item Conseguimos manipular e guardar reais.
	\end{itemize}

	Como calcular o $F_{n}$?
	\begin{itemize}
		\item Recorrência: $\sim n$ moedas.
		\item Fórmula direta: $\sim 2n$ moedas.
			\[
				F_n = \frac{1}{\sqrt5}\left( \left(\frac{1 + \sqrt5}{2}\right)^n - \left(\frac{1 - \sqrt{5}}{2}\right)^n \right)
			\]
	\end{itemize}
\end{frame}

\begin{frame}{Computador}
	Queremos contruir uma função $power(a, b)$, que calcula $a^b$.
	\begin{itemize}
		\item $power(a, 0) := 1$
		\item $power(a, 2n) := power(a, n)^2$
		\item $power(a, 2n+1) := power(a, 2n) \cdot a$
	\end{itemize}

	Quantas moedas gastamos?
	\begin{itemize}
		\item $\sim 2\log(n)$ moedas.
	\end{itemize}	
\end{frame}

\begin{frame}
	
	O que é um computador? \textbf{Modelo 2:}
	\begin{itemize}
		\item Uma moeda pra fazer operações: soma, multiplicação de inteiros.
		\item Conseguimos manipular e guardar inteiros.
	\end{itemize}

	Vamos criar uma nova classe dos números legais. Número legal é um número da forma $a + b\sqrt{5}$. Esse conjunto é chamado de $\mathbb{Z}[\sqrt{5}]$. No computador, a gente vai guardar o número $a + b\sqrt{5}$ como  $legal(a, b)$.

	\begin{itemize}
		\item $legal(a, b) + legal(c, d) := legal(a + c, b + d)$
		\item $legal(a, b) \cdot legal(c, d) := legal(ac + 5bd, ad + bc)$
		\item $(a + b\sqrt5)(c + d\sqrt5) = (ac + 5bd) + (ad + bc)\sqrt{5}$.
	\end{itemize}

	Calcular $power(legal(a, b), n)$ gasta $\sim 14\log n$ moedas. Por extenso:
	\begin{itemize}
		\item Primeiro calcule $legal(x, y) := power(legal(1, 1), n)$.
		\item $2^{n-1} := power(2, n-1)$.
		\item $F_n := y/2^{n-1}$.
	\end{itemize}
	Isso gasta algo na ordem de $\sim \log n$ moedas.
\end{frame}

\end{document}
