\documentclass[10pt,a4paper]{article}
\usepackage[utf8]{inputenc}
\usepackage[brazilian]{babel}
\usepackage{lmodern}
\usepackage[left=2.5cm, right=2.5cm, top=2.5cm, bottom=2.5cm]{geometry}
\usepackage[prob-boxed]{zeus}
\usepackage{parskip}
\usepackage{transparent}

\usepackage[printwatermark]{xwatermark}
\newwatermark[firstpage, angle=0,scale=3,xpos=-69,ypos=118]{{\transparent{0.3}\includegraphics[scale=0.55]{pensi_pdf}}}

\title{Problemas Sortidos II}
\author{Guilherme Zeus Dantas e Moura}
\mail{zeusdanmou@gmail.com}
\titlel{Turma Olímpica}
\titler{11 de Janeiro de 2021}

\begin{document}	
	\zeustitle
	
	\problem{math/metropolises/2018/4}

	\begin{sol}
		Vamos definir a seguinte sequência: $a_0 = 1$,  \[
			a_{n+1} = 2a_{n} + (-1)^n.
		\]

		É fácil demonstrar que essa sequência é crescente.

		Seja $m$ o menor inteiro positivo tal que $a_m$ não divide $4k$.

		Vamos provar que $a_m - 1, a_m + 1$ dividem $4k$, usando que os termos anteriores da sequência dividem $4k$.

		\begin{enumerate}[label = \textbullet]
			\item $m = 0$ nunca acontece.
			\item Se $m = 1$, então $3 \nmid 4k$, mas $2, 4\mid 4k$.
			\item Se $m > 1$, então 
				\[
					2a_{m-1} \mid 4k \implies a_m + (-1)^m \mid 4k.
				\]
				\[
					4a_{m-2} \mid 4k \implies 2a_{m-1} + 2(-1)^{m-1} \mid 4k \implies a_m - (-1)^m \mid 4k.
				\]
		\end{enumerate}
	\end{sol}

	\problem{math/putnam/2018/b1}
	\begin{sk}
		Trocar 100 por $4k$. A resposta são os vetores $
			\begin{pmatrix}
				1 \\\text{par}
			\end{pmatrix}
		$.
		Provar por indução em $k$.
	\end{sk}
	\problem{math/putnam/2019/b1}
	\begin{ans}
		$5n + 1$.
	\end{ans}

\end{document}
