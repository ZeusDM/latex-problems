\documentclass[10pt,a4paper]{article}
\usepackage[utf8]{inputenc}
\usepackage[brazilian]{babel}
\usepackage{amsmath}
\usepackage{amsfonts}
\usepackage{amssymb}
\usepackage{graphicx}
\usepackage{enumitem}
\usepackage{lmodern}
\usepackage{fullpage}

\usepackage[stylish, persection]{zeusproblems}
\usepackage{zeuscolor}
\usepackage{zeusall}

\colorlet{main}{DarkCyan}
\pagestyle{empty}

%\titleformat{\section}[hang]{\vspace{-0.5em}}{}{1em}{\large\bfseries\sffamily}[]

\title{Equação de Pell Generalizada}
\author{Guilherme Zeus; Lecturer: Rafael Filipe}
\nomail
\titler{27 de janeiro de 2020}
\titlel{Semana Olímpica 2020}

\renewcommand\playerA[1]{Guilherme}
\renewcommand\playerB[1]{Zeus}

\newcommand\sD{\sqrt{D}}

\begin{document}
	\zeustitle

	\section{Equação de Pell}

	Uma equação de Pell é uma equação do tipo $$x^2 - D \cdot y^2 = 1,$$ com D não quadrado perfeito, para variáveis $x$ e $y$ inteiros positivos.

	\begin{thm}
		Existe solução para toda equaçãoo de Pell.
	\end{thm}

	\begin{thm}
		Existe uma solução minimal $(x_0, y_0)$, que vamos representar como $x_0 + y_0 \sqrt{D}$.
		Todas as soluções são da forma $$x_n + y_n \sqrt{D} = (x_1 + y_1 \sqrt{D})^n$$
	\end{thm}
	
	\subsection{Como achar soluções para a equação de Pell}

	\begin{itemize}
		\item Usar frações contínuas.
	\end{itemize}

	\subsection{Olhar as soluções da Equação de Pell como uma recorrência}

	\begin{gather*}
	x_n + y_n \sD = (x_1 + y_1 \sD)^n \\ 
	x_n - y_n \sD = (x_1 - y_1 \sD)^n \\
	\end{gather*}

	Logo:

	\begin{gather*}
		x_n = \frac{(x_1 + y_1 \sD)^n + (x_1 + y_1 \sD)^n}{2} \\
		y_n = \frac{(x_1 + y_1 \sD)^n - (x_1 + y_1 \sD)^n}{2\sD}
	\end{gather*}

	Portanto, ambas as seqências $(x_n)$ e $(y_n)$ seguem a recorrência:

	\begin{equation*}
		u_{n+2} - 2x_1 u_{n+1} + u_n = 0
	\end{equation*}

	\section{Equação de Pell Generalizada}

	Agora, vamos resolver equações do tipo \[x^2 - D y^2 = C.\]

	Nesse caso, nem sempre tem solução (e.g., $x^2 - 3y^2 = 7$ não possui solução\footnote{Basta ver módulo 4.}). Além disso, mesmo quando existe uma solução, não existe necessariamente uma única solução minimal, mas na verdade várias soluções minimais e uma família de soluções para cada uma delas.

	\begin{thm}
		Seja $(x_0, y_0)$ a solução da equação de Pell $x^2 - D y^2 = 1$.

		Se $(x, y)$ é solução de $x^2 - D y^2 = C$, então \[x + y \sD = (u + v \sD) (x_1 + y_1 \sD)^k,\] com $k$ inteiro e $u^2 - D v^2 = C$, \[u + v \sD < (x_1 + y_1 \sD) \sqrt{|C|}\]
	\end{thm}

	Usando o teorema acima, podemos achar todas as soluções dessa generalização da equação de Pell.

	\section{Um Truque Especial}

	\begin{thm}
		Em uma recorrência de segunda ordem $a_{n+2} + ba_{n+1} + ca_n = 0$, seja $p$ um primo e $T_p$ o período dessa sequência módulo $p$. Podemos achar $T_p$ usando que:
		\begin{itemize}
			\item Se $\left(\frac{b^2 - 4c}{p}\right) = 1$, então $T_p | p-1$.
			\item Se $\left(\frac{b^2 - 4c}{p}\right) = 1$, então $T_p | p^2-1$.
			\item Se $b^2 - 4c \equiv 0 \pmod{p}$, então $p | T_p$ e $T_p | p(p-1)$.
		\end{itemize}
	\end{thm}
\end{document}
