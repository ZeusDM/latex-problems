\documentclass[10pt, a4paper]{article}
\usepackage[utf8]{inputenc}
\usepackage[brazilian]{babel}
\usepackage{lmodern}
\usepackage[left=2cm, right=2cm, top=2cm, bottom=2.5cm]{geometry}
\usepackage{indentfirst}
\usepackage[inline]{enumitem}

\usepackage{pgf,tikz,pgfplots}
\pgfplotsset{compat=1.15}
\usepackage{mathrsfs}
\usetikzlibrary{arrows}

\usepackage[pensi,
			problem-list
			]{zeus}

\title{Problemas Sortidos de Combinatória -- Live}
\author{Guilherme Zeus Moura}
\mail{zeusdanmou@gmail.com}
\titlel{Turma Olímpica}
\titler{{\footnotesize v. 1} -- 1 de Julho de 2020}

\renewcommand{\playerA}[1]{Guilherme}
\renewcommand{\playerB}[1]{Zeus}

\newcommand{\sep}{
	
	\begin{center}
		\vspace{-.6em}
		\rule{10cm}{.5pt}
		\vspace{-.3em}
	\end{center}

}

\begin{document}	
	\zeustitle

	\problem{math/imosl/2016/C1}

	\sep

	\begin{hnt}
		Como em muitos problemas de matemática, especialmente de combinatória, faça casos pequenos.
		
		(Inclusive, uma boa ideia é fazer casos pequenos para entender o enunciado.)
	\end{hnt}

	Vamos jogar? Eu vou assumir o papel de ambos líder e vice-líder, e você vai assumir o papel do competidor tentando adivinhar a sequência original.

	\textbf{Caso 1.} O líder anuncia $n=5$ e $k=2$, secretamente conta a sequência de $5$ bits escolhida para o vice-líder. Então, o vice-líder escreve num papel as sequências 00111, 11010, 01000, 01110, 10011, 00001, 11001, 01101, 11111, 00010.
	
	(João Arthur conseguiu acertar a sequência original (01011) com uma única tentativa.)

	\textbf{Caso 2.} O líder anuncia $n=2$ e $k=1$, secretamente conta a sequência de $2$ bits escolhida para o vice-líder. Então, o vice-líder escreve num papel as sequências 00, 11.

	\sep

	\textbf{Pergunta chave.} Quantas vezes o $i$-ésimo bit é invertido ou não é inverido, quando olhamos as sequências do vice-líder?
	\textit{Resposta.} Aparece invertido $\binom{n-1}{k-1}$ vezes, e aparece não-invertido $\binom{n-1}{k}$.

	Logo, se $\binom{n-1}{k-1} \neq \binom{n-1}{k}$, podemos identificar o $i$-ésimo bit vendo qual bit aparece $\binom{n-1}{k-1}$ vezes (que vai ser o bit invertido) e qual bit aparece $\binom{n-1}{k}$ vezes (que vai ser o bit original). 

	Como $\binom{n-1}{k-1} \neq \binom{n-1}{k} \iff n \neq 2k$, falta só analisarmos o caso em que $n = 2k$.

	\sep

	Estamos agora no caso $n = 2k$.
	
	Sejam $a$ e $b$ duas sequências de $2k$ bits. Suponha que $a$ e $b$ geram a mesma lista (a lista de sequências geradas pelo vice-líder). As sequências diferenciam em $\ell$ posições, i.e., são iguais em $2k - \ell$ posições.
	
	Vamos supor que $\ell \ge k$. Vamos criar uma sequência $c$ que é igual a $a$, com a diferença que, dentre as $\ell$ posições em que $a$ e $b$ diferenciam, $k$ posições são escolhidas para serem invertidas. Como $c$ é uma sequência derivada da sequência $a$, com exatamente $k$ posições inveritdas, $c$ está na lista do vice-líder. Mas, $c$ difere de $b$ em $\ell - k$ posições. Como $c$ também está na lista relativa a $b$, é necessário que $\ell - k = k \implies \ell = 2k \implies$ $a$ e $b$ são sequências inversas.

	Algo análogo pode ser feito no caso $\ell \le k$, que chegará na conclusão de que $a$ e $b$ são sequências iguais. (Minha recomendação é que você tente escrever essa parte por completo.)

	Além disso, uma sequência $a$ e sua sequência inversa $a'$ sempre geram a mesma lista. (Minha recomndação é que você mostre por completo porque isso acontece.)

	\sep

	Logo, o competidor pode garantir acertar com número de movimentos máximo $
	\begin{cases}
		2, \text{ se $n = 2k$}\\ 
		1, \text{ caso contrário}
	\end{cases}$.
	\vfill

	\note{Um problema interessante que usa essa noção de distância (\emph{diferir em $k$ posições}) é o (EGMO 2013, 6).}

	\newpage
	\problem{math/metropolises/2017/2}

	Seja $100 = k$. Vamos provar que $d = 2k$.

	Dizemos que $(v_0, v_1, v_2, \dots, v_t)$ é um caminho se e somente se $v_i$ tem um voo, sem escala, para $v_{i+1}, \forall i \in \{0, 1, 2, \dots, t-1\}$. Adicionalmente, dizemos que esse caminho liga $v_0$ e $v_t$, além de ter tamanho $t$ (número de voos).

	Para duas cidades $u$ e $v$, definimos $d(u, v)$ como o menor tamanho de caminho entre todos os caminhos que ligam $u$ e $v$. Definimos $d_2(u, v)$ como o menor tamanho de caminho entre todos os caminhos que ligam $u$ e $v$ que possuem tamanho par. 

	Definimos o \emph{parâmetro de um grafo $G$}, com símbolo $p(G)$, como o valor máximo de $d_2(u, v)$, variando $u$ e $v$ dentre as cidades (vértices) de $G$.

	Estamos procurando o menor $d$ para o qual se pode sempre afirmar que $d_2(u, v) \le d$, para quaisquer duas cidades $u$ e $v$ de qualquer grafo $G$ que satisfaça as condições iniciais do enunciado. Em outras palavras, estamosprocurando o menor $d$ para o qual $p(G) \le d$, para todo grafo $G$ que satisfaça as condições iniciais do enunciado. Em outras palavras (novamente), queremos achar o valor máximo de $p(G)$, variando $G$ dentre os grafos que satisfazem as confições iniciais do enunciado.

	\sep

	Vamos olhar para o grafo $G_1$ a seguir. O conjunto dos vértices de $G_1$ é $\{0, 1, 2, 3, \dots, k+1\}$.

	\vspace{-.3em}

	\begin{center}
\definecolor{ududff}{rgb}{0.30196078431372547,0.30196078431372547,1.}
	\begin{tikzpicture}[line cap=round,line join=round,>=triangle 45,x=1.0cm,y=1.0cm]
\clip(-1.1,-1.5) rectangle (10.3,1.5);
\draw [line width=.5pt] (-1.,1.)-- (-1.,-1.);
\draw [line width=.5pt] (0.,0.)-- (-1.,-1.);
\draw [line width=.5pt] (-1.,1.)-- (0.,0.);
\draw [line width=.5pt] (0.,0.)-- (1.4142135623730951,0.);
\draw [line width=.5pt] (1.4142135623730951,0.)-- (2.8284271247461903,0.);
\draw [line width=.5pt] (2.8284271247461903,0.)-- (4.242640687119286,0.);
\draw [line width=.5pt] (7.0710678118654755,0.)-- (8.485281374238571,0.);
\draw [line width=.6pt,dotted] (4.242640687119286,0.)-- (7.0710678118654755,0.);
\begin{scriptsize}
\draw [fill=ududff] (0.,0.) circle (2.5pt);
\draw[color=ududff] (0.281096557169435,0.39169648205450097) node {$2$};
\draw [fill=ududff] (-1.,-1.) circle (2.5pt);
\draw[color=ududff] (-0.5070863403145677,-1.0064864154295024) node {$1$};
\draw [fill=ududff] (-1.,1.) circle (2.5pt);
\draw[color=ududff] (-0.7170863403145677,1.4115790077446784) node {$0$};
\draw [fill=ududff] (1.4142135623730951,0.) circle (2.5pt);
\draw[color=ududff] (1.6915723905707434,0.39169648205450097) node {$3$};
\draw [fill=ududff] (2.8284271247461903,0.) circle (2.5pt);
\draw[color=ududff] (3.1020482239720515,0.39169648205450097) node {$4$};
\draw [fill=ududff] (4.242640687119286,0.) circle (2.5pt);
\draw[color=ududff] (4.5342236855795335,0.39169648205450097) node {$5$};
\draw [fill=ududff] (7.0710678118654755,0.) circle (2.5pt);
\draw[color=ududff] (7.35517535238215,0.39169648205450097) node {$k$};
\draw [fill=ududff] (8.485281374238571,0.) circle (2.5pt);
\draw[color=ududff] (8.8958489550205,0.39169648205450097) node {$k+1$};
\end{scriptsize}
\end{tikzpicture}
\end{center}

\vspace{-1.7em}

Confira que $G_1$ satisfaz as condições iniciais do enunciado, ou seja, 

\begin{itemize}
	\item confira que, para qualquer par de cidades, existe um caminho ligando elas com tamanho menor ou igual a $k$. Em outras palavras, confira que $d(u, v) \le k$.
	\item confira que, para qualquer par de cidades, existe um caminho ligando elas com tamanho par.
\end{itemize}

Confira que $d_2(k, k+1) = 2k$, e que essa é, de fato, o maior valor de $d_2(u, v)$ em $G_1$. Consequentemente, $p(G_1) = 2k \implies \max p(G) \ge 2k$.

\definecolor{qqzcbt}{rgb}{0.,0.511764705882353,0.9019607843137254}
\definecolor{ffxfqq}{rgb}{1.,0.4980392156862745,0.}
\definecolor{dcrutc}{rgb}{0.8627450980392157,0.0784313725490196,0.23529411764705882}
\definecolor{wzwebf}{rgb}{0.4117647058823529,0.43137254901960786,0.7490196078431373}
\definecolor{ududff}{rgb}{0.30196078431372547,0.30196078431372547,1.}
\sep

Vamos supor que existe um grafo $G_2$, que satizfaz as condiçoões iniciais do enunciado, e tal que $p(G_2) = 2t > 2k$. Isso significa que existem dois vértices (que chamaremos de $A$ e $B$) tal que $d_2(A, B) = 2t$. Um dos caminhos de tamanho $2t$ que liga $A$ e $B$ está desenhado abaixo em pontilhado (note que existem vértices dentro desse caminho que não estão desenhados). Seja $M$ o vértice que se encontra exatamente na metade desse caminho.

$M$ divide o caminho pontilhado naturalmente em dois subcaminhos. O primeiro subcaminho, que liga $A$ e $M$, está representado na figura com \textcolor{ffxfqq}{laranja}, enquanto o segundo subcaminho, que liga $M$ e $B$ está representado na figura com \textcolor{qqzcbt}{azul}. Ambos esses subcaminhos tem tamanho exatamente $t$.

Como $d(A, M) \le k$ (pelo enunciado), existe um caminho com tamanho \textcolor{dcrutc}{$x$} $\le k$ que liga $A$ e $M$. Analogamente, existe um caminho com tamanho \textcolor{green}{$y$} $\le k$ que liga $M$ e $B$.

Se juntarmos os caminhos \textcolor{dcrutc}{vermelho} e \textcolor{qqzcbt}{azul}, temos um caminho com tamanho $x + t$ (que é menor que $2t$) que liga $A$ e $B$. Como $d_2(A, B) = 2t$, $x + t$ deve ser ímpar. Analogamente, $y + t$ deve ser impar. Juntando, temos que $x + y$ é par.

Porém, podemos juntar os caminhos \textcolor{dcrutc}{vermelho} e \textcolor{green}{verde}, criando um caminho com tamanho $x + y$ (que é par e é menor que $2t$) que liga $A$ e $B$. Chegamos numa contradição!

\begin{center}
\begin{tikzpicture}[line cap=round,line join=round,>=triangle 45,x=.8cm,y=.6cm]
\clip(-0.5,-2.6) rectangle (10.5,0.7);
\draw [shift={(7.4949579389545935,-0.004873319879858882)},line width=.7pt,color=green]  plot[domain=-3.1396472516311587:0.0019454019586344629,variable=\t]({1.*2.505046801329921*cos(\t r)+0.*2.505046801329921*sin(\t r)},{0.*2.505046801329921*cos(\t r)+1.*2.505046801329921*sin(\t r)});
\draw [shift={(2.4949579389545935,-0.004873319879858882)},line width=.7pt,color=dcrutc]  plot[domain=3.1396393887246288:6.2812320423144214,variable=\t]({1.*2.4949626983983557*cos(\t r)+0.*2.4949626983983557*sin(\t r)},{0.*2.4949626983983557*cos(\t r)+1.*2.4949626983983557*sin(\t r)});
\draw [line width=.7pt,color=ffxfqq] (0.,0.)-- (5,0);
\draw [line width=.7pt,color=qqzcbt] (5,0)-- (10.,0.);
\draw [line width=.7pt,dash pattern=on 1pt off 2pt, opacity = 0.6] (0.,0.)-- (10.,0.);
\begin{scriptsize}
\draw [fill=ududff] (0.,0.) circle (2.5pt);
\draw[color=ududff] (0.19765507345988829,0.4659924630126601) node {$A$};
\draw [fill=ududff] (10.,0.) circle (2.5pt);
\draw[color=ududff] (10.202716273875386,0.4659924630126601) node {$B$};
\draw [fill=ududff] (4.989915877909187,-0.009746639759717764) circle (2.5pt);
\draw[color=ududff] (5.192839813755437,0.4659924630126601) node {$M$};
\draw[color=green] (7.5, -2) node {$y$};
\draw[color=dcrutc] (2.5, -2) node {$x$};
\draw[color=ffxfqq] (2.5952474436076627,0.27784214406626808) node {$t$};
\draw[color=qqzcbt] (7.590432183903212,0.26315042424186943) node {$t$};
\end{scriptsize}
\end{tikzpicture}
\end{center}

Portanto, não existe grafo $G_2$ com essa propriedade, o que significa que o valor máximo de $p(G)$ é exatamente $2k$.

\end{document}
