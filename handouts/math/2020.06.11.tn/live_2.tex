\documentclass[10pt, a4paper]{article}
\usepackage[utf8]{inputenc}
\usepackage[brazilian]{babel}
\usepackage{lmodern}
\usepackage[left=2cm, right=2cm, top=2cm, bottom=2.5cm]{geometry}
\usepackage{indentfirst}
\usepackage[inline]{enumitem}

\usepackage[stylish, pensi, hide]{zeus}

\title{Problemas Sortidos de Teoria dos Números}
\author{Guilherme Zeus Moura}
\mail{zeusdanmou@gmail.com}
\titlel{Turma Olímpica}
\titler{{\footnotesize v. 1} -- 11 de Junho de 2020}

\setlength{\parindent}{0pt}

\newcommand{\divides}{\ |\ }

\begin{document}	

	\setcounter{prob}{6}

	\problem{math/nz_training/2010/1_nt/6}

	Primeiro, por Fermat, $7^{p-1} \equiv 1 \pmod{p}$. Logo, $\frac{7^{p-1} - 1}{p}$ é inteiro pra todo $p \neq 7$.

	Vamos falar que $7^{p-1} - 1 = n^2p$.

	Olhando $\pmod{3}$, sabemos que $7^{p-1} - 1 \equiv 0 \equiv n^2p.$ Para $p \neq 3$, $n$ é múltiplo de $3$. Desse modo, olhando $\pmod{9}$, sabemos que $0 \equiv n^2p \equiv 7^{p-1} - 1$. Como em $\pmod{9}$, $7^1 \equiv 7$, $7^2 \equiv 4$, $7^3 \equiv 1$, temos que $p = 3k + 1$. Como $p$ é primo, $p = 6k+1$.

	\textbf{Conclusão.} $p = 3$ ou $p = 6k + 1$.

	\begin{itemize}
		\item Se $p=3$, temos que $\frac{7^2 - 1}{3} = 16$, que é quadrado perfeito.
		
		\item Se $p = 6k + 1$. Logo, $7^6 - 1 \ |\  7^{6k} - 1 = 7^{p-1} - 1$.

		Isto é, $7^6 - 1 = (7^3 - 1)(7^3 + 1) = (7 - 1)(7^2 + 7 + 1)(7^3 + 1) = 6 \times 57 \times 344 = 2^4 \times 3^2 \times 19 \times 43$

		Sabemos que $6$ divide $n^2$, logo $6$ divide $n$. Seja $n = 6m$.

		\begin{align*}
			n^2p &= 7^{6k} - 1\\
				 &= (7^{3k} - 1)(7^{3k} + 1)\\
				 &= (7^k-1)(7^{2k} + 7^k + 1)(7^k+1)(7^{2k} - 7^k + 1)
		\end{align*}

		\begin{enumerate}[label = \textbf{\alph*.}]
			\item $(7^k - 1, 7^k + 1) = (7^k - 1, 2) = 2$
			\item $(7^k - 1, 7^{2k} - 7^k + 1) = (7^k - 1, 1) = 1$
			\item $(7^k - 1, 7^{2k} + 7^k + 1) = (7^k - 1, 2\cdot7^k + 1) = (7^k - 1, -3) = 3$
			\item $(7^k + 1, 7^{2k} + 7^k + 1) = (7^k + 1, 1) = 1$
			\item $(7^k + 1, 7^{2k} - 7^k + 1) = (7^k + 1, -2\cdot7^k + 1) = (7^k + 1, 3) = 1$
			\item $(7^{2k} + 7^k + 1, 7^{2k} - 7^k + 1) = (7^{2k} + 7^{k} + 1, 2\cdot7^k) = 1$ 
		\end{enumerate}

		\[ m^2 p = \frac{(7^k-1)}{6} \frac{(7^{2k} + 7^k + 1)}{3} \frac{(7^k+1)}{2} (7^{2k} - 7^k + 1)\]

		\begin{enumerate}[label = \textbf{Caso \Roman*.}]
			\item \[\begin{cases}7^k - 1 = 6x^2 \\ 7^k + 1 = 2y^2\end{cases}\]

				\[\begin{cases} 7^k = 3x^2 + y^2 \\ 1 = y^2 - 3x^2 \end{cases}\]

				A solução minimal é $(y_1, x_1) = (2, 1)$. Logo, $(y_t, x_t)$ é tal que $y_t + \sqrt{3} x_t = \left(2 + \sqrt{3}\right)^t$. Logo:
				\[y_t = \frac{(2 + \sqrt{3})^t + (2-\sqrt{3})^t}{2}\]
				\[x_t = \frac{(2 + \sqrt{3})^t - (2-\sqrt{3})^t}{2\sqrt{3}}\]

				Ambos $y$ e $x$ seguem a recorrência $a_n - 4a_{n-1} + a_{n-2} = 0$.
		\end{enumerate}

		

		\newpage



		%\textbf{Tentativa 2.}

		%\begin{align*}	
		%	n^2p &= (7^{3k} - 1) (7^{3k} + 1)\\
		%	(3m)^2 p &= \frac{(7^{3k} - 1)}{2} \frac{(7^{3k} + 1)}{2}
		%\end{align*}

		%\begin{enumerate}[label = \textbf{Caso \Roman*.}]
		%	\item \[\begin{cases} 7^{3k} - 1 = 2 x^2 p\\ 7^{3k} + 1 = 2y^2 \end{cases}\]
		%		\[\begin{cases} 7^{3k} = x^2 p + y^2 \\ 1 = x^2 p - y^2\end{cases}\]
		%		
		%\end{enumerate}
	\end{itemize}

	%\[n^2p = (7-1)(7^{p-2} + \cdots + 7 + 1)\]

	%Vamos chamar $n = 6m$.

	%\[6m^2p = (7^{p-2} + \cdots 7 + 1)\]
	
	
	%\begin{prob}%[Colombia 2009]
	%	Encontre todas as triplas de inteiros positivos $(a, b, n)$ que satisfazem a \[a^b = 1 + b + \cdots + b^n.\]
	%\end{prob}
	\setcounter{prob}{8}
	\problem{math/nz_training/2010/1_nt/7}
	
	Sabemos que, $x^{12} \equiv 1 \text{ ou } 0 \pmod{13}$, para todo $x$. Consequentemente, $x^{6} \equiv -1, 0 \text{ ou } 1 \pmod{13}$. Logo, \[x^2 + y^2 \equiv -2, -1, 0, 1 \text{ ou } 2 \pmod{13}.\]

	\begin{lem}
		Se $n \equiv 3, 4, 5, 6, 7, 8, 9 \text{ ou } 10$, então $n$ não pode ser escrito como soma de potências sextas.
	\end{lem}

	\begin{lem}[Fermat's theorem on sums of two squares]
		$n$ é soma de dois quadrados se, e somente se, $\nu_p(n)$ é par para todo $p = 4k + 3$ primo.
	\end{lem}

	Queremos infinitos $n$ tais que \[ n = a^2 + b^2 = c^3 + d^3. \]

	\textbf{Solução 1.}

	Seja $p$ primo tal que $p \equiv 1 \pmod{4}$ e $p \equiv 6 \pmod{7}$. Por Dirichilet, existem infititos primos com essa propriedade.

	\begin{itemize}
		\item Sabemos que $p^3$ pode ser escrito como soma de quadrados, pelo Lema 2.
		\item Sabemos que $p^3 + 0^3$ é soma de dois cubos.
		\item Sabemos que $p^3 \equiv 6 \pmod{7}$, que implica que $p^3$ não é soma de duas potências sextas. 
	\end{itemize}

	\textbf{Solução 2.}

	\[1^3 + 2^3 = 9 = 3^2 + 0^2.\]

	\[k^6 (1^3 + 2^3) = k^6 \cdot 9 = k^6 (3^2 + 0^2)\]

	\[(k^2)^3 + (2 k^2)^3 = k^6 \cdot 9 = (3 k^3)^2 + 0^2\]

	Note que $k^6 \cdot 9 \equiv 2 \pmod{7}$, logo não é soma de duas potências sextas.

	\textbf{Solução 3.}

	\[ (5\cdot 5)^3 + (5 \cdot 10)^3 = 25 \cdot 3^2 \cdot 5^4 = (3 \cdot (3^2 \cdot 5^4))^2 + (4 \cdot (3^2 \cdot 5^4))^2 \]

	\setcounter{prob}{10}

	\newpage
	\problem{math/nz_training/2010/1_nt/9}

	Umas soluções encontradas foram $(p, q) = (11, 31)$ ou $(42, 127)$ ou $(19, 73)$ ou $(17, 257)$.

	\[\begin{cases} 11 \divides 2^5 + 1  \\ 31 \divides 2^5 - 1\end{cases}\qquad
	\begin{cases} 43 \divides 2^7 + 1  \\ 127 \divides 2^7 - 1\end{cases}\qquad
	\begin{cases} 19 \divides 2^9 + 1  \\ 73 \divides 2^9 - 1\end{cases}\qquad
	\begin{cases} 257 \divides 2^8 + 1 \\ 17 \divides 2^8 - 1\end{cases}\]

	\[\begin{cases} p \divides 2^n + 1  \\ q \divides 2^n - 1 \\ n \divides p - 1 \\ p-1 \divides q - 1 \end{cases}\]

	\begin{lem}
		Sejam $p$ e $q$ primos e $n$ natural. Se $q \divides 2^n - 1$, $n \divides p-1$ e $p-1 \divides q-1$, então esse par $(p, q)$ satisfaz o enunciado.
	\end{lem}

\end{document}
