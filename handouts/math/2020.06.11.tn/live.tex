\documentclass[10pt, a4paper]{article}
\usepackage[utf8]{inputenc}
\usepackage[brazilian]{babel}
\usepackage{lmodern}
\usepackage[left=2cm, right=2cm, top=2cm, bottom=2.5cm]{geometry}
\usepackage{indentfirst}
\usepackage[inline]{enumitem}

\usepackage[stylish, pensi, hide]{zeus}

\title{Problemas Sortidos de Teoria dos Números}
\author{Guilherme Zeus Moura}
\mail{zeusdanmou@gmail.com}
\titlel{Turma Olímpica}
\titler{{\footnotesize Ao Vivo em} 11 de Junho de 2020}

\renewcommand{\playerA}[1]{Guilherme}
\renewcommand{\playerB}[1]{Zeus}

\begin{document}	
	\zeustitle
	
	\begin{enumerate}[label = \textbullet]
		\item Esse foi o arquivo gerado ao vivo na aula de 11 de Junho de 2020.
		\item As soluções aqui não são soluções completas, são somente esboços de soluções.
			É esperado que você argumente um pouco melhor quando for escrever as suas soluções.
	\end{enumerate}

	\newpage
	\problem{math/nz_training/2010/1_nt/1}
	\[5^p + (2p^2)^2 = n^2\]
	\[5^p = n^2 - (2p^2)^2\]
	\[5^p = (n - 2p^2)(n + 2p^2)\]
	
	\[\begin{cases} 5^a = n - 2p^2 \\ 5^b = n + 2p^2 \\ a + b = p \end{cases}\]

	\[\begin{cases} 2n = 5^a + 5^b \\ 4p^2 = 5^b - 5^a \\ a + b = p \end{cases}\]

	Se $a, b > 0$, então $4p^2$ tem fator $5$, que implica $p = 5$.

	Se $a = 0$ e $p \neq 5$, então:

	\[\begin{cases} 2n = 5^p + 1 \\ 4p^2 = 5^p - 1 \end{cases} \]

	Olhando a segunda equação $\pmod{p}$: \[ 0 \equiv 4, \] logo $p = 2$.

	Testando $p = 5$ e $p = 2$, vemos que só $p = 5$ funciona.
	
	\newpage
	\problem{math/nz_training/2010/1_nt/2}
	
	%\[a_n = a + 40^{1!} + 40^{2!} + \cdots + 40^{n!}\]

	\[ 40^{\phi(2009)} \equiv 1 \pmod{2009}\]

	Para todo $n \ge \phi(2009)$, vale que $40^{n!} \equiv 40^{\phi(2009) \cdot k} \equiv 1 \pmod{2009}$.

	\newpage
	\problem{math/nz_training/2010/1_nt/3}

	\begin{defn}
		$(a, b)$ denota o maior divisor comum entre $a$ e $b$.
	\end{defn}

	\begin{thm}[Teorema Útil] 
		\[(a, b) = (a, b + ka)\]
	\end{thm}

	\begin{align*}
		1	&= (4n+1, kn+1)\\
			&= (4n + 1, n(k-4))\\
			&= (4n + 1, k-4)
	\end{align*}

	Seja $p$ um primo.
	
	Se $p = 4l + 1$, então jogando $n = l$, $p$ não divide $k-4$.

	Se $p = 4l + 3$, então $3p = 12l + 9 = 4(l + 2) + 1$. Jogando $n = l+2$, temos que $p$ não divide $k-4$.

	Logo, o único primo que pode dividir $k-4$ é $2$. Em outras palavras, $k - 4 = \pm 2^\alpha$, para $\alpha$ inteiro não negativo.

	Testando $k = 4 \pm 2^\alpha$, sempre funciona! 

	\newpage
	\problem{math/nz_training/2010/1_nt/4}

	Vamos escrever $n = p_1^{\alpha_1} \cdot p_2^{\alpha_2} \cdot \cdots \cdot p_k^{\alpha_k}$.

	Um divisor arbitrário de $n$ é da forma $p_1^{\beta_1} \cdot p_2^{\beta_2} \cdot \cdots \cdot p_k^{\beta_k}$, com $0 \le \beta_i \le \alpha_i$, para todo $i \in \{1, 2, \dots, k\}$.
	
	Logo, a soma dos divisores é \[s(n) = (1 + p_1 + p_1^2 + \cdots + p_1^{\alpha_1})\cdot(1 + p_2 + p_2^2 + \cdots + p_2^{\alpha_2})\cdot\cdots\cdot(1 + p_k + p_k^2 + \cdots + p_k^{\alpha_k}).\]

	Logo, o número de divisores é \[d(n) = (\alpha_1 + 1) \cdot (\alpha_2 + 1)\cdot\cdots\cdot(\alpha_k + 1).\]
	
	\[s(k) = 2^x\]

	\[(1 + p + p^2 + \cdots p^\alpha) = 2^y\]

	\[p^{\alpha + 1} - 1 = 2^y (p - 1)\]

Seja $\alpha + 1 = 2^j \cdot \ell$, $\ell$ ímpar.

	\[p^{2^j \ell} - 1 = 2^y (p - 1)\]

	\[(p^{2^{j-1}\ell} + 1) (p^{2^{j-1}\ell} - 1) = 2^y (p-1)\]

	\[(p^{2^{j-1}\ell} + 1) (p^{2^{j-2}\ell} + 1) (p^{2^{j-2}\ell} - 1) = 2^y (p-1)\]

	\[(p^{2^{j-1}\ell} + 1) (p^{2^{j-2}\ell} + 1) \cdots (p^{\ell} + 1) (p^{\ell} - 1) = 2^y (p-1)\]

	\[(p^{2^{j-1}\ell} + 1) \cdot (p^{2^{j-2}\ell} + 1) \cdots (p^{\ell} + 1) \cdot \frac{(p^{\ell} - 1)}{(p-1)} = 2^y \]

	\[  \frac{(p^{\ell} - 1)}{(p-1)} = p^{\ell - 1} + \cdots + 1 = 2^z\]

	\[ \ell = 1\]

	\[ \alpha + 1 = 2^j \]

	\[ d(n) = 2^h \]
	
	\newpage
	\problem{math/canada/1976/5}

	Seja $n = 2^j \ell$, com $\ell$ ímpar.

	\[ 2^j \ell = n + (n + 1) + (n + 2) + \cdots + (n + i), \text{ para algum $n$ e $i$ inteiros positivos}\]

	\[ 2^{j+1} \ell = (2n+i)(i+1)\]

	\textbf{Ida:} Como $2n + i$ e $i+1$ tem paridades diferentes, logo, $k$ não é potência de $2$.

	\textbf{Volta:} Seja $a$ e $b$ o maior e o menor entre ${2^j+1, \ell}$. Como $\ell \ge 3$, podemos fazer \[\begin{cases} a = 2n + i \\ b = i + 1 \end{cases}\] que possui solução pois $a$ e $b$ tem paridades diferentes.

	%\end{lem}

	%\begin{lem}[Volta]
	%	Se $k$ não é potência de $2$, então existe 

	%\end{lem}

	\newpage
	\problem{math/nz_training/2010/1_nt/5_alt}

	\textbf{Valores de $k$ encontrados.} (Casos Pequenos)

	\begin{itemize}
		\item $k = 3$: $(2, 2)$, $(2, 3)$, $(3, 6)$
		\item $k = 4$: $(1, 1)$, $(1, 2)$, $(2, 6)$
	\end{itemize}

	%\textbf{Quando $a$ e $b$ são coprimos?}

	%\[ \text{$a$ e $b$ são coprimos quando } (a, b) = 1\]

	%Logo $(a, b) = 1$ \emph{não implica} que $a \not|\ b$
	
	%\vspace{1em}

	%\textbf{Tentativa de Solução 1, usando mdc e divisibilidade.}

	%\[ k = \frac{m^2 + m + n^2 + n}{mn} \]

	%\[ mn \ |\ m + n + m^2 + n^2\]

	%\[ mn \ |\ m + n + m^2 + 2mn +  n^2\]

	%\[ mn \ |\  (m + n)(m + n + 1)\]
	
	%Seja $d = (m, n)$ e $m = dm_0$, $n = dn_0$.
	
	%\[d^2 m_0 n_0 \ |\ (dm_0 + dn_0)(dm_0 + dn_0 + 1)\]

	%\[d m_0 n_0 \ |\ (m_0 + n_0)(dm_0 + dn_0 + 1)\]

	%Isso implica que (note que não é se, e somente se)

	%\[\begin{cases} d \ |\ m_0 + n_0 \\ m_0 \ |\ dn_0 + 1 \\ n_0 \ |\ dm_0 + 1   \end{cases}\]

	%\newpage
	\textbf{Solução.}

	Suponha que existam $n_0 \ge n_1$ inteiros tais que a $(m, n) \mapsto (n_0, n_1)$ valida a equação.

	\[n_0n_1k = n_0^2 + n_0 + n_1^2 + n_1\]
	\[0 = n_0^2 - (n_1k - 1)n_0 + (n_1^2 + n_1)\]
	
	Sejam $n_0$ e $n_2$ as raízes do polinômio $P(x) = x^2 - (n_1k-1)x + (n_1^2 + n_1)$.

	Por soma e produto, \[ \begin{cases} n_0 + n_2 = n_1k - 1 \\ n_0 \cdot n_2 = n_1(n_1+1) \end{cases}\]

	Se $n_0 > n_1$, então $n_1 + 1 > n_2$, ou seja, $n_1 \ge n_2$.
	
	Além disso, \[0 = n_2^2 - (n_1k - 1)n_2 + (n_1^2 + n_1)\]

	Logo, $(m, n) \mapsto (n_1, n_2)$ também valida a equação.

	(Note que podemos definir o resto da sequência de forma análoga. Fazendo $n_3$ em função do par $(n_1, n_2)$, e assim por diante.)

	\textbf{Conclusão:} Se $n_0 > n_1$ tal que $(n_0, n_1)$ satisfaz a equação, então $n_1 \ge n_2$ tal que $(n_1, n_2)$ também satisfaz a equação. Em outras palavras, em três termos consecutivos, se o primero é maior que o segundo, então o segundo é maior ou igual que o terceiro.

	Isso implica que, enquanto $n_i > n_{i+1}$, $n_{i+1} \ge n_{i+2}$. Isto é, enquanto dois termos consecutivos são diferentes, a sequência não cresce. Como é uma sequência de inteiros positivos, essa sequência não pode decrescer para sempre, temos que $n_j = n_{j+1}$, para algum $j$. Como $(n_j, n_{j+1}) = (n, n)$ satisfaz a equação, concluímos que \[ k = \frac{2(n+1)}{n}, \]
	que só admite solução se $n = 1$ ou $n = 2$ e, respectivamente, $k = 4$ ou $k = 3$.  

	\begin{rem}
		Se quiser estudar mais sobre essa técnica, procure por \emph{Descida de Fermat} ou \emph{Vieta Jumping} ou \emph{root flipping}.
	\end{rem}

	%\newpage
	%\textbf{Tentativa de Solução 3.}

	%Suponha que existam $m$ e $n$ inteiros tais que a equação é válida.

	%\[mnk = m^2 + m + n^2 + n\]
	%\[0 = m^2 - (nk - 1)m + (n^2 + n)\]

	%\[2m = nk - 1 \pm \sqrt{(nk-1)^2 - 4(n^2+n)} \]

	%\[(nk-1)^2 - 4n(n+1) = t^2\]

	%\[(nk-1)^2 - t^2 = 4n(n+1)\]

	%\[(nk - 1 + t)(nk - 1 - t) = 4n(n+1)\]

	%\[\frac{nk - 1 + t}{2} \cdot \frac{nk - 1 - t}{2} = n(n+1)\]

\end{document}
