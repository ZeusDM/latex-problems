\documentclass[10pt, a4paper]{article}
\usepackage[utf8]{inputenc}
\usepackage[brazilian]{babel}
\usepackage{lmodern}
\usepackage[left=2cm, right=2cm, top=2cm, bottom=2.5cm]{geometry}

\usepackage{../../../commands/problems}
\renewcommand{\mypath}{../../../}

\title{Distâncias}
\author{Guilherme Zeus Moura}
\mail{zeusdanmou@gmail.com}
\titlel{}
\titler{}

\begin{document}	
	\zeustitle
	\section{O que é distância?}
	\begin{defn}[Função distância]
		Seja $S$ um conjunto. \emph{Distância} é uma função $d: S \times S \to [0, +\infty)$, tal que, para quaisquer $x, y, z \in S$:
		\begin{itemize}
			\item $d(x, y) = 0 \iff x = y$;
			\item $d(x, y) = d(y, x)$;
			\item $d(x, y) \le d(x, z) + d(z, y)$.
		\end{itemize} 
	\end{defn}
	\subsection{Distância Euclidiana}
	\begin{defn}[Distância Euclidiana]
		A distância euclidiana é definida como $d: \RR^n \times \RR^n \to [0, +\infty)$:
		$$d(x, y) = \sqrt{ \sum_{i=1}^n (x_i - y_i)^2 }.$$
	\end{defn}
	\subsection{Distância de Manhattan}
	\begin{defn}[Distância de Manhattan]
		A distância de Manhattan é definida como $d: \RR^n \times \RR^n \to [0, +\infty)$:
		$$d(x, y) = \sum_{i=1}^n |x_i - y_i| .$$
	\end{defn}

\end{document}
