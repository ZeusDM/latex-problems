\documentclass[11pt, a4paper]{article}
\usepackage[utf8]{inputenc}
\usepackage[brazilian]{babel}
\usepackage{lmodern}
\usepackage[left=2cm, right=2cm, top=2cm, bottom=2.5cm]{geometry}
\usepackage{indentfirst}
\usepackage[inline]{enumitem}

\usepackage{zeus}

%\title{Combinatória \& Divisibilidade}
\title{Briefing: Estudando Sistemas de Votações}
\author{Guilherme Zeus Dantas e Moura}
\mail{zeusdanmou@gmail.com}
\titlel{\includegraphics[width = .2\textwidth]{mm.png}}
%\titler{09 de Novembro de 2020}

\begin{document}	
	\zeustitle

	No Reino Animal, há uma onda de revoltas contra o poder absoluto da monarquia: os cidadãos querem mais controle sobre como são governados. Cedendo à pressão, o rei Leão abdica de seu trono e a (ex-)rainha Leoa será a responsável para instalar um sistema democrático que irá escolher quem assumirá o trono.

	A Leoa declara que:
	\begin{itemize}
		\item Todos os cidadãos ganham um, e somente um, voto.
		\item O candidato com mais votos ganha o trono.
	\end{itemize}
	
	Um sistema simples, justo e lógico. Certo? Os candidatos da primeira rodada do novo sistema são:
	\begin{itemize}
		\item o antigo rei Leão;
		\item o Leopardo;
		\item o Gorila;
		\item a Coruja;
		\item a Tartaruga.
	\end{itemize}

	É um tempo animador no Reino Animal, com todos os cidadãos ansiosos para exercerem seus recém-adiquiridos poderes de voto. Após as contagens dos votos, os resultados são os seguintes:
	\begin{itemize}
		\item Gorila: 30\%
		\item Leopardo: 25\%
		\item Leão: 20\%
		\item Coruja: 14\%
		\item Tartaruga: 11\%
	\end{itemize}

	E assim, o Gorila é coroado o novo líder da Democracia Animal. Porém:
	\begin{itemize}
		\item $70\%$ dos eleitores prefeririam outro líder que não o Gorila;
		\item Os eleitores dos Leopardo e Leão, que somam $55\%$ do eleitorado (maioria), compartilham os similares pensamentos do Partido Felino. Mesmo sendo maioria (e numa disputa de Felinos vs. Gorilas, claramente sairem na frente), eles perderam a eleição para o candidato Gorila.
	\end{itemize}

	\noindent \textbf{Questionamento.} Será que há um outro sistema que resolve esses problemas? Que novos problemas esse sistema cria? Existe um algoritmo que transforma preferências individuais em preferências de grupo de um jeito ``matematicamente superior'' (em algum sentido de ``superior'' que você gostaria de atribuir)?

\end{document}
