\documentclass[10pt,a4paper]{article}
\usepackage[utf8]{inputenc}
\usepackage[brazilian]{babel}
\usepackage[left=2cm, right=2cm, top=2cm, bottom=2cm]{geometry}
\usepackage[problem-list]{zeus}
\usepackage{parskip}
\usepackage{transparent}
\usepackage{lmodern}
\usepackage{multicol}
\usepackage[euler-digits, euler-hat-accent]{eulervm}

\setlength{\columnsep}{3em}

\title{Tutoria}
\author{Guilherme Zeus Dantas e Moura}
\mail{zeusdanmou@gmail.com}
\titlel{\rlap{Matematicamente Internacionais}\smash{\raisebox{-2.4cm}{\transparent{.3}\includegraphics[width = 3cm]{mm_2}}}}
\titler{31 de julho de 2021}

\pagestyle{empty}

\begin{document}	
 	\zeustitle

	\begin{prob}%[TST2 Cone Sul 2020, 4]
		Seja $ABC$ um triângulo, e $D$ um ponto em seu interior. Definimos o ponto $A_0$ como ponto médio do arco $BDC$, na circunferência que passa por $B$, $C$ e $D$. Da mesma maneira, defina $B_0$ como o ponto médio do arco $ADC$ e $C_0$ como o ponto médio do arco $ADB$. Prove que existe uma única circunferência que passa por $D$, $A_0$, $B_0$, $C_0$.
	\end{prob}

	\begin{prob}
		Seja $A$ um ponto fora de uma circunferência $\omega$. Por $A$ são traçadas duas retas, cada uma intersecta $\omega$ em dois pontos.
		A primeira intersecta $\omega$ em $B$ e $C$, a segunda intersecta em $D$ e $E$ ($D$ está entre $A$ e $E$). A reta por $D$ paralela a $BC$ corta $\omega$ novamente em $F$.
		A reta $AF$ corta $\omega$ novamente em  $T$, diferente de $F$. As retas $BC$ e $ET$ se encontram em $M$.
		O ponto $N$ é tal que $M$ é o ponto médio de $AN$. Seja $K$ o ponto médio de $BC$.
		Prove que os pontos $D$, $E$, $K$ e $N$ estão sobre uma mesma circunferência.
	\end{prob}

	\begin{prob}
		Seja $n$ um inteiro positivo dado.
		Determine o menor valor possível do inteiro positivo $m$ ($m > n$) para o qual o conjunto $M = \{n, n + 1, n + 2, \dots, m\}$ pode ser particionado em subconjuntos disjuntos de maneira que em cada um destes subconjuntos, existe um elemento que é igual à soma de todos os outros elementos.
	\end{prob}

	\begin{prob}%[TST3 Cone Sul 2020, 3]
		Seja $a_1$, $a_2$, $\dots$, $a_n$ uma sequência de inteiros positivos satisfazendo $a_1=1$ e $a_{n+1}=a_n+a_{\floor{\sqrt{n}}}$ para todo inteiro positivo $n$. Prove que para todo inteiro positivo $k$, existe um inteiro positivo $m$ tal que $a_m$ é divisível por k.
	\end{prob}

\end{document}
