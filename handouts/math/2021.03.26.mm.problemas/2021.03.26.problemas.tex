\documentclass[10pt,a4paper]{article}
\usepackage[utf8]{inputenc}
\usepackage[brazilian]{babel}
\usepackage[left=2cm, right=2cm, top=2cm, bottom=2cm]{geometry}
\usepackage[problem-list]{zeus}
\usepackage{parskip}
\usepackage{transparent}
\usepackage{lmodern}
\usepackage{multicol}
\usepackage[euler-digits, euler-hat-accent]{eulervm}

\setlength{\columnsep}{3em}

\title{Problemas sabor Combinatória \\ \normalsize Aviso aos alérgicos: \normalfont Pode conter Álgebra, Geometria e Teoria dos Números. Não contém glúten.}
\author{Guilherme Zeus Dantas e Moura}
\mail{zeusdanmou@gmail.com}
\titlel{\rlap{Matematicamente Internacionais}\smash{\raisebox{-2.4cm}{\transparent{.3}\includegraphics[width = 3cm]{mm_2}}}}
\titler{26 de Março de 2020}

\pagestyle{empty}

\begin{document}	
 	\zeustitle

	\problem{math/miklos_schweitzer/2017/1}
	\problem{math/miklos_schweitzer/2020/1}
	\renewcommand\playerA[1]{Bean}
	\renewcommand\playerB[1]{Dagm{\ae}r}
	\problem{math/imosl/2009/C5}
	\renewcommand\playerA[1]{Elfo}
	\renewcommand\playerB[1]{Luci}
	\problem{math/putnam/2020/b2}

	\begin{center}
		\section*{Problemas para os que estão entediados}
	\end{center}

	\problem{math/imosl/2014/C2}
	\problem{math/imosl/2014/C4}
	\problem{math/imosl/2009/C4}
	\problem{math/miklos_schweitzer/2008/3}

\end{document}
