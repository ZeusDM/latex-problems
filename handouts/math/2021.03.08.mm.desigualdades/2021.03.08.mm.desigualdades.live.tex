\documentclass[10pt,a4paper]{article}
\usepackage[utf8]{inputenc}
\usepackage[brazilian]{babel}
\usepackage[left=2cm, right=2cm, top=2cm, bottom=2cm]{geometry}
\usepackage[prob-boxed, hide]{zeus}
%\usepackage{parskip}
\usepackage{transparent}
\usepackage{lmodern}
\usepackage[euler-digits, euler-hat-accent]{eulervm}

\setlength{\columnsep}{3em}

\title{Desigualdades}
\author{Guilherme Zeus Dantas e Moura}
\mail{zeudanmou@gmail.com}
\titlel{\rlap{Matematicamente}\smash{\raisebox{-2.4cm}{\transparent{.3}\includegraphics[width = 3cm]{mm_2}}}}
\titler{Matemática para Pensar}

\begin{document}	
	%\twocolumn[
	\zeustitle

	\begin{center}
		\begin{minipage}{14cm}
			\itshape
			Os problemas desta lista foram majoritamente retirados das listas do Emiliano Augusto Chagas (Semana Olímpica 2015) e Onofre Campos (Semana Olímpica 2001).
		\end{minipage}
	\end{center}

	\vspace{1ex}
	%]

	\begin{prob}
		Seja $x > 1$ um número real. Prove que
		\[
			\frac{1}{\sqrt{x}} < \sqrt{x+1} - \sqrt{x-1}
		\]
	\end{prob}

	\begin{prob}
		Sejam $a, b$ reais positivos. Prove que
		\[
			\frac{a+b}{1+a+b} < \frac{a}{1+a} + \frac{b}{1+b}.
		\]
	\end{prob}

	\begin{prob}
		Seja $x$ um número real. Prove que $
			x^6 + 2 \ge x^4 + 2x.
		$
	\end{prob}

	\begin{prob}
		Sejam $a, b, c, d $ reais positivos. Prove que o número \[
			\frac{a}{a + b + d} + 
			\frac{b}{a + b + c} + 
			\frac{c}{b + c + d} + 
			\frac{d}{a + c + d}
		\]
		está estritamente entre $1$ e $2$.
	\end{prob}

	\begin{prob}
		Seja $n$ um inteiro maior que $2$. Prove que \[
			\frac{1}{1^2} + \frac{1}{2^2} + \cdots + \frac{1}{n^2} < \frac{7}{2} - \frac{1}{n}.
		\]
	\end{prob}

	\begin{prob}
		Seja $n$ um inteiro maior que $1$. Prove que \[
			\frac{1}{2\sqrt{n}} < \frac{1}{2}\cdot\frac{3}{4}\cdots\frac{2n-1}{2n} < \frac{1}{\sqrt{2n+1}}.
		\]
	\end{prob}

	\begin{prob}
		Sejam $x, y$ reais. Prove que \[
			x^2 - 2xy + 6y^2 - 12x + 2y + 41 \ge 0.
		\]
	\end{prob}

	\begin{prob}
		Sejam $a, b, c$ os tamanhos dos lados de um triângulo. Mostre que a função \[
			f(x) = b^2x^2 + \left(b^2 + c^2 - a^2\right)x + c^2
		\]
		é positiva, para todo real $x$.
	\end{prob}

	\begin{prob}\label{prob:2-mq-ma-mg-mh}
		Sejam $a, b$ reais positivos. Prove e determine o caso de igualdade das seguintes inequações
		\begin{enumerate}[label = (\alph*), itemsep = 0ex]
			\item \ \vspace{-\baselineskip}\[
					\frac{a+b}{2} \ge \sqrt{ab}.
				\]

			\item \ \vspace{-\baselineskip}\[
					\sqrt{ab} \ge \frac{2}{\frac{1}{a} + \frac{1}{b}}.
				\]

			\item \ \vspace{-\baselineskip}\[
					\sqrt{\frac{a^2 + b^2}{2}} \ge \frac{a + b}{2}.
				\]
		\end{enumerate}
	\end{prob}
	
	\begin{prob}
		Sejam $a, b$ reais tais que $a^3 + b^3 = 2$. Prove que $a + b \le 2$.
	\end{prob}

	\begin{prob}
		Se $a, b$ são reais positivos tais que $a + b = 1$, prove que $27ab^2 \le 4$ e determine quando ocorre a igualdade.
	\end{prob}

	\begin{prob}
		Se $x$ é um real positivo, qual é o valor mínimo que \[
			x + \frac1x
		\] pode atingir? Para que valores de $x$ esse mínimo é atingido?
	\end{prob}

	\begin{prob}
		Sejam $a, b, c$ reais positivos. Prove que \[
			\frac{a}{b^2} + \frac{b}{a^2} \ge \frac{1}{a} + \frac{1}{b}.
		\]
	\end{prob}

	\begin{prob}
		Sejam $a, b$ reais positivos tais que $ab = 1$. Prove que \[
			\left(1+\frac1a\right)
			\left(1+\frac1b\right)
			\ge 4.
		\]
	\end{prob}


	\begin{prob}
		Generalize as inequações do \cref{prob:2-mq-ma-mg-mh} para três reais positivos e prove as inequações generalizadas.
	\end{prob}

	\begin{prob}
		Sejam $a, b, c$ reais. Prove que \[
			a^2 + b^2 + c^2 \ge ab + bc + ca.
		\]
	\end{prob}

	\begin{prob}
		Sejam $a, b, c$ reais. Prove que \[
			a^4 + c^4 + b^4 \ge abc(a + b + c).
		\]
	\end{prob}

	\begin{prob}
		Sejam $a, b, c$ reais positivos. Prove que \[
			a + b + c \le \frac{bc}{a} + \frac{ca}{b} + \frac{ab}{c}.
		\]
	\end{prob}

	\begin{prob}
		Variando $a, b, c$ sobre os reais positivos, determine o valor mínimo de \[
			\frac{bc}{a} + 
			\frac{ac}{b} + 
			\frac{ab}{c}.
		\]
	\end{prob}
	
	\begin{prob}%[Nórdica]
		Determine todos os $x, y, z$ reais maiores que $1$ tais que
		\begin{gather*}
			x + y + z + \frac{3}{x-1} + \frac{3}{y-1} + \frac{3}{z-1}\\
			\rotatebox[origin=c]{270}{$=$}\\
			2\left(\sqrt{x+2} + \sqrt{y+2} + \sqrt{z+2}\right).
		\end{gather*}
	\end{prob}
	
	\begin{prob}
		Seja $n$ um inteiro positivo.
		Dados $n$ reais positivos, prove que a média aritmética desses termos é maior ou igual a média geométrica dos mesmos.
	\end{prob}
	
	\begin{prob}
		Sejam $a, b, c$ reais positivos. Prove que \[
			(a+b)(b+c)(c+a) \ge 8abc.
		\]
	\end{prob}

	\problem{math/brazil/mo/2001/1}

	\begin{prob}
		Sejam $a, b$ reais positivos tais que $a + b = 1$. Prove que \[
			\left(1+\frac1a\right)
			\left(1+\frac1b\right)
			\ge 9.
		\]
	\end{prob}
	
	%\begin{prob}
	%	Sejam $a > b$ reais positivos. Prove que \[
	%		a + \frac1{ab - b^2} \ge 3.
	%	\]
	%\end{prob}

	\begin{prob}
		Sejam $a, b, c, d$ reais positivos. Prove que \[
			\frac{a}{b+c} + \frac{b}{c+d} + \frac{c}{d+a} + \frac{d}{a+b} \ge 2.
		\]
	\end{prob}

	\begin{prob}
		Seja $P$ um ponto no interior de um triângulo e sejam $h_a, h_b, h_c$ as distâncias de $P$ aos lados $a, b, c$, respectivamente. Mostre que o valor mínimo de  \[
			\frac{a}{h_a} + \frac{b}{h_b} + \frac{c}{h_c} 
		\]
			ocorre quando $P$ é o incentro do triângulo $ABC$.
	\end{prob}

	\begin{prob}
		Sejam $a, b, c$ reais não negativos tais que $a + b + c = 3$. Prove que  \[
			\left(a^2 - ab + b^2\right)
			\left(b^2 - bc + c^2\right)
			\left(c^2 - ca + a^2\right)
			\le 12
		\]
	\end{prob}

	\begin{prob}
		Sejam $a, b, c$ números reais positivos. Prove que \[
			\sqrt{a^2 - ab + b^2} + \sqrt{b^2 - bc + c^2} \ge \sqrt{a^2 + ac + c^2}.
		\]
	\end{prob}

	\begin{prob}
		Se $a, b, c, d, e$ são números reais tais que $a + b + c + d + e = 8$ e $a^2 + b^2 + c^2 + d^2 + e^2 = 1$, determine o valor máximo de $e$.
	\end{prob}

	\problem{math/imo/2014/1}
	\problem{math/canada/2008/3}
	\problem{math/brazil/mo/2005/2/twocolumn}
	\problem{math/japan/2004/4}

	\begin{prob}
		Sejam $a_1, \dots, a_n, b_1, \dots, b_n$ reais dados. Prove que \begin{gather*}
			\left|a_1b_1 + \cdots + a_nb_n\right| \\
			\rotatebox[origin=c]{270}{$\le$} \\
			\sqrt{a_1^2 + \cdots + a_n^2} \sqrt{b_1^2 + \cdots + b_n^2}.
		\end{gather*}
	\end{prob}

	\begin{prob}
		Dados $n$ reais positivos $a_1, \dots, a_n$, prove que \[
			\frac{a_1^2 + \cdots + a_n^2}{n} \ge \left(\frac{a_1 + \cdots + a_n}{n}\right)^2.
		\]
	\end{prob}

	\problem{math/croatia/tst/2016/1/1/twocolumn}
	
	\begin{prob}%[Torneio das Cidades 1994]
		Prove que, para quaisquer $a_1, a_2, \dots, a_n$ reais positivos,
		\begin{gather*}
			\left(1 + \frac{a_1^2}{a_2}\right)
			\left(1 + \frac{a_2^2}{a_3}\right) \cdots
			\left(1 + \frac{a_n^2}{a_1}\right) \\
			\rotatebox[origin=c]{270}{$\ge$} \\
			\left(1 + a_1\right)
			\left(1 + a_2\right) \cdots
			\left(1 + a_n\right).
		\end{gather*}
	\end{prob}
	
	\problem{math/imo/2012/2}
	\problem{math/imo/2000/2}
	\begin{prob}% 2010 IMO Summer Training Adrian Tang
		Sejam $u, v, w$ reais positivos tais que $u + v + w + \sqrt{uvw} = 4$. Prove que  \[
			\sqrt{\frac{uv}{w}} + \sqrt{\frac{vw}{u}} + \sqrt{\frac{wu}{v}} \ge u + v + w.
		\]
	\end{prob}
	%\problem{math/japan/2010/4}
	%\problem{math/japan/2014/5}
	\problem{math/imo/2008/2}
	\problem{math/imo/2005/3}

\end{document}
