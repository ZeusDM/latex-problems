\documentclass[10pt,a4paper]{article}
\usepackage[utf8]{inputenc}
\usepackage[brazilian]{babel}
\usepackage[left=2cm, right=2cm, top=2cm, bottom=2cm]{geometry}
\usepackage[problem-list]{zeus}
\usepackage{parskip}
\usepackage{transparent}
\usepackage{lmodern}
\usepackage{multicol}
\usepackage[euler-digits, euler-hat-accent]{eulervm}

\setlength{\columnsep}{3em}

\title{Tutoria, 16:00}
\author{Guilherme Zeus Dantas e Moura}
\mail{zeusdanmou@gmail.com}
\titlel{\rlap{Matematicamente Internacionais}\smash{\raisebox{-2.4cm}{\transparent{.3}\includegraphics[width = 3cm]{mm_2}}}}
\titler{5 de Junho de 2021}

\pagestyle{empty}

\begin{document}	
 	\zeustitle

	\problem*{math/egmo/2021/2}
	\problem*{math/egmo/2021/1}
	\problem*{math/egmo/2021/5}

	\begin{prob}%[Livro do Davi]
		Sejam $\{a_1, \dots, a_n\}$ e $\{b_1, \dots,b_n\}$ dois multiconjuntos distintos, cada um deles formado por inteiros positivos.
		Se a igualdade dos seguintes multiconjuntos é verdadeira \[\{a_i+a_j; 1 \leq i<j \leq n\} = \{b_i+b_j; 1 \leq i<j \leq n\},\] prove que $n$ é uma potência de $2$.
	\end{prob}
\end{document}
