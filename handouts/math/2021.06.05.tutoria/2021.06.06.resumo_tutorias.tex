\documentclass[10pt,a4paper]{article}
\usepackage[utf8]{inputenc}
\usepackage[brazilian]{babel}
\usepackage[left=2cm, right=2cm, top=2cm, bottom=2cm]{geometry}
\usepackage[problem-list]{zeus}
\usepackage{parskip}
\usepackage{transparent}
\usepackage{lmodern}
\usepackage{multicol}
\usepackage[euler-digits, euler-hat-accent]{eulervm}

\setlength{\columnsep}{3em}

\title{Resumo das Tutorias}
\author{Guilherme Zeus Dantas e Moura}
\mail{zeusdanmou@gmail.com}
\titlel{\rlap{Matematicamente Internacionais}\smash{\raisebox{-2.4cm}{\transparent{.3}\includegraphics[width = 3cm]{mm_2}}}}
\titler{6 de Junho de 2021}

\pagestyle{empty}

\begin{document}	
 	\zeustitle

	\begin{prob}[Livro do Davi]
		Sejam $\{a_1, \dots, a_n\}$ e $\{b_1, \dots,b_n\}$ dois multiconjuntos distintos, cada um deles formado por inteiros positivos.
		Se a igualdade dos seguintes multiconjuntos é verdadeira \[\{a_i+a_j: 1 \leq i<j \leq n\} = \{b_i+b_j: 1 \leq i<j \leq n\},\] prove que $n$ é uma potência de $2$.
	\end{prob}

	\begin{sk}
		Davi Braga, Arthur e Nikolas sabem fazer esse problema.
	\end{sk}

	\problem{math/egmo/2021/5}

	\begin{sk}
		Davi Braga, Arthur, Nikolas e Giglio sabem fazer esse problema.
	\end{sk}

	\problem{math/imosl/2012/A4}
	\begin{sk}
		Rosalba, Eduardo e Thiago sabem fazer esse problema.
	\end{sk}

	\problem{math/imosl/2012/N4}
	\begin{sk}
		Rodrigo, Rosalba, Arthur e João Rafael sabem fazer esse problema.
	\end{sk}

	\begin{prob}[1\textsuperscript{o} Teste de Seleção do Brasil para a IMO 2021, Problema 4]
		\texttt{Problema censurado.}
	\end{prob}
	\begin{sk}
		Parece que o Rodrigo sabe fazer.
	\end{sk}

	\problem*{math/imosl/2015/C6}
	\begin{sk}
		Suamos, mas não terminamos. Perguntar avanços a Miguel, João Rafael, Rodrigo e João Marcelo.
	\end{sk}

	\problem{math/imosl/2018/G3}
	\begin{sk}
		Arthur, Giglio e Rosalba sabem fazer.
	\end{sk}


\end{document}
