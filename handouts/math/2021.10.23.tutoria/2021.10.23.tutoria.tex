\documentclass[10pt,a4paper]{scrartcl}
\usepackage[utf8]{inputenc}
\usepackage[brazilian]{babel}
\usepackage{lmodern}
\usepackage{euler}
\usepackage[left=2.5cm, right=2.5cm, top=2.5cm, bottom=2.5cm]{geometry}

\usepackage[problem-list, hide]{zeus}

\setlength{\columnsep}{4em}

\title{Banco de Problemas para a Tutoria}
\author{Guilherme Zeus Dantas e Moura}
\mail{\href{https://guilhermezeus.com}{\texttt{guilhermezeus.com}}}
\titlel{\includegraphics[width=3cm]{MM_logo_c}}
\titler{23 de Outubro de 2021}

\begin{document}	
	%\twocolumn[\zeustitle]
	\zeustitle

	%\sloppy

	\problem{math/usa/jmo/2021/4}
	\problem{math/usa/tst/2002/6}
	\problem{math/usa/tstst/2014/3}
	\begin{prob}
		Mostre que todo racional positivo pode ser escrito como soma de inversos de inteiros positivos distintos. Por exemplo, \(7/3 = 1/1 + 1/2 + 1/3 + 1/4 + 1/5 + 1/20\).
	\end{prob}
	\problem{math/usa/tst/2000/6}
	\begin{prob}%[Treinamento Cone Sul, Lista 4, Problema 6]
		Sejam $a, b, c, d$ quatro elementos distintos do conjunto $\{1,2,3, \ldots, 2017\}$ tais que a soma de quaisquer três deles é divisível pelo quarto. Determine o maior valor possível de $a+b+c+d$.
	\end{prob}
	\begin{prob}
		Encontre todos os inteiros $n$ tais que $\sqrt{n}+\sqrt{n+2019}$ também é inteiro.
	\end{prob}
\end{document}
