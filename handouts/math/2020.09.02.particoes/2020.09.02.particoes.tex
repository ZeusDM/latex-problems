\documentclass[10pt, a4paper]{article}
\usepackage[utf8]{inputenc}
\usepackage[brazilian]{babel}
\usepackage{lmodern}
\usepackage[left=2cm, right=2cm, top=2cm, bottom=2.5cm]{geometry}
\usepackage{indentfirst}
\usepackage[inline]{enumitem}

\usepackage[section]{zeus}
%\usepackage{cite}

%\usepackage{csquotes}
\usepackage{natbib}

\title{Partições}
\author{Guilherme Zeus Moura}
\mail{zeusdanmou@gmail.com}
\titlel{Turma Olímpica}
\titler{{\footnotesize v. 1} -- 02 de Setembro de 2020}

\DeclareMathOperator\Spec{Spec}

\begin{document}	
	\zeustitle

	\nocite{PartitionsofIntegers-JLaurendi}
	\nocite{Particoes-CShine}
	\nocite{Particoes-GLucas}
	\nocite{NT-DSantos}
	\nocite{Combinatorics3-YZhao}

	\section*{Algumas ideias}

	\begin{itemize}
		\item \emph{Casos pequenos}.
		\item \emph{Provar quantidades iguais:} criar uma bijeção pode ser útil.
		\item \emph{Pense recursivamente:} para representar $x$ como soma de elementos de $A$, olhe para os números $x - a$, $a \in A$.
		\item \emph{Casos grandes:} pensar assintoticamente pode ser útil.
		\item \emph{Provar existência de representação:} casa dos pombos ou algoritmo guloso podem ser uma solução rápida.
		\item \emph{Quantidade de parcelas:} usar contagem pode ser útil para fazer estimativas.
		\item \emph{Funções geratrizes} podem ser úteis.
		\item \emph{Teoria aditiva:} ao estudar $A + A$, pode ser útil estudar também $A - A$. 
	\end{itemize}

	\section*{Definição}

	\defn{Uma \emph{partição} de um inteiro positivo $n$ é uma forma de decomposição de $n$ como soma de interos positivos. Duas somas são consideradas iguais se e somente se possuem as mesmas parcelas, mesmo que em ordem diferente.

	Rigorosamente uma partição de um inteiro positivo $n$ é uma sequência de inteiros positivos $(x_1, x_2, \dots, x_m)$ tais que \[x_1 \ge x_2 \ge \cdots \ge x_m \text{\ e\ } x_1 + x_2 + \cdots + x_m = n.\]}

	Chamamos $x_1, x_2, \dots, x_n$ de \emph{partes} desta partição.

	\section{Exercícios Elementares}

	\begin{prob}
		Seja $p(n)$ o número de partições de $n$. Prove que o número de partições de $n$ com todas as partes maiores que $1$ é $p(n) - p(n-1)$.
	\end{prob}
	
	\begin{prob}
		Mostre que o número de partições de um inteiro $n$ em partes tal que a maior parte tem tamanho exatamente $r$ é igual ao número de partições em exatamente $r$ partes.
	\end{prob}

	\begin{prob}
		Prove que o número de partições de $n$ em que apenas as partes ímpares podem ser repetidas é igual ao número de partições de $n$ em que nenhum parte aparece mais do que $3$ vezes.
	\end{prob}

	\begin{prob}
		Prove que o número de partições de $n$ em partes distintas é igual ao número de partições de $n$ em partes ímpares.
	\end{prob}

	\begin{prob}
		O conjunto $A$ é um subconjunto de $\{1, 2, 3, \dots, n\}$ tal que $A+A = \{a+b : a, b \in A\}$ não intersecta $A$. Ache, em função de $n$, o número máximo de elementos de $A$.
	\end{prob}

	\section{Questões Divertidas}

	\begin{prob}
		Seja $n$ um inteiro positivo.
		Alice e Bruno jogam o seguinte jogo: eles constroem uma partição de $n$ da seguinte forma: Inicialmente, Alice escolhe um inteiro positivo $a_1 < n$.
		Depois Bruno escolhe um inteiro positivo $a_2 \le a_1$ tal que $a_1 + a_2 \le n$.
		Em seguida, Alice escolhe um inteiro positivo $a_3 \le a_2$ tal que $a_1 + a_2 + a_3 \le n$.
		O jogo continua, alternando os jogadores, até obtermos uma partição $a_1 + a_2 + \cdots + a_k$ de $n$.
		Se $k$ é ímpar, Alice vence; caso contrário, Bruno vence.
		Determine, em função de $n$, quem tem a estratégia vencedora.
	\end{prob}

	\problem{math/imo/1997/6}
	
	\problem{math/imo/1992/6}

	\begin{prob}[Yufei Zhao \cite{Combinatorics3-YZhao}]
		Determine se existe um subconjunto $S$ dos inteiros positivos com a seguinte propriedade: para todo inteiro positivo $n$, o número de partições de $n$, onde cada parte aparece no máximo duas vezes, é igual ao número de partições de $n$ em partes que são elementos de $S$.
	\end{prob}

	\section{Problemas Interessantes}

	\problem{math/miklos_schweitzer/2009/3}

	\problem{math/imosl/2015/C6}
	
	\problem{math/apmo/2020/3}


	\begin{prob}
		Definimos o \emph{espectro} de um número real $\alpha$ como a sequência \[\Spec(\alpha) = (\floor{\alpha}, \floor{2\alpha}, \floor{3\alpha}, \dots).\]
		\begin{enumerate}[label = (\alph*)]
			\item \textbf{(Beatty's Theorem, 1926)} Se $\alpha > 1$ é um irracional e $\frac{1}{\alpha} + \frac{1}{\beta} = 1$, mostre que as sequências $\Spec(\alpha)$ e $\Spec(\beta)$ particionam os inteiros positivos. Em outras palavras, mostre que $\Spec(\alpha) \cup \Spec(\beta) = \ZZ_{>0}$ e $\Spec(\alpha) \cap \Spec(\beta) = \varnothing$. 
			\item \textbf{(Bang’s Theorem, 1957)} Prove a recíproca do teorema acima.
		\end{enumerate}
	\end{prob}

	\section{Desafio Final}

	\problem{math/imosl/2010/C7}

	\bibliographystyle{plain}
	%\bibliography{mybib}{}

	\bibliography{mybib}
	%\printbibliography
\end{document}
