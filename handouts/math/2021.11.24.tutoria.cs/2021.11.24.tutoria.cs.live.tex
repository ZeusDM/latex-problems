\documentclass[10pt,a4paper]{scrartcl}
\usepackage[utf8]{inputenc}
\usepackage[brazilian]{babel}
\usepackage{lmodern}
\usepackage{euler}
\usepackage[left=2.5cm, right=2.5cm, top=2.5cm, bottom=2.5cm]{geometry}
\usepackage{parskip}

\usepackage[problem-list]{zeus}

\setlength{\columnsep}{4em}

\title{Treinamento de Velocidade --- Live}
\author{Guilherme Zeus Dantas e Moura}
\mail{\href{https://guilhermezeus.com}{\texttt{guilhermezeus.com}}}
\titlel{\includegraphics[width=3cm]{MM_logo_c}}
\titler{24 de Novembro de 2021}

\begin{document}	
	%\twocolumn[\zeustitle]
	\zeustitle

	\problem{math/korea/junior/2021/2}
	\begin{sk}
		A resposta é \(0\). Defina \[
			S_k = a_1 - a_2 + a_3 - a_4 + \dots + (-1)^{k_1} a_k,
		\]
		e \(b_k = \frac{S_k}{k}\).

		Note que \(a_{k+1}\) é o único número tal que \(0 \leq a_{k+1} \leq k\) e \(S_k + (-1)^{k}a_{k+1} \equiv 0 \pmod{k+1}\).
		Isso é equivalente a \(a_{k+1} = (-1)^{k}b_k \pmod{k+1}\).

		Para \(k \geq 2021^{2021}\), dá pra cotar  \(-k < b_l < k\).

		\begin{itemize}
			\item \underline{Se \(k\) é par,} concluímos que
				\[
					a_{k+1} =
					\begin{cases}
						(b_k) & b_k \geq 0 \\
						(b_k + k + 1) & b_k < 0.
					\end{cases}
				\]
				Logo, \[
					(k+1)b_{k+1} = S_{k+1} = S_{k} + (-1)^{k} a_{k+1} =
					\begin{cases}
						(k+1)b_k & b_k \geq 0 \\
						(k+1)(b_k+1) & b_k < 0,
					\end{cases}
				\]
				ou seja, \[
					b_{k+1} =
					\begin{cases}
						b_k & b_k \geq 0 \\
						b_k + 1 & b_k < 0.
					\end{cases}
				\]

			\item \underline{Se \(k\) é ímpar,} concluímos que
				\[
					a_{k+1} =
					\begin{cases}
						-b_k + k + 1 & b_k > 0 \\
						-b_k & b_k \leq 0.
					\end{cases}
				\]
				Logo, \[
					(k+1)b_{k+1} = S_{k+1} = S_{k} + (-1)^{k} a_{k+1} =
					\begin{cases}
						(k+1)(b_k-1) & b_k > 0 \\
						(k+1)(b_k) & b_k \leq 0,
					\end{cases}
				\]
				ou seja, \[
					b_{k+1} =
					\begin{cases}
						b_k - 1 & b_k > 0 \\
						b_k & b_k \leq 0.
					\end{cases}
				\]
		\end{itemize}

		Seja \(N = 2021^{2021}\).
		Seja \(k \geq N\) qualquer.
		Portanto, concluímos que, se \(b_k \neq 0\), então \(|b_{k + 2}| = |b_{k}| - 1\); e, se \(b_k = 0\), então \(b_{k+1} = 0\).
		Como \(b_{N} < N\), isso implica que \(b_{3N} = 0\), que finalmente implicará que \(S_k = b_{k} = a_{k} = 0\) para todo \(k > 3N\). 
	\end{sk}
	\problem{math/korea/junior/2021/4}
	\begin{sol}
		Seja \(E\) a intersecção de \(BC\) e \(AD\).
		\begin{align*}
			\text{\(K\), \(B\), \(D\) são colineares}
			&\iff \angle ABK = 180^\circ - \angle DBA \\
			&\iff \angle APK = \angle ACD \\
			&\iff \angle APK = \angle ACB + \angle BCD \\
			&\iff \angle APK = \angle ACE + \angle BAD \\
			&\iff \angle APK = \angle ACE + \angle DAC \\
			&\iff \angle APK = \angle ACE + \angle EAC \\
			&\iff \angle APK = \angle AEB \\
			&\iff \angle APC = \angle AEC \\
			&\iff \text{\(APEC\) é cíclico.}
		\end{align*}
	\end{sol}
	\problem{math/korea/junior/2021/5}
	\begin{sol}
		Jogando \((x, y) \mapsto (x, 0)\) na equação original, temos que \(f(0) = 0\).

		Observe que, se \(f\) é solução, então \(-f\) também é solução, pois a equação original com \((x, y) \mapsto (x, -y)\) implica \[
			(-f)((-f)(x+y) - (-f)(x - y)) = y^2 (-f)(x).
		\]

		A função identicamente nula funciona. Suponha que \(f\) não é a função identicamente nula. Portanto, existe \(c\) tal que \(f(c) \neq 0\). Temos que \(c \neq 0\), pois \(f(0) = 0\). Pela observação do parágrafo anterior, podemos supor, sem perda de generalidade, que \(f(c) > 0\).

		Seja \(y \geq 0\) qualquer real não-negativo. Defina \(z_y = \sqrt{\frac{y}{f(c)}}\). Jogando \((x, y) \mapsto (c, z_y)\) na equação original, temos que \[
			f(f(c+z_y) - f(c-z_y)) = y,
		\] portanto, \(y\) está na imagem de \(f\).

		Defina por \(w_y\) o número tal que \(f(2w_y) = y\), que existe, como vimos acima. Jogando \((x, y) \mapsto (w_y, w_y)\) e \((x, y) \mapsto (w_y, -w_y)\) na equação original, temos que
		\begin{align*}
			f(y) = f(f(2w_y)) &= w_y^2f(w_y) \\
							  &= (-w_y)^2f(w_y) = f(-f(2w_y)) = f(-y),
		\end{align*}
		para todo \(y \geq 0\).

		Portanto, trocando \(y \mapsto -y\) caso \(y\) seja negativo, concluímos que \[
			f(y) = f(-y)
		\] para todo \(y\) real; ou seja, \(f\) é uma função par.

		Finalmente, jogando \((x, y) \mapsto (x, c)\) e \((x, y) \mapsto (c, x)\) na equação original, concluímos que 
		\begin{align*}
			c^2f(x) &= f(f(x+c) - f(x-c)) \\
					&= f(f(c+x) - f(c-x)) = x^2f(c),
		\end{align*}
		para todo \(x\) real, e portanto, \[
			f(x) = \frac{f(c)}{c^2}x^2.
		\]

		Basta testarmos a função \(f(x) = kx^2\), para uma constante \(k\). Concluímos que \(k = 0\), \(k = 1/4\) ou \(k = -1/4\), e portanto, as funções que satisfazem o enunciado são \[
			f(x) = \frac{-x^2}{4}, \quad f(x) = 0, \quad f(x) = \frac{x^2}{4}.
		\]
	\end{sol}
	\problem{math/hmic/2021/1}
	\begin{sol}
		Seja \(n = 2021\) e \(k = 1000\). A resposta é \(k(n-k)\).

		Cada pessoa possui um contador de passos orientados.
		Cada vez em que uma pessoa é movimentada no sentido anti-horário, seu contador de passos aumenta em \(1\).
		Cada vez em que uma pessoa é movimentada no sentido horário, seu contador de passos diminui em \(1\).
		Seja \(p_i\) o valor do contador de passos da pessoa \(i\).

		Em cada operação, uma pessoa é movimentada no sentido anti-horário; e outra é movimentada no sentido horário.
		Portanto, \(\sum p_i\) é constante em qualquer momento; isto é \(\sum p_i = 0\).
		Note também que, em cada operação, \(\sum |p_i|\) aumenta em no máximo  \(2\) unidades.

		Suponha que, após \(M\) movimentos, cada pessoa se encontra \(k\) posições no sentido horário de onde começou.
		Isso implica que, para cada pessoa \(i\), \[
			p_i \equiv k \pmod{n},
		\] ou seja, \[
			p_i \in \{\dots, k-2n, k-n, k, k+n, \dots \}.
		\]

		Como \(M\) movimentos foram feitos, concluímos que \(\sum |p_i| \leq 2M\).
		Sejam \[ \mathcal P = \{i : f(i) \geq 0\} \qquad \text{and} \qquad \mathcal N = \{i : f(i) < 0\}.  \]
		Portanto,
		\begin{align*}
			0 &= \sum_{i \in \{1, \dots, n\}} p_i \\
			  &= \sum_{i \in \mathcal P} p_i   + \sum_{i \in \mathcal N} p_i   \\
			  &= \sum_{i \in \mathcal P} |p_i| - \sum_{i \in \mathcal N} |p_i|.
		\end{align*}

		Seja \(X = \sum_{i \in \mathcal P} |p_i| = \sum_{i \in \mathcal N} |p_i|\).
		Note que \(X = \sum_{i \in \mathcal P} |p_i| \geq |\mathcal P| \cdot k\) e \(X = \sum_{i \in \mathcal N} |p_i| \geq |\mathcal N| (n-k)\).
		Portanto,
		\begin{align*}
		2M \geq	\sum_{i \in \{1, \dots, n\}} |p_i|
			&= \sum_{i \in \mathcal P} |p_i| + \sum_{i \in \mathcal N} |p_i| \\
			&= 2X \\
			&= 2\left(\frac{kX + (n-k)X}{n}\right) \\
			&\geq 2\left(\frac{k(n-k)|\mathcal N| + (n-k)k|\mathcal P|}{n}\right) \\
			&= 2k(n-k)\frac{|\mathcal P| + |\mathcal N|}{n} = 2k(n-k),
		\end{align*}
		ou seja, o número de movimentos é maior ou igual a \(k(n-k)\).

		Tal número é atingível com a seguinte sequências de movimentos:
		\begin{itemize}
			\item[\(0\).] No início, estão todos em fila: \[
					1, 2, 3, \dots, k, k+1, k+2, k+3, \dots, n
			\]
		\item[\(1\).] Faça \(k\) movimentos, trocando o \(k+1\) com \(k\), então trocando \(k+1\) com \(k-1\), e assim por diante, até trocar \(k+1\) com \(1\). Teremos, então, a seguinte sequência: \[
			k+1, 1, 2, \dots, k-1, k, k+2, k + 3, \dots, n
		\]
		\item[\(2\).] Faça outros \(k\) movimentos, trocando o \(k+2\) com \(k\), então trocando \(k+2\) com \(k-1\), e assim por diante, até trocar \(k+2\) com \(1\). Teremos, então, a seguinte sequência: \[
			k+1, k+2, 1, \dots, k-2, k-1, k, k+3, \dots, n
		\]
		\item[\(\vdots\)]
		\item[\(n-k\).] Por fim, faça outros \(k\) movimentos, trocando o \(n\) com \(k\), então trocando \(n\) com \(k-1\), e assim por diante, até trocar \(n\) com \(1\). Teremos, então, a seguinte sequência: \[
			k+1, k+2, \dots, n-1, n, 1, 2, \dots, k
		\]
		\end{itemize}
	\end{sol}
\end{document}
