\documentclass[10pt,a4paper]{article}
\usepackage[utf8]{inputenc}
\usepackage[brazilian]{babel}
\usepackage{lmodern}
\usepackage[left=2.5cm, right=2.5cm, top=2.5cm, bottom=2.5cm]{geometry}

\usepackage{../../../commands/problems}
\renewcommand{\mypath}{../../../}

\usepackage{wrapfig}
\usepackage{multicol}

\title{Algoritmos}
\author{Guilherme Zeus Moura}
\mail{zeusdanmou@gmail.com}
\titlel{Turma Olímpica}
\titler{\today}

\renewcommand\playerA[1]{Guilherme}
\renewcommand\playerB[1]{Zeus}

\begin{document}	
	\zeustitle
	\begin{defn}
		Um \emph{algoritmo} é um conjunto de procedimentos sistemáticos que resolvem um problema.
	\end{defn}
	
	Por definição, um algoritmo deve ter duas propriedades:
	\begin{itemize}
		\item Resolver o problema.
		\item Terminar.
	\end{itemize}

	A segunda propriedade parece uma besteira, mas às vezes não é trivial mostrar que um algoritmo acaba.
	\subsection*{Situações em que algoritmos podem ser úteis}
	
	\begin{itemize}
		\item Uma operação é dada no problema, e queremos organizar uma sequência de operações para alcançar uma meta.
		\item Você quer construir um exemplo.
		\item Temos uma configuração complicada e queremos reduzi-la a algo mais simples.
		\item Você quer representar um conjunto de alguma outra forma.
		\item Você quer empacotar coisas ou separá-las em grupos.
		\item Jogos com estratégias vencedoras.
	\end{itemize}

	\subsection*{Técnicas}

	\begin{itemize}
		\item Você pode usar uma indução para provar que um algoritmo funciona.
		\item Uma maneira de demonstrar que um algoritmo acaba é encontrar um \emph{monovariante}, ou seja, algo que sempre diminui ou sempre aumenta.
		\item Às vezes um invariante pode ajudar nas duas tarefas.
		\item \emph{Ordenar} coisas facilita a indução e a organização do seu algoritmo.
		\item Um algoritmo simples e efetivo em muitas situações é o \emph{algoritmo guloso}, em que sempre escolhemos a melhor opção localmente. Nem sempre o algoritmo guloso é o melhor, porém. Tome cuidado!
		\item Teste seu algoritmo com \emph{casos pequenos}.
	\end{itemize}

	\subsection*{Exemplos}

	\begin{exmp}
		Considere o \emph{algoritmo de Euclides}: sejam $a$ e $b$ inteiros positivos com $a > b$. Para calcular $\mdc(a, b)$, fazemos os seguintes passos:
		\begin{enumerate}
			\item Dividimos $a$ por $b$, obtendo resto $r$. Se $r = 0$, o $\mdc(a, b)$ é $b$.
			\item Se $r > 0$, trocamos $(a, b)$ por $(b, r)$ e voltamos ao passo anterior.
		\end{enumerate}
	\end{exmp}

	\subsection*{Problemas}
	\begin{prob}
		Há pedras de massa total $9$ toneladas. Caminhões, cada um com capacidade de $3$ toneladas, estão disponíveis para carregar essas pedras. Sabe-se somente que nenhuma pedra pesa mais que $1$ tonelada. Qual é a menor quantidade de caminhões necessários para carregar todas as pedras? Cada caminhão só faz uma viagem.
	\end{prob}

	\begin{prob}
		Sejam $A_1, A_2, \dots, A_n$ subconjuntos de $\{1, 2, \dots n\}$, todos com $3$ elementos. Prove que é possível pintar $\floor{\frac{n}{3}}$ números de $\{1, 2, \dots, n\}$ de modo que cada $A_i$ tenha pelo menos um número não pintado.
	\end{prob}	

	\begin{prob}
		Uma quantidade finida de cartas é colocada em cada um dos pontos $A_1, A_2, \dots, A_n$, $O$, sendo $n \ge 3$. Em cada passo é permitido fazer uma das seguintes operações:
		
		\begin{enumerate}[label = (\roman*)]
			\item Se há mais de duas cartas em $A_i$, tirar três cartas de $A_i$ e colocá-las em $A_{i-1}$, $A_{i+1}$ e $O$, uma carta em cada ponto; os indíces são cíclicos, isto é, $A_{n+1} = A_1$ e $A_0 = A_n$.
			\item Se há pelo menos $n$ cartas em $O$, tirar $n$ cartas de $O$ e colocá-las em $A_1, A_2, \dots, A_n$, uma carta em cada ponto.
		\end{enumerate}	

		Sendo a quantidade de cartas pelo menos $n^2 + 3n + 1$, prove que é possível deixar, após uma quantidade finita de operações, pelo menos $n+1$ cartas em cada ponto.
	\end{prob}

	\begin{prob}
		Temos um tabuleiro $m \times n$. Atribui-se inicialmente um número inteiro não negativo a cada uma das casas. No tabuleiro é permitido efetuar a seguinte operação: em qualquer par de casas com um lado em comum, pode-se modificar os dois números somando-lhes um mesmo número inteiro (que pode ser negativo), sempre que ambos resultados sejam não negativos.

		Que condições devem ser satisfeitas inicialmente na atribuição dos números, para deixar, mediante aplicações reiteradas da operação, zero em todas as casas?
	\end{prob}

	\begin{prob}
		Prove que o algoritmo de Euclides termina após no máximo $\ceil{\log_\phi\max(a, b)}$ passos, em que $\phi = \frac{1 + \sqrt{5}}{2}$ é a razão áurea.
	\end{prob}

	\begin{prob}(Frações Egípcias)
		Prove que todo racional positivo menor que $1$ pode ser escirto como soma de frações de numerador $1$ e denominadores todos distintos.
	\end{prob}

	\begin{prob}
		Em um grafo, cada vértice tem grau menor ou igual a $\Delta$. Prove que é possível pintar os vértices do grafo com $\Delta + 1$ cores, de modo que dois vértices conectados têm cores diferentes.
	\end{prob}

	\begin{prob}
		Uma loja vende garrafas com as seguintes capacidades: $1$ litro, $2$ litros, $\dots$, $1966$ litros. Os preços das garrafas satisfazem duas condições:
		\begin{itemize}
			\item Duas garrafas têm o mesmo preço se e somente se suas capacidades $m$, $n$ ($m > n$) satisfazem $m - n = 1000$.
			\item Cada garrafa de $m$ litros de capacidade ($1 \le m \le 1000$) custa $1996 - m$ dinheiros.
		\end{itemize}
		
		Ache todos os pares de garrafas de $m$ e $n$ litros tais que:
		\begin{enumerate}[label = (\alph*)]
			\item $m + n = 1996$.
			\item o custo total do par seja o menor possível;
			\item com o par se possa medir $k$ litros, para todo $k$ inteiro desde $1$ até $1996$.
				\begin{rem}
					As operações permitidas para medir são:
					\begin{itemize}
						\item Encher ou esvaziar qualquer das duas garrafas.
						\item Passar líquido de uma garrafa para a outra.
					\end{itemize}
					Ter-se-á conseguido medir $k$ litros quando a quantidade de litros de uma garrafa mais a quantidade de litros da outra é igual a $k$.
				\end{rem}
		\end{enumerate}
	\end{prob}

	\begin{prob}
		Em cada casa de um tabuleiro quadriculado $2000 \times 2000$ deve-se escrever um dos três números: $-1$, $0$ ou $1$. Se, em seguida, somam-se os números escritos em cada linha e colina, obtém-se $4000$ resultados. Mostre é que possível preencher o tabuleiro de modo que os $4000$ ressultados sejam todos distintos. 
	\end{prob}

	\begin{prob}
		Determine se existe uma sequência infinita $a_0, a_1, \dots$ de inteiros positivos que satisfaz as seguintes condições:
		\begin{itemize}
			\item Todos os números inteiros positivos aparecem na sequência uma única vez.
			\item Todos os números inteiros positivos aparecem na sequência $|a_0 - a_1|, |a_1 - a_2|, \dots$ uma única vez.	
		\end{itemize}
	\end{prob}

	\begin{prob}
		Seja $A_1A_2B_1B_2$ um quadrilátero convexo. Nos vértices $A_1$ e $A_2$, adjacentes, há duas cidades argentinas. Nos vértices $B_1$ e $B_2$, há duas cidades brasileiras. No interior do quadrilátero, existem $a$ cidades argentinas e $b$ cidades brasileiras, sem haver três cidades colineares.
		Determine se é possível, independentemente da posição das cidades, construir estradas retas, cada uma das quais conecta duas cidades argentinas ou duas cidades brasileiras, de modo que:
		\begin{itemize}
			\item Não existam duas estradas que se intersectem em um ponto que não seja uma cidade.
			\item De qualquer cidade argentina, seja possível chegar a qualquer outra cidade argentina.
			\item De qualquer cidade brasileira, seja possível chegar a qualquer outra cidade brasileira.
		\end{itemize}

		Se for sempre possível, monte um algoritmo que construa uma possível configuração.
	\end{prob}

	\begin{prob}
		Prove que, dado $k$ inteiro positivo, $m$ e unicamente representado na forma
		$$ m = \binom{a_k}{k} + \binom{a_{k-1}}{k-1} + \cdots + \binom{a_t}{t}$$
		em que $a_k > a_{k-1} > \cdots > a_t \ge t \ge 1$.
	\end{prob}

	\begin{prob}
		Um conjunto de inteiros não negativos $\{x, y, z\}$ com $x < y < z$ é \emph{histórico} se satisfaz $\{z-y, y-x\} = \{1776, 2001\}$.
		Mostre que podemos expressar os inteiros não negativos como união disjunta de conjuntos históricos.
	\end{prob}

	\begin{prob}
		Um conjunto $A$ de inteiros é \emph{soma-cheia} se $A \subseteq A + A$, ou seja, cada elemento de $A$ é a soma de dois elementos (não necessáriamente distintos) de $A$. Um conjunto $A$ de inteiros é \emph{livre de soma zero} se $0$ é o único inteiro que não pode ser expresso como soma dos elementos de um subconjunto finito não vazio de $A$.

		Determine se existe um conjunto soma-cheia e livre de soma zero.
	\end{prob}

	\begin{prob}[Casamentos estáveis]
		Considere $n$ garotos e $n$ garotas.
		Cada garoto ordena as garotas e cada garota ordena os garotos.
		Mostre que é possível formar $n$ casais, cada um com um garoto e uma garota, de modo que não existam dois casais $(a, b)$ e $(c, d)$ tais que $a$ apareça acima de $c$ na ordem de $d$ e $d$ apareça acima de $b$ na ordem de $a$.
	\end{prob}

	\begin{prob}
		Há $n$ inteiros positivos escritos na lousa, $n \ge 2$.
		Em cada passo, podemos escolher dois números escritos na lousa, apagá-los e substituí-los por sua soma (ou seja, trocamos $a$ e $b$ por $a + b$ e $a + b$).
		Encontre todos os valores de $n$ para os quais é sempre possível deixar os $n$ números na lousa iguais após uma quantidade finita de passos.
	\end{prob}

	\begin{prob}
		O senhor Patinhas tem três contas bancárias, cada uma com uma quantidade inteira de patacas.
		Ele só transfere dinheiro de uma conta para outra se o saldo da segunda conta é dobrado.
		Prove que o senhor Patinhas consegue deixar todo seu rico dinheirinho em duas contas.
	\end{prob}

	\begin{prob}
		Dada um família $F$ de subconjuntos de $S = \{1, 2, . . . , n\}$ ($n \ge 2$), uma jogada permitida é escolher dois conjuntos disjuntos $A$ e $B$ de $F$ e agregar $A \cup B$ a $F$, mantendo $A$ e $B$ em $F$.
		
		Inicialmente, $F$ tem exatamente todos os subconjuntos unitários de $S$. O objetivo é obter, mediante jogadas permitidas, que F tenha todos os subconjuntos de $n - 1$ elementos de $S$.

		Determine o menor número de jogadas necessárias para alcançar o objetivo.
	\end{prob}

	\begin{prob}
		Determine se existe uma sequência infinita $a_0, a_1, a_2, a_3, \dots$ de inteiros não negativos que satisfaz as seguintes condições:
		\begin{itemize}
			\item Todos os inteiros não negativos aparecem na sequência uma única vez.
			\item A sequência $$b_n = a_n + n, n \ge 0$$ é formada por todos os números primos, cada um aparecendo exatamente uma vez.
		\end{itemize}
	\end{prob}

	%18 ~ 24, Shine

	\problem{math/brazil/mo/2018/2}

	\problem{math/imo/2019/5}

	\problem{math/imo/2019/3}

	\newpage
	\section*{Referências}
	\begin{enumerate}[label = { [\arabic*] }]
		\item \emph{Algoritmos,} Carlos Shine.
	\end{enumerate}
\end{document}
