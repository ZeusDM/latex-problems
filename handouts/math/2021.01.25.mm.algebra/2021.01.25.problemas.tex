\documentclass[10pt,a4paper]{article}
\usepackage[utf8]{inputenc}
\usepackage[brazilian]{babel}
\usepackage[left=2cm, right=2cm, top=2cm, bottom=2cm]{geometry}
\usepackage[prob-boxed, dem-boxed]{zeus}
\usepackage{parskip}
\usepackage{transparent}

%\usepackage{lmodern}
%\usepackage{eulervm}
%\usepackage[scaled]{helvet}
%\usepackage[T1]{fontenc}
%\usepackage{mdsymbol}
%\usepackage[italic]{mathastext}

%\renewcommand{\familydefault}{\sfdefault}
%\usepackage[eulergreek, EULERGREEK]{sansmath}
%\sansmath

\usepackage[printwatermark]{xwatermark}
\newwatermark[firstpage, angle=0,scale=3,xpos=-70,ypos=122]{{\transparent{0.3}\includegraphics[width = 1cm]{mm_2}}}

\title{Miscelânea de Álgebra}
\author{Guilherme Zeus Dantas e Moura}
\mail{zeusdanmou@gmail.com}
\titlel{Semana Matematicamente Olímpica}
\titler{29 de Janeiro de 2021}

\begin{document}	
	\zeustitle

	\hfill 
	\begin{minipage}{7.4cm}
		{\textit{A proof is something that satisfies the audience.} \\ --- \href{https://youtu.be/nJP7hkYuiFg?t=1535}{Rob Morris, 2019}}
	\end{minipage}


	\section*{O polinômio $x^3 - 3x + 1$.}

	\begin{prob}[$x^3 - 3x + 1 = 0$, $x \in \RR$]\label{prob:achar-raizes}
		Determine as raízes reais do polinômio $x^3 - 3x + 1$.
	\end{prob}
	\problem{math/imc/2020/6}

	Em vários problemas, é mais útil saber que um número $\alpha$ é raiz de um certo polinômio, em comparação com saber escrever $\alpha$ explicitamente usando operações, funções trigonométricas, entre outras coisas --- principalmente quando o polinômio é bonitinho. Veja mais sobre essa ideia no \href{https://youtu.be/8EoVi_GV-vM}{Matematicamente Ao Vivo \#32}, e nos problemas da secção ``\nameref{sec:root}''. 

	Um problema interessante em que o polinômio $x^3 - 3x + 1$ aparece é o \nameref{math/brazil/mo/2017/6}. Saber os problemas acima ajuda a resolvê-lo, mas também é útil saber sobre Extensão de Corpos. Recomendo o \href{https://www.obm.org.br/content/uploads/2018/01/Carlos_Shine_-algebra_abstrata.pdf}{material do Carlos Shine sobre Extensão de Corpos da Semana Olímpica de 2018}.
	
	\problem{math/brazil/mo/2017/6}

	\section*{Seja $\alpha$ uma raiz de um certo polinômio.}\label{sec:root}
	\problem{math/apmo/2006/2}

	\section*{Polinômio traiçoeiro.}

	\begin{prob}[$P(k) = 2^k$, $P(n+1) =\ ?$] \label{P(k)=2^k}
		Seja $P(x)$ um polinômio de grau $n$ com coeficientes reais tal que $P(k) = 2^k$, para $k \in \{0, 1, 2, \dots, n\}$. Determine $P(n+1)$.
	\end{prob}
	
	\begin{prob}[$P(k) = F_k$, $P(2n+3) =\ ?$]\label{P(k)=F_k}
	Seja $P(x)$ um polinômio de grau $n$ com coeficientes reais tal que $P(k) = F_k$, para $k \in \{n+2, n+3, \dots, 2n+2\}$, em que $F_k$ denota o $k$-ésimo número de Fibonacci\footnote{$F_0 = 0$, $F_1 = 1$, $F_{k} = F_{k-1} + F_{k-2}$ para $k \ge 2$.}. Determine $P(2n+3)$.
	\end{prob}

	\newpage
	\section*{Determine todas as funções.}

	\problem{math/brazil/rio/2017/N3/6}
	\problem{math/imo/2009/5}

	\section*{Problemas para os entediados.}

	\begin{prob}[Canada Winter Camp 2020, Buffet, A1]
		Prove que existe um polinômio $P$ tal que, para qualquer inteiro positivo $n$, \[
			\floor{2\sqrt{1}} + \floor{2\sqrt{2}} + \dots + \floor{2\sqrt{n^2}} = P(n)
		\]
	\end{prob}

	\problem{math/imo/2009/3}

	\begin{prob}[Canada Winter Camp 2020, Buffet, A2]
		Ache todas as funções $f \colon \ZZ \to \ZZ$ tais que  \[ f( - f(x) - f(y) ) = 1 - x - y, \]para quaisquer $x, y$ inteiros.
	\end{prob}

	\begin{prob}[Canada Winter Camp 2020, Buffet, A3]
		Seja $P_0 = x^3 - 4x$. Uma sequência de polinômios é definida pela recorrência \[  P_{n+1} = P_n(1+x)P_n(1-x) - 1, \]
		para inteiros $n \ge 0$. Prove que $x^{2016} \mid P_{2016}(x)$.
	\end{prob}

	\begin{prob}[Canada Winter Camp 2020, Buffet, A4]
		Considere uma sequência de reais positivos $a_1, a_2, \dots$, tal que $a_1 = 1$, $a_2 = 2$,  \[
			a_{mn} = a_ma_n, \qquad \qquad a_{m+n} \le 2020(a_m + a_n),
		\]
			para todos $m, n$ inteiros positivos. Prove que $a_n = n$ para todo $n$ inteiro positivo.
	\end{prob}

	%\newpage
	%
	\newpage
	\section*{Algumas Soluções.}
	\begin{sk}[para \nameref{prob:achar-raizes}]
		Ache $\cos(3\theta) = 4\cos^3\theta - 3\cos\theta$. Portanto, $2\cos(3\theta) = 1$ se, e somente se, $2\cos(\theta)$ é raiz de $x^3 - 3x + 1$. Achamos, portanto, $2\cos(20^\circ), 2\cos(40^\circ), 2\cos(80^\circ)$ como raízes.
	\end{sk}

	\begin{sk}[para \nameref{math/imc/2020/6}]
		Note que se $\alpha$ é raíz de $P(x)$, então $\alpha^2 - 2$ também é raíz --- podemos ver que isso acontece no problema \nameref{prob:achar-raizes} e provar algebricamente. Logo, como $a \pmod{p}$ é a única raíz, é necessário $a \equiv a^2 - 2 \pmod{p} \iff a \equiv 2$ or $a \equiv -1$. Para $a \equiv 2$, $P(2) \equiv 2^3 - 3\cdot2 + 1 \equiv 3 \equiv 0 \pmod{p} \implies p = 3$; para $a \equiv -1$,  $P(-1) \equiv (-1)^3 - 3\cdot(-1) + 1 \equiv -3 \equiv 0 \pmod{p} \implies p = 3$.

		É necessário verificar que $p = 3$ funciona.
	\end{sk}

	\begin{sol}[para \nameref{math/apmo/2006/2}]
		Casos pequenos:
		\[1 = \phi^0\]
		\[2 = \phi^{-2} + \phi^{-1} + \phi^{0} = \phi^{-2} + \phi^1 \]
		\[3 = \phi^{-2} + \phi^0 + \phi^1 = \phi^{-2} + \phi^2\]
		\[4 = \phi^{-2} + \phi^0 + \phi^2\]

		Quem é $\phi$? Ora, $\phi = \frac{1 + \sqrt{5}}{2}$. Porém, um jeito por vezes mais útil é ver $\phi$ como raiz de $x^2 - x - 1$. Ou seja:
		\[\phi^2 = \phi + 1.\]

		Defina $n \equiv S$, sendo $n$ um inteiro e $S$ um suconjunto finito dos inteiros se, e somente se, \[n = \sum_{i \in S} \phi^i.\]

		Os casos pequenos ficam, então:
		\[1 \equiv \{0\}\]
		\[2 \equiv \{-2, -1, 0\} \equiv \{-2, 1\}\]
		\[3 \equiv \{-2, 0, 1\} \equiv \{-2, 2\}\]
		\[4 \equiv \{-2, 0, 2\}\]
		\[5 \equiv \{-4, -3, -2, -1, 0, 2\}\]

		A equação $\phi^2 = \phi + 1$ é traduzida como uma operação:

		Se $n, n+1 \in S$ e $n+2 \not\in S$, então \[S \equiv (S - \{ n, n+1\}) \cup \{n+2\},\]
		em outras palavras, podemos trocar $n$ e $n+1$ por $n+2$, ou, em outras palavras, \[\{n, n+1\} \equiv \{n+2\}.\]

		\begin{lem}
		Se $n \equiv S$, então existe $R$ tal que $n \equiv R$ e $R$ não possui consecutivos.
		\end{lem}
		\begin{dem} 
		Se $S$ não possui consecutivos, acabou! Suponha que $S$ possui consecutivos, pegue o maior par de consecutivos $(n, n+1)$.  $n+2$ não está em $S$, pois, se estivesse, $(n+1, n+2)$ seria um par de consecutivos maior.

		Sabemos que $\phi^{n+2} = \phi^{n+1} + \phi^n$.
		Logo, \[n \equiv S' = (S - \{n, n+1\}) \cup \{ n+2\}.\]

		Repetimos esse algoritmo enquanto houverem consecutivos. Esse algoritmo acaba pois, em cada etapa, o número de elementos do conjunto diminui; e esse número é sempre inteiro não-negativo. (E ele começa como um inteiro finito).
		\end{dem}

		Vamos provar por indução que todo $n$ possui a propriedade desejada. (Base: OK.)

		Suponha que $n - 1$ possui a propriedade. Pelo Lema, existe $S \equiv n - 1$, $S$ sem consecutivos. 
		Seja $-2k$ o maior par não-positivo tal que $-2k \not\in S$ (como $S$ é finito, esse número existe).
		Logo, $-2k+2, -2k+4, \dots, -2, 0$ estão em $S$. Como $S$ no possui consecutivos, isso implica que $-2k+1, -2k+3, \dots, -3, -1$ não estão em $S$. Note que:
		\begin{align*}
		1 &= \phi^{-1} + \phi^{-2}\\
		 &= \phi^{-1} + \phi^{-3} + \phi^{-4}\\
		 &= \phi^{-1} + \phi^{-3} + \phi^{-5} + \phi^{-6}\\
		 &\vdots\\
		 &= \phi^{-1} + \phi^{-3} + \cdots + \phi^{-2k+3} + \phi^{-2k+1} + \phi^{-2k}.
		\end{align*}

		Logo, \[n \equiv S \cup \{-2k, -2k+1, -2k+3, \dots, -3, -1\}.\]

	\end{sol}

	\begin{sk}[para \nameref{P(k)=2^k}]
		Basta notar que \[
			P(k) = \binom{x}{0} + \binom{x}{1} + \cdots + \binom{x}{n}.
		\]

		Logo, $P(n+1) = 2^{n+1} - 1$.
	\end{sk}
	
	\begin{sk}[para \nameref{P(k)=F_k}]Definimos $\Delta Q(x) = Q(x+1) - Q(x)$.
		\begin{gather*}
			P(k) = F_k, \text{\ para\ } k \in \{n + 2, n+3, \dots, 2n+2\} \text{\ e tem grau\ }n.\\
			\Delta P(k) = F_{k-1}, \text{\ para\ } k \in \{n + 2, n+3, \dots, 2n+1\} \text{\ e tem grau\ }n - 1.\\
			\Delta^2 P(k) = F_{k-2}, \text{\ para\ } k \in \{n + 2, n+3, \dots, 2n\} \text{\ e tem grau\ }n - 2.\\
			\vdots\\
			\Delta^{n-1}P(k) = F_{k-n+1}, \text{\ para\ } k \in \{n + 2, n + 3\} \text{\ e tem grau\ }1.\\
			\Delta^{n}P(k) = F_{k-n}, \text{\ para\ } k \in \{n + 2\} \text{\ e tem grau\ }0.
		\end{gather*}

		Logo, $\Delta^n P(k) = F_{2}$, para todo $k$. Em especial, isso é válido para $k = n+3$.
		\begin{align*}
			\Delta^n P(n+3) = F_{2}	&\implies \Delta^{n-1} P(n+4) - \Delta^{n-1} P(n+3) = F_2\\
									&\implies \Delta^{n-1} P(n+4) = F_2 + F_4\\
									&\implies \Delta^{n-2} P(n+5) = F_2 + F_4 + F_6\\
									&\ \ \ \vdots\\
									&\implies \Delta^1 P(2n+2) = F_2 + F_4 + \cdots + F_{2n}\\
									&\implies P(2n+3) = F_2 + F_4 + \cdots + F_{2n} + F_{2n+2}\\
									&\implies P(2n+3) = F_{2n+3} - 1.
		\end{align*}
	\end{sk}




\end{document}
