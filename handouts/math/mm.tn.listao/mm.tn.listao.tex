\documentclass[10pt,a4paper]{article}
\usepackage[utf8]{inputenc}
\usepackage[brazilian]{babel}
\usepackage[left=2.5cm, right=2.5cm, top=2.5cm, bottom=2.5cm]{geometry}
%\usepackage[hide]{zeus}
\usepackage[section, prob-boxed]{zeus}
\usepackage{parskip}
\usepackage{transparent}
\usepackage{euler}

\title{Agregado de Problemas}
\author{\vspace{1.3ex}}
\nomail
\titlel{\includegraphics[width=.2\textwidth]{mm}}
\titler{Workshop de Teoria dos Números}
\pagestyle{empty}

\begin{document}	
	\zeustitle

	\section{Lista 1, PECI}

	\begin{prob}
		Calcule $123123 \div 1001$.
	\end{prob}
	
	\begin{prob}
		Pense em um inteiro de dois algarismos. Subtraia a soma desses algarismos. Divida o resultado por $9$. O que você obtém? Explique!
	\end{prob}
	
	\begin{prob}
		Existe algum número de $2$ algarismos tal que se subtraírmos o número formado pelos mesmos algarismos na ordem inversa o resultado é um número primo?
	\end{prob}
	
	\begin{prob}
		Encontre um número inteiro positivo menor do que $100$ que aumenta em $20\%$ quando a ordem de seus algarismos é invertida.
	\end{prob}
	
	\begin{prob}
		Explique por que o seguinte método funciona: Para elevar ao quadrado um número terminado em $5$, basta apagar esse último algarismo, multiplicar o número assim obtido por seu sucessor e acrescentar os algarismos $25$ no final do resultado.
	\end{prob}
	
	\begin{prob}
		Quantos são  os  números inteiros de $2$ algarismos que são iguais ao dobro do produto de seus algarismos?
	\end{prob}
	
	\begin{prob}
		O número $N$ tem três algarismos. O produto dos algarismos de $N$ é $126$ e a soma dos dois últimos algarismos de $N$ é $11$. O algarismo das centenas de $N$ é:
	\end{prob}
	
	\begin{prob}
		Determine o menor número inteiro positivo cujo primeiro algarismo é $1$ e que fica multiplicado por $3$ quando esse algarismo é transferido para o final do número.
	\end{prob}
	
	\begin{prob}
		Encontre todos os números cujo primeiro algarismo é $6$ e que ficam divididos por $25$ quando este algarismo é apagado.
	\end{prob}
	
	\begin{prob} Encontre todos os primos que são soma e diferença de dois primos.
	\end{prob}
	
	\begin{prob}
		Determine um número inteiro positivo de $6$ algarismos distintos tal que os resultados de suas multiplicações por $2, 3, 4, 5$ e $6$ são números formados pelos mesmos algarismos em outra ordem.
	\end{prob}
	
	\begin{prob}
		Qual é o maior inteiro positivo $n$ tal que os restos das divisões de $154$, $238$ e $334$ por n são iguais? 
	\end{prob}
	
	\begin{prob}
		Hoje é sábado. Que dia da semana será daqui a $99$ dias?
	\end{prob}
	
	\begin{prob} Prove que a fração $\dfrac{21n+4}{14n+3}$ é irredutível para todo inteiro positivo $n$.
	\end{prob}
	
	\begin{prob} 
	Prove que as expressões $2x+3y$ e $9x+5y$ são divisíveis por $17$ para o mesmo conjunto de valores inteiros de $x$ e $y$.
	\end{prob}
	
	\begin{prob} Determine todos os valores inteiros de $x$ tais que $\dfrac{15x^2-11x+37}{3x+2}$ é inteiro.
	\end{prob}
	
	\begin{prob} Qual é o maior inteiro $n$ para o qual $n^3+100$ é divisível por $n+10$?
	\end{prob}
	
	\begin{prob} Prove que, para todo natural $n$, \[(n!+1,(n+1)!+1)=1.\]
	\end{prob}
	
	\begin{prob} Prove que, para inteiros positivos $m$ e $a>1$, \[\left(\frac{a^m-1}{a-1},a-1\right)=1.\]
	\end{prob}
	
	\begin{prob}
		Os números na sequência $101, 104, 109, 116, \dots$ são gerados pela fórmula $a_n=100+n^2$, $n=1,2,3,\dots$. Sendo $d_n=(a_n,a_{n+1})$, qual é o valor máximo assumido por $d_n$?
	\end{prob}
	
	\begin{prob}
		Prove que existem exatamente três triângulos retângulos cujos lados têm por medida números inteiros e cuja área é numericamente igual ao dobro do perímetro.
	\end{prob}
	
	\begin{prob}
		Prove que $n^2$ é divisor de $(n+1)^n-1$, para todo inteiro positivo $n$.
	\end{prob}
	
	\begin{prob}
		Determine o maior inteiro $k$ tal que $2009^k$ é divisor de $2008^{2009^{2010}}+2010^{2009^{2008}}.$
	\end{prob}
	
	\begin{prob}[Números de Fermat]
		Seja $F_k=2^{2^k}+1$, com $k$ inteiro maior que $1$.
		\begin{enumerate}[label = (\alph*)]
			\item Prove que $(F_k,F_j)=1$ se, e somente se, $k\neq j$.
			\item Prove que os números primos da forma $2^n+1$ são números de Fermat.	
		\end{enumerate}
	\end{prob}
	
	\begin{prob}
			Prove que se $m$ e $n$ são números naturais e $m$ é ímpar, então $\left(2^m-1,2^n+1\right)=1$.
	\end{prob}
	
	\begin{prob}
			Encontre todas as ternas $(x, y, z)$ de números naturais distintos, dois a dois primos entre si, tais que a soma de quaisquer dois é divisível pelo terceiro.
	\end{prob}
	
	\begin{prob}  Um retângulo de lados inteiros $m,n$ é dividido em quadrados de lado $1$. Um raio de luz entra no retângulo por um dos vértices, na direção da bissetriz do ângulo reto, e é refletido sucessivamente nos lados do retângulo. Quantos quadrados são atravessados pelo raio de luz?
	\end{prob}
	
	\begin{prob} Ao converter a fração $m/n$ em um número decimal, onde $m$ e $n$ são inteiros positivos e $n$ é menor do que $100$, Vladimir encontrou um quociente que continha, em uma determinada posição após a vírgula decimal, os algarismos $167$, nesta ordem. Prove que Vladimir cometeu um erro nessa divisão.
	\end{prob}
	
	\begin{prob}[O jogo de Euclides]
		Um juiz fornece um conjunto de dois números inteiros positivos $C_1=\{x_1,y_1\}$ a dois jogadores e indica quem faz o primeiro lance. Um lance consiste em substituir $C_n=\{x_n,y_n\}$ por $C_{n+1}=\{x_{n+1},y_{n+1}\}$ tal que $x_{n+1}=\min C_n$ e $y_{n+1}= \max C_n -kx_{n+1}$, para algum $k$ inteiro positivo. 

	Ganha o jogador que obtiver pela primeira vez $y_{n+1}=0$. Determine os valores de $x_1/y_1$ para os quais o primeiro jogador possui uma estratégia vencedora, e descreva essa estratégia.
	\end{prob}
	
	\begin{prob}
		Qual é o dígito das unidades do número $3^{2009}$?
	\end{prob}
	

	\subsection*{Fatos que Ajudam}

	\begin{defn}[Divisibilidade]
		Sejam $a, b$ inteiros. Dizemos que \emph{$a$ divide $b$} se existe $k$ inteiro tal que $b=ka$. Neste caso usamos a notação $a  \mid  b$.
	\end{defn}

	\begin{prop}[Desigualdade]
		Se $a \mid b$ e $b \neq 0$, então $ |a| < |b| $.
	\end{prop}

	\begin{defn}[Máximo Divisor Comum]
		Denominamos \emph{máximo divisor comum de $a$ e $b$} ao maior inteiro positivo que é divisor tanto de $a$ como de $b$. Representaremos esse número por $(a,b)$.
	\end{defn}

	\begin{prop}[Algoritmo de Euclides]
		$(a,b)=(a, ax+b)$, para todo $x$.
	\end{prop}

	\begin{thm}[Teorema Fundamental da Aritmética]
		Se $(a,b)=1$ e $a  \mid  bc$, então $a  \mid  c$.
	\end{thm}

	\begin{defn}[Primos]
		Um número inteiro positivo é dito \emph{primo} quando tem exatamente dois divisores positivos.
	\end{defn}

	\begin{thm}[Fatoração Única]
		Todo inteiro positivo maior que $1$ pode ser expresso, de forma única salvo a ordem dos fatores, como produto de números primos.
	\end{thm}

	\begin{thm}[Teorema de Bezout]
		$(a, b)$ é o menor inteiro positivo da forma $ax + by$, com $x, y \in \ZZ$.
	\end{thm}

	\newpage
	\section{Lista 2, PECI}

	\begin{prob}[IMO]
		Quando $4444^{4444}$ é escrito em notação decimal, a soma dos seus algarismos vale $A$. Seja $B$ a soma dos algarismos de $A$. Determine a soma dos algarismos de $B$.
	\end{prob}

	\begin{prob}
		É possível, para algum inteiro positivo $n$, escrever os números $n, n^2$ e $n^3$ utilizando apenas uma vez cada um dos algarismos $0,1,2,3,4,5,6,7,8$ e $9$?
	\end{prob}

	\begin{prob}[OBM 1989]
		Se $n$ é um inteiro positivo tal que $\frac{n(n+1)}3$ é inteiro e quadrado perfeito, prove que $n$ é múltiplo de $3$ e que tanto $n+1$ como $\frac n3$ são quadrados perfeitos.
	\end{prob}

	\begin{prob}[OBM 2001]
		Dado um inteiro $a_0>1$, define-se a sequência $(a_n)_{n\geq 0}$ da seguinte maneira: para cada $k\geq0$, $a_{k+1}$ é o menor inteiro maior que $a_k$ tal que $mdc(a_{k+1},a_0a_1\cdots a_k)=1$. Determine todos os valores de $a_0$ para os quais todos os termos da sequência são primos ou potências de primos.
	\end{prob}

	\begin{prob}[IMO 1997]
		Encontre todos os pares $(a,b)$ de inteiros positivos tais que $$a^{b^2}=b^a.$$
	\end{prob}

	\begin{prob}[IMO 1992]
		Encontre todos os inteiros $a,b,c$ com $1<a<b<c$ tais que o produto $(a-1)(b-1)(c-1)$ é um divisor de $abc-1$.
	\end{prob}

	\begin{prob}[IMO 1994]
		Determine todos os pares ordenados $(m,n)$ de inteiros positivos tais que $$\dfrac{n^3+1}{mn-1}$$ é inteiro.
	\end{prob}

	\begin{prob}[IMO 1998]
		Determine todos os pares $(a,b)$ de inteiros positivos tais que $ab^2+b+7$ é um divisor de $a^2b+a+b$.
	\end{prob}

	\begin{prob}[IMO 2006]
		Determine todos os pares de inteiros $(x,y)$ tais que $$1+2^x+2^{2x+1}=y^2.$$
	\end{prob}

	\begin{prob}
		Resolva as seguintes equações em congruências:
		\begin{enumerate}[label = (\alph*)]
			\item $2x \equiv 1 \imod{17}$

			\item $3x \equiv 6 \imod{18}$

			\item $25x \equiv 15 \imod{29}$

			\item $36x \equiv 8 \imod{102}$

			\item  $14x \equiv 36 \imod{18}$
		\end{enumerate}
	\end{prob}

	\begin{prob} Encontre todas as soluções inteiras das equações 
		\begin{enumerate}[label = (\alph*)]
			\item $48x + 7y = 17$	

			\item $9x + 16y = 35$	

			\item $5x - 53y = 17$	

			\item $75x - 131y = 6$	

			\item $12x + 25y = 331$
		\end{enumerate}
	\end{prob}

	\begin{prob} Resolva o sistema de congruências
		\[ \left \{ \begin{array}{l}
			x \equiv 8 \imod{9}\\ 
			x \equiv 2 \imod{3} \\
			x \equiv 5 \imod{7} \\
		\end{array} \right. \]
	\end{prob}

	\newpage
	\section{Lista 3, PECI}

	\begin{prob}
		Determine o último dígito do número $\left\lfloor10^{1992}\over 10^{83}+7\right\rfloor$.
	\end{prob}

	\begin{prob}
		Determine os dois últimos dígitos de $9^{(9^{9})}$ e de $14^{(14^{14})}$.
	\end{prob}

	\begin{prob}
		Determine os três últimos dígitos de $7^{9999}$.
	\end{prob}

	\begin{prob}
		Sejam $x$, $y$, $z$ inteiros tais que $x^3+y^3-z^3$ é um múltiplo de $7$. Mostre que um desses números é múltiplo de $7$.
	\end{prob}

	\begin{prob}
		Mostre que
		\begin{enumerate}[label = (\alph*)]
			\item Se $n$ é um inteiro positivo maior do que $1$ e $2^n+n^2$ é primo, então $n\equiv3\pmod{6}$.
			\item Seja $x\equiv 23\pmod{24}$, se $a$, $b$ são inteiros positivos tais que $ab=x$, então $a+b$ é um múltiplo de $24$.
			\item Se $n^2+m$ e $n^2-m$ são quadrados perfeitos, então $m$ é divisível por $24$.
			\item Se $2n+1$ e $3n+1$ são quadrados perfeitos, então $n$ é divisível por $40$.  
		\end{enumerate}
	\end{prob}

	\begin{prob} Seja $S$ um conjunto de números primos, tal que se $a$, $b\in S$,
	então $ab+4\in S$.  Mostre que $S$ deve ser o conjunto vazio.
	\end{prob}

	\begin{prob} Prove que a sequência $11, 111, 1111, \ldots$ não contém
	quadrados.
	\end{prob}

	\begin{prob} Seja $d$ um inteiro positivo diferente de $2$, $5$ e $13$. Mostre
	que podemos escolher $a$ e $b$ distintos, $a, b\in \{2, 5, 13, d\}$,
	tais que $ab-1$ não seja um quadrado perfeito.
	\end{prob}

	\begin{prob} Sejam $d_1, d_2,\ldots,d_k$ todos os divisores positivos de um
	natural $n$, onde $1=d_1<d_2<\cdots<d_k=n$. Encontre todos os naturais
	$n$ para os quais $k\geq 4$ e $d^2_1+d^2_2+d^2_3+d^2_4=n$.
	\end{prob}

	\begin{prob} Para $n$ e $k\in N$, definimos $F(n, k)=\sum\nolimits^n_{r=1}
	r^{2k-1}$. Prove que $F(n, 1)$ divide $F(n, k)$.
	\end{prob}

	\begin{prob} Prove que existe uma sucessão $a_0, a_1,\ldots,a_k,\ldots$, onde
	$a_i\in \{0, 1, 2,\ldots,9\}$, com $a_0=6$ tal que para cada inteiro
	positivo $n$, sendo $x_n=a_0+10a_1+100a_2+\cdots+10^{n-1}a_{n-1}$,
	$x^2_n-x_n$ é divisível por $10^n$.
	\end{prob}

	\begin{prob} Existe um múltiplo de $5^{100}$ que não contém zeros em sua
	representação decimal?
	\end{prob}

	\begin{prob} Para cada natural $n$, prove que há $n$ naturais consecutivos,
	nenhum dos quais é potência inteira de um número primo.
	\end{prob}

	\begin{prob} Existem um milhão de inteiros consecutivos cada um dos quais é
	divisível pelo quadrado de um número primo?
	\end{prob}

	\begin{prob} Existem 14 inteiros positivos consecutivos cada um dos quais é
	divisível por um ou mais primos $p$ do intervalo $2\leq p\leq 11$?
	\end{prob}

	\begin{prob} Existem 21 inteiros positivos consecutivos cada um dos quais é
	divisível por um ou mais primos $p$ do intervalo $2\leq p\leq 13$?
	\end{prob}

	\begin{prob} Mostre que existem infinitos conjuntos de 1983 inteiros positivos
	consecutivos, cada um dos quais é divisível por algum número da forma
	$a^{1983}$, onde $a$ é um inteiro positivo.
	\end{prob}

	\begin{prob}
		Um ponto do reticulado é dito visível se $(x, y)=1$. É verdade que dado um inteiro positivo $n$, existe um ponto do reticulado cuja distância a todo ponto visível é maior ou igual a $n$?
	\end{prob}

	\begin{prob}
		Existe um número ímpar $n>1$ que não é da forma $2^k+p$, onde $k$ é um número natural e $p$ é um primo ou o número $1$?
	\end{prob}

	\begin{prob}
		Prove que existe um inteiro positivo $k$ tal que $k\cdot2^n+1$ é composto para todo inteiro positivo $n$.
	\end{prob}

	\begin{prob}[IMO]
		Encontre um par de inteiros positivos $a$ e $b$ tais que $ab(a+b)$ não é divisível por $7$ e $(a+b)^7-a^7-b^7$ é divisível por $7^7$.
	\end{prob}

	\begin{prob}
		Mostre que, para todo inteiro positivo $n$, a sequência $$2,2^2,2^{2^2},\dots (\mod n)$$ é constante a partir de um certo ponto.
	\end{prob}

	\begin{prob}[OBM]
		Mostre que existe um número da forma $199 \cdots 91$, com mais de dois noves, que é múltiplo de $1991$.
	\end{prob}

	\begin{prob}
		Os três últimos dígitos de $1978^m$ são iguais aos três últimos dígitos de $1978^n$ $(1 \leq m <n; \mbox{\;\;}m,n \in \NN)$. Determine $m$ e $n$ tais que $m+n$ seja mínimo.
	\end{prob}

	\begin{prob}
		Prove que o conjunto $\{2^k-3\mid k=2, 3,\ldots\}$ contém um subconjunto infinito cujos membros são primos dois a dois.
	\end{prob}

	\begin{prob}
		Seja $n>3$ um inteiro ímpar. Prove que existe um primo $p$ tal que $p$ divide $2^{\phi(n)}-1$, mas não divide $n$.
	\end{prob}

	\begin{prob}
		Mostre que para todo inteiro positivo $n$ existe uma potência de $2$ com uma string de pelo menos $n$ zeros sucessivos.
	\end{prob}

	\begin{prob}
		Um inteiro positivo é denominado número duplo se sua representação decimal consiste de um bloco de dígitos não iniciado por zero, seguido imediatamente de um bloco idêntico. Então, por exemplo, $360360$ é um número duplo, mas $36036$ não é. Mostre que existem infinitos números duplos que são quadrados perfeitos.
	\end{prob}

	\begin{prob}
		Prove que existe uma potência de $2$ cujos últimos 1000 dígitos são todos iguais a $1$ ou $2$.
	\end{prob}

	\begin{prob}
		Sendo $k\geq2$ e $n_1, n_2, \ldots, n_k\geq1$ números naturais tais que $n_2 \mid 2^{n_1}-1$, $n_3 \mid 2^{n_2}-1$, $\dots$, $n_k \mid 2^{n_{k-1}}-1$ e $n_1 \mid 2^{n_k}-1$.

		Mostre que $n_1=n_2=\cdots=n_k=1$.
	\end{prob}

	\begin{prob}
		Mostre que se $n$ é um inteiro maior do que $1$, então $n$ não divide $2^n-1$.
	\end{prob}

	\begin{prob}
		Determine todos os inteiros $n\geq1$ tais que $({2^n+1})/n^2$ seja inteiro.
	\end{prob}

	\begin{prob}
		Existe um inteiro positivo $n$ com exatamente $2000$ fatores primos distintos tal que $n \mid 2^n+1$?
	\end{prob}

	\begin{prob}
		Determine todos os pares $(n,p)$ de inteiros estritamente positivos tais que
		\begin{enumerate}[label = (\textit{\roman*})]
			\item $p$ é primo,
			\item $n\leq 2p$, e
			\item $(p-1)^n+1$ é divisível por $n^{p-1}$.
		\end{enumerate}
	\end{prob}

	\begin{prob} Se $p$ é um primo $(p>5)$, então o número $(p-1)!+1$ não é uma
	$k$-ésima potência de $p$, $k$ natural.
	\end{prob}

	\begin{prob} Se $p$ é um primo ímpar, mostre que
	$$1^{p-1}+2^{p-1}+\cdots+(p-1)^{p-1}\equiv p+(p-1)!\pmod{p^2}.$$
	\end{prob}

	\begin{prob} (a) Encontre todos os primos $p$ tais que $p$ divide $5^{p^2} +
	1$.
	\end{prob}

	\begin{prob}{(b)} Mostre que existem infinitos pares de primos $(p;q)$ tais
	que $p$ divide $q^{p^2} + 1$.
	\end{prob}

	\begin{prob} Mostre que se $n$ é ímpar então $2^{n!} - 1$ é divisível por $n$.
	\end{prob}

	\begin{prob} Prove que, dado um primo $p$, existem infinitos números da forma
	$2^n - n$, $n$ natural, divisíveis por $p$.
	\end{prob}

	\begin{prob} Seja $a > 1$ um número inteiro. Encontre todos os números naturais
	que dividem algum dos números da forma $a_n = a^n + a^{n-1} + \cdots +
	a + 1$.
	\end{prob}

	\begin{prob} Um número é dito {\sl alternado} se seus dígitos na base decimal
	são alternadamente nulo e não nulo, e se seu dígito das unidades é não
	nulo. Por exemplo, $4050201$ é alternado, mas $4050$ não é. Encontre
	todos os números inteiros positivos que não dividem nenhum número
	alternado.
	\end{prob}

	\begin{prob} Determine todas as ternas de números $(a;m;n)$ tais que $a^m + 1$
	divide $(a + 1)^n$.
	\end{prob}

	\begin{prob} Seja $A(m,n) = m^{3^{4n} + 6} - m^{3^{4n}} - m^5 + m^3$. Encontre
	todos os valores de $n$ para os quais $A(m,n)$ é divisível por $1992$
	para todo $m$ inteiro.
	\end{prob}

	\begin{prob} Seja $p$ um número primo. Demonstre que existe um número primo $q$
	tal que, para todo inteiro $n$, o número $n^p - p$ não é divisível por
	$q$.
	\end{prob}

	\subsection*{Fatos que ajudam}

	\begin{defn}[Função $\phi$ de Euler]
		A função $\phi\colon Z^*_+\longrightarrow Z^*_+$ conta o número de inteiros positivos menores ou iguais a $n$, primos com $n$.  $$\phi(n)=n\prod_{{p \mid n\atop p {\rm\ primo}}} \left(1-{1\over p}\right).$$
	\end{defn}

	\begin{thm}[Teorema de Euler-Fermat]
	Sendo $\mdc(m,a)=1$, então
	$$a^{\phi(m)}\equiv1\pmod{m}$$

	Em particular, quando $m$ é primo, temos o Pequeno Teorema de Fermat:
	$$a^m\equiv a\pmod{m}.$$
	\end{thm}

	\begin{defn}[Menor Expoente]
		Seja $d$ o menor expoente tal que $a^d\equiv1 \pmod{m}$. Então qualquer outro expoente $t$ para o qual $a^t\equiv1\pmod{m}$ é um múltiplo de $d$.
	\end{defn}

	\begin{thm}[Teorema de Wilson]
		$p$ é primo se, e somente se, $(p-1)!\equiv-1 \pmod{p}$
	\end{thm}

	\newpage
	\section{Lista Singular de Teoria dos Números, Matematicamente}

	\emph{Aqui só estão os problemas que nào apareceram acima.}

	\begin{prob}
		Sejam $a$ e $b$ inteiros positivos. Prove que se $4ab - 1$ divide $(4a^2 - 1)^2$, então $a = b$.
	\end{prob}

	\begin{prob}
		Determine todas as ternas $(p, x, y)$ de inteiros positivos, com $p$ primo, tais que $x^{p-1} + y$ e $x + y^{p-1}$ são potências de $p$.
	\end{prob}

	\newpage
	\section{Achados do Zeus}

	\problem{math/bangladesh/mo/2019/1}
	\problem{math/canada/1970/1}
	\problem{math/canada/1970/4}
	\problem{math/canada/1970/7}
	\problem{math/canada/1970/10}
	\problem{math/book/andrei_negut/problems_for_the_mathematical_olympiads/N5}
	\problem{math/balkan/2017/3}
	\problem{math/putnam/2018/a1}
	\problem{math/putnam/2017/a1}
	\problem{math/putnam/2017/b2}


\end{document}
