\documentclass[10pt,a4paper]{article}
\usepackage[utf8]{inputenc}
\usepackage[brazilian]{babel}
\usepackage{lmodern}
\usepackage[left=2.5cm, right=2.5cm, top=2.5cm, bottom=2.5cm]{geometry}
\usepackage[prob-boxed]{zeus}
\usepackage{parskip}
\usepackage{transparent}

%\usepackage[printwatermark]{xwatermark}
%\newwatermark[firstpage, angle=0,scale=3,xpos=-69,ypos=118]{{\transparent{0.3}\includegraphics[scale=0.55]{pensi_pdf}}}

\title{Minha sequência favorita}
\author{Guilherme Zeus Dantas e Moura}
\mail{zeusdanmou@gmail.com}
\titlel{}
\titler{}

\begin{document}	
	\zeustitle

	\begin{prob}
		Considere uma coloração dos inteiros não-negativos em duas cores, azul e vermelho. É verdade que existe uma progressão aritimética infinita monocromática?
	\end{prob}

	\begin{prob}
		Seja $t$ um inteiro positivo.

		É possível achar uma partição do conjunto $\{0, 1, \dots, 2^n - 1 \}$ em dois conjuntos disjuntos $I$ e $J$ tal que \[
			\sum_{i\in I} i^k = \sum_{j \in J} j^k,
		\]
		para $k \in \{0, 1, \dots, t\}$.
	\end{prob}

	\begin{prob}
		Um alfabeto é um conjunto finito de letras. Uma palavra é uma sequência (finita ou infinita) de letras.

		Uma palavra é dita \emph{uniformemente recorrente} se toda subpalavra aparece infinitas vezes.

		Existe uma palavra infinita que é uniformemente recorrente, mas não periódica?
	\end{prob}

	\begin{prob}
		Um \emph{quadrado} é uma palavra que resulta da concatenação de duas palavras finitas iguais. Por exemplo, \texttt{AABCAABC} é um quadrado.

		Uma palavra é dita \emph{livre de quadrados} se nenhuma subpalavra é um quadrado. Por exemplo, \texttt{ABCBAC} é livre de quadrados, mas \texttt{BAAC} não.

		Determine se existe uma palavra infinita e livre de quadrados num alfabeto com $2$ letras. E com $3$ letras?
	\end{prob}

	\begin{ques}
		Num colégio com com $2n$ alunos, o professor de educação física foi substituido por um professor de matemática. O professor de matemática escolheu \playerA{$\mathcal{A}$} e \playerB{$\mathcal{B}$} como capitões de duas equipes para uma partida de queimada. 

		Em alguma ordem, \playerA{$\mathcal{A}$} e \playerB{$\mathcal{B}$} escolhem um aluno por vez para se juntar à sua equipe.

		Determine qual é a ordem ``mais justa'' das escolhas de \playerA{$\mathcal{A}$} e \playerB{$\mathcal{B}$}.
	\end{ques}
\end{document}
