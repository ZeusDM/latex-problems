\documentclass[10pt,a4paper]{article}
\usepackage[utf8]{inputenc}
\usepackage[brazilian]{babel}
\usepackage{lmodern}
\usepackage[left=2.5cm, right=2.5cm, top=2.5cm, bottom=2.5cm]{geometry}

\usepackage[]{zeus}

\title{Problemas do Curso de Combinatória do IMPA \\[1mm] Anotações de Aula}
\author{Guilherme Zeus Dantas e Moura}
\mail{\texttt{zeusdanmou@gmail.com}}
\titlel{Aula Particular}
\titler{14 de julho de 2021}

\DeclareMathOperator\ex{ex}

\begin{document}	
	%\twocolumn[\zeustitle]
	\zeustitle

	\section*{\textsf{Definições}}

	\begin{itemize}
		\item $[n] = \{1, 2, \dots, n\}$;
		\item $\mathcal{P}(S)$ é o conjunto de todos os subconjuntos de $S$.
		\item $\binom{S}{k}$ é o conjunto de todos os subconjuntos de $S$ que possuem exatamente $k$ elementos.
		\item $\mathcal{S} \subset \mathcal{P}([n])$ é uma \emph{cadeia} se $A \subset B$ ou $B \subset A$, para quaisquer $A, B \in \mathcal{S}$, $A \neq B$.
		\item $\mathcal{S} \subset \mathcal{P}([n])$ é uma \emph{anti-cadeia} se $A \not\subset B$, para quaisquer $A, B \in \mathcal{S}$, $A \neq B$.
		\item $\mathcal{S} \subset \mathcal{P}([n])$ é \emph{intersectante} se $A \cap B \neq \varnothing$, para quaisquer  $A, B \in \mathcal{S}$.
		\item Seja $\ex(n, H)$ o número máximo de arestas que um grafo $G$ com $n$ vértices pode ter de modo que não existam cópias de $H$ em $G$.
	\end{itemize}

	\section*{\textsf{Problemas}}


\begin{prob}[Sperner, 1910's] \label{thm:sperner}
	Se $\mathcal{S} \subset \mathcal P([n])$ é uma anti-cadeia, então $|\mathcal{S}| \le \binom{n}{n/2}$.
\end{prob}


\begin{sol}[of \nameref{thm:sperner}]
	The example is $\binom{[n]}{n/2}$.

	\begin{lem}[LYMB, 1960's]\label{lem:lymb}
		If $\mathcal{S} \subset \mathcal P([n])$ is an anti-chain, then $\sum_{A \in \mathcal{S}} \frac{1}{\binom{n}{|A|}} \leq 1$.
	\end{lem}

	\begin{dem}[of \nameref{lem:lymb}]
		Let's count the pairs $(\pi, A)$ such that $\pi$ is a permutation of $[n]$, $A \in \mathcal{S}$, and  $\{\pi(1), \pi(2), \dots, \pi(|A|)\} = A$.

		For each $A \in \mathcal{S}$, the number of $\pi$ such that $\{\pi(1), \pi(2), \dots, \pi(|A|)\}$ is equal to $|A|!(n - |A|)!$.

		For each $\pi$, the number of $A \in \mathcal{S}$ such that $\{\pi(1), \dots, \pi(|A|)\}$ is at most $1$, since $\mathcal{S}$ is an anti-chain.

		Therefore, \[
			\sum_{A \in \mathcal{S}} |A|!(n-|A|)! \le n!
			\implies
			\sum_{A \in \mathcal{S}} \frac{1}{\binom{n}{|A|}} \le 1.
		\]
	\end{dem}
	We know that $\binom{n}{k} \le \binom{n}{n/2}$. Thus, by \nameref{lem:lymb}, \[
		1 \ge \sum_{A\in \mathcal{S}} \frac{1}{\binom{n}{|A|}} \ge \sum_{A\in \mathcal{S}} \frac{1}{\binom{n}{n/2}} = \frac{|\mathcal{S}|}{\binom{n}{n/2}}
	\]
\end{sol}

\begin{prob}
	Suponha que $\mathcal{S} \subset \mathcal{P}([n])$ é intersectante. Prove que $|\mathcal{S}| \le 2^{n-1}$.
\end{prob}

\begin{sk}
	The example is $\{A \in \mathcal{P}([n]) : 1 \in A\}$.
	At most one of $(A, \overline{A})$ can be on $\mathcal{S}$. Therefore, $|\mathcal{S}| \le 2^{n-1}$.
\end{sk}

\begin{prob}[Erd\H{o}s-Ko-Rado, 1961]
	Sejam $k$ e $n$ inteiros positivos tais que $k < \frac{n+1}{2}$.
	Suponha que $\mathcal{S} \subset \binom{[n]}{k}$ é intersectante. Prove que $|\mathcal{S}| \le \binom{n-1}{k-1}$.
\end{prob}

\begin{sk}
	O exemplo é $\{A \in \binom{[n]}{k} : 1 \in A\}$.

	A ideia é contar os pares $(\pi, A)$ tais que $\pi$ é uma permutação cíclica, $A \in \mathcal{S}$, e $A = \{\pi(t+1), \pi(t+2), \dots, \pi(t+k)\}$, for some $t$.
	Contando por $\pi$, a quantidade é no máximo $k (n-1)!$. Contando por $A$, a quantidade é exatamente $\left|S\right| k!(n-k)!$. Portanto, $|S| \leq \binom{k-1}{n-1}$.

	Mais detalhes em \href{http://tcs.nju.edu.cn/wiki/index.php/Combinatorics_(Fall_2010)/Extremal_set_theory#Katona.27s_proof}{Department of Computer Science and Technology at Nanjing University}.
\end{sk}

\begin{prob}
	Sejam $k$ e $r$ inteiros positivos. Todo inteiro positivo é pintado com uma de $r$ cores. Prove que existe uma progressão aritmética monocromática com $k$ termos.
\end{prob}

\begin{sol}
	We will use induction on $k$. Note that $W(r, 1) = 1$.

	We shall find $r$ color-focused $(k-1)$-arithmetic progressions.
	\begin{lem}
		There exists $n = n(s, r)$ such that, for every coloring $c\colon [n] \to [r]$, there exists a monochromatic $k$-arithmetic progression or $s$ color-focused $(k-1)$-arithmetic progressions.
	\end{lem}
	\begin{dem}
		Induction on $s$. If $s = 1$, then $n(1, r) = W(r, k-1) < \infty$.

		Suppose $s > 1$. Let $N = 2n(s-1, r)$. Consider $W(r^N, k-1) < \infty$ blocks of size $N$. There is an arithmetic progression of equally-colored blocks of size $k-1$, let $D$ be the distance of consecutive blocks in the arithmetic progression of blocks. Since the first half of the block has $n(s-1, r)$ elements, there exists a monochromatic $k$-arithmetic progression (which means we're done), or $s-1$ color-focused $(k-1)$-arithmetic progressions -- their focus $f$ surely lies inside the block of size $N$.

		Let the $s-1$ color-focused $(k-1)$-arithmetic progressions in the first block be $PA_{k-1}(a_1, d_1)$, $\dots$, $PA_{k-1}(a_{s-1}, d_{s-1})$, with focus $f_1$. The proposed $s$ color-focused $(k-1)$-arithmetic progressions are $PA_{k-1}(a_1, d_1 + d), \dots, PA_{k-1}(a_{s-1}, d_{s-1} + d), PA_{k-1}(f_1, d)$.

		Therefore,  \[
			n(s, r) \le 2 \cdot W(r^{2n(s-1, r)}, k-1) \cdot 2n(s-1, r).
		\]
	\end{dem}

	Therefore, for suitable large $n$, there must exist a large $k$-arithmetic progression.
\end{sol}


%\begin{prob}
%	Em contrapartida ao problema anterior, prove que existe uma coloração com $2$ cores para a qual não existe uma progressão aritmética monocromática com infinitos termos.
%\end{prob}

%\begin{prob}
%	Determine $\ex(n, K_3)$.
%\end{prob}

%\begin{prob}
%	Mostre que $\ex(n, C_4) \leq \frac{n^{3/2}}{2}$.
%\end{prob}

%\begin{prob}
%	Seja $T$ uma árvore com $k$ vértices. Mostre que $\frac{k-2}{2}n \leq \ex(n, T) \leq (k-1)n$.
%
%	\rem{Se for ajudar, você pode supor que $n$ é múltiplo de $d$, para um $d$ da sua escolha.}
%\end{prob}

\end{document}
