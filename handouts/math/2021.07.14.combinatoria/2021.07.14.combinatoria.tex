\documentclass[10pt,a4paper]{article}
\usepackage[utf8]{inputenc}
\usepackage[brazilian]{babel}
\usepackage{lmodern}
\usepackage[left=2.5cm, right=2.5cm, top=2.5cm, bottom=2.5cm]{geometry}

\usepackage[]{zeus}

\title{Problemas do Curso de Combinatória do IMPA}
\author{Guilherme Zeus Dantas e Moura}
\mail{\texttt{zeusdanmou@gmail.com}}
\titlel{Aula Particular}
\titler{14 de julho de 2021}

\DeclareMathOperator\ex{ex}

\begin{document}	
	%\twocolumn[\zeustitle]
	\zeustitle

	\section*{\textsf{Definições}}

	\begin{itemize}
		\item $[n] = \{1, 2, \dots, n\}$;
		\item $\mathcal{P}(S)$ é o conjunto de todos os subconjuntos de $S$.
		\item $\binom{S}{k}$ é o conjunto de todos os subconjuntos de $S$ que possuem exatamente $k$ elementos.
		\item $\mathcal{A} \subset \mathcal{P}([n])$ é uma \emph{anti-cadeia} se $A \not\subset B$, para quaisquer $A, B \in \mathcal{A}$, $A \neq B$.
		\item $\mathcal{A} \subset \mathcal{P}([n])$ é \emph{intersectante} se $A \cap B \neq \varnothing$, para quaisquer  $A, B \in \mathcal{A}$.
		\item Seja $\ex(n, H)$ o número máximo de arestas que um grafo $G$ com $n$ vértices pode ter de modo que não existam cópias de $H$ em $G$.
	\end{itemize}

	\section*{\textsf{Problemas}}


\begin{prob}%[Sperner, 1910's] \label{thm:sperner}
	Se $\mathcal A \subset \mathcal P([n])$ é uma anti-cadeia, então $|\mathcal A| \le \binom{n}{n/2}$.
\end{prob}

%The example is $\binom{[n]}{n/2}$.

%\begin{dem}[of \nameref{thm:sperner}]
%	We know that $\binom{n}{k} \le \binom{n}{n/2}$. Thus, by \nameref{lem:lymb}, \[
%		1 \ge \sum_{A\in \mathcal{A}} \frac{1}{\binom{n}{|A|}} \ge \sum_{A\in \mathcal{A}} \frac{1}{\binom{n}{n/2}} = \frac{|\mathcal{A}|}{\binom{n}{n/2}}
%	\]
%\end{dem}

%\begin{lem}[LYMB, 1960's]\label{lem:lymb}
%	If $\mathcal A \subset \mathcal P([n])$ is an anti-chain, then $\sum_{A \in \mathcal A} \frac{1}{\binom{n}{|A|}} \leq 1$.
%\end{lem}

%\begin{dem}[of \nameref{lem:lymb}]
%	Let's count the pairs $(\pi, A)$ such that $\pi$ is a permutation of $[n]$, $A \in \mathcal{A}$, and  $\{\pi(1), \pi(2), \dots, \pi(|A|)\} = A$.

%	For each $A \in \mathcal{A}$, the number of $\pi$ such that $\{\pi(1), \pi(2), \dots, \pi(|A|)\}$ is equal to $|A|!(n - |A|)!$.

%	For each $\pi$, the number of $A \in \mathcal{A}$ such that $\{\pi(1), \dots, \pi(|A|)\}$ is at most $1$, since $\mathcal{A}$ is an anti-chain.

%	Therefore, \[
%		\sum_{A \in \mathcal{A}} |A|!(n-|A|)! \le n!
%		\implies
%		\sum_{A \in \mathcal{A}} \frac{1}{\binom{n}{|A|}} \le 1.
%	\]
%\end{dem}

\begin{prob}
	Suponha que $\mathcal{A} \subset \mathcal{P}([n])$ é intersectante. Prove que $|A| \le 2^{n-1}$.
\end{prob}

%\begin{sk}
%	At most one of $(S, \overline{S})$ can be on $\mathcal{A}$.
%\end{sk}

\begin{prob}%[Erd\H{o}s-Ko-Rado, 1961]
	Sejam $k$ e $n$ inteiros positivos tais que $k < \frac{n+1}{2}$.
	Suponha que $\mathcal{A} \subset \binom{[n]}{k}$ é intersectante. Prove que $|A| \le \binom{n-1}{k-1}$.
\end{prob}

\begin{prob}
	Sejam $k$ e $r$ inteiros positivos. Todo inteiro positivo é pintado com uma de $r$ cores. Prove que existe uma progressão aritmética monocromática com $k$ termos.
\end{prob}

\begin{prob}
	Em contrapartida ao problema anterior, prove que existe uma coloração com $2$ cores para a qual não existe uma progressão aritmética monocromática com infinitos termos.
\end{prob}

\begin{prob}
	Determine $\ex(n, K_3)$.
\end{prob}

\begin{prob}
	Mostre que $\ex(n, C_4) \leq \frac{n^{3/2}}{2}$.
\end{prob}

\begin{prob}
	Seja $T$ uma árvore com $k$ vértices. Mostre que $\frac{k-2}{2}n \leq \ex(n, T) \leq (k-1)n$.

	\rem{Se for ajudar, você pode supor que $n$ é múltiplo de $d$, para um $d$ da sua escolha.}
\end{prob}

\end{document}
