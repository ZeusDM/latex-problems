\documentclass[10pt, a4paper]{article}
\usepackage[utf8]{inputenc}
\usepackage[brazilian]{babel}
\usepackage{lmodern}
\usepackage[left=2cm, right=2cm, top=2cm, bottom=2.5cm]{geometry}
\usepackage{indentfirst}

\usepackage[pensi]{zeuscolor}
\usepackage{zeusall}

\title{Contagem Dupla}
\author{Guilherme Zeus Moura}
\mail{zeusdanmou@gmail.com}
\titlel{Turma Olímpica}
\titler{{\footnotesize v. 1} -- 05 de Março de 2020}

\begin{document}	
	\zeustitle

	\section{Introdução}

	Em poucas palavras, contagem dupla é algo que você já fez várias vezes: calcular algo de duas maneiras. Nos problemas a seguir, vamos contar algo de duas maneiras e igualar.

	\subsection{Alerta}

	Se você já conhece algum desses problemas, ótimo! Essa é uma oportunidade de treinar a sua escrita de problemas.

	\section{Problemas}

	Os problemas não estão ordenados por dificuldade!

	\begin{prob} %Davi
		Uma bola de futebol é feita com 32 peças de couro. 12 delas são pentágonos regulares e as outras 20 são hexágonos também regulares. Os lados dos pentágonos são iguais aos dos hexágonos de forma que possam ser costurados. Cada costura une dois lados de duas deças peças. Quantas costuras são feitas na fabricação de uma bola de futebol?
	\end{prob}

	\begin{prob} %Luciano
		Em uma casa térrea, todos os cômodos têm um número par de portas. Prove que o número de portas que ligam a casa ao exterior é par.	
	\end{prob}

	\begin{prob} %Luciano
		Em uma escola, há $b$ professores e $c$ estudantes que satisfazem as seguintes condições:
		
		\begin{itemize}
			\item Cada professor ensina a exatamente $k$ estudantes.
			\item Para cada dois estudantes distintos, existem exatamente $h$ professores que ensinam a ambos.
		\end{itemize}

		Prove que $bk(k-1) = hc(c-1)$.
	\end{prob}

	\begin{prob} %Victoria Krakovna
		Prove a seguinte identidade: \[\binom{n}{1} + 2 \binom{n}{2} + 3 \binom{n}{3} + \cdots + n\binom{n}{n} = n2^{n-1}.\]
	\end{prob}

	\begin{prob}[China Hong Kong 2007]
		Em uma escola com $2007$ meninas e $2007$ meninos, cada estudante faz parte de, no máximo, $100$ clubes. Sabemos que qualquer dois estudantes de gêneros opostos estão em um mesmo clube. Mostre que existe um clube com pelo menos $11$ meninos e $11$ meninas.
	\end{prob}

	%\begin{prob}[China Hong Kong 2007]
		%In a school there are $2007$ girls and $2007$ boys. Each student joins no more than $100$ clubs in the school. It is known that any two students of opposite genders have joined at least one common club. Show that there is a club with at least $11$ boys and $11$ girls.
	%\end{prob}

	\begin{prob} %Luciano
		Dado $n$ inteiro, seja $d(n)$ o número de divisores de $n$. Seja $D(n)$ o número médio de divisores dos números entre $1$ e $n$, isto é, \[D(n) = \frac{1}{n} \sum_{j=1}^{n} d(j).\]

		Mostre que \[\sum_{i=2}^n \frac{1}{i} \le D(n) \le \sum_{i=1}^n \frac{1}{i}.\]
	\end{prob}

	\begin{prob}[OBM] %Luciano
		Em um torneio de xadrez, cada participante joga com cada um dos outros exatamente uma vez. Uma vitória vale $1$ ponto, um empate vale $\frac{1}{2}$ pontos e uma derrota vale $0$ pontos. Cada jogandor ganhou a mesma quantidade de pontos contra homens e contra mulheres. Prove que a quantidade de participantes é um quadrado perfeito.
	\end{prob}

	\problem{math/imo/1987/1}

	\begin{prob}[Irã 2010, 6]
		Uma escola possui $n$ estudantes, e cada estudante possui liberdade para escolher assistir as aulas que quiser. Cada aula possui pelo menos dois estudantes. Sabemos que, se duas aulas diferentes tem pelo menos dois estudantes em comum, então o número de estudantes nessas aulas são diferentes. Prove que o número de aulas é menor ou igual a $(n - 1)^2$.
	\end{prob} %Victoria

	%\begin{prob}[Irã 2010, 6]
		%A school has $n$ students, and each student can take any number of classes. Every class has at least two students in it. We know that if two different classes have at least two common students, then the number of students in these two classes is different. Prove that the number of classes is not greater that $(n - 1)^2$.
	%\end{prob} %Victoria

	\begin{prob}[Teorema de Euler] %Shine
		Um poliedro é um sólido delimitado por polígonos. Sejam $V$, $A$ e $F$ as quantidades de vértices, arestas e faces de um poliedro convexo. Prove que \[V - A + F = 2.\]
	\end{prob}

	\problem{math/imo/1998/2}
	
	\begin{prob}[MOP Practice Test 2007]
		Em uma matriz $n \times n$, cada um dos números em $\{1, 2, \dots , n\}$ aparece exatamente $n$ vezes. Mostre que existe uma linha ou coluna com pelo menos $\sqrt{n}$ números distintos.
	\end{prob}
	
	%\begin{prob}[MOP Practice Test 2007]
	%In a $n \times n$ array, each of the numbers in $\{1, 2, \dots , n\}$ appears exactly $n$ times. Show that there is a row or a column in the array with at least $\sqrt{n}$ distinct numbers.
	%\end{prob}

	\problem{math/usa/mo/1995/5}

	\begin{prob} %Victoria
		Considere um grafo com $n$ vértices que não possui ciclos de tamanho $4$. Mostre que o número de arestas é no máximo $\frac{n}{4}(1 + \sqrt{4n - 3})$.
	\end{prob}

	%\begin{prob} %Victoria
		%Consider an undirected graph with $n$ vertices that has no cycles of length $4$. Show that the number of edges is at most $\frac{n}{4}(1 + \sqrt{4n - 3})$.
	%\end{prob}

	\problem{math/imo/1989/3}
	
	\problem{math/treinamentoconesul/2020/L1/10}

	\problem{math/book/andrei_negut/problems_for_the_mathematical_olympiads/C1}

	\begin{prob}[Pequeno Teorema de Fermat]
		Sejam $a$ um inteiro e $p$ um primo, então $a^p \equiv a \pmod{p}$.
	\end{prob}

	\begin{prob}[Lema de Sperner] %Shine
		Dividimos um triângulo grande em triângulos menores de modo que qualquer dois dentre os triângulos menores ou não têm ponto em comum, ou têm vértice em comum, ou têm um lado (completo) em comum. Os vértices do triângulos são numerados: $1$, $2$, $3$. Os vértices dos triângulos menores também são numerados: $1$, $2$ ou $3$. A numeração é arbitrária, exceto que os vértices sobre os vértices do triângulo maior oposto ao vértice $i$ não podem receber o número $i$. Mostre que entre os triângulo menores existe um com os vértices $1$, $2$ e $3$.
	\end{prob}

	\section{Referências}

	\begin{itemize}
		\item Combinatória. Luciano Monteiro de Castro. Treinamento para IMO 2018. Eleva.
		\item Contagem Dupla. Carlos Shine. Curso de Combinatória -- Nível 3. POTI.
		\item Problemas em Contagem Dupla. Davi Lopes. Colóquio de Matemática 2018. Farias Brito.
		\item Contagem Dupla. Lucas Barros. Treinamento Cone Sul 2018.
		\item Double Couting. Victoria Krakovna. Canada IMO Summer Training 2010.
	\end{itemize}

\end{document}
