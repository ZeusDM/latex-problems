\documentclass[10pt,a4paper]{scrartcl}
\usepackage[utf8]{inputenc}
\usepackage[brazilian]{babel}
\usepackage{lmodern}
\usepackage{euler}
\usepackage[left=2.5cm, right=2.5cm, top=2.5cm, bottom=2.5cm]{geometry}

\usepackage[problem-list]{zeus}

\setlength{\columnsep}{4em}

\title{Tutoria para RMM}
\author{Guilherme Zeus Dantas e Moura}
\mail{\href{https://guilhermezeus.com}{\texttt{guilhermezeus.com}}}
\titlel{\includegraphics[width=3cm]{MM_logo_c}}
\titler{07 de Outubro de 2021}

\begin{document}	
	%\twocolumn[\zeustitle]
	\zeustitle

	%\sloppy

	\problem{math/rmm/2017/1}
%%%%\begin{sk}
%%%%	Seja \(\mathcal{S} = \{(m_1, m_2, \dots, m_{2k+1}) : k \geq 0, 0 \leq m_1 < m_2 < \cdots < m_{2k+1}\}\), i.e., \(S\) é o conjunto das sequências estritamente crescentes de inteiros não negativos com uma quantidade ímpar de termos.

%%%%	Considere a função \(f\colon S \to \mathbb{Z}_{>0}\) dada por \[
%%%%		f(m_1, \dots, m_{2k+1}) = \sum_{j=1}^{2k+1} (-1)^{j-1} 2^{m_j}\text{.}
%%%%	\]

%%%%	O item (a) está resolvido caso mostrarmos que \(f\) é uma bijeção. Vamos, portanto, mostrar que \(f\) é uma bijeção.

%%%%	Primeiro, \(f\) é de fato uma função bem definida, pois \[
%%%%		f(m_1, \dots, m_{2k+1}) = 2^{m_1} + \left(2^{m_3} - 2^{m_2}\right) + \cdots + \left(2^{m_{2k+1}} - 2^{m_{2k}}\right)\text{,}
%%%%	\] que é uma soma de uma quantidade positiva de inteiros positivos, portanto é um inteiro positivo. A equação acima também implica que \[
%%%%		f()
%%%%	\]

%%%%	Note também que \[
%%%%		f(m_1, \dots, m_{2k]
%%%%	\]

%%%%	Suponha, por indução, que para todo \(1 \leq n < N\), existe uma única sequnência em \(S\) cuja imagem é \(n\). Para \(N = 1\), temos \(1 = 2^0\).

%%%%	Suponha que \(N \geq 2\).
%%%%	Seja \(\ell\) menor inteiro positivo tal que \(N \leq 2^\ell\). Como \(N \geq 2\), \(\ell \geq 1\); e portanto, \(2^{\ell - 1} + 1 \leq N \leq 2^\ell\).
%%%%	Considere o número \(M = 2^\ell - N + 1\). Como \(1 \leq (2^\ell - N) + 1 = M = 2(2^{\ell - 1} + 1) - N - 1 \leq 2N - N - 1 < N\), pela hipótese de indução, temos que existe uma única sequência \((m_1, m_2, \dots, m_{2k+1})\) tal que  \[f(m_1, m_2, \dots, m_{2k+1}) = M = 2^\ell - N + 1\text{.}\]

%%%%	Se \(N\) é par, então \(M\) é ímpar, que implica que \(m_1 = 0\) e, portanto, \[
%%%%		N = f(m_2, m_3, \dots, m_{2k+1}, \ell)
%%%%	\]


%%%%\end{sk}
	\problem{math/rmm/2018/2}
	\problem{math/rmm/2018/3}
	\problem{math/rmm/2020/4}
	%\problem{math/rmm/2019/6}
	%\problem{math/rmm/2020/6}

\end{document}
