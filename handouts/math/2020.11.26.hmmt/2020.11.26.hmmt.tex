\documentclass[10pt,a4paper]{article}
\usepackage[utf8]{inputenc}
\usepackage[brazilian]{babel}
\usepackage{lmodern}
\usepackage[left=2.5cm, right=2.5cm, top=2.5cm, bottom=2.5cm]{geometry}

\usepackage[section, boxed, prob-boxed]{zeus}

\title{Problemas do HMMT November 2020}
\author{Guilherme Zeus Moura}
\mail{zeusdanmou@gmail.com}
\titlel{Turma Olímpica}
\titler{26 de novembro de 2020}

\begin{document}	
	%\twocolumn[\zeustitle]
	\zeustitle

	\section{Introdução: Valor Esperado}

	\begin{defn}[Valor Esperado]
		Dada uma variável aleatória $X$, o valor esperado de $X$, denotado por $\EE[X]$, é \[ \EE[X] = 
		\sum_{x} x \cdot \PP[X = x].\]	
	\end{defn}

	\begin{thm}[Linearidade da Esperança]
		Dadas variáveis aleatórias $X_1$ e $X_2$ (que podem ser dependentes!), \[\EE[X_1 + X_2] = \EE[X_1] + \EE[X_2].\]
	\end{thm}

	\begin{prob}
		No Pensi, existem $n$ alunos, e cada um dos alunos possui uma etiqueta com seu nome. Guilherme recolhe e embaralha as etiquetas, entregando aleatoriamente uma etiqueta para cada aluno. Seja $S$ o número de alunos que recebem a etiqueta com seu próprio nome. Prove que o valor esperado de $S$ é $1$.
	\end{prob}

	\begin{prob}
		Em um berçário, $2006$ bebês sentam em um círculo. De repente, cada bebê cutuca aleatóriamente o bebê imediatamente a sua direita ou o bebê imediatamente a sua esquerda. Qual é o valor esperado do número de bebês que não foram cutucados?
	\end{prob}

	\begin{prob}[NIMO 4.3]
One day, a bishop and a knight were on squares in the same row of an infinite chessboard, when a huge meteor storm occurred, placing a meteor in each square on the chessboard independently and randomly with probability $p$.
Neither the bishop nor the knight were hit, but their movement may have been obstructed by the meteors.
For what value of $p$ is the expected number of valid squares that the bishop can move to (in one move) equal to the expected number of squares that the knight can move to (in one move)?
	\end{prob}

	\begin{prob}[SJSU M179 Midterm]
		Prove that any subgraph of $K_{n,n}$ with at least $n^2 - n+1$ edges has a perfect matching.
	\end{prob}
	
	\newpage

	\section{Não-Combinatória}

	\problem{math/hmmt/nov/2020/guts/7}
	\problem{math/hmmt/nov/2020/guts/18}
	\problem{math/hmmt/nov/2020/guts/25}
	
	\section{Combinatória}
	\problem{math/hmmt/nov/2020/guts/9}
	\problem{math/hmmt/nov/2020/guts/12}
	\problem{math/hmmt/nov/2020/guts/13}
	\problem{math/hmmt/nov/2020/guts/23}
	\problem{math/hmmt/nov/2020/guts/28}
	
\end{document}
