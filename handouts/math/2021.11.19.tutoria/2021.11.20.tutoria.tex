\documentclass[10pt,a4paper]{scrartcl}
\usepackage[utf8]{inputenc}
\usepackage[brazilian]{babel}
\usepackage{lmodern}
\usepackage{euler}
\usepackage[left=2.5cm, right=2.5cm, top=2.5cm, bottom=2.5cm]{geometry}

\usepackage[hide, problem-list]{zeus}

\setlength{\columnsep}{4em}

\title{Banco de Problemas para a Tutoria}
\author{Guilherme Zeus Dantas e Moura}
\mail{\href{https://guilhermezeus.com}{\texttt{guilhermezeus.com}}}
\titlel{\includegraphics[width=3cm]{MM_logo_c}}
\titler{19 de Novembro de 2021}

\begin{document}	
	%\twocolumn[\zeustitle]
	\zeustitle

	%\sloppy

	\problem{math/apmo/2011/1}
	\begin{sol}
		Sem perda de generalidade, \(a \geq b \geq c\). Portanto, \[
			a^2 < a^2 + b + c \leq a^2 + a + a < (a+1)^2,
		\] logo \(a^2 + b + c\) não é quadrado.
	\end{sol}
	\problem{math/apmo/2011/2}
	\begin{sk}
		Divida nos possíveis casos para os feixes convexos; pentágono, quadrilátero, ou triângulo.
		Marque os ângulos minimais dos vértices do feixe convexo.
		Você marcará, respectivamente, \(15\), \(12\), ou \(9\) ângulos.
		A soma desses ângulos será, respectivamente, \(540^\circ\), \(360^\circ\), ou \(180^\circ\).
		Portanto, existirá um ângulo com medida menor ou igual que, respectivamente, \(36^\circ\), \(30^\circ\), ou \(20^\circ\).
		Portanto, o mínimo de tais ângulos é menor ou igual a \(36^\circ\).

		Tal mínimo é obtido no pentágono regular.
	\end{sk}
	\problem{math/apmo/2012/1}
	\begin{sol}
		Sejam \(a = [CDP]\), \(b=[AEP]\), \(c = [BFP]\). Teorema de Ceva implica que \(abc = 1\). Note que
		\begin{align*}
			\frac{1+a}{1} &= \frac{[BAP]}{[BDP]} \\
						  &= \frac{BC}{BD} \\
						  &= \frac{[ABC]}{[ABD]} \\
						  &= \frac{3 + a + b + c}{2 + c} \\
						  &= 1 + \frac{1 + a + c}{2 + c}.
		\end{align*}

		Logo, \(a + ca = 1 + b\). Somando ciclicamente, \[
			ab + bc + ca = 3,
		\]
		portanto, pela igualdade da MA-MG, temos que \(ab = bc = ca\), e portanto, \(a = b = c\), o que implica que \(P\) é o baricentro. Finalmente, a área de \(ABC\) é \(6\).
	\end{sol}
	\problem{math/apmo/2012/2}
	\begin{sk}
		Resolva para \(3 \times 3\). A resposta é \(5\).

		Para qualquer tabuleiro válido \(m \times n\), com \(n \geq 4\), dá pra criar um tabuleiro válido \(m \times n'\), com \(n' < n\) com a mesma soma do tabuleiro original.

		Isso prova que a resposta \(n \times n\) para \(n \geq 4\) é \(5\) também.
	\end{sk}
\end{document}
