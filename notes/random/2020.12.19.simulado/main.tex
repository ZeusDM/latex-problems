\documentclass[10pt, a4paper]{article}
\usepackage[utf8]{inputenc}
\usepackage[brazil]{babel}
%\usepackage{fullpage}

\usepackage[prob-boxed]{zeus}
\usetikzlibrary{patterns}

\usepackage{fullpage}
\usepackage{pgfplots}
\pgfplotsset{compat=1.15}
\usepackage{mathrsfs}
\usetikzlibrary{arrows}

\begin{document}

\begin{center}
	\textbf{Simulado OBM -- 2020.12.19}\\
\end{center}

\begin{prob}
	Qual é a maior quantidade de subconjuntos de $5$ elementos do conjunto $\{1, 2, \dots, 20\}$ que podemos escolher de modo que quaisquer dois compartilhem exatamente $1$ elemento.
\end{prob}

\begin{sk}
	Isso lembra planos projetivos finitos. Os pontos são os elementos de $\{1, 2, \dots, 20\}$ e as retas são os subconjuntos.
\end{sk}

\begin{sol}
	A resposta é $16$. Seja $S = \{1, 2, \dots, 20\}$.

	Para todo $x \in S$, $x$ pertence a, no máximo, $4$ conjuntos. \emph{Prova a cargo do leitor.}

	Usando contagem dupla em $(x \in S, C)$, com $C$ um dos conjuntos selecionados e $x \in C$, temos que \[\#(C) \cdot 5 \le 20 \cdot 4,\]
	isto é, $\#(C) \le 16$.

	Eis um exemplo com $16$ conjuntos:
	\begin{gather*}
		\{	1,  2,  3,  4,  17 \};
	\{	5,  6,  7,  8,  17 \};
	\{	9,  10, 11, 12, 17 \};
	\{	13, 14, 15, 16, 17 \};\\
	\{	1,  5,  9,  13, 18 \};
	\{	2,  6,  10, 14, 18 \};
	\{	3,  7,  11, 15, 18 \};
	\{	4,  8,  12, 16, 18 \};\\
	\{	1,  6,  11, 16, 19 \};
	\{	2,  5,  12, 15, 19 \};
	\{	3,  8,  9,  14, 19 \};
	\{	4,  7,  10, 13, 19 \};\\
	\{	1,  4,  12, 14, 20 \};
	\{	2,  8,  11, 13, 20 \};
	\{	3,  5,  10, 16, 20 \};
	\{	4,  6,  9,  15, 20 \}.
	\end{gather*}
\end{sol}

\begin{prob}
	Seja $ABC$ um triângulo e $\Gamma$ seu circuncírculo. Os pontos $D$ e $E$ estão no segmento $BC$ de modo que  $\angle BAD = \angle CAE$. O círculo  $\omega$ é tangente a $AD$ em $A$ e seu centro está em $\Gamma$. Seja $A'$ a reflexão de $A$ por $BC$ e sejam $L$ e $K$ as intersecções de $A'E$ com $\omega$. Prove que $BL$ e $CK$ ou $BK$ e $CL$ se intersectam em $\Gamma$.
\end{prob}

\begin{sk}
	Vamos reduzir o problema para mostrar que $ABC \sim ALK$.

	Então,  $A$ é o centro da homotetia que leva $BC$ em $LK$. Portanto, também é o centro da homotetia que leva $BL$ em $CK$. Logo, se $X = BL \cap CK$, $ABCX$ e $ALXK$ são cíclicos, que termina o problema. 
\end{sk}

\begin{prob}
	Seja $d(n)$ a quantidade de divisores positivos de $n$ e $\sigma(n)$ a soma dos divisores positivos de $n$. Determine todos os inteiros positivos para os quais \[d(n) \mid 2^{\sigma(n)} - 1.\]
\end{prob}

\begin{sk}
	Como $\sigma(n) \ge 1$, $2^{\sigma(n)} - 1$ é ímpar. Logo,  $d(n)$ deve ser ímpar, ou seja, $n$ é um quadrado perfeito.

	Seja $n = m^2$
\end{sk}

\end{document}
