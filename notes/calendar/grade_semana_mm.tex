%%%%%%%%%%%%%%%%%%%%%%%%%%%%%%%%%%%%%%%%%
% Weekly Timetable Calendar
% LaTeX Template
% Version 1.2 (19/9/18)
%
% This template has been downloaded from:
% http://www.LaTeXTemplates.com
%
% Original author:
% Evan Sultanik with modifications by 
% Vel (vel@LaTeXTemplates.com)
%
% License:
% CC BY\\ Nível Z-NC-SA 3.0 (http://creativecommons.org/licenses/by-nc-sa/3.0/)
%
% Important note:
% This template requires the calendar.sty file to be in the same directory as the
% .tex file. The calendar.sty file provides the necessary structure to create the
% calendar.
%
%%%%%%%%%%%%%%%%%%%%%%%%%%%%%%%%%%%%%%%%%

%----------------------------------------------------------------------------------------
%	PACKAGES AND OTHER DOCUMENT CONFIGURATIONS
%----------------------------------------------------------------------------------------

\documentclass[11pt]{article} % Can also use 11pt for a larger overall font size

\usepackage{calendar2} % Use the calendar.sty style

\usepackage[landscape, a4paper, margin=1cm]{geometry} % Page dimensions and margins
\usepackage[brazilian]{babel}
\usepackage{xcolor}


\usepackage{transparent}
%\usepackage[printwatermark]{xwatermark}
%\newwatermark[firstpage, angle=0,scale=1,xpos=0,ypos=0]{{\transparent{0.15}\includegraphics[width = 13cm]{mm}}}

%\usepackage{palatino} % Use the Palatino font
\usepackage{helvet}
\renewcommand{\familydefault}{\sfdefault}
%\usepackage[default, angular]{comicneue}
%\usepackage[T1]{fontenc}

\begin{document}

\pagestyle{empty} % Disable default headers and footers

\setlength{\parindent}{0pt} % Stop paragraph indentation

\StartingDayNumber=1 % Calendar starting day, default of 1 means Sunday, 2 for Monday, etc


%----------------------------------------------------------------------------------------
%	TITLE SECTION
%----------------------------------------------------------------------------------------

\hspace{0pt}\vfill

%\begin{minipage}
%\end{minipage}
%\begin{minipage}{.7\textwidth}
\begin{center}
	\textsc{\LARGE Semana Matematicamente Olímpica}\\ % Title text
	\textsc{\large 24--29 de Janeiro de 2021}\\ % Subtitle text
	\textsc{Grade Provisória --- Atualizado em: \today} % Subtitle text
\end{center}
%\end{minipage}

%----------------------------------------------------------------------------------------

\begin{calendar}{\textwidth} % Calendar to be the entire width of the page
	\setcounter{calendardate}{24}

%----------------------------------------------------------------------------------------
%	FIRST DAY
%----------------------------------------------------------------------------------------

\day{}{
	\textbf{20:00--22:00} \daysep \textsl{Palestra ``Você quer ser ouro na IMO?''}\\ Prof. Luciano
}

%----------------------------------------------------------------------------------------
%	SECOND DAY
%----------------------------------------------------------------------------------------

\day{}{
	\textbf{08:00--10:00} \daysep \textsl{Círculo dos Nove Pontos}\\ Cícero Thiago\\ Nível 2 \timesep
	\textbf{10:00--12:00} \daysep \textsl{Indução sem Fronteiras} \\ Rafael Filipe dos Santos\\ Nível 3 \timesep
	\textbf{14:00--16:00} \daysep \textsl{Dominós e Tabuleiros} \\ Emiliano Augusto Chagas\\ Níveis P e 1 \timesep
	\textbf{16:00--18:00} \daysep \textsl{Miscelânea de Álgebra}\\Guilherme Zeus\\ Nível 3
} 

%----------------------------------------------------------------------------------------
%	THIRD DAY
%----------------------------------------------------------------------------------------

\day{}{ % Tuesday
	\textbf{08:00--10:00} \daysep \textsl{Princípio da Casa dos Pombos} \\ Samuel Barbosa Feitosa\\ Nível 2\timesep
	\textbf{10:00--12:00} \daysep \textsl{Inversão} \\ Régis Prado \\ Níveis 2 e 3 \timesep
	\textbf{13:00--15:00} \daysep \textsl{Problemas do Canguru} \\ Luiz Felipe Lins \\ Níveis P e 1 \timesep
	\textbf{16:00--18:00} \daysep \textsl{Sequências e Padrões} \\ Marcio Watanabe \\ Nível 1
} 

%----------------------------------------------------------------------------------------
%	FOURTH DAY
%----------------------------------------------------------------------------------------

\day{}{ % Wednesday
	\textbf{08:00--10:00} \daysep \textsl{Jogos: usando a matemática para vencer} \\ Marcelo Xavier \\ Níveis 1 e 2 \timesep
	\textbf{10:00--12:00} \daysep \textsl{Você resolve!} \\ Prof. Luciano \\ Todos os níveis \timesep
	\textbf{14:00--16:00} \daysep \textsl{Transformações Geométricas} \\ Davi Lopes \\ Níveis 2 e 3 \timesep
	\textbf{16:00--18:00} \daysep \textsl{Polinômios e Raízes da Unidade} \\ Ana Paula Chaves\\ Nível 3
} 

%----------------------------------------------------------------------------------------
%	FIFTH DAY
%----------------------------------------------------------------------------------------

\day{}{ % Thursday
	\textbf{08:00--10:00} \daysep \textsl{Quem consegue resolver um problema tão difícil como este?} \\ Prof. Luciano \\ Nível 3 \timesep
	\textbf{10:00--12:00} \daysep \textsl{A Matemática por trás do seu dinheiro} \\ Pablo Ganassim\\ Todos os níveis \timesep
	\textbf{14:00--16:00} \daysep \textsl{Combinatória} \\ Diego Eloi \\ Nível P \timesep
	\textbf{16:00--18:00} \daysep \textsl{Descobrindo Critérios de Divisibilidade} \\ Kellem Corrêa Santos\\ Níveis P e 1
} 

%----------------------------------------------------------------------------------------
%	SIXTH DAY
%----------------------------------------------------------------------------------------

\day{}{ % Friday
	\textbf{08:00--10:00} \daysep \textsl{Exposição a Martingales} \\ Guilherme Zeus \\ Níveis 2 e 3 \timesep
	\textbf{10:00--12:00} \daysep \textsl{Método Probabilístico} \\ Matheus Secco \\ Nível 3 \timesep
	\textbf{14:00--16:00} \daysep \textsl{Apresentando Pontos Notáveis do Triângulo}\\ Karol Borges \\ Nível 1  \timesep
	\textbf{16:00--18:00} \daysep \textsl{Combinatória} \\ Gabriel Oliveira \\ Nível 2 \timesep
	\textbf{19:00--20:15} \daysep \textsl{Bate-papo de Encerramento} \timesep
	\textbf{20:15--21:00} \daysep \textsl{Círculo dos Nove Pontos (Cont.)}\\ Cícero Thiago\\ Nível 2 \timesep
} 

%----------------------------------------------------------------------------------------
 
\finishCalendar
\end{calendar}

\hspace{0pt}\vfill

\end{document}
