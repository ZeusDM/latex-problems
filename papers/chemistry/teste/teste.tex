\documentclass[10pt,a4paper]{article}
\usepackage[utf8]{inputenc}
\usepackage[brazil]{babel}

\PassOptionsToPackage{inline}{enumitem}

\usepackage{../../../commands/problems}
\renewcommand{\mypath}{../../../}
\usepackage{../../../commands/BraunChem}

\usepackage{lmodern}
\usepackage[left=1.5cm, right=1.5cm, top=2.5cm, bottom=1.5cm]{geometry}
\usepackage{siunitx}
\usepackage{fancyhdr}
\usepackage{icomma}
\usepackage{calc}

\begin{document}

\begin{prob}
	O gás \ch{SO2} é formado na queima de combustíveis fósseis.
Sua liberação na atmosfera é um grave problema ambiental, pois através de uma série de reações ele irá se transformar em \ch{H2SO4{(aq)}}, um ácido muito corrosivo, no fenômeno conhecido como chuva ácida.
A sua formação pode ser simplificadamente representada por:

\begin{Scheme}
	\ch{ S{(s)} + O2{(g)} -> SO2{(g)} }
\end{Scheme}

Quantas toneladas de dióxido de enxofre serão formadas caso ocorra a queima
de uma tonelada de enxofre? 

\begin{enumerate}[label = (\scalealph{\alph*})]
	\item 1 tonelada		
	\item 2 toneladas
	\item 3 toneladas	
	\item 4 toneladas		
	\item 5 toneladas
\end{enumerate}

Dados: \ch{S} = \SI{32}{\gram\per\mol} e \ch{O} = \SI{16}{\gram\per\mol}.
\end{prob}

\begin{center}
	\chemfig{=[:270]-[:330]-[:270]-[:210](-[:270](-[:330])(<[:105.1,1.951]\ch{O}>[:65.5,1.227])-[:210])-[:150]-[:90](-[:30])}
\end{center}

\end{document}
