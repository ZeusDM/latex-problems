\documentclass[10pt,a4paper]{article}
\usepackage[utf8]{inputenc}
\usepackage[brazil]{babel}
\usepackage{amsmath}
\usepackage{amsfonts}
\usepackage{amssymb}
\usepackage{graphicx}
\usepackage{enumitem}
\usepackage{lmodern}
\usepackage[left=3cm, right=3cm, top=3cm, bottom=3cm]{geometry}

\usepackage{../../commands/problems}
\renewcommand{\mypath}{../../}

\title{OMERJ 2012, Questão 5}
\date{\today}
\author{Guilherme Zeus Moura}

\begin{document}
	\maketitle

	\section{Enunciado}

	Ache todas as funções $f: \RR \to \RR$ tais que
	$$f(xy - f(x)) = xf(y)$$
	para todos $x$ e $y$ reais.

	\section{Solução}

	Vamos chamar $f(xy - f(x)) = xf(y)$ de $P(x, y)$.

	\begin{enumerate}
		\item[$P(0, y)$:] $f(-f(0)) = 0$
		\item[$P(x,0)$:] $f(-f(x)) = x f(0)$
		\item[$P(-f(0), 1)$:] $f(-f(0)-f(-f(0))) = -f(0)f(1) \implies 0 = f(0)f(1)$\
			$$f(0) = 0 \text{ ou } f(1) = 0$$ 
	\end{enumerate}	

	Sabemos que $f(x) \equiv 0$ é solução. 

	Portanto, vamos supor que $f(k) \neq 0$.

	 \begin{enumerate}
		 \item[$P(x, k)$] $f(xk - f(x)) = xf(k)$
	\end{enumerate}

	Como $xf(k)$ percorre todos os reais, $f$ é sobrejetora.

	Vamos dividir em dois casos:

	\begin{enumerate}[label = (Caso (\alph*))]
		\item Se $f(0) = 0$:
			\begin{enumerate}
				\item[$P(x,0):$] $f(-f(x)) = 0$
			\end{enumerate}

			Mas, como $f$ é sobrejetora, $-f(x)$ cobre todos os reais; então $f(x) \equiv 0$. Absurdo.
		
		\item Se $f(1) = 0$ e $f(0) \neq 0$:

			Vamos mostrar que $f$ é injetora fora de zero:

			Se $f(x) = f(y) = c \neq = 0$:
	
			\begin{enumerate}
				\item[$P(x,y)$:] $f(xy - c) = xc$ 
				\item[$P(y,x)$:] $f(xy - c) = yc$
			\end{enumerate}

			Que implicam que $x = y$. Logo, $f$ é injetora fora de zero.
			
			\begin{enumerate}
				\item[$P(-f(x),y)$:] $f(-f(x)y-f(-f(x))) = -f(x)f(y)$
				\item[$P(-f(y),x)$:] $f(-f(y)x-f(-f(y))) = -f(x)f(y)$
			\end{enumerate}

			$$ f(-f(x)y - xf(0)) = f(-f(y)x - yf(0)) $$
			
			Se $f(x) \neq 0$ e $f(y) \neq 0$, podemos igualar as entradas das funções:

			$$ -f(x)y - xf(0) = -f(y)x $$

	\end{enumerate}

	
\end{document}
