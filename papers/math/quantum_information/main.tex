\documentclass[10pt,a4paper]{article}
\usepackage[utf8]{inputenc}
\usepackage[english]{babel}
\usepackage{amsmath}
\usepackage{amsfonts}
\usepackage{amssymb}
\usepackage{graphicx}
\usepackage{enumitem}
\usepackage{lmodern}
\usepackage{fullpage}

\usepackage{zeusall}

\title{Quantum Information}
\author{MIT Lecture}
\mail{}
\titler{}
\titlel{}

\newcommand{\FF}{\mathbb{F}}
\newcommand{\ket}[1]{|#1\rangle}

\begin{document}
	\zeustitle

	\setcounter{section}{-1}
	\section{Outline}
	\begin{enumerate}
		\item Classical game
		\item Simple quantum system
		\item Beating the game
		\item Quantum computing today\footnote{If the time permits}
	\end{enumerate}

	\section{Classical game}
	\subsection{The game}

	Alice and Bob receive, respectly, $a$ and $b$ randomly chosen in $\FF_2$ (Alice does not know $b$ and Bob does not know $a$) and they choose, respectly, $p$ and $q$ (without talking to each other).

	Their goal is to make $ p + q = a \cdot b $ true.
	
	\subsection{Solution}

	A strategy is for both Alice and Bob always pick $0$. The probability of them winning is $\frac{3}{4}$.

	No determinant strategy was a probability higher than $\frac{3}{4}$.

	\section{Simple quantum system}

	\begin{defn}[Qubit]
		2 \emph{linearly indemendent} states.
		\[\ket{0} = (1, 0)\]
		\[\ket{1} = (0, 1)\]
	\end{defn}

\end{document}
