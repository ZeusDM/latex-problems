\documentclass[11pt]{article}
\usepackage{amsfonts} 
\usepackage{amsmath}

\topmargin -1in
\textheight 9.5in
\oddsidemargin -.25in
\evensidemargin -.25in
\textwidth 7in

\begin{document}

% ========== Edit your name here
\author{Mark Helman}
\title{Generating Functions and the Residue Theorem \\Treinamento IMO}
\date{19 de agosto de 2019}
\maketitle

\medskip

% ========== Begin answering questions here
\begin{enumerate}

\item
(HMMT 2007) Let $S$ denote the set of all triples $(i, j, k)$ of positive integers where $i + j + k = 17$. Compute
\[\sum_{(i,j,k)\in S} ijk\]

\item
Is it possible to partition the set of positive integers into a finite number of (more than one) arithmetic progressions with distinct ratio?

\item
(IMO Shortlist 1998) Let $a_{0},a_{1},a_{2},\ldots $ be an increasing sequence of nonnegative integers such that every nonnegative integer can be expressed uniquely in the form $a_{i}+2a_{j}+4a_{k}$, where $i,j$ and $k$ are not necessarily distinct. Determine $a_{1998}$.
\item
For $n \geq 0$, compute
\[\sum_{k\geq 0} \binom{2k}{k} \binom{n}{k} \bigg ( -\frac{1}{4} \bigg )^k\]

\item
(Putnam 2003) For a set $S$ of nonnegative integers, let $r_S(n)$ denote the number of ordered pairs $(s_1, s_2)$ such that $s_1 \in S$, $s_2 \in S$, $s_1 \neq s_2$, and $s_1 + s_2 = n$. Is it possible to partition the nonnegative integers into two sets $A$ and $B$ in such a way that $r_A(n) = r_B(n)$ for all $n$?

\item
(IMO 1995) Let $p$ be an odd prime number. How many $p$-element subsets $A$ of $\{1, 2,\ldots, 2p\}$
are there such that the sums of its elements are divisible by $p$?

\item
(PUMaC 2015). Let $p$ be an odd prime. Prove that $p^2|2^p-2$ if and only if
\[\frac{1}{1\cdot2}+\frac{1}{3\cdot4}+\cdots+\frac{1}{(p-2)(p-1)}\equiv0\pmod{p}.\]

\item
Let $b(n)$ be the $n^{th}$ Bell number, i.e., $b(n)$ is the number of ways of partitioning the set $\{1,2,\ldots,n\}$ into disjoint subsets. Find its exponential generating function and use it to find a recurrence formula for the Bell numbers.
\item
 (IMO Shortlist 2014 N6) Let $a_1 < a_2 <  \cdots <a_n$ be pairwise coprime positive integers with $a_1$ being prime and $a_1 \ge n + 2$. On the segment $I = [0, a_1 a_2  \cdots a_n ]$ of the real line, mark all integers that are divisible by at least one of the numbers $a_1 ,   \ldots , a_n$ . These points split $I$ into a number of smaller segments. Prove that the sum of the squares of the lengths of these segments is divisible by $a_1$.

\item
(Putnam 2018) Let $S$ be the set of sequences of length 2018 whose terms are in the set $\{1, 2, 3, 4, 5, 6, 10\}$ and sum to 3860. Prove that the cardinality of $S$ is at most
\[2^{3860} \cdot \left(\frac{2018}{2048}\right)^{2018}.\]
% ========== Continue adding items as needed

\end{enumerate}

\end{document}
