\documentclass[10pt,a4paper]{article}
\usepackage[utf8]{inputenc}
\usepackage[brazilian]{babel}
\usepackage{lmodern}
\usepackage[left=2.5cm, right=2.5cm, top=2.5cm, bottom=2.5cm]{geometry}

\usepackage{../../../commands/problems}
\renewcommand{\mypath}{../../../}

\title{Combinatória Simples}
\author{Guilherme Zeus Moura}
\mail{zeusdanmou@gmail.com}
\titlel{Turma Olímpica}
\titler{\today}

\begin{document}	
	\zeustitle
	\section{Ideias úteis}
	\begin{enumerate}[label = (\alph*)]
		\item \emph{Casos pequenos.} Mesmo que sejam trabalhosos.
		\item \emph{Princípio Extremal:} Analisar propriedades extremais. Usualmente olhamos para o objeto ou exemplo minimal ou maximal, em algum sentido.
	\end{enumerate}
	\section{Problemas}
	\prob{
		Pinte todas as arestas de um grafo com $6$ vértices de preto ou branco. Mostre que existem pelo menos dois triângulos monocromáticos.
	}
	\problem{math/mit/putnam/hidden_independence_and_uniformity/1}
	\problem{math/mit/putnam/hidden_independence_and_uniformity/2}
	\problem{math/mit/putnam/hidden_independence_and_uniformity/3}
	\problem{math/mit/putnam/hidden_independence_and_uniformity/5}
	\problem{math/conesul/2001/4}
	\problem{math/ibero/1997/6}
	\problem{math/lusophon/2018/6}
	\problem{math/mit/putnam/hidden_independence_and_uniformity/4}
	\problem{math/mit/putnam/hidden_independence_and_uniformity/6}
	\problem{math/mit/putnam/hidden_independence_and_uniformity/7}
	\problem{math/mit/putnam/hidden_independence_and_uniformity/8}
	\problem{math/mit/putnam/hidden_independence_and_uniformity/9}
\end{document}
