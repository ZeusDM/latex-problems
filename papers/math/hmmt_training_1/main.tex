\documentclass[10pt,a4paper]{article}
\usepackage[utf8]{inputenc}
\usepackage[english]{babel}
\usepackage{amsmath}
\usepackage{amsfonts}
\usepackage{amssymb}
\usepackage{graphicx}
\usepackage{enumitem}
\usepackage{lmodern}
\usepackage{fullpage}
\usepackage{titlesec}

\usepackage[classic, mm]{zeus}

\titleformat{\section}[hang]{\vspace{-0.5em}}{}{1em}{\large\bfseries\sffamily}[]

\title{Treinamento para Provas de Velocidade em Equipe, \#1 \\ \normalfont ``Mini-Guts''}
\author{}
\nomail
\titler{8 de janeiro de 2020}
\titlel{Matematicamente Internacionais}

\newcommand{\round}[1]{

	\noindent\dotfill\section{#1}
}
\begin{document}
	\zeustitle

	\vspace{1em}\noindent\hspace{1em}{\large \sffamily \bfseries Instruções:}

	\begin{itemize}
		\item Tamanho esperado da equipe: 8 pessoas.
		\item Tempo disponível: 65 minutos.
		\item São 6 rodadas, com pontuações de, respectivamente, 5, 7, 10, 12, 15 e 20 pontos por problema.
	\end{itemize}

	\round{Round 1 (5 points each)}	
	\problem*{math/pumac/2018/live/1.1}
	\problem*{math/pumac/2018/live/1.2}
	\problem*{math/pumac/2018/live/1.3}
	\round{Round 2 (7 points each)}	
	\problem*{math/pumac/2018/live/2.1}
	\problem*{math/pumac/2018/live/2.2}
	\problem*{math/pumac/2018/live/2.3}
	\round{Round 3 (10 points each)}	
	\problem*{math/pumac/2018/live/4.1}
	\problem*{math/pumac/2018/live/4.2}
	\problem*{math/pumac/2018/live/4.3}
	\newpage
	\section{Round 4 (12 points each)}
	\problem*{math/pumac/2018/live/5.1}
	\problem*{math/pumac/2018/live/5.2}
	\problem*{math/pumac/2018/live/5.3}
	\round{Round 5 (15 points each)}	
	\problem*{math/pumac/2018/live/7.1}
	\problem*{math/pumac/2018/live/7.2}
	\problem*{math/pumac/2018/live/7.3}
	\round{Round 6 (20 points each)}	
	\problem*{math/pumac/2018/live/8.1}
	\problem*{math/pumac/2018/live/8.2}
	\problem*{math/pumac/2018/live/8.3}
\end{document}
