\documentclass[10pt, a4paper]{article}
\usepackage[utf8]{inputenc}
\usepackage[brazilian]{babel}
\usepackage{lmodern}
\usepackage[left=2cm, right=2cm, top=2cm, bottom=2.5cm]{geometry}
\usepackage{indentfirst}
\usepackage[inline]{enumitem}
\usepackage[dvipsnames]{xcolor}

\usepackage[%mm,
			%problem-list
			]{zeus}

\title{Solução de Rússia 2018, Problema 11.8}
\author{Guilherme Zeus Moura}
\mail{zeusdanmou@gmail.com}
\titlel{Matematicamente Internacionais}
\titler{Treinamento para a IMO 2020}

\newcommand{\X}[1]{X_{#1}}
\newcommand{\Y}[1]{Y_{#1}}
\newcommand{\Z}[1]{Z_{#1}}

%\newcommand{\X}[1]{\color{Red}X_#1\color{black}}
%\newcommand{\Y}[1]{\color{blue}Y_#1\color{black}}
%\newcommand{\Z}[1]{\color{Green}Z_#1\color{black}}

%\renewcommand{\playerA}[1]{\textcolor{Red}{#1}}
%\renewcommand{\playerB}[1]{\textcolor{blue}{#1}}

\begin{document}	
	\zeustitle

	\problem{math/russia/2018/11.8}

	\begin{sol}
		Vamos pintar o tabuleiro como o de xadrez.
		Para não confundir com as cores vermelho e azul (relativas aos saltadores), cada casa do tabuleiro terá uma \emph{pintura} branca ou preta. Cada vez que um saltador salta, ele muda de pintura. Como os saltadores começam em cores distintas, vale o seguinte lema.

		\begin{lem}
			Os saltadores estão em pinturas distintas se, e somente se, o próximo a jogar é \playerA{Roger}.
		\end{lem}

		Sejam $\X{0}$ e $\Y{0}$ as casas iniciais dos saltadores vermelho e azul, respectivamente. Generalizando, seja $\X{n}$ ($\Y{n}$) a casa do saltador vermelho (azul) imediatamente após a $n$-ésima jogada de \playerA{Roger} (\playerB{Bazil}).

		\begin{lem} \label{lem:seq}
			Existe uma série de movimentos que começa em $\Y{0}$ e termina em $\X{1}$, que não passa pela memsa casa duas vezes e \emph{com certa liberdade}.
		\end{lem}
		\begin{dem}
			A demonstração ficará para mais tarde, devido a incerteza da liberdade.
		\end{dem}

		Sejam $\Z{0} = \Y{0}, \Z{1}, \Z{2}, \dots, \Z{t} = \X{1}$ as casas visitadas por uma das série de movimentos garantida pelo \cref{lem:seq}. Pela pintura, $t$ é par. 

		A primeira jogada de \playerB{Bazil} será $\Y{1} = \Z{1}$.

		Seja $k$ o menor número tal que $\X{k}$ não é $\X{0}$ ou $\X{1}$. (Dizemos que $k = \infty$, se não existir nenhum número com tal propriedade.)

		Já sabemos se $k = 2$ ou se $k > 2$.

		\begin{lem}
			Se $3 \le k \le t$, então \playerB{Bazil} ganha.
		\end{lem}

		\begin{dem}
			Sabemos que $\X{2} = \X{0}, \X{3} = \X{1}, \dots, \X{k-1} = \X{k-1 \pmod{2}}$ e $X_k \not\in \{X_0, X_1\}$.
		\end{dem}
	\end{sol}
\end{document}
