\documentclass[10pt, a4paper]{article}
\usepackage[utf8]{inputenc}
\usepackage[brazilian]{babel}
\usepackage{lmodern}
\usepackage[left=2cm, right=2cm, top=2cm, bottom=2.5cm]{geometry}
\usepackage{indentfirst}
\usepackage[inline]{enumitem}

\usepackage{pgf,tikz,pgfplots}
\pgfplotsset{compat=1.15}
\usepackage{mathrsfs}
\usetikzlibrary{arrows}
%\pagestyle{empty}

%\PassOptionsToPackage{section}{zeus-amsthm}

\usepackage[section]{zeus}
\usepackage{cite}
\usepackage[h]{esvect}

\title{Problemas usando Transformações}
\author{Guilherme Zeus Moura}
\mail{zeusdanmou@gmail.com}
\titlel{Turma Olímpica}
\titler{{\footnotesize v. 1} -- 26 de Agosto de 2020}

\begin{document}	
	\zeustitle
	
	\nocite{Transformations-VKrakovna}
	\nocite{GeometricTransformations-AMustata}
	\nocite{Homotetia-Regis}
	\nocite{Homotetia-DLopes}
	\nocite{EGMO-EChan}

	\begin{center}
		\begin{minipage}{12cm}
		\slshape Sugestões e correções de erros (mesmo que mínimos) são muito bem vindas. Envie-as para \href{mailto:zeusdanmou+tex@gmail.com}{\texttt{zeusdanmou+tex@gmail.com}}.
		\end{minipage}
	\end{center}

	\vspace{.3cm}

	\begin{center}
		\begin{minipage}{12cm}	
			\tableofcontents
		\end{minipage}
	\end{center}

	\vspace{.5cm}

	\section{Reflexões}

	\begin{prob}
		Seja $ABC$ um triângulo com ortocentro $H$ e circumcentro $O$. Seja $M$ o ponto médio de $BC$. Prove que $\vv{AH} = 2\vv{OM}$.
	\end{prob}

	\begin{prob}[Reta de Euler]
		Seja $ABC$ um triângulo com ortocentro $H$, circumcentro $O$ e baricentro $G$. Prove que $H$, $O$, $G$ são colineares.
	\end{prob}

	\begin{prob}
		Seja $ABC$ um triângulo com ortocentro $H$ e circumcírculo $\Omega$. Seja $P$ um ponto em $\Omega$. Sejam $P_A$, $P_B$, $P_C$ as reflexões de $P$ pelas retas $BC$, $CA$ e $AB$, respectivamente. Prove que $P_A$, $P_B$, $P_C$, $H$ são colineares.
	\end{prob}

	\problem{math/egmo/2012/7}

	\problem{math/imo/2011/6}

	\problem{math/balkan/2018/1}

	\section{Homotetias}
	
	\begin{prob}[Círculo de Nove Pontos]
		Seja $ABC$ um triângulo. Prove que o ponto médio dos lados, o pé das alturas e os pontos médios dos segmentos que ligam os vértices do triângulo com o ortocentro $H$ estão todos sobre um mesmo círculo. Qual é o raio deste círculo?
	\end{prob}

	\begin{prob}
		Seja $ABC$ um triângulo com incírculo $\omega$. Seja $D$ a intersecção de $\omega$ com $BC$. Seja $T$ o ponto diametralmente oposto a $D$ em $\omega$. Seja $X$ a intersecção de $AT$ com $BC$.
		
		Prove que $BD = CX$.
	\end{prob}

	\begin{prob}[Lema da Estrela da Morte]
		Sejam $\Gamma_1$ e $\Gamma_2$ circunferências tangentes, com $\Gamma_1$ no interior de $\Gamma_2$. Sejam $T$ o ponto de tangência e $AB$ uma corda de $\Gamma_2$ que tangencia $\Gamma_1$ em $U$.
		
		Prove que $TU$ bissecta o arco $AB$.
	\end{prob}

	\problem{math/imo/1978/4}

	\problem{math/imo/1981/5}

	\problem{math/imo/1982/2}

	\problem{math/brazil/mo/2012/2}
	
	\problem{math/japan/2007/3}

	\problem{math/brazil/mo/2014/6}
	
	\problem{math/brazil/mo/2017/5}

	\problem{math/imo/1983/2}

	\problem{math/imo/1992/4}

	\problem{math/imo/1999/5}

	\problem{math/imo/2008/6}

	\section{Rotações}
	
	\begin{prob}[Ponto de Fermat]\ 
		\begin{enumerate}[label = (\alph*), before = \leavevmode, after = \vspace{-\baselineskip}]
			\item Seja $ABC$ um triângulo com ângulos menores ou iguais a $120^\circ$. Construa o triângulo equilátero $BCD$, com $D$ e $A$ em semiplanos distintos em relação a reta $BC$. Construa os triângulos equiláteros $CAE$ e $ABF$ de forma análoga. Prove que $AD$, $BE$, $CF$ possuem tamanhos iguais e são concorrentes.
			
				Vamos chamar essa intersecção de \emph{ponto de Fermat}.
			\item Seja $ABC$ um triângulo com ângulos menores ou iguais a $120^\circ$. Seja $P$ um ponto no interior de $ABC$. Prove que $AP + BP + CP$ é mínimo quando $P$ é o ponto de Fermat.
		\end{enumerate}%
	\end{prob}%

	\begin{prob}
		Dado um paralelogramo $ABCD$, construa externamente quadrados em seus quatro lados. Prove que os centros desses quadrados foram um quadrado.
		%Given a parallelogram $ABCD$, construct squares externally on its four sides. Prove that the centres of these squares form a square.
	\end{prob}

	\begin{prob}[Mathematical Olympiad Challenges]
		Sejam $ABC$ e $BCD$ triângulos equiláteros, com $A$ e $D$ distintos. Uma reta passa por $D$, intersectando o prolongamendo de $AC$ em $M$ e $AB$ em $N$. Seja $P$ a intersecção de $CN$ e $BM$. Prove que $\angle BPC = 60^\circ$.
		%Let $ABC$ and $BCD$ be equilateral triangles. $A$ and $D$ are distinct. A line passes through $D$, intersecting $AC$ extended at $M$ and $AB$ at $N$. Let $CN$ intersect $BM$ at $P$. Prove that $\angle BPC = 60^\circ$.
	\end{prob}

	\problem{math/rmmsl/2019/original_P4}

	\section{Roto-homotetias}

	\begin{prob}
		Sejam $AB$ e $CD$ segmentos, e seja $X$ a intersecção entre $AC$ e $BD$. Se os circuncírculos de $ABX$ e $CDX$ se intersectam em $O$, então $O$ é o centro da única roto-homotetia que leva $AB$ em $CD$.
	\end{prob}

	\problem{math/imo/2016/1}
	
	\problem{math/imo/2010/2}

	\problem{math/egmo/2017/6}	

	\bibliographystyle{plain}
	\bibliography{mybib}{}

\end{document}
