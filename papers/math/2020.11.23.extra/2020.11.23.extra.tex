\documentclass[10pt,a4paper]{article}
\usepackage[utf8]{inputenc}
\usepackage[brazilian]{babel}
\usepackage{lmodern}
\usepackage[left=2.5cm, right=2.5cm, top=2.5cm, bottom=2.5cm]{geometry}

\usepackage[boxed, prob-boxed]{zeus}

\title{Aula Extra}
\author{Guilherme Zeus Moura}
\mail{zeusdanmou@gmail.com}
\titlel{Turma Olímpica}
\titler{23 de novembro de 2020}

\begin{document}	
	\zeustitle

	\problem{math/brazil/mo/2016/2}
	\ans{$2(3 \cdot 4 \cdot 7)^2 + 1$.}
	\problem{math/brazil/mo/2016/5}
	\begin{sk}~
		\begin{enumerate}[label = (\alph*)]
			\item A raiz $r$ negativa de $P(x) - 2016x$ funciona. 
			\item Seja $f(x) = \frac{P(x)}{2016} = \frac{4x^2 + 12x - 3015}{2016}.$

				Use o fato de que $(-3/2, -3/2)$ é o mínimo de $f$ e tambem é raíz $f(x) - x$ para mostrar que para qualquer $x \in (-3/2, 1005/2)$, \[\lim_{n\to\infty} f^n(x) = -3/2.\]
		\end{enumerate}
	\end{sk}
	\problem{math/brazil/mo/2014/5}
	\begin{sk}
		A soma dos elementos $\bmod{\ 3}$ é constante.

		Vamos provar por indução em $m \cdot n$ que, se $3 \mid mn$, então a condição necessária e suficiente é que a soma dos números iniciais é múltipla de $3$ e, se $3 \nmid mn$, então sempre funciona.

		Bases:  $2 \times 2$, $2 \times 4$ e $4 \times 4$.

		Se $3 \nmid mn$, escreva $2m = 6k + k_0$ e $2n = 6l + l_0$, com $k_0, l_0 \in \{2, 4\}$. Divida em $4$ (ou possivelmente $2$) tabuleiros menores de tamanhos $6k \times 6l$, $6k \times l_0$,  $k_0 \times 6l$ e $k_0 \times l_0$ e resolva usando a hipótese. 

		Se $3 \mid mn$, sem perda de generalidade, $3 \mid m$. Escreva $2m = 6k$ e divida em $2$ tabuleiros $4k \times 2n$ e $2k \times 2n$ e resolva usando a hipótese.
	\end{sk}
	\problem{math/brazil/mo/2013/5}
	\begin{sk}
		Prove por indução que $p(k) \ge k+1$.

		O passo indutivo se resume a $p(k-1) = p(k) = 1$. Nesse caso, construiremos um grafo ordenado em que $V = \{\text{sequências distintas de tamanho $k-1$ em $x$}\}$ e $s \to s'$ se, e somente se, $s = a_{j+1}\dots a_{j+k-1}$ e $s' = a_{j+2} \dots a_{j+k}$, para algum $j$.

		Dá pra provar que $\mathrm{outdeg}(v) = 1$, para todo $v \in V$ (usando $p(k-1) = p(k) = k$) e depois mostrar que isso implica que $x$ é, na verdade, um racional (a sequência dos $a$'s é periódica).	
	\end{sk}

	\problem{math/imosl/2010/A4}

	\begin{sk}
		Vamos provar por indução forte em $k$ que, para todo $n \le 4k$, \[S_n = x_1 + x_2 + \cdots + x_n \ge 0.\]
	\end{sk}

	\begin{prob}[HMMO November 2020, Team, 9]
		Alice and Bob take turns removing balls from a bag contaiting $10$ black balls and $10$ white balls, with Alice going first. Alice always removes a black ball if there is one, while Bob removes one of the remaining balls uniformly at random. Once all balls have been removed, the expected number of black balls which Bob has can be expressed as $\frac{a}{b}$, where $a$ and $b$ are relatively prime positive integers. Compute $100a + b$.
	\end{prob}

	\begin{sk}
		Defina $E(b, p)$ como o esperado de bolas pretas que Bob terá no final do jogo, caso o jogo comece com $b$ bolas brancas e $p$ bolas pretas, e seja a vez do Bob de jogar. O problema pede para calcular $E(10, 9)$.

		Construa a equação de recorrência de $E(p, b)$ e faça casos pequenos para conjecturar que \[E(b, p) = \frac{p(p+1)}{2(b+p)}.\]

		Prove a equação anterior usando indução.
	\end{sk}

\end{document}
