\documentclass[10pt, a4paper]{article}
\usepackage[utf8]{inputenc}
\usepackage[brazilian]{babel}
\usepackage{lmodern}
\usepackage[left=2cm, right=2cm, top=2cm, bottom=2.5cm]{geometry}
\usepackage{indentfirst}
\usepackage[inline]{enumitem}

\usepackage[stylish, pensi]{zeus}

\title{Aquecimento}
\author{Guilherme Zeus Moura}
\mail{zeusdanmou@gmail.com}
\titlel{Turma Olímpica}
\titler{{\footnotesize v. 1} -- 22 de Junho de 2020}

\renewcommand{\playerA}[1]{Guilherme}
\renewcommand{\playerB}[1]{Zeus}

\newcommand{\eps}{\varepsilon}

\begin{document}	
	\zeustitle

	\problem{math/brazil/mo/1999/2}

	Suponha que todos os algarismos são zero. Então, \[\sqrt{2} = 1,414\dots a_{999999}\underbrace{00000\cdots00000}_{20000001\ \text{algarismos}} a_{3000001}\dots.\]

\[\sqrt{2} = \frac{N}{10^{10^6-1}} + \varepsilon,\ \text{com}\ \varepsilon < \frac{1}{10^{3\cdot10^6}}\ \text{e}\ N \in \ZZ \]

\begin{align*}
	2 &= \left( \frac{N}{10^{10^6-1}} + \varepsilon \right)^2\\
	  &= \frac{N^2}{10^{2(10^6 - 1)}} + \frac{2N\varepsilon}{10^{10^6 - 1}} + \varepsilon^2
\end{align*}

\[ 2\cdot(10^{2(10^6 - 1)}) - N^2 = 2N\varepsilon (10^{10^6-1}) + \varepsilon^2 (10^{2(10^6 - 1)})\]

\begin{itemize}
	\item $2\cdot(10^{2(10^6-1)}) = 2\underbrace{00\cdots00}_{2(10^6-1)}$.
	\item $N^2$ é inteiro, aproximadamente, na ordem de $10^{2(10^6-1)}$.
	\item $2N\varepsilon(10^{10^6 - 1}) < \frac{2N}{10^{2(10^6-1)}}$, na ordem de $\frac{1}{10^{10^6}}$
	\item $\varepsilon^2 (10^{2(10^6 - 1)}) < \frac{1}{10^{4(10^6 - 1)}}$
\end{itemize}

$LE$ é inteiro. $LD$ é muito pequeno. Logo, como $LD = LE$, $LD = LE = 0$.

\[LD = 0 \implies 2 = \left(\frac{N}{10^{10^6-1}}\right)^2 \implies \sqrt{2} \text{\ racional. Absurdo!}\]

\end{document}
