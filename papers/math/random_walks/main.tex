\documentclass[10pt, a4paper]{article}
\usepackage[utf8]{inputenc}
\usepackage[brazilian]{babel}
\usepackage{lmodern}
\usepackage[left=2cm, right=2cm, top=2cm, bottom=2.5cm]{geometry}
\usepackage{indentfirst}
\usepackage[inline]{enumitem}

\usepackage[stylish, pensi, problem-list]{zeus}

\title{Passeios Aleatórios}
\author{Guilherme Zeus Moura}
\mail{zeusdanmou@gmail.com}
\titlel{Turma Olímpica}
\titler{{\footnotesize v. 1} -- 31 de Junho de 2020}

\renewcommand{\playerA}[1]{Guilherme}
\renewcommand{\playerB}[1]{Zeus}

\newcommand{\coin}{\text{§}\hspace{.3ex}}

\renewcommand{\arraystretch}{1.3}

\begin{document}	
	\zeustitle

	\begin{itemize}
	\item \textbf{Isso é uma adaptação (com algumas modificações) para o português de um material entitulado \emph{Random Walks}, em uma aula ministrada por \emph{Paul Zeitz} no \emph{Berkeley Math Circle} em \emph{6 de maio de 2014}.}
	\item \coin\ é o símbolo para \emph{simoleon}, moeda utilizada nos jogos \emph{The Sims} e \emph{SimCity}.
	\end{itemize}
		
	\begin{center} \rule{15cm}{0.5pt} \end{center}

	\section*{Probabilidade Recursiva e Valor Esperado}

	\begin{prob}
		Dois jogadores jogam uma moeda, de forma alternada. O primeiro jogador que conseguir uma \emph{cara}, ganha. Qual é a probabilidade de:
		\begin{enumerate}[label = (\alph*)]
			\item o jogo nunca acabar?
			\item o primeiro jogador ganhar?
			\item o segundo jogador ganhar?
		\end{enumerate}
	\end{prob}

	\begin{prob}
		Um bilhete de loteria em Sunset Valley no custa $\coin 1$. O comprador raspa o bilhete para ver o prêmio. Calcule o lucro esperado do governo de Sunset Valley por bilhete vendido, considerando os seguintes cenários de premiação:
		\begin{enumerate}[label = (\alph*)]
			\item
				\begin{tabular}{r||c|c}
					Prêmio & $\coin 1$ & $\coin 10$ \\\hline
					Probabilidade & $\frac{1}{10}$ & $\frac{1}{100}$
				\end{tabular}
			\item
				\begin{tabular}{r||c|c|c}
					Prêmio & $\coin 1$ & $\coin 10$ & um bilhete de loteria grátis \\\hline
					Probabilidade & $\frac{1}{10}$ & $\frac{1}{100}$ & $\frac{1}{5}$ 
				\end{tabular}

		\end{enumerate}
	\end{prob}

	\begin{prob}
		\begin{enumerate}[label = (\alph*)]
			\item Em média, quantas vezes um dado de $6$ faces deve ser lançado até obter um $6$?
			\item Em média, quantas vezes um dado de $6$ faces deve ser lançado para ver todos os $6$ resultados possíveis?
		\end{enumerate}
	\end{prob}

	\begin{prob}[Uma formiga num cubo.]
		Imagine uma formiga que caminha pelas bordas de um cubo. A formiga não muda de direção enquanto viaja em uma borda. Dois vértices adjacentes, $C$ e $V$, têm comida e veneno, respectivamente. Se a formiga atingir um desses vértices, ela para de viajar.

Sempre que a formiga alcança um dos outros seis vértices, ela tem três arestas para escolher e escolhe aleatoriamente (i.e., com probabilidade de $1/3$ para cada escolha). Para cada um desses seis vértices iniciais, calcule a probabilidade de a formiga viver (ou seja, atingir $C$ antes de atingir $V$).
	\end{prob}

	\begin{prob}[Uma história real.]
		Quando organizei a Bay Arena Mathematical Olympiad pela primeira vez, eu precisava enviar formulários de inscrição com números de identificação aleatórios para os participantes. Então, eu fiz uma lista dos números de $1$ a $1000$ e, em seguida, usei meu software de amostragem para obter uma amostra aleatória de tamanho $1000$ desses números. No entanto, estupidamente esqueci de verificar o botão \emph{amostra sem substituição} e, em vez disso, fiz uma \emph{amostra com substituição}. Quantos números de identificação distintos foram produzidos?
	\end{prob}

	\begin{prob}[Outra história real.]
		No SF Math Circle para crianças do ensino fundamental, 11 crianças de 8 anos estavam em círculo. Eles escreveram seus nomes em um pedaço de papel, e o instrutor os colocou em uma caixa e sacudiu a caixa. Então cada criança escolheu aleatoriamente um nome. O instrutor entregou a uma criança uma bola de saquinho de feijão e a criança jogou a bola para a pessoa cujo nome eles tinham. E assim continuou. Se nem todas as crianças jogaram uma bola para elas, o instrutor deu a bola a uma dessas crianças deixadas de fora e o processo continuou.
		\begin{enumerate}[label = (\alph*)]
			\item Se uma criança acabasse jogando uma bola para si mesma, ela chorava. Em média, quantas crianças choram?
			\item Qual é a probabilidade de nenhuma criança chorar?
			\item O instrutor queria que todas as crianças pudessem jogar a bola sem intervenção. Em outras palavras, idealmente, todas as 11 crianças formarão um "ciclo". Qual é a probabilidade de isso acontecer?
			\item Se todas as crianças não estiverem em um ciclo, o instrutor solicitou que mudassem de nome para que isso fosse alcançado. O instrutor só permitia que duas crianças de cada vez trocassem seus pedaços de papel. Em média, quantas trocas são necessárias?
			\item Outro cenário desejável para o instrutor era que a maioria das crianças estivesse em um ciclo. Caso contrário, as crianças têm birras. Qual é a probabilidade de uma birra?
		\end{enumerate}
	\end{prob}

\end{document}
