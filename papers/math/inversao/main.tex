\documentclass[10pt, a4paper]{article}
\usepackage[utf8]{inputenc}
\usepackage[brazilian]{babel}
\usepackage{lmodern}
\usepackage[left=2cm, right=2cm, top=2cm, bottom=2.5cm]{geometry}

\usepackage{../../../commands/problems}
\renewcommand{\mypath}{../../../}

\title{Inversão}
\author{Guilherme Zeus Moura}
\mail{zeusdanmou@gmail.com}
\titlel{}
\titler{}

\begin{document}	
	\zeustitle
	\begin{defn}[Inversão]
		Seja $O$ um ponto e $r > 0$ um real. A inversão $I$ com centro $O$ e raio $r$ é uma transformação que leva o ponto $P \neq O$ em um ponto $P'$ tal que:
		\begin{enumerate}[label = (\roman*)]
			\item $P'$ está na semirreta $\overrightarrow{OP}$;
			\item $OP \cdot OP' = r^2$.
		\end{enumerate}
	\end{defn}

	\begin{prop}[Preservação de ângulos]
		\[ \angle OAB = \angle OB'A'. \]
	\end{prop}

	\begin{prop}
		Uma inversão leva uma reta que passa por $O$ em si mesma.
	\end{prop}

	\begin{prop}
		Uma inversão leva uma reta que não passa por $O$ em um círculo que passa por $O$.
	\end{prop}

	\begin{prop}
		Uma inversão leva um círculo que passa por $O$ em uma reta que não passa por $O$.
	\end{prop}

	\begin{prop}
		Uma inversão leva um círculo que não passa por $O$ em um círculo que não passa por $O$.
	\end{prop}

	\section*{Problemas}
	\begin{prob}[Teorema de Ptolomeu]
		Em um quadrilátero inscritível, o produto das medidas das diagonais é igual à soma dos produtos das medidas dos lados opostos.
	\end{prob}
	\problem{math/ibero/1998/2}
	\problem{math/brazil/mo/2011/5}	
\end{document}
