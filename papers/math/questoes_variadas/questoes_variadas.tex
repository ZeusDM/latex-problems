\documentclass[10pt,a4paper]{article}
\usepackage[utf8]{inputenc}
\usepackage[brazil]{babel}
\usepackage{lmodern}
\usepackage{fullpage}

\usepackage{../../../commands/problems}
\renewcommand{\mypath}{../../../}

\title{Questões Variadas (ou não tão variadas assim)}
\date{}
\author{Guilherme Zeus Moura}
\mail{zeusdanmou@gmail.com}

\begin{document}
	\zeustitle

	As questões \textbf{não} estão em ordem de dificuldade. Aproveite para treinar a escrever as soluções de maneira organizada. Faça os desenhos de geometria com \textit{régua e compasso.}
	\problem{math/british/2017/round1/1}	
	\problem{math/british/2017/round1/2}
	\problem{math/british/2019/round2/1}
	\problem{math/british/2017/round1/3}
	\problem{math/british/2017/round1/4}
	\problem{math/british/2019/round2/2}
	\problem{math/british/2017/round1/5}
	\problem{math/british/2017/round1/6}
	\problem{math/british/2019/round2/3}
	\problem{math/british/2019/round2/4}
\end{document}
