\documentclass[10pt, a4paper]{article}
\usepackage[utf8]{inputenc}
\usepackage[english]{babel}
\usepackage{lmodern}
\usepackage[left=2cm, right=2cm, top=2cm, bottom=2.5cm]{geometry}
\usepackage{indentfirst}
\usepackage[inline]{enumitem}

\usepackage[table,xcdraw]{xcolor}

\usepackage[section]{zeus}
%\usepackage{cite}

%\usepackage{csquotes}
\usepackage{natbib}


\title{Treinamento de Velocidade em Equipes, com Respostas}
\author{Guilherme Zeus Moura}
\mail{zeusdanmou@gmail.com}
\titlel{Turma Olímpica}
\titler{{\footnotesize v. 1} -- 10 de Setembro de 2020}

\begin{document}	
	\zeustitle

	\setcounter{section}{-1}

	\section{Pontuação e Instruções}
	
	A pontuação segue a seguinte regra:
	\begin{itemize}
		\item A $k$-ésima equipe a acertar a questão $n.i.$, na sua $t$-ésima tentativa, ganha
			\[\ceil{\frac{n+2}{kt}}\]
			pontos.
	\end{itemize}

	\begin{exmp}
		~

		A equipe $A$ acerta a questão $1.1$, na primeira tentativa. $A$ ganha $3$ pontos.

		A equipe $B$ chuta, mas erra a questão $1.1$. A pontuação não muda.

	A equipe $B$ acerta a questão $1.1$, na sua segunda tentativa. $B$ ganha $\ceil{\frac{1+2}{2\cdot 2}} = 1$ ponto.

		A equipe $A$ acerta a questão $3.2$, na primeira tentativa. $A$ ganha $5$ pontos.

		A equipe $B$ acerta a questão $8.1$, na primeira tentativa. $B$ ganha $10$ pontos.
	\end{exmp}

	\subsection{Definições}

	\begin{defn}
		O \emph{valor esperado} de uma variável aleatória $X$ é \[\EE(X) = \sum_{x} x\cdot \PP(X = x).\]
	\end{defn}

	\begin{exmp}
		O valor esperado da variável aleatória $D_6$, determinada pela face superior um dado comum de seis faces é 
	\begin{align*}
		\EE(D_6) & = 1\cdot\PP(D_6 = 1) + 2 \cdot \PP(D_6 = 2) + \cdots + 6\cdot \PP(D_6 = 6) \\
				 & = 1\cdot\frac{1}{6} + 2 \cdot \frac{1}{6} + \cdots + 6 \cdot \frac{1}{6} \\
				 & = \frac{7}{2}.
	\end{align*}
	\end{exmp}

	\subsection{Soluções com explicações.}

	\begin{itemize}
		\item \href{https://static1.squarespace.com/static/570450471d07c094a39efaed/t/58b0dac82e69cf75a401b718/1487985353030/PUMaC2016_CombinatoricsASol.pdf}{Combinatória;}
		\item \href{https://static1.squarespace.com/static/570450471d07c094a39efaed/t/58b0dae6a5790a958b9aacdf/1487985382725/PUMaC2016_GeometryASol.pdf}{Geometria;}
		\item \href{https://static1.squarespace.com/static/570450471d07c094a39efaed/t/58b0db0086e6c04706c33ee1/1487985408927/PUMaC2016_NumberTheoryASol.pdf}{Teoria dos Números.}
	\end{itemize}

	\newpage

	\section{Round 1}	

	\problem{math/pumac/2016/geometry/a/1}
	\ans{19.}
	\problem{math/pumac/2016/number_theory/a/1}
	\ans{294.}
	\problem{math/pumac/2016/combinatorics/a/1}
	\ans{3.}

	\section{Round 2}

	\problem{math/pumac/2016/combinatorics/a/2}
	\ans{247.}
	\problem{math/pumac/2016/geometry/a/2}
	\ans{36.}
	\problem{math/pumac/2016/number_theory/a/2}
	\ans{481.}

	\section{Round 3}

	\problem{math/pumac/2016/number_theory/a/3}
	\ans{161.}
	\problem{math/pumac/2016/combinatorics/a/3}
	\ans{1440.}
	\problem{math/pumac/2016/geometry/a/3}
	\ans{36.}

	\section{Round 4}

	\problem{math/pumac/2016/geometry/a/4}
	\ans{3023.}
	\problem{math/pumac/2016/combinatorics/a/4}
	\ans{10080.}
	\problem{math/pumac/2016/number_theory/a/4}
	\ans{540.}

	\section{Round 5}

	\problem{math/pumac/2016/combinatorics/a/5}
	\setcounter{prob}{0}
	\begin{prob}
		Seja $a_1, a_2, a_3, \dots$ uma sequência infinita em que, para todo inteiro positivo $i$, o número $a_i$ é um inteiro positivo escolhido aleatóriamente (de maneira uniforme) entre $1$ e $2016$, inclusive. Seja $S$ o conjunto de todos os inteiros positivos $k$ com a seguinte propriedade: para todo inteiro positivo $j$, com $j < k$, vale que $a_j \neq a_k$. (Portanto, $1 \in S$;  $2 \in S$ se e somente se $a_1 \neq a_2$; $3 \in S$ se e somente se $a_1 \neq a_3$ e $a_2 \neq a_3$.) Seja $\dfrac{p}{q}$ (com $p$ e $q$ coprimos) o valor experado da quantidade de inteiros positivos $m$ tais que ambos $m$ e $m+1$ estão em $S$. Calcule $pq$.
	\end{prob}	
	\ans{4030.}
	\problem{math/pumac/2016/number_theory/a/5}
	\ans{895.}
	\problem{math/pumac/2016/geometry/a/5}
	\ans{14.}

	\section{Round 6}

	\problem{math/pumac/2016/number_theory/a/6}
	\ans{528.}
	\problem{math/pumac/2016/geometry/a/6}
	\ans{9.}
	\problem{math/pumac/2016/combinatorics/a/6}
	\ans{1016060.}

	\section{Round 7}

	\problem{math/pumac/2016/geometry/a/7}
	\ans{421.}
	\problem{math/pumac/2016/combinatorics/a/7}
	\ans{525.}
	\problem{math/pumac/2016/number_theory/a/7}
	\ans{30576.}

	\section{Round 8}

	\problem{math/pumac/2016/number_theory/a/8}
	\ans{1410.}
	\problem{math/pumac/2016/geometry/a/8}
	\ans{175.}
	\problem{math/pumac/2016/combinatorics/a/8}
	\ans{4830.}

\end{document}
