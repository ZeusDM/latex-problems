\documentclass[11pt, a4paper]{article}
\usepackage[utf8]{inputenc}
\usepackage[brazil]{babel}
\usepackage{lmodern}
\usepackage[left=2cm, right=2cm, top=2cm, bottom=2.5cm]{geometry}
\usepackage{indentfirst}
\usepackage[inline]{enumitem}

\usepackage{zeus}
%\usepackage{cite}

%\usepackage{csquotes}
\usepackage{natbib}

\title{Problemas Sortidos de Combinatória -- Live}
\author{Guilherme Zeus Dantas e Moura}
\mail{zeusdanmou@gmail.com}
\titlel{Turma Olímpica}
\titler{{\footnotesize v. 3} -- 29 de Outubro de 2020}

\begin{document}	
	\zeustitle

	\setcounter{prob}{3}
	\problem{math/imosl/2012/C2}

	Seja $k(n)$ a resposta do problema, i.e. qual a quantidade máxima de pares com a propriedade desejada. 

	Vamos juntar $(n - 1, 1)$, $(n - 3, 2)$, $(n - 5, 3)$, $(n - 7, 4)$, $(n - 9, 5)$, $\dots$.

	Essa lista tem $\floor{\frac{n}{3}}$ pares. Portanto, \[k(n) \ge \floor{\frac{n}{3}}.\]

	\textcolor{gray}{\textsl{(Ideia não continuada)} Para provar a desigualdade reversa, uma boa ideia pode ser usar indução de uma forma parecida com indução em grafos: começar com uma configuração que funciona para $n$, obter outra configuração menor (no sentido de ``menor'' que for mais apropriado) e aplicar a hipótese de indução.}

	\vspace{1em}

	Vamos usar contagem dupla. Suponha que existem $k$ pares que satisfazem o enunciado. Sejam eles $(a_1, b_1), (a_2, b_2), \dots, (a_k, b_k)$.

	Por um lado, a soma de todos os $2k$ números selecionados é \[\sum = (a_1 + b_1) + \cdots + (a_k + b_k)\le n + (n - 1) + (n - 2) + \cdots + (n - k + 1) = \frac{(2n - k + 1)k}{2}.\]

	Por outro lado, vale \[\sum = a_1 + b_1 + \cdots + a_k + b_k \ge 1 + 2 + 3 + \cdots + 2k = \frac{(2k+1)2k}{2}.\]

	Logo,
	\begin{align*}
		2n - k + 1 & \ge 4k + 2\\
		2n - 1 & \ge 5k\\
		\frac{2n - 1}{5} & \ge k.
	\end{align*}

	Isso implica que \[ k(n) \le \frac{2n-1}{5}.\]

	\vspace{1em}

	Se $n = 5m + 3$, $k = 2m + 1$, a gente vai tentar usar os números $1, 2, \cdots 4m + 2$ para obter as somas $5m + 3, 5m + 2, \cdots, 3m+3$. Para isso, vamos formar 2 grupos de pares:
\begin{enumerate}[label = \textbullet]
	\item $(4m + 2, m+1)$ com soma $5m + 3$;
	\item $(4m, m+2)$ com soma $5m + 2$;
	\item $(4m - 2, m+3)$ com soma $5m + 1$;
	\item $(4m - 4, m+4)$ com soma $5m$;
	\item[] $\vdots$
	\item $(2m + 2, 2m+1)$, com soma $4m + 3$;
	\item[] e
	\item $(4m + 1, 1)$ com soma $4m + 2$;
	\item $(4m - 1, 2)$ com soma $4m + 1$;
	\item $(4m - 3, 3)$ com soma $4m$;
	\item $(4m - 5, 4)$ com soma $4m - 1$;
	\item[] $\vdots$
	\item $(2m+3, m)$ com soma $3m+3$.
\end{enumerate}

	\newpage

	\setcounter{prob}{1}
	\begin{prob}[Irã 1997-98]
		Cada casa de um tabuleiro $n\times n$ está preenchida com um dos números $-1, 0, 1$. Toda linha e toda columa possui exatamente um $1$ e um $-1$. Prove que as linhas e colunas podem ser reordenadas tal que, no tabuleiro resultante, cada número seja trocado por seu oposto. 
	\end{prob}

	Vamos trocar $1$ por $+$, $-1$ por $-$ e $0$ por uma casa vazia.

	Vamos chamar de tabela principal de tamanho $n$ a tabela onde a diagonal principal possui $+$ e as casas acima da diagonal principal possuem $-$. Aqui, representamos a tabela principal de tamanho $5$:	
	\[
	 \begin{bmatrix}
		 + & - &   &   &   \\
		   & + & - &   &   \\
		   &   & + & - &   \\
		   &   &   & + & - \\
		 - &   &   &   & + 
	\end{bmatrix}
	\]

	Ao trocar as linhas $(1, 2, 3, \dots, n-2, n-1, n)$ por $(n, n-1, n-2, \dots, 3, 2, 1)$ e trocar $(1, 2, 3, \dots, n-2, n-1, n)$ por $(1, n, n-2, \dots, 4, 3, 2)$, mostramos que o problema é verídico para qualquer tabela principal de tamanho $n$. Vamos chamar a permutação de linhas e colunas acima de $\psi$.

	Se conseguirmos uma permutação $\phi$ de linhas e colunas que leva um tabuleiro $T$ na tabela principal, podemos fazer $\phi^{-1} \circ \psi \circ \phi$ para levar o tabuleiro $T$ no tabuleiro com entradas opostas. (Isso é falso. Não dá pra levar qualquer tabuleiro em uma tabela principal, mas dá pra levar qualquer tabuleiro em uma ``concatenação'' de tabelas principais. De qualquer modo, vamos abordar o problema usando grafos.)
	
	\vspace{2em}

	Vamos criar um grafo orientado $G(T)$, em função de $T$. Os vértices serão as colunas e $i \to j$ se o $+$ da coluna $i$ está na mesma linha que o $-$ da coluna $j$. Por exemplo, se $T$ é a tabela principal de tamanho $n$, $G(T)$ é o ciclo $1 \to 2 \to 3 \to \cdots \to n \to 1$.

	Observe como esse grafo varia quando permutamos as linhas e colunas (fizemos isso durante a aula).

	Os seguintes dois lemas (que provamos durante a aula) acabam o problema.
	\begin{lem}
		Se $T$ é um tabuleiro e $T'$ é a tabuleiro com entradas opostas, então existe um tabuleiro $S$ tal que:
		\begin{itemize}
			\item Existe uma permutação de colunas que leva $T$ em $S$, e;
			\item $G(S) = G(T')$.
		\end{itemize}
	\end{lem}
	\begin{lem}
		Se $A$ e $B$ são dois tabuleiros com $G(A) = G(B)$, então existe uma permutação de linhas que leva $A$ em $B$.
	\end{lem}
	
	%\begin{prob}[Irã 1997-98]
	%	An $n\times n$ table is filled with numbers $-1, 0, 1$ in such a manner that every row and column contain exactly one $1$ and one $-1$. Prove that the rows and columns can be reordered so that in the resulting table each number has been replaced with its negative.
	%\end{prob}

	%\begin{prob}[Torneio das Cidades 1999, Junior A, 6]
	%	A rook is allowed to move one cell either horizontally or vertically. After $64$ moves the rook visited all cells of the $8 \times 8$ chessboard and returned back to the initial cell. Prove that the number of moves in the vertical direction and the number of moves in the horizontal direction cannot be equal.
	%\end{prob}
	%\problem{math/brazil/mo/2014/5}
	%\problem{math/brazil/mo/1995/6}
	%\begin{prob}
	%	Let $n$ be an positive integer. Find the smallest integer $k$ with the following property:
		
	%	Given any real numbers $a_1 , \cdots , a_d $ such that $a_1 + a_2 + \cdots + a_d = n(n+1)$ and $0 \le a_i \le 1$ for all $i \in \{1,2,\cdots ,d\}$, it is possible to partition these numbers into $k$ groups (some of which may be empty) such that the sum of the numbers in $i^\mathrm{th}$ group is at most $i$, for all $i \in \{1,2,\cdots ,k\}$.
	%\end{prob}
	%\begin{prob}[San Diego Power Contest Fall 2020-21, 7] 
	%	Alice is wandering in the country of Wanderland. Wanderland consists of a finite number of cities, some of which are connected by two-way trains, such that Wanderland is connected: given any two cities, there is always a way to get from one to the other through a series of train rides.
    
%Alice starts at Riverbank City and wants to end up at Conscious City. Every day, she picks a train going out of the city she is in uniformly at random among all of the trains, and then boards that train to the city it leads to. Show that the expected number of days it takes for her to reach Conscious City is finite.
%	\end{prob}
%	\begin{prob}[North Macedonia Junior 2020, 5]
%		Let $T$ be a triangle whose vertices have integer coordinates, such that each side of $T$ contains exactly $m$ points with integer coordinates. If the area of $T$ is less than $2020$, determine the largest possible value of $m$.
%	\end{prob}
%	\problem{math/ciim/2020/3}
%	\problem{math/usemo/2020/2}
%	\problem{math/ibero/1994/3}
%	\problem{math/elmo/2019/3}
\end{document}
