\documentclass[10pt,a4paper]{article}
\usepackage[utf8]{inputenc}
\usepackage[brazil]{babel}
\usepackage{amsmath}
\usepackage{amsfonts}
\usepackage{amssymb}
\usepackage{amsthm}
\usepackage{graphicx}
\usepackage{enumitem}
\usepackage{lmodern}
\usepackage[left=3cm, right=3cm, top=3cm, bottom=3cm]{geometry}

\theoremstyle{definition}
\newtheorem{problem}{Problema}

\usepackage{../../../commands/problems}
\renewcommand{\mypath}{../../../}

\newcommand{\problema}[1]{
	\begin{problem}
		\exercise{#1}
	\end{problem}
}

\title{Equações Funcionais}
\date{22 de agosto de 2019}
\author{Guilherme Zeus Moura}

\begin{document}	
	{\noindent \, \large \sc Turma Olímpica}
	\hfill
	{\sc \today}
	\vspace{3mm}
	\hrule
 	\hrule
 	\vspace{0mm}
	\begin{center}
		{\Large Equações Funcionais}\\\vspace{1mm}
		{\large Guilherme Zeus Moura}\\\vspace{0mm}
		{zeusdanmou@gmail.com}
	\end{center}
	\vspace{0mm}
	\hrule
	\hrule
 	\vspace{2mm}

	\section{Ideias úteis}
	\begin{enumerate}[label = (\alph*)]
		\item Procurar algumas soluções para saber o que tentar provar.
		\item Injetividade, sobretetividade, paridade, $\dots$
		\item Explorar quasi-simetrias.
		\item Resolver para $\mathbb{Z}$, extender para $\mathbb{Q}$ e extender para $\mathbb{R}$.
	\end{enumerate}
	\section{Problemas}
	\begin{problem}
		Encontre todas as funções $f: \RR \to \RR$ tais que
		$$ 2f(x) + f(1-x) = 1 + x$$
		para todo $x$ real.
	\end{problem}
	\problema{math/brazil/rio/2006/4}
	\begin{problem}
		(Cauchy)
		Encontre todas as funções $f: \mathbb{Q} \to \mathbb{Q}$ tais que
		$$ f(x+y) = f(x) + f(y).$$
	\end{problem}
	\problema{math/imo/2019/1}
	\begin{problem}
		Encontre todas as funções $f: \RR \in \RR$ tais que $f(0) = 1$ e
		$$ f(xy + 1) = f(x)f(y) - f(y) - x + 2$$
		para todos os $x$ e $y$ reais.
	\end{problem}
	\begin{problem}
		(2018 IMO Canada Training)
		Encontre todas as funções $f: \mathbb{Z} \to \mathbb{Z}$ tais que
		$$ f(x - f(y)) = f(f(x)) - f(y) - 1$$
		para todos os $x$ e $y$ inteiros.
	\end{problem}
	\problema{math/imo/2010/1}
	\problema{math/imo/2002/5}
	\problema{math/brazil/mo/2010/1}
	\problema{math/brazil/mo/1998/5}
	\problema{math/imo/2004/2}
	\problema{math/imo/2009/5}
	\problema{math/imo/1999/6}
	\problema{math/brazil/mo/2013/3}	
	\problema{math/brazil/mo/2012/6}
	\problema{math/brazil/mo/2006/3}

\end{document}
