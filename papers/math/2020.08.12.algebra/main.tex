\documentclass[10pt, a4paper]{article}
\usepackage[utf8]{inputenc}
\usepackage[brazilian]{babel}
\usepackage{lmodern}
\usepackage[left=2cm, right=2cm, top=2cm, bottom=2.5cm]{geometry}
\usepackage{indentfirst}
\usepackage[inline]{enumitem}

\usepackage{pgf,tikz,pgfplots}
\pgfplotsset{compat=1.15}
\usepackage{mathrsfs}
\usetikzlibrary{arrows}
\pagestyle{empty}
\newcommand{\degre}{\ensuremath{^\circ}}

\usepackage[pensi,
			problem-list
			]{zeus}

\title{Problemas Sortidos de Álgebra}
\author{Guilherme Zeus Moura}
\mail{zeusdanmou@gmail.com}
\titlel{Turma Olímpica}
\titler{{\footnotesize v. 1} -- 12 de Agosto de 2020}

\renewcommand{\playerA}[1]{Chen}
\renewcommand{\playerB}[1]{Rodrigo}

\begin{document}	
	\zeustitle

	\begin{prob}[XXII Semana Olímpica, George Lucas]
		Sejam $a$, $b$ e $c$ números reais positivos. Prove a desigualdade \[ \sqrt{a^2 - ab + b^2} +  \sqrt{b^2 - bc + c^2} \ge \sqrt{a^2 + ac + c^2}, \]
		e ache os casos de igualdade.
	\end{prob}

	\begin{prob}[Harvard Math Review, Zachary Abel]
		\playerA{Bert} is thinking of an ordered quadruple of integers $(a, b, c, d)$. \playerB{Ernie}, hoping to determine these integers, hands \playerA{Bert} a $4$-variable polynomial $P(w, x, y, z)$ with integer coefficients, and \playerA{Bert} returns the value of $P(a, b, c, d)$. From this value alone, \playerB{Ernie} can always determine \playerA{Bert}’s original ordered quadruple. Construct, with proof, one polynomial that \playerB{Ernie} could have used.
	\end{prob}

	\setcounter{prob}{1}

	\begin{prob}[Problema 2, com menos \playerA{Bert} e \playerB{Ernie}]
		Ache, com prova, um polinômio $P \in \ZZ[x, y, z, w]$ tal que $P: \ZZ^4 \to \ZZ$ é uma injetiva.
	\end{prob}

	\begin{prob}[The Mandelbrot Problem Book, Sam Vandervelde]
		Resolva os seguintes itens:
		\begin{enumerate}[label = \textbf{\color{main} (\alph*)}]
			\item Se $P(x)$ é um polinômio de grau $n$ tal que $P(0) = 1, P (1) = -1, P (2) = 1, P (3) = -1, \dots, P (n) = (-1)^n$. Determine $P(n + 1)$.
			\item Se $P(x)$ é um polinômio de grau $n$ tal que $P(1) = 1, P (2) = 3, P(4) = 9, \dots, P (2^n) = 3^n$. Determine $P (2^{n+1} )$.
		\end{enumerate}
	\end{prob}


\end{document}
