\documentclass[11pt, a4paper]{article}
\usepackage[utf8]{inputenc}
\usepackage[brazilian]{babel}
\usepackage{lmodern}
\usepackage{euler}
\usepackage[left=2cm, right=2cm, top=2cm, bottom=2.5cm]{geometry}
\usepackage{indentfirst}
\usepackage[inline]{enumitem}

\usepackage{zeus}

\title{Combinatória \& Divisibilidade}
%\title{Briefing: Estudando Sistemas de Votações}
\author{Guilherme Zeus Dantas e Moura}
\mail{zeusdanmou@gmail.com}
\titlel{\includegraphics[width = .2\textwidth]{mm.png}}
\titler{09 de Novembro de 2020}

\begin{document}	
	\zeustitle

	\section{Problema Motivador}

	\problem{math/imo/1995/6}

	\section{Pensando combinatóriamente}

	O objetivo é que, no final dessa aula, vocês entendam como métodos combinatóricos podem ajudar a entender divisibilidades.

	\begin{idea}
		Para mostrar que uma quantidade de objetos é múltipla de $n$, basta dividir esses objetos em vários grupos de tamanho $n$, ou em $n$ grupos de mesmo tamanho. 
	\end{idea}

	\begin{prob}[Pequeno Teorema de Fermat]
		Sejam $p$ um primo e $a$ um inteiro positivo. Prove, usando argumentos de combinatória, que $a^p - a$ é múltiplo de $p$. 
	\end{prob}

	\begin{prob}
		Seja $p$ um primo ímpar. Quantos subconjuntos do conjunto $\{1, 2, \dots, p\}$ possuem soma de seus elementos múltipla de $p$.
	\end{prob}
	\begin{rem}
		A soma dos elementos do conjunto vazio é zero.
	\end{rem}

	\section{De Volta ao Problema Motivador}

	\setcounter{prob}{0}
	\problem{math/imo/1995/6}
	\setcounter{prob}{3}

	\section{Desafio}
	
	Fica como desafio para casa, para treinar a precisão e formalização, o seguinte problema:

	\begin{prob}[Generalização do IMO 1995, 6]
		Sejam $p$ um primo ímpar e $m < n$ inteiros positivos. Quantos subconjuntos de tamanho $mp$ do conjunto $\{1, 2, \dots, np\}$ possuem soma de seus elementos múltipla de $p$?
	\end{prob}

\end{document}
