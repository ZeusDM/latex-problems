\documentclass[10pt,a4paper]{article}
\usepackage[utf8]{inputenc}
\usepackage[brazil]{babel}
\usepackage{amsmath}
\usepackage{amsfonts}
\usepackage{amssymb}
\usepackage{graphicx}
\usepackage{enumitem}
\usepackage{lmodern}
\usepackage{fullpage}

\usepackage[stylish, persection]{zeusproblems}
\usepackage{zeuscolor}
\usepackage{zeusall}

\colorlet{main}{DarkCyan}

\titleformat{\section}[hang]{}{}{1em}{\large\bfseries\sffamily}[]

\title{Treinamento para Provas de Velocidade em Equipe, \#2}
\author{Alunos Matematicamente Internacionais}
\nomail
\titler{8 de janeiro de 12020 HE}
\titlel{Matematicamente Internacionais}

\pagestyle{empty}

\renewcommand\playerA[1]{Guilherme}
\renewcommand\playerB[1]{Zeus}
\newcommand{\round}[1]{

	\noindent\dotfill

	\section{#1}
}
\begin{document}
	\zeustitle

	\vspace{1em}\noindent\hspace{1em}{\large \sffamily \bfseries Instruções:}

    \begin{itemize}
        \item Tamanho esperado da equipe: entre 6 e 8 pessoas.
        \item Tempo disponível: 80 mintuos.
    \end{itemize}
	
	\round{Round 1}	
	\problem*{math/pumac/2018/algebraa/1}
	\problem*{math/pumac/2018/combinatoricsa/1}
	\problem*{math/pumac/2018/geometrya/1}
	\problem*{math/pumac/2018/numbertheorya/1}
	\round{Round 2}	
	\problem*{math/pumac/2018/algebraa/2}
	\problem*{math/pumac/2018/combinatoricsa/2}
	\problem*{math/pumac/2018/geometrya/2}
	\problem*{math/pumac/2018/numbertheorya/2}
	\newpage
	\section{Round 3}	
	\problem*{math/pumac/2018/algebraa/3}
	\problem*{math/pumac/2018/combinatoricsa/3}
	\problem*{math/pumac/2018/geometrya/3}
	\problem*{math/pumac/2018/numbertheorya/3}
	\round{Round 4}	
	\problem*{math/pumac/2018/algebraa/4}
	\problem*{math/pumac/2018/combinatoricsa/4}
	\problem*{math/pumac/2018/geometrya/4}
	\problem*{math/pumac/2018/numbertheorya/4}
	\newpage
	\section{Round 5}	
	\problem*{math/pumac/2018/algebraa/5}
	\problem*{math/pumac/2018/combinatoricsa/5}
	\problem*{math/pumac/2018/geometrya/5}
	\problem*{math/pumac/2018/numbertheorya/5}
	\round{Round 6}	
	\problem*{math/pumac/2018/algebraa/6}
	\problem*{math/pumac/2018/combinatoricsa/6}
	\newpage
	\section{Round 7}	
	\problem*{math/pumac/2018/algebraa/7}
	\problem*{math/pumac/2018/combinatoricsa/7}
	\problem*{math/pumac/2018/geometrya/7}
	\problem*{math/pumac/2018/numbertheorya/7}
	\round{Round 8}	
	\problem*{math/pumac/2018/algebraa/8}
	\problem*{math/pumac/2018/combinatoricsa/8}
	\problem*{math/pumac/2018/geometrya/8}
	\problem*{math/pumac/2018/numbertheorya/8}









\end{document}
