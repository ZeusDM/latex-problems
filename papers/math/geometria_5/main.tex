\documentclass[12pt, a4paper]{article}
\usepackage[utf8]{inputenc}
\usepackage[brazilian]{babel}
\usepackage{lmodern}
\usepackage[left=2cm, right=2cm, top=1.5cm, bottom=2.5cm]{geometry}

\usepackage[mm]{zeuscolor}
\usepackage{zeusall}

\title{Rodadas de Geometria}
\author{} % Régis
\nomail
\titlel{}
\titler{}

\begin{document}	
	\zeustitle
	\section{Rodada 1}
	\problem{math/apmo/2007/2}
	\begin{prob}[Rússia 2008]
		Um ponto $K$ é escolhido sobre a diagonal $BD$ do quadrilátero inscritível $ABCD$ tal que $\angle AKB = \angle ADC$. Denote por $I$ e $I'$ os incentros dos triângulos $ACD$ e $ABK$, respectivamente. Os segmentos $II'$ e $BD$ se intersectam no ponto $X$. Prove que $A$, $X$, $I$ e $D$ são concíclicos.
	\end{prob}
	\problem{math/ibero/2013/2}
	\begin{prob}[Rússia 2008]
		O incírculo $\omega$ do triângulo $ABC$ tangencia os lados $BC$, $CA$ e $AB$ nos pontos $A'$, $B'$ e $C'$, respectivamente. Dois pontos distintos $K$ e $L$ são escolhidos sobre $\omega$ tal que $\angle AKB' + \angle BKA' = \angle ALB' + \angle BLA' = 180^\circ$. Prove que a reta $KL$ é equidistante dos pontos $A'$, $B'$ e $C'$.
	\end{prob}
	\problem{math/usa/tst/2017/5}
	\problem{math/china/tst/2013/5}

	\section{Rodada 2}
	\problem{math/imo/2008/1}
	\problem{math/japan/2009/4}
	\begin{prob}[Bulgária 1998] % not Bulgaria, 98
		Um quadrilátero convexo $ABCD$ tem $AD = CD$ e $\angle DAB = \angle ABC < 90^\circ$. A reta por $D$ e pelo ponto médio de $BC$ corta $AB$ no ponto $E$. Prove que $\angle BEC = \angle DAC$.
	\end{prob}
	\problem{math/imo/2007/2}
	\begin{prob}[Rússia 2007]
		Dado um triângulo $ABC$, uma circunferência passa pelos vértices $B$ e $C$ e intersecta os lados $AB$ e $AC$ nos pontos $D$ e $E$, respectivamente. Os segmentos $CD$ e $BE$ se intersectam no ponto $O$. Denote os incentros dos triângulos $ADE$ e $ODE$ por $M$ e $N$, respectivamente. Prove que o ponto médio do menor arco $DE$ está sobre a reta $MN$.
	\end{prob}
	\begin{prob}[Rússia 2012]
		O ponto $E$ é o ponto médio do segmento conectando o ortocentro do triângulo escaleno $ABC$ e o ponto $A$. O incírculo do triângulo $ABC$ tangencia os lados $AB$ e $AC$ nos pontos $C'$ e $B'$, respectivamente. Seja $F$ o simétrico do ponto $E$ em relação à rela $B'C'$ e sejam $I$ e $O$ o incentro e o circumcentro do triângulo $ABC$, respectivamente. Prove que $F$, $I$ e $O$ são colineares.
	\end{prob}

	\section{Rodada 3}

	\problem{math/imo/2014/4}
	\problem{math/rioplatense/2008/5}
	\problem{math/tuymaada/2012/3}
	\problem{math/conesul/2010/5}
	\begin{prob}[Bulgária 2013]
		Considere um triângulo acutângulo $ABC$ com alturas $AA_1$, $BB_1$ e $CC_1$. Considere o ponto $C'$ no prolongamento de $B_1A_1$ além do ponto $A_1$ tal que $A_1C' = B_1C_1$. Analogamente, considere o ponto $B'$ no prolongamento de $A_1C_1$ além do ponto $C_1$ tal que $C_1B' = A_1B_1$ e o ponto $A'$ no prolongamento de $C_1B_1$ além do ponto $B_1$ tal que $B_1A' = C_1A_1$. Denote $A''$, $B''$ e $C''$ os pontos simétricos de $A'$, $B'$ e $C'$ em relação aos pontos $BC$, $CA$ e $AB$, respectivamente. Prove que se $R$, $R'$ e $R''$ são os circunraios dos triângulos $ABC$, $A'B'C'$ e $A''B''C''$, então $R$, $R'$ e $R''$ são lados de um triângulo com área igual a metade da área do triângulo $ABC$.
	\end{prob}
	\begin{prob}[Bulgária 2014]
		O quadrilátero $ABCD$ está inscrito na circunferência $\omega$. As retas $AC$ e $BD$ se intersectam no ponto  $E$ e as semirretas $\overrightarrow{CB}$ e  $\overrightarrow{DA}$ se encontram no ponto $F$. Mostre que a reta pelos incentros de $ABE$ e $ABF$ e a reta pelos incentros de $CDE$ e $CDF$ se encontram em um ponto sobre a circunferência $\omega$.
	\end{prob}
\end{document}
