\documentclass[10pt, a4paper]{article}
\usepackage[utf8]{inputenc}
\usepackage[brazilian]{babel}
\usepackage{lmodern}
\usepackage[left=2.5cm, right=2.5cm, top=2.5cm, bottom=2.5cm]{geometry}

\usepackage[persection]{../../../commands/problems}
\renewcommand{\mypath}{../../../}

\title{Régua e Compasso}
\author{Guilherme Zeus Moura\\Migual Batista}
\mail{zeusdanmou@gmail.com}
\titlel{Turma Olímpica}
\titler{5 de setembro de 2019}

\begin{document}	
	\zeustitle
	\section{Construções Simples}
	\prob{Dados dois pontos $A$ e $B$ no plano, desenhe o ponto médio do segmento $AB$.}
	\prob{Dados dois pontos $A$ e $B$ no plano, desenhe a mediatriz do segmento $AB$.}
	\prob{Dadas duas retas secantes $r$ e $s$, desenhe o par de bissetrizes das retas $r$ e $s$.}
	\prob{Dado um ponto $A$ e uma reta $r$, desenhe a perpependicular a $r$ por $A$.}
	\prob{Dado um ponto $A$ numa reta $r$, desenhe a perpependicular a $r$ por $A$.}
	\prob{Dados dois pontos $A$ e $B$ e dois segmentos de tamanho $p$ e $q$, desenhe o ponto $P$ no segmento $AB$ tal que $AP : PB = p : q$.}
	\prob{Dado um circunferência, ache seu centro.}
	\prob{Dados um ponto $A$ e uma circunferência $\Gamma$ que passa por $A$, desenhe a reta tangente a $\Gamma$ por $A$.}

	\section{Construções não tão simples}
	\prob{Dados uma reta $r$ e dois pontos $A$ e $B$ no mesmo semiplano definido por $r$, desenhe o círculo tangente a $r$ que passa por $A$ e $B$.}
\end{document}
