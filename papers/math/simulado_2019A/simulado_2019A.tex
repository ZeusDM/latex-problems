\documentclass[10pt, a4paper]{article}
\usepackage[utf8]{inputenc}
\usepackage[brazilian]{babel}
\usepackage{lmodern}
\usepackage[left=2.5cm, right=2.5cm, top=2.5cm, bottom=2.5cm]{geometry}

\usepackage{../../../commands/problems}
\renewcommand{\mypath}{../../../}

\title{Simulado}
\author{}
\mail{}
\titlel{Turma Olímpica}
\titler{}

\newcommand{\rulesep}{
	\vspace{4mm}

	\hrule

	\vspace{3.5mm}
}

\begin{document}
	\pagestyle{empty}
	\hrule\vspace{.5mm}
	
	\begin{center}
		{\bfseries {\large S}{IMULADO}} \vspace{1mm}

		Nível 3 (Ensino Médio) \vspace{0.5mm}
		
		\large Primeiro Dia
	\end{center}
	
	\vspace{.5mm}\hrule
	
	\vspace{4mm}

	\problem*{math/jbmosl/2018/C2}
	
	\rulesep

	\problem*{math/imosl/1998/N2}
	
	\rulesep
	
	\problem*{math/hmic/2016/2}

	\rulesep

	{\hfill \slshape Cada problema vale 7 pontos.}

	{\hfill \slshape Tempo: 4 horas e 30 minutos.}

	\newpage

	\hrule\vspace{.5mm}
	
	\begin{center}
		{\bfseries {\large S}{IMULADO}} \vspace{1mm}

		Nível 3 (Ensino Médio) \vspace{0.5mm}
		
		\large Segundo Dia
	\end{center}
	
	\vspace{.5mm}\hrule
	
	\vspace{4mm}
	
	\problem*{math/hmmt/2019/T/2}
	\rulesep
	\problem*{math/jbmosl/2018/G1}
	\rulesep
	\begin{prob} %putnam92
		Quatro pontos são escolhidos uniformemente em uma esfera. Qual é a probabilidade de que o centro da esfera esteja no interior do tetraedro formado por esses pontos?
	\end{prob}
	\rulesep

	{\hfill \slshape Cada problema vale 7 pontos.}
	
	{\hfill \slshape Tempo: 4 horas e 30 minutos.}
\end{document}
