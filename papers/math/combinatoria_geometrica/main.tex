\documentclass[10pt,a4paper]{article}
\usepackage[utf8]{inputenc}
\usepackage[brazilian]{babel}
\usepackage{lmodern}
\usepackage[left=1.5cm, right=1.5cm, top=1.5cm, bottom=2.5cm]{geometry}

\usepackage{../../../commands/problems}
\renewcommand{\mypath}{../../../}

\title{Combinatória Geométrica}
\author{Guilherme Zeus Moura}
\mail{zeusdanmou@gmail.com}
\titlel{Turma Olímpica}
\titler{\today}

\renewcommand\playerA[1]{Guilherme}
\renewcommand\playerB[1]{Zeus}

\begin{document}	
	\zeustitle
	\begin{prob}
		Dados $2n$ pontos no plano, sem três colineares, sendo $n$ vermelhos e $n$ azuis. Prove que existe um pareamento entre os pontos vermelhos e os pontos azuis tal que os segmentos unindo cada par não se intersectam dois a dois.
	\end{prob}
	\problem{math/conesul/2001/4}
	\begin{prob}[Ibero 1997]
		Seja $\mathcal{P} = \{P_1, P_2, \dots, P_{1997}\}$ um conjunto de 1997 pontos no interior de um círculo de raio $1$, com $P_1$ sendo o centro do círculo. Para $k = 1, 2, \dots, 1997$ seja $x_k$ a distância de $P_k$ ao ponto de $\mathcal{P}$ mais próximo de $P_k$. Mostre que $$ x_1^2 + x_2^2 + \cdots + x_{1997}^2 \le 9.$$
	\end{prob}
	\begin{prob}
		São desenhadas $n \ge 3$ retas no plano tais que:
		\begin{enumerate}[label = (\roman*)]
			\item Quaisquer duas retas são concorrentes;
			\item Por todo ponto de interseção de duas retas passa pelo menos mais uma reta.
		\end{enumerate}
		Prove que todas as retas passam por um mesmo ponto.
	\end{prob}
	\begin{prob}
		Seja $C$ um círculo de raio $16$ e $A$ um anel com raio interior $2$ e raio exterior $3$. Agora, suponha que um conjunto $S$ de $650$ pontos são selecionados no interior de $C$. Prove que podemos colocar o anel $A$ no plano de modo que ele cubra pelo menos $10$ pontos de $S$.
	\end{prob}
	\problem{math/brazil/mo/2010/5}
	\begin{prob}[Banco IMO 1989]
		Temos um conjunto finito de segmentos no plano, de medida total 1. Prove que existe uma reta $\ell$ tal que a soma das medidas das projeções destes segmentos a reta $\ell$ é menor que $2/\pi$.
	\end{prob}
	\problem{math/brazil/mo/2018/6}
	\begin{prob}
		Prove que existe um conjunto $S$ de $3^{1000}$ pontos no plano tal que, para cada ponto $P$ de $S$, existem pelo menos $2000$ pontos em $S$ cuja distância para $P$ é exatamente uma unidade.
	\end{prob}
	\begin{prob}[Turquia]
		Mostre que o plano não pode ser coberto por um número finito de parábolas.
	\end{prob}
	\begin{prob}[Ibero 2002]
		Seja $S$ um conjunto de nove pontos, sem três pontos colineares. Se $P$ é um ponto de $S$, mostre que a quantidade de triângulos cujos vértices estão em $S - {P}$ e $P$ está em seu inteirior é par.
	\end{prob}
	\begin{prob}
		No interior de um quadrado são escolhidos $1000$ pontos. Estes pontos, juntamente com os $4$ vértices do quadrado, formam um conjunto $P$ onde não há três pontos colineares. Alguns destes pontos são ligados por segmentos de modo a particionar o quadrado em vários triângulos. Não há polígonos com mais de três lados nessa partição e todo ponto é vértice de ao menos um triângulo. Ache a quantidade de triângulos da partição.
	\end{prob}
	\begin{prob}[IMO 1989]
		Sejam $n$ e $k$ dois inteiros positivos e seja $S$ um conjunto de $n$ pontos num plano tais que
		\begin{enumerate}[label = (\roman*)]
			\item Não haja três pontos de $S$ que sejam colineares;
			\item Para qualquer ponto $P$, há pelo menos $k$ pontos de $S$ que são equidistantes de $P$.
		\end{enumerate}

		Prove que $$k < \frac{1}{2} + \sqrt{2n}.$$
	\end{prob}
	\begin{prob}[Irã 2010]
		Temos $n$ pontos no plano de modo que não existam três deles colineares. Prove que o número de triângulos com vértices nestes pontos e cuja área é exatamente $1$ não pode ser maior que $\frac{2}{3}(n^2 - n)$.
	\end{prob}
	\begin{prob}[China 2000, HongBin Yu]
		Considere $n$ pontos colineares e todas as distâncias entre dois deles. Suponha que cada distância aparece no máximo duas vezes. Prove que existem pelo menos $\floor{n/2}$ destas distâncias que aparecem 1 vez cada.
	\end{prob}
	\begin{prob}[China 2011]
		Seja $S$ um conjunto de $n$ pontos no plano tal que não há quatro pontos colineares. Sejam $d_1 < d_2 < \cdots < d_k$ as diferentes distâncias entre os pares de pontos distintos de $S$ e seja $m_i$ a quantidade de pares de pontos cuja distância é exatamente $d_i$. Prove que $$\sum_{i = 1}^{k} m_i^2 \le n^3 - n^2.$$
	\end{prob}
\end{document}
