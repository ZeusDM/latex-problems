\documentclass[10pt,a4paper]{article}
\usepackage[utf8]{inputenc}
\usepackage[brazilian]{babel}
\usepackage{lmodern}
\usepackage[left=1.5cm, right=1.5cm, top=2.5cm, bottom=2.5cm]{geometry}

\usepackage{zeusall}
\usepackage{zeusproblems}
\usepackage{zeustitle}

\title{Alguns Problemas de Combinatória}
\author{Guilherme Zeus Moura}
\mail{zeusdanmou@gmail.com}
\titlel{}
\titler{}

\renewcommand\playerA[1]{Guilherme}
\renewcommand\playerB[1]{Zeus}

\begin{document}	
	\zeustitle
	\section{Ideias úteis}
	\begin{enumerate}[label = (\alph*)]
		\item \emph{Casos pequenos.} Mesmo que sejam trabalhosos.
		\item \emph{Princípio Extremal:} Analisar propriedades extremais. Usualmente olhamos para o objeto ou exemplo minimal ou maximal, em algum sentido.
		\item \emph{Indução} e \emph{Indução Forte}.
		\item \emph{Pareamento} e \emph{Agrupamento}: Juntar objetos com propriedades que se relacionam em algum sentido.
	\end{enumerate}
	\section{Problemas}
	\problem{math/others/3}
	\problem{math/mit/putnam/hidden_independence_and_uniformity/1}
	\problem{math/mit/putnam/hidden_independence_and_uniformity/2}
	\problem{math/others/1}
	\problem{math/mit/putnam/hidden_independence_and_uniformity/3}
	\problem{math/brazil/rio/2018/N3/5}
	\problem{math/mit/putnam/hidden_independence_and_uniformity/5}
	\problem{math/conesul/2001/4}
	\problem{math/ibero/1997/6}
	\problem{math/lusophon/2018/6}
	\problem[/alternate]{math/imo/2018/4}
	\problem{math/imo/2019/5}
	\problem{math/imo/2015/1}
	\problem{math/mit/putnam/hidden_independence_and_uniformity/4}
	\problem{math/mit/putnam/hidden_independence_and_uniformity/6}
	\problem{math/mit/putnam/hidden_independence_and_uniformity/7}
	\problem{math/mit/putnam/hidden_independence_and_uniformity/8}
	\problem{math/mit/putnam/hidden_independence_and_uniformity/9}
\end{document}
