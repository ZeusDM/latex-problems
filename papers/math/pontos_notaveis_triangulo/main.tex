\documentclass[10pt, a4paper]{article}
\usepackage[utf8]{inputenc}
\usepackage[brazilian]{babel}
\usepackage{lmodern}
\usepackage[left=2.5cm, right=2.5cm, top=2.5cm, bottom=2.5cm]{geometry}

\usepackage[persection, zeusnotation]{../../../commands/problems}
\renewcommand{\mypath}{../../../}

\title{Compilado de Geometria}
\author{Guilherme Zeus Moura}
\mail{zeusdanmou@gmail.com}
\titlel{Turma Olímpica}
\titler{\today}

\begin{document}	
	\zeustitle
	\section{Minha Notação}
		Essa notação não é usual. Para usar essa notação em qualquer outro lugar, é necessário explicar o que significa cada símbolo.
		\begin{itemize}
			\item $\col{A}{B}{\dots}{Z}$ significa que existe uma reta que passa por $A, B, \dots, Z$.
			\item $\conc{A}{B}{\dots}{Z}$ significa que existe um círculo que passa por $A, B, \dots, Z$ e também denota, quando existe, o círculo que passa por esses pontos.
		\end{itemize}
			
	\section{Algumas definições}
	\defn{O incentro $I$ de um triângulo $ABC$ é o encontro das bissetrizes dos ângulos $\angle CAB$, $\angle ABC$ e $\angle BCA$.}
	\prob{Prove que as três bissetrizes são concorrentes.}
	\cor{$I$ é o centro do círculo contido no interior de $ABC$ tangente às retas $BC$, $CA$ e $AB$, o incírculo de $ABC$.}	
	
	\defn{O baricentro $G$ de um triângulo $ABC$ é o encontro das medianas relativas a $A$, $B$ e $C$.}
	\prob{Prove que as três medianas são concorrentes e que este ponto divide cada mediana na razão $2:1$.}
	
	\defn{O circumcentro $O$ de um triângulo $ABC$ é o encontro das mediatrizes de $BC$, $CA$ e $AB$.}
	\prob{Prove que as três mediatrizes são concorrentes.}
	\cor{$O$ é o centro do círculo que passa por $A$, $B$ e $C$, o circuncírculo de $ABC$.}
	
	\defn{O ortocentro $H$ de um triângulo $ABC$ é o encontro das alturas relativas a $A$, $B$ e $C$.}
	\prob{Prove que as três alturas são concorrentes.}

	\section{Círculo de Nove Pontos}
	\thm{Existe um círculo que passa pelo ponto médio dos lados, pelos pés das alturas e pelo ponto médio dos segmentos $AH$, $BH$ e $CH$, o círculo de nove pontos, cujo centro, $N$, é o ponto médio de $OH$. \label{thm:novepontos}}
	\defn{Este círculo é o círculo de nove pontos.}
	\prob{Prove o \cref{thm:novepontos}}.

	\section{Simedianas}
	\defn{A simediana relativa a $A$ é a reta que liga $A$ com o encontro das tangentes ao circuncírculo de $ABC$ por $B$ e por $C$.}
	\thm{A simediana relativa a $A$ é a reflexão da mediana relativa a $A$ pela bissetriz de $\angle CAB$, isto é, a simediana é a isogonal da mediana. \label{thm:simediana-isogonal-mediana}}
	\prob{Prove o \cref{thm:simediana-isogonal-mediana}.}
	\defn{O ponto simediano, ou alternativamente Ponto de Lemoine, $L$, é o encontro das simedianas relativas a $A$, $B$ e $C$.\label{defn:ponto-simediano}}
	\defn{O conjugado isogonal de um ponto $P$ com respeito ao triângulo $ABC$, $P^*$ é a interseção das reflexões das retas $PA$, $PB$ e $PC$ em relação às bissetrizes de $\angle CAB$, $\angle ABC$ e $\angle BCA$, isto é, a interseção das isogonais de $PA$, $PB$ e $PC$.\label{defn:conjugado-isogonal}}
	\proof{A existência de $P^*$ na \cref{defn:conjugado-isogonal} é uma consequência direta do Teorema de Ceva Trigonométrico.}
\cor{Usando a \cref{defn:conjugado-isogonal} e o \cref{thm:simediana-isogonal-mediana}, o ponto $L$, definido em \cref{defn:ponto-simediano}, existe.}

	\section{Incírculos}
	%definir pontos de tangência
	%definir gregonne
	%prove gregonne
	\section{Excírculos}
	\defn{Seja $I_A$ é a interseção da bissetriz de $\angle CAB$ com as bissetrizes externas de $\angle ABC$ e $\angle CBA$.}
	\prob{Prove que as bissetrizes acima se intersectam.}
	\defn{O excírculo relativo a $A$ é um círculo fora do triângulo $ABC$, tangente ao lado $BC$ e ao prolongamento dos lados $AB$ e $AC$.}
	\cor{$I_A$ é o centro do excírculo relativo a $A$.}
	
	\subsection{Nagel}
	\defn{Vamos chamar de $P_A$ o ponto de tangência do excírculo relativo a $A$ com o lado $BC$, com $P_B$ e $P_C$ definidos analogamente.}
	\thm{O ponto $P_A$ também é conhecido como ponto isoperimétrico, pois o caminho $ABD$ tem mesmo comprimento que o caminho $ACD$.\label{thm:ponto-isoperimetrico}}
	\prob{Prove o \cref{thm:ponto-isoperimetrico}.}

	%definir ponto de nagel
	%provar ponto de nagel
	
	\subsection{Círculo \texorpdfstring{$(B, C, I, I_A)$}{(B, C, I, IA)}}	

	\thm{Seja $M$ a interseção da bissetriz de $\angle A$ com o circuncírculo. Então $\conc{B}{C}{I}{I_A}$, com centro $M$. \label{thm:circulo-bciia}}
	\prob{Prove o \cref{thm:circulo-bciia}.}

	\section{Ponto de Fermat}

	\prob{Seja $P$ um ponto no plano. Como se constrói um triângulo $ABC$ equilátero tal que os $PA = 2$, $PB = 3$ e $PC = 4$?}
	
	%defina potno de fermat
	
	%lema usando o problema: ache os segmentos iguais

	%prob: prove o lema e prove que os segmentos de fermat intersectam

	%cor: AFB = 120


\end{document}
