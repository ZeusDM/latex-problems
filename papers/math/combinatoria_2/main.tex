\documentclass[10pt,a4paper]{article}
\usepackage[utf8]{inputenc}
\usepackage[brazilian]{babel}
\usepackage{lmodern}
\usepackage[left=1.5cm, right=1.5cm, top=2.5cm, bottom=2.5cm]{geometry}

\usepackage{../../../commands/problems}
\renewcommand{\mypath}{../../../}

\title{Alguns Problemas de Combinatória 2}
\author{Guilherme Zeus Moura}
\mail{zeusdanmou@gmail.com}
\titlel{Turma Olímpica}
\titler{\today}

\renewcommand\playerA[1]{Guilherme}
\renewcommand\playerB[1]{Zeus}

\begin{document}	
	\zeustitle
	\begin{prob}[Desafio PUC]
		\playerA{Rosencrantz} e \playerB{Guildenstern} gostam de jogar cara ou coroa.
		\playerA{Rosencrantz} sempre aposta em cara e \playerB{Guildenstern} sempre aposta emcoroa. 
		Eles gostam de inventar novas maneiras de jogar.
		
		A última maneira que eles inventaram usa uma moeda comum. 
		Eles combinam um número inteiro positivo $N$ e jogam a moeda várias vezes contando as ocorrências até que tenham saído exatamente $N$ caras.
		Cada coroa vale um ponto para \playerB{Guildenstern} e cadacara vale um ponto para \playerA{Rosencrantz}.
		Quando o jogo terminar, quem tiver mais pontos ganha.
		
		Por exemplo, eles combinaram $N = 5$ e obtiveram os seguintes resultados (com $H$ para cara e $T$ para coroa): $$HTHTHHH$$ e com isso o jogo acabou com um placar de $5$ a $2$.
		
		Em função de $N$, responda:
		\begin{enumerate}[label = (\alph*)]
			\item Qual é a probabilidade de que haja empate (ou seja, uma placar de $N$ a $N$)?
			\item Qual é o placar final mais provável?
				(Se houver mais de um placar final com igual probabilidade máxima indique quais são estes placares.)
			\item Qual é a probabilidade de que \playerA{Rosencrantz} ganhe (por qualquer placar)?
		\end{enumerate}
	\end{prob}
	\problem{math/brazil/mo/2005/4}
	\problem{math/brazil/mo/2002/4}
\end{document}
