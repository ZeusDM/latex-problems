\documentclass[10pt, a4paper]{article}
\usepackage[utf8]{inputenc}
\usepackage[brazilian]{babel}
\usepackage{lmodern}
\usepackage[left=2.5cm, right=2.5cm, top=2.5cm, bottom=2.5cm]{geometry}

\usepackage{../../../commands/problems}
\renewcommand{\mypath}{../../../}

\title{Alguns Problemas de Geometria}
\author{Guilherme Zeus Moura}
\mail{zeusdanmou@gmail.com}
\titlel{Turma Olímpica}
\titler{\today}

\begin{document}	
	\zeustitle
	\section{Ideias úteis}
	\begin{enumerate}[label = \textbullet]
		\item \textbf{Faça uma boa figura!} O que é uma boa figura?
		\begin{enumerate}[label = --]
			\item é feita com régua e compasso;
			\item é grande (uma folha inteira);
			\item deixa um bom espaço para marcar ângulos e traçar segmentos adicionais;
			\item não deixa pontos muito próximos um do outro;
			\item não é próxima de casos particulares notáveis (triângulo equilátero, isósceles, retângulo).
		\end{enumerate}
	\item \textbf{Não hesite em fazer várias figuras!} Muitas vezes, depois de progredir no problema, algumas partes da figura são inúteis, e devem ser descartadas.
	\item Marque vários ângulos, procure semelhanças, colineariedades e quadriláteros cíclicos.
	\item Tome cuidado ao abordar um problema por uma técnica de contas, como geometria analítica ou complexos. Tenha noção de quanto tempo irá levar para resolver o problema usando essas técnicas.

		Mesmo caso decida fazer o problema com contas, faça uma boa figura, pois fatos sintéticos podem simplificar o seu trabalho.
	\end{enumerate}
	\section{Problemas}
	\problem*{math/others/4}
	\problem*{math/bosnia/jbmo_tst/2012/1}
	\problem*{math/greece/jbmo_tst/2017/2}
	\problem*{math/imo/2008/1}
	\problem*{math/jbmo/2016/1}
	\problem*{math/imo/2012/1}
\end{document}
