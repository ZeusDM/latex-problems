\documentclass[10pt,a4paper]{article}
\usepackage[utf8]{inputenc}
\usepackage[brazil]{babel}
\usepackage{amsmath}
\usepackage{amsfonts}
\usepackage{amssymb}
\usepackage{graphicx}
\usepackage{enumitem}
\usepackage{lmodern}
\usepackage{fullpage}

\usepackage{../../commands/problems}
\renewcommand{\mypath}{../../}

\title{Questões Variadas\\{\large(não tão variadas assim)}}
\date{}
\author{Guilherme Zeus Moura -- \texttt{zeusdanmou@gmail.com}}

\showid

\begin{document}
	\maketitle
	As questões \textbf{não} estão em ordem de dificuldade. Aproveite para treinar a escrever as soluções de maneira organizada. Faça os desenhos de geometria com \textit{régua e compasso.}
	\begin{enumerate}
		\item \exercise{math/british/2017/round1/1}
		\item \exercise{math/british/2017/round1/2}
		\item \exercise{math/british/2019/round2/1}
		\item \exercise{math/british/2017/round1/3}
		\item \exercise{math/british/2017/round1/4}
		\item \exercise{math/british/2019/round2/2}
		\item \exercise{math/british/2017/round1/5}
		\item \exercise{math/british/2017/round1/6}
		\item \exercise{math/british/2019/round2/3}
		\item \exercise{math/british/2019/round2/4}
	\end{enumerate}
\end{document}