Yesterday, $n\ge 4$ people sat around a round table. Each participant remembers only who his two neighbours were, but not necessarily which one sat on his left and which one sat on his right. Today, you would like the same people to sit around the same round table so that each participant has the same two neighbours as yesterday (it is possible that yesterday’s left-hand side neighbour is today’s right-hand side neighbour). You are allowed to query some of the participants: if anyone is asked, he will answer by pointing at his two neighbours from yesterday.

a) Determine the minimal number $f(n)$ of participants you have to query in order to be certain to succeed, if later questions must not depend on the outcome of the previous questions. That is, you have to choose in advance the list of people you are going to query, before effectively asking any question.

b) Determine the minimal number $g(n)$ of participants you have to query in order to be certain to succeed, if later questions may depend on the outcome of previous questions. That is, you can wait until you get the first answer to choose whom to ask the second question, and so on.