In the land of Heptanomisma, four different coins and three different banknotes are used, and their denominations are seven different natural numbers. The denomination of the smallest banknote is greater than the sum of the denominations of the four different coins. A tourist has exactly one coin of each denomination and exactly one banknote of each denomination, but he cannot afford the book on numismatics he wishes to buy. However, the mathematically inclined shopkeeper offers to sell the book to the tourist at a price of his choosing, provided that he can pay this price in more than one way.

(The tourist can pay a price in more than one way if there are two different subsets of his coins and notes, the denominations of which both add up to this price.)

(a) Prove that the tourist can purchase the book if the denomination of each banknote is smaller than $49$.

(b) Show that the tourist may have to leave the shop empty-handed if the denomination of the largest banknote is $49$.