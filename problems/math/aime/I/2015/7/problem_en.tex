In the diagram below, $ABCD$ is a square. Point $E$ is the midpoint of $\overline{AD}$. Points $F$ and $G$ lie on $\overline{CE}$, and $H$ and $J$ lie on $\overline{AB}$ and $\overline{BC}$, respectively, so that $FGHJ$ is a square. Points $K$ and $L$ lie on $\overline{GH}$, and $M$ and $N$ lie on $\overline{AD}$ and $\overline{AB}$, respectively, so that $KLMN$ is a square. The area of $KLMN$ is $99$. Find the area of $FGHJ$.

\begin{center}
	\begin{asy}
		size(6cm);
		import olympiad;
		pair A,B,C,D,E,F,G,H,J,K,L,M,N;
		B=(0,0);
		real m=7*sqrt(55)/5;
		J=(m,0);
		C=(7*m/2,0);
		A=(0,7*m/2);
		D=(7*m/2,7*m/2);
		E=(A+D)/2;
		H=(0,2m);
		N=(0,2m+3*sqrt(55)/2);
		G=foot(H,E,C);
		F=foot(J,E,C);
		draw(A--B--C--D--cycle);
		draw(C--E);
		draw(G--H--J--F);
		pair X=foot(N,E,C);
		M=extension(N,X,A,D);
		K=foot(N,H,G);
		L=foot(M,H,G);
		draw(K--N--M--L);
		label("$A$",A,NW);
		label("$B$",B,SW);
		label("$C$",C,SE);
		label("$D$",D,NE);
		label("$E$",E,dir(90));
		label("$F$",F,NE);
		label("$G$",G,NE);
		label("$H$",H,W);
		label("$J$",J,S);
		label("$K$",K,SE);
		label("$L$",L,SE);
		label("$M$",M,dir(90));
		label("$N$",N,dir(180));
	\end{asy}
\end{center}