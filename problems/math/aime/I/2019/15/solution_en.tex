Use $PX+PY=11$ and $PX\cdot PY=15$ to compute $PX,PY=\frac{11\pm\sqrt{61}}{2}$.

Invert about $P$ with radius $1$. Let $T^*$ denote the inverse of $T$. Then $PA^*=\frac{1}{5}$, $PB^*=\frac{1}{3}$, $PX^*=\frac{2}{11+\sqrt{61}}$, $PY^*=\frac{2}{11-\sqrt{61}}$. Note that $Q^*$ is the intersection of the tangents to $\omega^*$ from $A^*$ and $B^*$. Let $Q^*X^*=a$ and $Q^*A^*=b$. Power of a Point on $Q^*$ and $\omega^*$ tells us that \[a\left(a+\frac{11}{15}\right)=b^2.\] Letting $k=\frac{2}{11+\sqrt{61}}$, Stewart's Theorem on $\triangle{Q^*A^*B^*}$ and cevian $Q^*P$ tells us that \[\frac{8}{225}+\frac{8}{15}\left(a+k\right)^2=\frac{8}{15}b^2.\] Using this allows us to solve for $a=\frac{k^2+\frac{1}{15}}{\frac{11}{15}-2k}$, so \[PQ^*=a+k=\frac{-k^2+\frac{11}{15}k+\frac{1}{15}}{\frac{11}{15}-2k}.\] Using the fact that $k^2-\frac{11}{15}k+\frac{1}{15}=0$, we have that \[PQ=\frac{1}{PQ^*}=\frac{\frac{11}{15}-2k}{-k^2+\frac{11}{15}k+\frac{1}{15}}=\frac{11-30k}{2}=\frac{\sqrt{61}}{2}\] from which the answer $\boxed{065}$ is found.