Given a positive integer $n$, it can be shown that every complex number of the form $r+si$, where $r$ and $s$ are integers, can be uniquely expressed in the base $-n+i$ using the integers $1,2,\ldots,n^2$ as digits. That is, the equation\[ r+si=a_m(-n+i)^m+a_{m-1}(-n+i)^{m-1}+\cdots +a_1(-n+i)+a_0  \]is true for a unique choice of non-negative integer $m$ and digits $a_0,a_1,\ldots,a_m$ chosen from the set $\{0,1,2,\ldots,n^2\}$, with $a_m\ne 0$. We write \[ r+si=(a_ma_{m-1}\ldots a_1a_0)_{-n+i}  \]to denote the base $-n+i$ expansion of $r+si$. There are only finitely many integers $k+0i$ that have four-digit expansions \[ k=(a_3a_2a_1a_0)_{-3+i}~~~~a_3\ne 0.  \]Find the sum of all such $k$.