We have that $\left\{\tau\left(n\right),\tau\left(n+1\right)\right\}$ is one of $\left\{1,6\right\},\left\{2,5\right\},\left\{3,4\right\}$. Note that $\tau\left(m\right)=1$ if and only if $m=1$, so $\left\{1,6\right\}$ is impossible.
\begin{itemize}
	\item $\left\{\tau\left(n\right),\tau\left(n+1\right)\right\}=\left\{2,5\right\}$. Note that $\tau\left(m\right)=2$ if and only if $m=p$ for a prime $p$ and $\tau\left(m\right)=5$ if and only if $m=q^4$ for a prime $q$. So we need solutions to $\left|p-q^4\right|=1$. So one of $p,q^4$ is even. Checking all possibilities gives the only solution to be $n=16$.
	\item $\left\{\tau\left(n\right),\tau\left(n+1\right)\right\}=\left\{3,4\right\}$. Note that $\tau\left(m\right)=3$ if and only if $m=p^2$ for a prime $p$ and $\tau\left(m\right)=4$ if and only if $m=q^3$ for a prime $q$ or $m=qr$ for primes $q,r$. So we need solutions to $\left|p^2-q^3\right|=1$ or $\left|p^2-qr\right|=1$. If $p=2$ we have no solutions, so $p^2$ is odd and thus we can say $q=2$. In the case of $p^2-q^3$, we get a solution of $n=8$. In the case of $p^2-qr$, we can compute $p^2$ for $p=3,5,7,11,13,17,19$ and check if $p^2\pm1$ is twice a prime. We get $p=3,5,11,19$ have that $p^2+1$ is twice a prime but $p^2-1$ is not twice a prime by difference of squares. So the valid $n$ are $n=8,9,25,121,361$.
\end{itemize}
Combining these, the sum of the first six $n$ is $8+9+16+25+121+361=\boxed{540}$ as desired.