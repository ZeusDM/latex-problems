Seja $\ell$ o eixo radical de $\omega_1$ e $\omega_3$. 

\begin{lem*}
	$\ell$ é $B_2D_2$.
\end{lem*}

\begin{dem}
	$B_2 \in \ell$, pois $B_2A_1 \cdot B_2A_2 = B_2A_3 \cdot B_2A_4$.

	Seja $X$ o encontro de $B_2D_2$ com $A_2A_3$.

	Usando potência do ponto $X$ em relação aos círculos $\Omega_2$ e $\Omega_3$, temos $XA_2 \cdot XB_1 = XD_2 \cdot XB_2$ e $XA_3 \cdot XB_3 = XD_2 \cdot XB_2$.

Logo, $XA_2 \cdot XB_1 = XA_3 \cdot XB_3 \implies X \in \ell$, e portanto, $\ell$ é a reta $B_2X$, ou seja, a reta $B_2D_2$.
\end{dem}
