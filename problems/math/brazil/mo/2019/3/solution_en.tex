\begin{lem*}
    $f(x)\geq x$, para qualquer $x$ real positivo
\end{lem*}

\begin{dem}
	Suponha que $f(x) < x$. Tomando $y = \frac{x - f(x)}{x} > 0$, temos $$ f(x) = f(f(x)f(y)) + x \implies f(x) > x,$$ um absurdo.

	Logo, $f(x) \ge x$.
\end{dem}

Se $f(x) \equiv x$, temos uma solução.

Suponha que existe $b$ tal que $f(b) > b$.

Seja $z = by + f(b)$, em outras palavras, tome $y = \frac{z - f(b)}{b}$. Temos, portanto, $$f(z) = f\left(f\left(b\right)f\left(\frac{z - f(b)}{b}\right)\right) + b$$.

\begin{lem*}
Existe $Z$ tal que para todo $z > Z$, vale que $$ $$
\end{lem*}

\begin{dem}
a	
\end{dem}

