Vamos chamar de $S(x_1, x_2, \dots, x_{2n})$ a soma gerada pelos números $x_1, \dots, x_{2n}$.

\begin{lem*}
	Suponha que $(x_1, x_2, \dots, x_{2n})$ é uma sequência que maximiza a soma de \playerA{Esmeralda}. Então, uma das duas sequências $(x_1, \dots, x_{i-1}, 0, x_{i+1}, \dots, x_{2n})$ ou $(x_1, \dots, x_{i-1}, 1, x_{i+1}, \dots, x_{2n})$ também gera uma soma máxima.
\end{lem*}

\begin{dem*}
	Fixando todos os elementos, menos $x_i$, podemos ver que a soma é linear em $x_i$. Logo
	\begin{equation*}
		S(x_1, x_2, \dots, x_{2n}) \text{ está entre } S(x_1, x_2, \dots, x_{i-1}, 0, x_{i+1}, \dots, x_{2n}) \text{ e } S(x_1, x_2, \dots, x_{i-1}, 1, x_{i+1}, \dots, x_{2n}),
	\end{equation*}
	que implica que uma das duas sequências gera uma soma máxima.
\end{dem*}

\begin{cor*}
	Existe uma sequênia $(x_1, \dots, x_{2n})$, tal que $x_i = 0$ ou $x_i = 1$ para todo $i$ que maximiza a soma.
\end{cor*}

\textcolor{red}{SOLUÇÃO INCOMPLETA}
