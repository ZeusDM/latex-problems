\begin{lem*}
	Existem duas retas que dividem o plano em quatro regiões, com $n$ pontos cada.
\end{lem*}

\begin{dem}
	\textcolor{red}{INCOMPLETO}
\end{dem}

Seja $X$ a interseção dessas retas. Todo quadrilátero formado for um ponto em cada região contém $X$ em seu interior. Desse modo, usando somente os vértices desse quadrilátero, conseguimos achar $2$ triângulos.

Porém, cada triângulo aparece em todos os $n$ quadriláteros formados com os 3 vértices do triângulo. Logo, o ponto $X$ está no interior de
$$ \frac{2n^4}{n} = 2n^3$$
triângulos, como queríamos demonstrar.
