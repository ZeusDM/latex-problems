É conhecido que $BU = p - c$,  $UC = p - b$ e os análogos, onde $a$, $b$, $c$ são as medidas dos lados de $ABC$ e $p = \frac{a+b+c}{2}$.
Sejam $P = r_u \cap r_v$, $I_C$ o centro do ex-incírculo relativo a $C$ e $H$ o ortocentro de $ABC$.

Por um lado, a projeção perpendicular do segmento $PI_C$ na reta $CB$ é $b$, enquanto a projeção perpendicular do segmento $PI_C$ na reta $CA$ é $a$. 
Por outro lado, a projeção perpendicular do segmento $CH$ na reta $CB$ é $b\cos C$, enquanto a projeção perpendicular do segmento $CH$ na reta $CA$ é $a\cos C$. 

Como essas proporções formam o mesmo raio, concluímos que $AB \perp CH \parallel PI_C$, e então, $PW$ é de fato perpendicular a $AB$. Portanto, as retas $r_u$, $r_v$ e $r_w$ se intersectam em $P$.
