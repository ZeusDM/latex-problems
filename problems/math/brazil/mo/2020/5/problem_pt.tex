Sejam $n$ e $k$ números inteiros positivos com $k \leq n$. Em um grupo de $n$ pessoas, cada uma ou sempre fala  a  verdade  ou  sempre  mente.   \playerA{Arnaldo}  pode  fazer  perguntas  para  quaisquer  dessas  pessoas desde que essas perguntas sejam do tipo:  “No conjunto $A$, qual a paridade de pessoas que falam averdade?”, onde $A$ é um subconjunto de tamanho $k$ do conjunto das $n$ pessoas. A resposta só pode ser ``par'' ou ``ímpar''.
\begin{enumerate}[label = (\alph*)]
	\item Para quais valores de $n$ e $k$ é possível determinar quais pessoas falam a verdade e quais pessoas sempre mentem?
	\item Qual o número mínimo de perguntas necessárias para determinar quais pessoas falam a verdade e quais pessoas sempre mentem, quando esse número é finito?
\end{enumerate}
