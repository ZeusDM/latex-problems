Sejam $r_A$, $r_B$ e $r_C$ semirretas de origem $P$. O círculo $\omega_a$, de centro $X$, é tangente a $r_B$ e $r_C$; o círculo $\omega_b$, de centro $Y$, é tangente a $r_C$ e $r_A$; e o círculo $\omega_c$, de centro $Z$, é tangente a $r_A$ e $r_B$.  Suponha que $P$ está no interior do triângulo $XYZ$, de modo que $r_A$, $r_B$ e $r_C$ sejam tangentes comuns internas aos círculos correspondentes. Sejam $s_A$ a reta tangente internamente a $\omega_b$ e $\omega_c$ que não contém $r_A$; $s_B$ a reta tangente internamente a $\omega_c$ e $\omega_a$ que não contém $r_B$; e $s_C$ a reta tangente internamente a $\omega_a$ e $\omega_b$ que não contém $r_C$. Prove que $s_A$, $s_B$ e $s_C$ têm um ponto comum $Q$, e prove que $P$ e $Q$ são conjugados isogonais no triângulo $XYZ$, ou seja, as retas $XP$ e $XQ$ são simétricas com relação à bissetriz de $\angle YXZ$ e as retas $YP$ e $YQ$ são simétricas com relação à bissetriz de $\angle XYZ$.

