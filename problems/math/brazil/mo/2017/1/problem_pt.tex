\mbox{}
\begin{itemize}
	\item[(a)] Mostre que podemos escolher dois racionais $p$ e $q$ entre $0$ e $1$ tal que, a partir das representações decimais $p = 0.p_1p_2p_3\dots$ e $q = 0.q_1q_2q_3\dots$, é possível construir um número irracional $\alpha = 0.a_1a_2a_3\dots$ tal que, para cada $i = 1, 2, 3, \dots$ temos $a_i = p_i$ ou $a_i = q_i$.

	\item[(b)] Mostre que existem um racional $s = 0.s_1s_2s_3\dots$ e um irracional $\beta = 0.b_1b_2b_3\dots$ tal que, para cada $N \geq 2017$, o número de índices $1 \leq i \leq N$ tais que $s_i \neq b_i$ é menor ou igual a $\frac{N}{2017}$.
\end{itemize}
