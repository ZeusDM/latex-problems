Sejam $L$, $M$ e $N$ os pés das bissetrizes por $A$, $B$ e $C$.

\begin{align*}
	\text{$F$, $A$, $I$ colineares} & \implies \triangle FAE \sim \triangle FIC \text{\ \ e\ \ } \triangle FAD \sim \triangle FIB \\
	& \implies \frac{AE}{CI} = \frac{FA}{FI} = \frac{AD}{BI} \\
	& \implies \frac{AE}{CI} = \frac{AD}{BI}.
\end{align*}

\begin{align*}
	BI || AD \implies \triangle ADN \sim \triangle BIN & \implies \frac{AD}{BI} = \frac{AN}{NB} \\
	CI || AE \implies \triangle AEM \sim \triangle CIM & \implies \frac{AE}{CI} = \frac{AM}{MC} \\
	         										   & \implies \frac{AN}{NB} = \frac{AM}{MC}.
\end{align*}

Mas, usando que as bissetrizes concorrem, isto é, por Ceva:

\begin{align*}
	\frac{AN}{NB} \cdot \frac{BL}{BL} \cdot \frac{CM}{MA} = 1	& \implies BL = LC \\
												& \implies \text{a bissetriz também é mediana} \\
												& \implies \text{o triângulo é isósceles} \\
												& \implies AB = AC.
\end{align*}
