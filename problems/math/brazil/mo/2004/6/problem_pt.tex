Sejam $a$ e $b$ números reais. Considere a função $f_{a,b}: \RR^2 \to \RR^2$ definida por $f_{a,b}(x;y) = (a-by-x^2;x)$. Sendo $P = (x;y) \in \RR^2$, definimos $f_{a,b}^0(P) = P$ e $f_{a,b}^{k+1}(P) = f_{a,b}(f_{a,b}^{k}(P))$, para $k$ inteiro não negativo.

O conjunto $per(a;b)$ dos \textit{pontos periódicos} da função $f_{a,b}$ é o conjunto dos pontos $P$ de $\RR^2$ para os quais existe um inteiro positivo $n$ tal que $f_{a,b}^n(P) = P$.

Fixando o real $b$, prove que o conjunto $A_b = \{a \in \RR\ |\ per(a,b) \neq \varnothing \}$ tem um menor elemento. Calcule esse menor elemento.