Dizemos que um número inteiro positivo é \textit{chapa} quando ele é formado apenas por algarismos não nulos e a soma dos quadrados de todos os seus algarismos é também um quadrado perfeito.
Por exemplo, $221$ é chapa pois $2^2 + 2^2 + 1^2 = 9$ e todos os seus algarismos são não nulos, $403$ não é chapa, pois, apesar de $4^2 + 0^2 + 3^2 = 52$, um de seus algarismos de $403$ é nulo e $12$ não é chapa pois $1^2 + 2^2 = 5$ não é quadrado perfeito. 

Prove que, para todo inteiro positivo n, existe um número chapa com exatamente n algarismos.