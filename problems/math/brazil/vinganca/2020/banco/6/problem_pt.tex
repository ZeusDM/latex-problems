Uma dupla de ladrões está tentando roubar um banco usando um helicóptero. O banco é um plano completamente coberto por moedas de ouro. Porém, mesmo que o banco exponha suas moedas de ouro, roubá-lo não é uma tarefa fácil. O banco é vigiado por $n$ guardas, cada um com um estiligue mortal que é ativado assim que o guarda vê o ladrão, imobilizando-o e parando o roubo. Em contrapartida, os ladrões escolhem a calada da noite para realizar o roubo e só será possível ver o ladrão se ele for iluminado por uma das lanternas que os guardas possuem. O feixe da lanterna ilumina a região no interior\footnote{Além da borda, i.e., na semirreta.} de ângulo com vértice no ponto em que o guarda está e ângulo $\frac{2\pi}{n}$.

Independente da posição dos guardas e da direção das lanternas, é possível que o ladrão sempre consiga pousar em um ponto do plano sem ser visto?
