\playerA{O professor Piraldo} participa de jogos de futebol em que saem muitos gols e tem uma maneira peculiar de julgar um jogo.
Um jogo com placar de $m$ gols a $n$ gols, $m \ge n$, é dito equilibrado quando $m \le f(n)$, sendo $f(n)$ definido por $f(0) = 0$ e, para $n \ge 1$, $f(n) = 2n - f(r) + r$, onde $r$ é o maior inteiro tal que $r < n$ e $f(r) \le n$.

Sendo $\phi = \frac{1+\sqrt{5}}{2}$, prove que um jogo com placar de $m$ gols a $n$, $m\ge n$, está equiliblado se $m \le \phi n$ e não está equilibrado se $m \ge \phi n + 1$