Dado um triângulo $ABC$, o exincentro relativo ao vértice A é o ponto de interseção das bissetrizes externas de $DB$ e $DC$.
Sejam $I_A$, $I_B$ e $I_C$ os exincentros do triângulo escaleno $ABC$ relativos a $A$, $B$ e $C$, respectivamente, e $X$, $Y$ e $Z$ os pontos médios de $I_BI_C$, $I_CI_A$ e $I_AI_B$, respectivamente.
O incírculo do triângulo $ABC$ toca os lados $BC$, $CA$ e $AB$ nos pontos $D$, $E$ e $F$, respectivamente. Prove que as retas $DX$, $EY$ e $FZ$ têm um ponto em
comum pertencente à reta $IO$, sendo $I$ e $O$ o incentro e o circuncentro do triângulo $ABC$, respectivamente.