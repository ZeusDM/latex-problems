A avenida Providência tem infinitos semáforos igualmente espaçados e sincronizados. A distância entre dois semáforos consecutivos é de 1.500m. Os semáforos ficam abertos por 1min 30s, depois fechados por 1 min, depois abertos por 1 min 30s e assim sucessivamente.

Suponha que um carro trafegue com velocidade constante igual a $v$, em m/s, pela avenida Providência.

Para quais valores de $v$ é possível que o carro passe por uma quantidade arbitrariamente grande de semáforos sem parar em qualquer um deles?