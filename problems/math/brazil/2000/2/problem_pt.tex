Seja $\sigma(n)$ a soma de todos os divisores positivos de $n$, onde $n$ é um inteiro positivo (por exemplo, $\sigma(6) = 12$ e $\sigma(11) = 12$).
Dizemos que $n$ é \textit{quase perfeito} se $\sigma(n) = 2n-1$ (por exemplo, $4$ é quase perfeito, pois $\sigma(4) = 7$) Sejam $n \mod{k}$ o resto da divisão de $n$ por $k$ e $s(n) = \displaystyle\sum_{k=1}^n n \mod{k}$ (por exemplo, $s(6) = 0 + 0 + 0 + 2 + 1 + 0 = 3$ e $s(11) = 0 + 1 + 2 + 3 + 1 + 5 + 4 + 3 + 2 + 1 + 0 = 22$).

Prove que $n$ é quase perfeito se, e somente se, $s(n) = s(n-1)$.