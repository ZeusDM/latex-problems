Dado um número $n$ natural, fazemos uma diamante com os números $1, 2, \dots, 2n - 1$ do seguinte modo,
na primeira linha aparece um número $1$, na segunda aparecem dois números $2$, e assim por diante até que na $n$-ésima linha aparecem $n$ números $n$, já na $(n+1)$-ésima linha aparecem $n - 1$ números $n + 1$, até que na $(2n - 1)$-ésima linha aparece um número $2n - 1$.
Abaixo temos exemplos de diamantes para $n = 1, 2, 3, 4$.

\[
\left|
\begin{tabular}{c}
	1
\end{tabular}
\right|\ 
\left|
\begin{tabular}{ccc}
	 &1&  \\
	2& &2 \\
	 &3&  
\end{tabular}
\right|\ 
\left|
\begin{tabular}{ccccc}
	 & &1& &  \\
	 &2& &2&  \\
	3& &3& &3 \\
	 &4& &4&  \\
	 & &5& &  
\end{tabular}
\right|\ 
\left|
\begin{tabular}{ccccccc}
	 & & &1& & &  \\
	 & &2& &2& &  \\
	 &3& &3& &3&  \\
	4& &4& &4& &4 \\
	 &5& &5& &5&  \\
	 & &6& &6& &  \\
	 & & &7& & &  \\
\end{tabular}
\right|
\]

Qual a soma de todos os elementos no $n$-ésimo diamante?
