Considere um triângulo equilátero $ABC$ de lado 1. Um círculo $C_1$ é construído tangenciando os lados $AB$ e $AC$.
Um círculo $C_2$, de raio maior que o raio de $C_1$, é construído tangenciando os lados AB e AC e tangenciando externamente o círculo $C_1$.
Sucessivamente, para $n$ inteiro positivo, o círculo $C_{n+1}$, de raio maior que o raio de $C_n$, tangencia os lados $AB$ e $AC$ e tangencia externamente o círculo $C_n$.
Determine os possíveis valores para o raio de $C_1$ de forma que caibam 4, mas não 5 círculos dessa sequência, inteiramente contidos no interior do triângulo $ABC$.