Sabemos que $f(1), f(3), f(5), \dots, f(2n-1)$ tem a mesma paridade e $f(2), f(4), f(6), \dots, f(2n)$ tem a mesma paridade. Vamos dividir em 4 casos:

\begin{itemize}
	\item $f(\text{ímpar}) = \text{ímpar}$ e $f(\text{par}) = \text{ímpar}$:

		Para cada um os $n$ ímpares, podemos escolher entre $1$, $3$ e $5$ para ser o seu respectivo valor de $f$. Para cada ímpar, há $3$ maneiras de escolher, portanto, para todos os ímpares, há $3^n$ maneiras de escolher.

		Da mesma maneira, há $3^n$ maneiras de escolher o $f$ dos pares.
	
		Logo, a quantidade é $3^n \cdot 3^n$.

	\item $f(\text{ímpar}) = \text{ímpar}$ e $f(\text{par}) = \text{par}$:

		Podemos escolher o $f$ dos ímpares entre $1$, $3$ e $5$ e o $f$ dos pares entre $2$ e $4$.

		Analogamente, a quantidade é $3^n \cdot 2^n$.

	\item $f(\text{ímpar}) = \text{par}$ e $f(\text{par}) = \text{ímpar}$:

		Analogamente, a quantidade é $2^n \cdot 3^n$.

	\item $f(\text{ímpar}) = \text{par}$ e $f(\text{par}) = \text{par}$:

		Analogamente, a quantidade é $2^n \cdot 2^n$.
\end{itemize}

Portanto, a quantidade total é $3^n \cdot 3^n + 3^n \cdot 2^n + 2^n \cdot 3^n + 2^n \cdot 2^n = (3^n + 2^n)^2$, que é um quadrado perfeito, como queríamos demonstrar.
