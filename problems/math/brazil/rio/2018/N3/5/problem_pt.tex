Sejam n um inteiro positivo e $\sigma = (a_1, \dots, a_n)$ uma permutação de $\{1, \dots, n\}$. O \textit{número
de cadência} de $\sigma$ é o número de blocos decrescentes maximais. Por exemplo, se $n = 6$ e
$\sigma = (4, 2, 1, 5, 6, 3)$, então o número de cadência de $\sigma$ é $3$, pois $\sigma$ possui $3$ blocos $(4, 2, 1)$, $(5)$,
$(6, 3)$ descrescentes e maximais. Note que os blocos $(4, 2)$ e $(2, 1)$ são decrescentes, mas não
são maximais, já que estão contidos no bloco $(4, 2, 1)$.

Calcule a soma das cadências de todas as permutações de $\{1, \dots, n\}$.
