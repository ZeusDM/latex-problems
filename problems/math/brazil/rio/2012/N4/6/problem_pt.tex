Arnaldo e Bernaldo jogam um jogo matemático.
Numa primeira etapa, Arnaldo começa escolhendo um número inteiro não negativo e, em seguida, após ver o número que Arnaldo escolheu, Bernaldo escolhe outro número inteiro não negativo.
O processo se repete uma vez, com Arnaldo escolhendo um novo número não negativo e, em seguida, Bernaldo escolhendo um último número não negativo.

Após esse processo, Arnaldo escolhe um subconjunto de $3$ números $a$, $b$ e $c$ dentre os $4$ números originais e monta a equação $ax^2 + bx + c = 0$.

Seja $d$ o número não escolhido por Arnaldo. Bernaldo ganha caso consiga encontrar $d$ números inteiros (possivelmente negativos) distintos $t_i$ $(i = 1, 2, ..., d)$ tais que a equação $(a + t_i)x^2 + (b + t_i)x + (c + t_i) = 0$
tenha raiz real. Caso contrário, Arnaldo ganha. Qual jogador possui estratégia vencedora?

\rem{Se $d = 0$ Bernaldo ganha. E uma equação da forma $0x^2 + 0x + 0 = 0$ tem raiz real.}

\noindent\textit{Exemplo.} Arnaldo começa escolhendo o número $1$, então Bernaldo escolhe $4$, depois Arnaldo escolhe $4$ e, por fim, Bernaldo escolhe $2$.
Então, Arnaldo monta a equação $4x^2 + x + 4 = 0$.
E Bernaldo ganha porque consegue $2$ (o número que sobrou) inteiros, $-3$ e $-4$, tais que as equações $(4 - 3)x^2 + (1 - 3)x + (4 - 3) = 0$ e $(4 - 4)x^2 + (1 - 4)x + (4 - 4) = 0$ têm raiz real.
