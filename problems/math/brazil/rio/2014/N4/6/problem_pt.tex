Seja $ABC$ um triângulo acutângulo com $AB \neq AC$. Um ponto $P$ interior ao triângulo é dito \emph{$B$-bom} se $\angle PBC = \angle PCA$, e é dito \emph{$C$-bom} se $\angle PCB = \angle PBA$.
Sejam $D$ o ponto $B$-bom mais próximo de $A$ e $E$ o ponto $C$-bom mais próximo de $A$.
Defina $F$ o ponto de interseção entre as retas $BD$ e $CE$, e $G$ o ponto de interseção, distinto de $F$, entre os circuncírculos de $BEF$ e $CDF$.

\begin{enumerate}[label = (\alph*)]
	\item Prove que a reta $FG$ é perpendicular a reta $BC$.
	\item Prove que $A$, $E$, $D$ e $G$ são concíclicos.
\end{enumerate}
