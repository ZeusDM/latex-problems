Um grafo cujo conjunto de vértices $V$ tem $n$ elementos é bacana se existir um conjunto $D \subset N$ e uma função injetiva $f: V \to [1, n^2/4] \cap N$ tal que os vértices $p$ e $q$ são ligados por uma aresta se e somente se $|f(p)-f(q)| \in D$.

Mostre que existe $n_0 \in N$ tal que para todo $n \ge n_0$ existem grafos com n vértices que não são bacanas.

\textit{Observação:} Um grafo com conjunto de vértices $V$ é um par $(V, E)$ onde $E$ é um conjunto de subconjuntos de $V$, todos com exatamente dois elementos.
Um conjunto $\{p,q\}$ é chamado de \textit{aresta} se pertencer a $E$ e neste caso dizemos que esta aresta liga os vértices $p$ e $q$.