Encontrar o maior valor de $n$ para o qual existe uma sequência $(a_0, a_1, \dots, a_n)$ de algarismos não nulos (ou seja, $a_i \in \{1,2,3,4,5,6,7,8,9\}$) tal que, para todo $k$, $1\le k \le n$, o número de $k$ dígitos $(a_{k-1}a_{k-2}\dots a_0) = a_{k-1}10^{k-1} + a_{k-2}10^{k-2} + \dots + a_0$ divide o número de $k+1$ dígitos $(a_{k}a_{k-1}\dots a_0)$.