Seja $N$ um inteiro maior do que 2.
Arnaldo e Bernaldo disputam o seguinte jogo:
há $N$ pedras em uma pilha.
Na primeira jogada, feita por Arnaldo, ele deve tirar uma quantidade $k$ de pedras da pilha com $1 \le k < N$.
Em seguida, Bernaldo deve retirar uma quantidade de pedras $m$ da pilha com $1\le m \le 2k$, e assim por diante, ou seja, cada jogador, alternadamente, tira uma quantidade de pedras da pilha entre 1 e o dobro da última quantidade que seu oponente tirou, inclusive.
Ganha o jogador que tirar a última pedra.

Para cada valor de $N$ determine qual jogador garante a vitória, independe de como o outro jogar, e explique qual é a estratégia vencedora para cada caso.