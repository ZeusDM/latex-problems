Seja $k$ um inteiro positivo fixado. \playerA{Alberto} e \playerB{Beralto} participam do seguinte jogo:
dado um número inicial $N_0$ e começando por \playerA{Alberto}, eles alternadamente fazem a seguinte operação:
trocar um número $n$ por um número $m$ tal que $m < n$ e $n$ e $m$ diferem, na sua representação em base 2, exatamente $l$ dígitos consecutivos para algum $l$ tal que $1 \le l \le k$. Quem não conseguir jogar perde.

Dizemos que um inteiro não negativo $t$ e \textit{vencedor} quando o jogador que recebe $t$ tem uma estratégia vencedora, ou seja, consegue escolher os números seguintes de modo a garantir a vitória, não importando como o outro jogador faça suas escolhas. Caso contrário, dizemos que $t$ é \textit{perdedor}.

Prove que, para todo $N$ inteiro positivo, a quantidade de inteiros não-negativos perdedores e menores do que $2^N$ é $2^{N-\floor{\log_2(\mathrm{min}\{k, N\})}}$.

\textit{Observação:} $\floor{x}$ é o maior inteiro menor ou igual a $x$. Por exemplo, $\floor{3,14} = 3, \floor{2} = 2 \e \floor{-4,6} = -5$.