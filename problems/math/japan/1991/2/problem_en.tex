Let $ N$ be the set of the whole of positive integers. The mapping from $ N$ to $ N$ is defined as follows: $ p(1)=2,\ p(2)=3,\ p(3)=4,\ p(4)=1,\ \ \ p(n)=n\ (n\geq 5)$,$ q(1)=3,\ q(2)=4,\ q(3)=2,\ q(4)=1,\ \ \ p(n)=n\ (n\geq 5)$. Answer the following questions.

(1) If you make a mapping $ f: N\rightarrow N$ sucessfully, we have $ f$ such that $ f(f(n))=p(n)+2$. Give an example.

(2) Prove that it is impossible that  $ f(f(n))=q(n)+2$ holds in regardless of any definition for $ f: N\rightarrow N$.