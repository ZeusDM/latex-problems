Primeiro, note que $u_n = 0$ ou $u_n = 1$, para todo $n \in \NN$.

Podemos provar que $u_n$ tem a mesma paridade que soma dos dígitos de $n$ em sua representação binária. Este exercício fica para o leitor, mas este fato só será usado no item (c).

\begin{enumerate}[label = (\alph*)]
	\item
		\begin{gather*}
			u_{2020} = u_{1010}
					 = u_{505}
					 = 1 - u_{252}
					 = 1 - u_{126}
					 = 1 - u_{63}
					 = u_{31}
					 = 1 - u_{15}
					 = u_{7}
					 = 1 - u_{3}
					 = u_{1}
					 = 1 - u_0\\
			u_{2020} = 1
		\end{gather*}
	\item
		Observe que existe exatamente um $0$ entre $u_{2n}$ e $u_{2n+1}$, pois $u_{2n+1} = 1 - u_{2n}$.

		Logo, olhando para os 1010 pares $(u_0, u_1), (u_2, u_3), \dots, (u_{2018}, u_{2019})$, exitem exatamente $1010$ $0$'s.

		Portanto, sabendo que $u_{2020} = 1$, temos que existem $1010$ índices $n \le 2020$ tal que $u_n = 0$.

	\item
		Se $m = 0$, $N = 0$ e $u_N = 0$. Se $m \geq 0$, vamos usar que $u_N \equiv \text{soma dos dígitos de $N$ na base $2$} \pmod{2}$.
		\[N = (2^m - 1)^2 = 2^{2m} - 2^{m+1} + 1 = \underbrace{11\cdots1}_{\text{$m-1$}}\underbrace{000\cdots0}_{\text{$m$}}1\quad\text{na base $2$.}\]
		Logo, \[u_N \equiv \text{soma dos dígitos de $N$ na base $2$} = m \pmod{2}.\]
		Ou seja, \[ u_N = 
			\begin{cases}
				1, \text{se $m$ é ímpar}\\
				0, \text{se $m$ é par}
			\end{cases}.\]
\end{enumerate}
