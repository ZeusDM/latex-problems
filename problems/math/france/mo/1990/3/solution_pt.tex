\begin{enumerate}[label = (\alph*)]
	\item
		Sem perda de generalidade, $a \le b \le c$.

		Se $a \le 2$, então 
		\begin{align*}
			\frac{1}{4} &= \frac{1}{a^2} + \frac{1}{b^2} + \frac{1}{c^2}\\
			\frac{1}{4} &\ge \frac{1}{4} + \frac{1}{b^2} + \frac{1}{c^2}\\
			0			&\ge \frac{1}{b^2} + \frac{1}{c^2}.
		\end{align*}
		\hfill \textit{Absurdo.}

		Se $a \ge 4$, então
		\begin{align*}
			\frac{1}{4} &= \frac{1}{a^2} + \frac{1}{b^2} + \frac{1}{c^2}\\
						&\le 3\cdot\frac{1}{16}
		\end{align*}
		\hfill \textit{Absurdo.}

		Logo, $a = 3$.

		Se $b \ge 4$, então
		\begin{align*}
			\frac{1}{4} &= \frac{1}{9} + \frac{1}{b^2} + \frac{1}{c^2}\\
						&\le \frac{1}{9} + \frac{1}{16} + \frac{1}{16} = \frac{1}{9} + \frac{1}{8} < \frac{1}{4}
		\end{align*}
		\hfill \textit{Absurdo.}

		Logo, $b = 3$.
		\begin{align*}
			\frac{1}{4} &= \frac{1}{9} + \frac{1}{9} + \frac{1}{c^2}\\
			\frac{1}{36}&= \frac{1}{c^2}
		\end{align*}

		Logo, $c = 3$. Todas as soluções para $(a, b, c)$ são $(3, 3, 6)$, $(3, 6, 3)$ e $(6, 3, 3)$.

	\item Observe que
		\[\frac{1}{x^2} = \frac{1}{(2x)^2} + \frac{1}{(2x)^2} + \frac{1}{(2x)^2} + \frac{1}{(2x)^2}.\] 

		Logo, se $n$ satisfaz a condição do enunciado, $n+3$ também satisfaz.

		Como $1$, $6$ e $8$ satisfazem a condição do enunciado,
		\begin{align*}
			1 &= \frac{1}{1^2}\\
			1 &= \frac{1}{2^2} + \frac{1}{2^2} + \frac{1}{2^2} + \frac{1}{3^2} + \frac{1}{3^2} + \frac{1}{6^2}\\
			1 &= \frac{1}{2^2} + \frac{1}{2^2} + \frac{1}{3^2} + \frac{1}{3^2} + \frac{1}{3^2} + \frac{1}{3^2} + \frac{1}{6^2} + \frac{1}{6^2}
		\end{align*}
		automaticamente, qualquer número da forma $3k+1$, $3k+6$ e $3k+8$ também satisfazem o enunciado.\

		Sobram os números $2$, $3$ e $5$.

		\begin{enumerate}[label = \bfseries \Alph*.]
			\item \textbf{Se $n = 2$ ou $n = 3$,}
				
				Se $x_i = 1$ para algum $i$, então $LD > 1$. \hfill \textit{Absurdo!}

				Logo, $x_i \ge 2$ para todo $i$.
				
				Mas, se $x_i \ge 2$ para todo $i$, então $LD \le \frac{n}{4} < 1$. \hfill \textit{Absurdo!}

			\item \textbf{Se $n = 5$.}

				Se $x_i = 1$ para algum $i$, então $LD > 1$. \hfill \textit{Absurdo!}

				Logo, $x_i \ge 2$ para todo $i$.

				Mas, se $x_i = 2$ para mais que $3$ valores de $i$, então $LD > 1$. \hfill \textit{Absurdo!}

				Logo, há, no máximo, $3$ valores de $i$ tal que $x_i = 2$ e, para os outros índices $i$, $x_i \ge 3$.
				
				Portanto, $LD \le 3\frac{1}{2^2} + 2\frac{1}{3^2} < 1$. \hfill \textit{Absurdo!}
		\end{enumerate}

		Deste modo, todos os números diferentes de $2$, $3$ ou $5$ funcionam.
	
\end{enumerate}
