Primeiro, uma desigualdade\footnote{Para provar essa igualdade, basta usar que $(x-y)^2 \geq 0$.} que usaremos é
\begin{equation*}
	x^2 + y^2 \geq \frac{1}{2}(x+y)^2,
\end{equation*}
com igualdade se, e somente se, $x=y$.

\begin{align*}
	BL^2 + CM^2 + AN^2 	&= (BP^2 - PL^2) + (CP^2 - PM^2) + (AP^2 - PN^2)\\
						&= (BP^2 - PN^2) + (CP^2 - PL^2) + (AP^2 - PM^2)\\
						&= BN^2 + AM^2 + CL^2\\
						&= \frac{1}{2}\left((BL^2 + LC^2) + (CM^2 + MA^2) + (AN^2 + NB^2)\right)\\
						&\geq \frac{1}{4}\left((BL + LC)^2 + (CM + MA)^2 + (AN + NB)^2\right)\\
						&\geq \frac{1}{4}\left(BC^2 + CA^2 + AB^2\right)
\end{align*}

A igualdade ocorre se, e somente se, $BL = LC$, $CM = MA$ e $AN = NB$, ou seja, quando $P$ é o circumcentro de $ABC$.
