Todo número $k$ pode ser escrito unicamente como $m\cdot2^\alpha$, com $m$ ímpar. Como todo número $k$ só ``interage'' com os números $2k$ e $k/2$, podemos dividir em grupos em relação ao $m$. Isto é, dividir nos seguintes grupos:

\begin{gather*}
	G_1 = \{1, 2,  4,  8, \dots\}\\
	G_3 = \{3, 6,  12, 24, \dots\}\\
	G_5 = \{5, 10, 20, 40, \dots\}\\
	G_7 =\{7, 14, 28, 56, \dots\}\\
	\vdots\\
	G_m =\{m, 2m, 4m, 8m, \dots\}
\end{gather*}

Se o grupo $G_m = \{m, 2m, 4m, 8m, \dots, 2^{\alpha-1}m\}$ tem tamanho $\alpha$, a quantidade máxima de números desse grupo em $A$ é $\floor{\frac{\alpha+1}{2}}$.

Esse máximo pode ser obtido colocando $m, 4m, 16m, \dots$ em $A$, para todo $m$. Desse modo, $k \in A \iff k = m\cdot2^\alpha$, com $\alpha$ par.

Podemos calcular quantos números até $2^p$ tem ``$\alpha$'' par\footnote{Para os que conhecem a notação, queremos a quantidade de números $k$ tal que $\nu_2(k)$ é par}.

\begin{enumerate}[label = (\alph*)]
	\item Se $p$ é par:
		\begin{align*}
			\max|A| &= \underbrace{2^{p-1}}_{\alpha = 0} + \underbrace{2^{p-3}}_{\alpha = 2} + \cdots + \underbrace{2^{p-1-2i}}_{\alpha = 2i} + \cdots + \underbrace{2^1}_{\alpha = p - 2} + \underbrace{1}_{\alpha=p}\\
					&= \frac{2^{p+1}+1}{3}
		\end{align*}
	\item Se $p$ é ímpar:
		\begin{align*}
			\max|A| &= \underbrace{2^{p-1}}_{\alpha = 0} + \underbrace{2^{p-3}}_{\alpha = 2} + \cdots + \underbrace{2^{p-1-2i}}_{\alpha = 2i} + \cdots + \underbrace{2^0}_{\alpha = p - 1}\\
					&= \frac{2^{p+1}-1}{3}
		\end{align*}	
\end{enumerate}

Para simplificar, \[\max |A| = \frac{2^{p+1} + (-1)^p}{3}.\]
