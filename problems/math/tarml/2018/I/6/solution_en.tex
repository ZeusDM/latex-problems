Let the exterior angle bisectors at $B,C$ meet at $I_A$. Then $\angle{CBI_A}=90^\circ-\frac{B}{2}$ and $\angle{BCI_A}=90^\circ-\frac{C}{2}$, so $\angle{BI_AC}=90^\circ-\frac{A}{2}$. Let the $20^\circ$ line (call it $\ell$ for short) meet the exterior angle bisectors at $B,C$ at $X,Y$, then $\angle{XAB}=20^\circ$ and $\angle{XBA}=90^\circ-\frac{B}{2}$, so $\angle{AXB}=70^\circ+\frac{B}{2}$. Similarly, $\angle{YAC}=160^\circ-A$ and $\angle{YCA}=90^\circ-\frac{C}{2}$, so $\angle{AYC}=\frac{A}{2}-\frac{B}{2}+20^\circ$. Thus, there are $6$ cases to consider: when $\left(A,B\right)$ is one of $\left(90^\circ-\frac{A}{2},\frac{A}{2}-\frac{B}{2}+20^\circ\right)$, $\left(90^\circ-\frac{A}{2},70^\circ+\frac{B}{2}\right)$, $\left(\frac{A}{2}-\frac{B}{2}+20^\circ,90^\circ-\frac{A}{2}\right)$, $\left(\frac{A}{2}-\frac{B}{2}+20^\circ,70^\circ+\frac{B}{2}\right)$, $\left(70^\circ+\frac{B}{2},90^\circ-\frac{A}{2}\right)$, and $\left(70^\circ+\frac{B}{2},\frac{A}{2}-\frac{B}{2}+20^\circ\right)$. The first equality gives $\left(A,B,C\right)=\left(60^\circ,\frac{100}{3}^\circ,\frac{260}{3}^\circ\right)$. This is a valid triangle. The second equality gives $C=-20^\circ$, which is invalid. The third and fourth equalities give $A=-100^\circ$, which is invalid. The fifth and sixth equalities give $\left(A,B,C\right)=\left(92,44,44\right)$, which is valid. So the possible values of $\angle{BAC}$ are $\boxed{60,92}$.