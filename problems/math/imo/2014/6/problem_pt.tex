Um conjunto de retas está em posição geral se não há duas retas paralelas, nem três retas passando por um mesmo ponto.
Um conjunto de retas em posição geral corta o plano em regiões, algumas das quais possuem área finita; chamaremos essas de regiões finitas.

Prove que, para todo $n$ suficientemente grande, em qualquer conjunto de $n$ retas em posição geral é possível pintar de azul pelo menos $\sqrt{n}$ retas de modo que nenhuma região finita possui fronteira completamente azul.

\rem{Resultados obtidos trocando $\sqrt{n}$ por $c\sqrt{n}$ receberão pontos dependendo do valor da constante $c$.}
