Dado um triângulo $ABC$, o ponto $J$ é o centro da circunferência ex-inscrita oposta ao vértice $A$.
Esta circunferência ex-inscrita\footnote{A circunferência ex-inscrita de $ABC$ oposta ao vértice $A$ é a circunferência tangente ao segmento $BC$, ao prolongamento do segmento $AB$ no sentido de $A$ para $B$ e ao prolongamento do segmento $AC$ no sentido de $A$ para $C$.} é tangente ao lado $BC$ em $M$, e às retas $AB$ e $AC$ em $K$ e $L$, respectivamente.
As retas $LM$ e $BJ$ intersectam-se em $F$, e as retas $KM$ e $CJ$ intersectam-se em $G$.
Seja $S$ o ponto de interseção das retas $AF$ e $BC$, e seja $T$ o ponto de interseção das retas $AG$ e $BC$.

Prove que $M$ é o ponto médio de $ST$.
