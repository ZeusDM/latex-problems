The Bank of Bath issues coins with an $H$ on one side and a $T$ on the other. Harry has $n$ of these coins arranged in a line from left to right. He repeatedly performs the following operation: if there are exactly $k > 0$ coins showing $H$, then he turns over the $k$th coin from the left; otherwise, all coins show $T$ and he stops. For example, if $n=3$ the process starting with the configuration $THT$ would be $THT \to HHT  \to HTT \to TTT$, which stops after three operations.

\begin{enumerate}[label = (\alph*)]
	\item Show that, for each initial configuration, Harry stops after a finite number of operations.
	\item For each initial configuration $C$, let $L(C)$ be the number of operations before Harry stops. For example, $L(THT) = 3$ and $L(TTT) = 0$. Determine the average value of $L(C)$ over all $2^n$ possible initial configurations $C$.
\end{enumerate}
