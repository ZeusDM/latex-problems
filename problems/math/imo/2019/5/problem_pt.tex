O Banco de Bath emite moedas com um $H$ num lado e um $T$ no outro.
Harry possui $n$ dessas moedas colocadas em linha, ordenadas da esquerda para a direita.
Ele repetidamente realiza a seguinte operação: se há exatamente $k > 0$ moedas mostrando $H$, então ele vira a $k$-ésima moeda contada da esquerda para a direita; caso contrário, todas as moedas mostram $T$ e ele para.
Por exemplo, se $n = 3$ o processo começando com a configuração $THT$ é $THT \to HHT \to HTT \to TTT$, que acaba depois de três operações.
(a) Mostre que, para qualquer configuração inicial, Harry para após um número finito de operações.
(b) Para cada configuração inicial $C$, seja $L(C)$ o número de operações antes de Harry parar. Por
exemplo, $L(THT) = 3$ e $L(TTT) = 0$. Determine a média de $L(C)$ sobre todas as $2^n$ possíveis
configurações iniciais $C$.