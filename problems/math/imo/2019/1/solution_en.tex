The answers are $f(x)=0$ or $2x+c$ for any $c\in\mathbb{Z}$. The former clearly works, while the latter works because \[2(2a)+c+2(2b+c)=2(2(a+b)+c)+c\] holds true for any $a,b,c\in\mathbb{Z}$.

If $a+b$ is fixed, then $f(2a)+2f(b)=f(f(a+b))$ is fixed too. So with $(a,b)=(0,x+1)$ and $(1,x)$ we get that \[f(0)+2f(x+1)=f(2)+2f(x)\] and thus \[f(x+1)=f(x)+\frac{f(2)-f(0)}{2}\] for all $x\in\mathbb{Z}$. It follows that $f$ is linear; say that it is of the form $mx+c$ for some constants $m$ and $c$. Since \[f(0)+2f(x)=f(f(x)),\] we deduce that \[2mx+3c=m^2x+(m+1)c\] and thus $m^2=2m$ and $(m+1)c=3c$. If $m=0$ then $c=0$, then $f\equiv0$. Otherwise $m=2$ and $c=f(0)\in\mathbb{Z}$.
