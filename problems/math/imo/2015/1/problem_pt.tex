Dizemos que um conjunto finito $\mathcal{S}$ de pontos no plano é \textit{balanceado} se, para quaisquer dois pontos distintos $a$ e $B$ em $\mathcal{S}$, existe um ponto $C$ em $\mathcal{S}$ tal que $AC = BC$. Dizemos que $\mathcal{S}$ é \textit{livre de centro} se, para quaisquer três pontos distintos $A$, $B$ e $C$ em $\mathcal{S}$, não existe ponto $P$ em $\mathcal{S}$ tal que $PA=PB=PC$.

\begin{enumerate}[label = (\alph*)]
	\item Mostre que, para todos os inteiros $n\ge 3$, existe um conjunto balanceado com exatamente $n$ pontos.
	\item Determine todos os inteiros $n\ge 3$ para os quais existe um conjunto balanceado livre de centro com exatamente $n$ pontos.
\end{enumerate}
