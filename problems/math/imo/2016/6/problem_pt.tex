Há $n\ge 2$ segmentos no plano tais que cada par de segmentos se intersecta num ponto interior a ambos e não há três segmentos que tenham um ponto em comum.
\PlayerA{Geoff} deve escolher um dos extremos de cada segmento e colocar sobre ele um sapo, virado para o outro extremo.
Depois \playerA{Geoff} baterá palmas $n - 1$ vezes. 
Cada vez que \playerA{Geoff} bater as mãos, cada sapo saltará imediatamente para a frente até o próximo ponto de intersecção sobre o seu segmento. Os sapos nunca mudam a direção dos seus saltos.
\PlayerA{Geoff} deseja colocar os sapos de tal forma que dois sapos nunca ocupem ao mesmo tempo o mesmo ponto de intersecção.
\begin{enumerate}[label = (\alph*)]
	\item Prove que se $n$ é ímpar, \PlayerA{Geoff} sempre tem uma maneira de realizar o seu desejo.
	\item Prove que se $n$ é par, \PlayerA{Geoff} nunca realiza o seu desejo.
\end{enumerate}
