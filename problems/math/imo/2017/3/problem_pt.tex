Um coelho invisível e um caçador jogam da seguinte forma no plano euclidiano. O ponto de partida $A_0$ do coelho e o ponto de partida $B_0$ são iguais. Depois de $n-1$ rodadas do jogo, o coelho encontra-se no ponto $A_{n-1}$ e o caçador encontra-se no potno $B_{n-1}$. Na $n$-ésima jogada do jogo, ocorrem três coisas na seguinte ordem: 
\begin{itemize}
	\item O coelho move-se de forma invisível para um ponto $A_n$ tal que a distância entre $A_{n-1}$ e $A_n$ é exatamente $1$.
	\item Um aparelho de localização informa um ponto $P_n$ ao caçador. A única informação garantida pelo aparelho ao caçador é que a distância entre $P_n$ e $A_n$ é menor ou igual a 1.
	\item O caçador move-se de forma visível para um ponto $B_n$ tal que a distância entre $B_{n-1}$ e $B_n$ é exatamente $1$.
\end{itemize}

É sempre possível que, qualquer que seja a maneira em que se mova o coelho e quaisquer que sejam os pontos informados pelo aparelho de localização, o caçador possa escolher os seus movimentos de modo que depois de $10^9$ rodadas o caçador possa garantir que a distância entre ele e o coelho seja menor ou igual que $100$?
