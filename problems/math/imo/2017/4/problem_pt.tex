Sejam $R$ e $S$ pontos distintos sobre a circunferência $\Omega$ tal que $RS$ não é um diâmetro. Seja $\ell$ a reta tangente a $\Omega$ em $R$. O ponto $T$ é tal que $S$ é o ponto médio do segmento $RT$. O ponto $J$ escolhe-se no menor arco $RS$ de $\Omega$ de maneira que $\Gamma$, a circunferência circunscrita ao triângulo $JST$, intersecta $\ell$ em dois pontos distintos. Seja $A$ o ponto comum de $\Gamma$ e $\ell$ mais próximo de $R$. A reta $AJ$ intersecta pela segunda vez $\Omega$ em $K$. Demonstre que a reta $KT$ é tangente a $\Gamma$.
