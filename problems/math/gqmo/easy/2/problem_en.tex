The Bank of Zürich issues coins with an $H$ on one side and a $T$ on the other side. Alice has $n$ of these coins arranged in a line from left to right. She repeatedly performs the following operation: if some coin is showing its $H$ side, Alice chooses a group of consecutive coins (this group must contain at least one coin) and flips all of them; otherwise, all coins show $T$ and Alice stops. For instance, if $n = 3$,  Alice may perform the following operations: $THT \to HTH \to HHH \to TTH \to TTT$. She might also choose to perform the operation $THT \to TTT$.

For each initial configuration $C$,  let $m(C)$ be the minimal number of operations that Alice must perform. For example, $m(THT) = 1$ and $m(TTT) = 0$. For every integer $n \geq 1$,  determine the largest value of $m(C)$ over all $2^n$ possible initial configurations $C$.Massimiliano Foschi, Italy