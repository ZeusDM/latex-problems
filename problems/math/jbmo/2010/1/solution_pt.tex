Somando as equações, temos \[abc+bcd+cda+dab=a+b+c+d\] e, portanto, $a$, $b$, $c$ e $d$ são as raízes de um polinômio $P(x)=x^4-Mx^3+Nx^2-Mx+K$, em que $M$, $N$ e $K$ são constantes.

Suponha que $M=0$. Então, $P(x)=x^4+Nx^2+K$. Portanto, $\{a,b,c,d\}=\{p,-p,q,-q\}$ para números reais $p,q$.

Desse modo, \[ \frac{p^2q^2}{p}-p = \alpha \text{ e } \dfrac{p^2q^2}{-p} + p = \beta,\] em que $\alpha, \beta$ são elementos distintos de $\{1,2,3,-6\}$.

Porém, a ultima condição implica que $\alpha = -\beta$, que não pode ser satisfeito. Portanto, $M = a+b+c+d \neq 0$.
