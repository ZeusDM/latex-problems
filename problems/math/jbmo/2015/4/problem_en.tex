An L-shape is one of the following four pieces, each consisting of three unit squares:[asy]

size(300);

defaultpen(linewidth(0.8));

path P=(1,2)--(0,2)--origin--(1,0)--(1,2)--(2,2)--(2,1)--(0,1);

draw(P);

draw(shift((2.7,0))*rotate(90,(1,1))*P);

draw(shift((5.4,0))*rotate(180,(1,1))*P);

draw(shift((8.1,0))*rotate(270,(1,1))*P);

[/asy]

A $5\times 5$ board, consisting of $25$ unit squares, a positive integer $k\leq 25$

 and an unlimited supply of L-shapes are given. Two players A and B, 

play the following game: starting with A they play alternatively mark a 

previously unmarked unit square until they marked a total of $k$ unit squares.

We say that a placement of L-shapes on unmarked unit squares is called $\textit{good}$ if the L-shapes do not overlap and each of them covers exactly three unmarked unit squares of the board.

B wins if every $\textit{good}$ placement of L-shapes leaves uncovered at least three unmarked unit squares.  Determine the minimum value of $k$ for which B has a winning strategy.