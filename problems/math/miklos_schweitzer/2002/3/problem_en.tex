Put $\mathbb{A}=\{ \mathrm{yes}, \mathrm{no} \}$. A function $f\colon \mathbb{A}^n\rightarrow \mathbb{A}$ is called a \textit{decision function} if
(a) the value of the function changes if we change all of its arguments; and
(b) the values does not change if we replace any of the arguments by the function value.
A function $d\colon \mathbb{A}^n \rightarrow \mathbb{A}$ is called a \textit{dictatoric function}, if there is an index $i$ such that the value of the function equals its $i$th argument.
The \textit{democratic function} is the function $m\colon \mathbb{A}^3 \rightarrow \mathbb{A}$ that outputs the majority of its arguments.
Prove that any decision function is a composition of dictatoric and democratic functions.