We proceed by the method of complex coordinates.

Let $A=a$, $B=b$, and $C=c$ be on the unit circle. Then $H=a+b+c$ and $M=\frac{b+c}{2}$. If $N$ is the midpoint of $AH$ (center of the circle with diameter $AH$), then $N=a+\frac{b+c}{2}$. Thus, the circle with diameter $AH$ is \[\left|z-\left(a+\frac{b+c}{2}\right)\right|^2=\left|\frac{b+c}{2}\right|^2,\] which reduces to \[z\overline{z}-d\overline{z}-\overline{d}z=\frac{\left(b+c\right)^2}{4bc}-d\overline{d},\] where $d=a+\frac{b+c}{2}$.

Now, consider a line through $M$; it is in the form \[kz+\overline{z}=\frac{kb}{2}+\frac{kc}{2}+\frac{1}{2b}+\frac{1}{2c}\] for some complex number $k$ with $\left|k\right|=1$. Plugging in $\overline{z}$ to the equation for the circle above gives that \[-kz^2+\left(\frac{kb}{2}+\frac{kc}{2}+\frac{1}{2b}+\frac{1}{2c}+dk-\overline{d}\right)z+\mathcal{E}=0,\] where $\mathcal{E}$ is some complex number. Thus, Viete's Formula gives us that \begin{align*}p+q&=\frac{b}{2}+\frac{c}{2}+\frac{1}{2bk}+\frac{1}{2ck}+d-\frac{\overline{d}}{k}\\&=\frac{b}{2}+\frac{c}{2}+\frac{1}{2bk}+\frac{1}{2ck}+a+\frac{b}{2}+\frac{c}{2}-\frac{1}{ak}-\frac{1}{2bk}-\frac{1}{2ck}\\&=a+b+c-\frac{1}{ak},\end{align*} where $P=p$ and $Q=q$. But note that the circumcenter of $\triangle{APQ}$ is $N=a+\frac{b+c}{2}$ and the centroid of $\triangle{APQ}$ is $\frac{a+p+q}{3}$, so the orthocenter of $\triangle{APQ}$ is \[p+q-\left(a+b+c\right)=-\frac{1}{ak}.\] But $\left|-\frac{1}{ak}\right|=1$, so this lies on the unit circle and thus the orthocenter of $\triangle{APQ}$ lies on the circumcircle of $\triangle{ABC}$.