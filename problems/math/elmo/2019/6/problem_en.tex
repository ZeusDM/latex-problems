Carl chooses a \textit{functional expression}* $E$ which is a finite nonempty string formed from a set $x_1, x_2, \dots$ of variables and applications of a function $f$, together with addition, subtraction, multiplication (but not division), and fixed real constants. He then considers the equation $E = 0$, and lets $S$ denote the set of functions $f \colon \mathbb R \to \mathbb R$ such that the equation holds for any choices of real numbers $x_1, x_2, \dots$. (For example, if Carl chooses the functional equation

$$ f(2f(x_1)+x_2) - 2f(x_1)-x_2 = 0, $$
then $S$ consists of one function, the identity function.

(a) Let $X$ denote the set of functions with domain $\mathbb R$ and image exactly $\mathbb Z$. Show that Carl can choose his functional equation such that $S$ is nonempty but $S \subseteq X$.

(b) Can Carl choose his functional equation such that $|S|=1$ and $S \subseteq X$?

*These can be defined formally in the following way: the set of functional expressions is the minimal one (by inclusion) such that (i) any fixed real constant is a functional expression, (ii) for any positive integer $i$, the variable $x_i$ is a functional expression, and (iii) if $V$ and $W$ are functional expressions, then so are $f(V)$, $V+W$, $V-W$, and $V \cdot W$.

