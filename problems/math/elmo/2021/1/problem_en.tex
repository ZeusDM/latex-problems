Let $Z$ be the intersection between $BY$ and $CX$.
Let $M$ be the midpoint of $BC$.

Note that, by angle chasing,
\begin{gather*}
	\angle BAD = \angle PAD = \angle PDB = \angle PYB;\\
	\angle DAC = \angle DAQ = \angle BDY = \angle BPY.
\end{gather*}

Futhermore, by looking into the angles of the triangles $\triangle ABC$ and $\triangle BPY$, 
\begin{gather*}
	\angle CBA + \angle BAD + \angle DAC + \angle ACB = \pi; \\ 
	\angle CBA + \angle PYB + \angle BYP + \angle YBC = \pi,
\end{gather*}
so $\angle ABC = \angle YBC \implies AC \parallel BY$.
Analogously, $AB \parallel CX$.
Therefore, $ABZC$ is a parallelogram; the midpoint of $AZ$ is the same of $BC$, which is $M$.

Note that, by the construction of $E$, $M$ is also the midpoint of $DE$.

By a $\pi$-rotation of center $M$, we conclude that $\angle BAD = \angle CZE = \angle XZE$.
By angle chasing, $\angle BAD = \angle PAD = \angle PDB = \angle XDE$. Thus, $\angle XZE = \angle XDE \implies X$ lies on the circumcircle of $ZDE$. Analogously, $Y$ lies on the circumcircle of $ZDE$. Therefore, $X$, $Y$, $D$ and $E$ lie on the circumcircle of $ZDE$. 
