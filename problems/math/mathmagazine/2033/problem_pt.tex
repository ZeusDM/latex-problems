Um baralho é uma coleção de 52 pares (\emph{cartas}) da forma $(n, s)$ onde $1 \le n \le 13$ é o número da carta, e o naipe $s$ da carta é um dos símbolos: ouro, copas, paus e espada.

Dada uma partição qualquer do baralho em $13$ conjuntos $S_1, S_2, \dots, S_{13}$ de $4$ cartas cada, prove que existe uma partição correspondente $C_1, C_2, C_3, C_4$ do baralho em $4$ conjuntos de $13$ cartas cada, tal que, para cada parte $C_i\ (1 \le i \le 4)$ vale:

\begin{itemize}
	\item $C_i$ tem uma carta de $S_j$ para $1 \le j \le 13$;
	\item todas as cartas em $C_i$ tem números diferentes.
\end{itemize}
