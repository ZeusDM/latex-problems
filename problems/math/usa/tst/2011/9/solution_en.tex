The answer is yes.

Let $N$ be a large positive integer and consider the $2011^2$-element subsets of $\{1,\ldots,N\}$. For such a subset $S$, consider the quantity $\displaystyle\sum_{s\in S}\binom{s}{i}$. Observe that \[\sum_{s\in S}\binom{s}{i}\leq\sum_{s\in S}\binom{N}{i}=2011^2\binom{N}{i}\] and this quantity is non-negative, so there are at most $2011^2\binom{N}{i}+1=O(N^i)$ possible values of this quantity. Thus there are at most $O(N^0)\cdot O(N^1)\cdot O(N^2)\cdots O(N^{2011})=O(N^{2011\cdot1006})$ possible values of $\displaystyle\sum_{s\in S}\binom{s}{i}$ over all $i=0,1,\ldots,2011$. But there are $\binom{N}{2011^2}=O(N^{2011^2})$ possible values of $S$, so for large enough $N$, there exist two such subsets $S$ and $T$ such that $\displaystyle\sum_{s\in S}\binom{s}{i}=\displaystyle\sum_{t\in T}\binom{t}{i}$ for all $i=0,1,\ldots,2011$.

Now let $M$ be a large positive integer. Consider the sets $A=S+M$ and $B=T+M$. I claim that this works for large enough $M$. Let \[P(x)=\sum_{s\in S}x^s-\sum_{t\in T}x^t.\] Taking the $i$th derivative, we have that \[P^{(i)}(1)=\sum_{s\in S}i!\binom{s}{i}-\sum_{t\in T}i!\binom{t}{i}=0\] for $i=0,1,\ldots,2011$, so $(1-x)^{2012}$ divides $P$. Let $P(x)=Q(x)(1-x)^{2012}$ for a polynomial $Q$. We need \[(1-x)^{2011}>\left|\sum_{a\in A}x^a-\sum_{b\in B}x^b\right|=\left|x^MP(x)\right|=\left|x^M(1-x)^{2012}Q(x)\right|\] so \[\left|Q(x)x^M(1-x)\right|<1\] for all $0<x<1$. But choosing $M$ large enough allows us to do this, so we are done.