Let $\alpha \geq 1$ be a real number. Hephaestus and Poseidon play a turn-based game on an infinite grid of unit squares. Before the game starts, Poseidon chooses a finite number of cells to be flooded. Hephaestus is building a levee, which is a subset of unit edges of the grid (called walls) forming a connected, non-self-intersecting path or loop. More formally, there must exist lattice points $\mbox{\footnotesize \(A_0, A_1, \dotsc, A_k\)}$,  pairwise distinct except possibly $\mbox{\footnotesize \(A_0 = A_k\)}$,  such that the set of walls is exactly $\mbox{\footnotesize \(\{A_0A_1, A_1A_2, \dotsc , A_{k-1}A_k\}\)}$. Once a wall is built it cannot be destroyed; in particular, if the levee is a closed loop (i.e. $\mbox{\footnotesize \(A_0 = A_k\)}$) then Hephaestus cannot add more walls. Since each wall has length $\mbox{\footnotesize \(1\)}$,  the length of the levee is $\mbox{\footnotesize \(k\)}$.

The game then begins with Hephaestus moving first. On each of Hephaestus’s turns, he adds one or more walls to the levee, as long as the total length of the levee is at most $\alpha n$ after his $n$th turn. On each of Poseidon’s turns, every cell which is adjacent to an already flooded cell and with no wall between them becomes flooded as well. Hephaestus wins if the levee forms a closed loop such that all flooded cells are contained in the interior of the loop — hence stopping the flood and saving the world. For which $\alpha$ can Hephaestus guarantee victory in a finite number of turns no matter how Poseidon chooses the initial cells to flood?
