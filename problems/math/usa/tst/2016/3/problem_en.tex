Let $p$ be a prime number.  Let $\mathbb F_p$ denote the integers modulo $p$, and let $\mathbb F_p[x]$ be the set of polynomials with coefficients in $\mathbb F_p$.  Define $\Psi : \mathbb F_p[x] \to \mathbb F_p[x]$ by \[ \Psi\left( \sum_{i=0}^n a_i x^i \right) = \sum_{i=0}^n a_i x^{p^i}. \] Prove that for nonzero polynomials $F,G \in \mathbb F_p[x]$, \[ \Psi(\gcd(F,G)) = \gcd(\Psi(F), \Psi(G)). \] Here, a polynomial $Q$ divides $P$ if there exists $R \in \mathbb F_p[x]$ such that $P(x) - Q(x) R(x)$ is the polynomial with all coefficients $0$ (with all addition and multiplication in the coefficients taken modulo $p$), and the gcd of two polynomials is the highest degree polynomial with leading coefficient $1$ which divides both of them.  A non-zero polynomial is a polynomial with not all coefficients $0$.  As an example of multiplication, $(x+1)(x+2)(x+3) = x^3+x^2+x+1$ in $\mathbb F_5[x]$.