A \emph{snake of length $k$} is an animal which occupies an ordered $k$-tuple $(s_1, \dots, s_k)$ of cells in a $n \times n$ grid of square unit cells. These cells must be pairwise distinct, and $s_i$ and $s_{i+1}$ must share a side for $i = 1, \dots, k-1$. If the snake is currently occupying $(s_1, \dots, s_k)$ and $s$ is an unoccupied cell sharing a side with $s_1$, the snake can \emph{move} to occupy $(s, s_1, \dots, s_{k-1})$ instead. The snake has \emph{turned around} if it occupied $(s_1, s_2, \dots, s_k)$ at the beginning, but after a finite number of moves occupies $(s_k, s_{k-1}, \dots, s_1)$ instead.

Determine whether there exists an integer $n > 1$ such that: one can place some snake of length $0.9n^2$ in an $n \times n$ grid which can turn around.