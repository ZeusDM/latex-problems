Let $M_A$ be the midpoint of minor arc $BC$ on $\Gamma$ and $P\in\Gamma$ be such that $M_AP\parallel AB$. Then $M_A$ is the center of $\omega$ by Fact 5, so \[AP=BM_A=CM_A\] and thus $AM_A\parallel PC$. Now, line $MXOY$ is the perpendicular bisector of $M_AP$, so \[PX=XM_A=YM_A=PY\] and thus $M_AXPY$ is a rhombus.

Since $\measuredangle{AQM}=\measuredangle{ACB}=\measuredangle{AOM}$ and $Q\neq O$, we have $AMQO$ cyclic. Then \[\measuredangle{AQC}=\measuredangle{AQO}=\measuredangle{AMO}=\frac{\pi}{2},\] so $Q$ lies on the circle with diameter $AC$, call it $\Omega$ with center $N$ (midpoint of $AC$). Then the radical axis of $\omega$ and $\Omega$ is $CQ$, while the center line is $M_AN$, so $CO\perp M_AN$. By the perpendicularity lemma, \[CM_A^2-CN^2=OM_A^2-ON^2=OC^2-ON^2=CN^2,\] so \[PM_A^2=AC^2=4CN^2=2CM_A^2=2YM_A^2=YP^2+YM_A^2\] and thus $PY\perp YM_A$, so $M_AXPY$ is a square. Then $P$ is the pole of $XY$ with respect to $\omega$.

Now, observe that \[\measuredangle{PAD}=\measuredangle{APC}=\measuredangle{ABC}=\measuredangle{ABD},\] so $PA$ is tangent to $\gamma$. Next, observe that \[AP^2=BM_A^2=YM_A^2=PY^2=PM_A^2-YM_A^2,\] so the power of $P$ with respect to $\gamma$ and $\omega$ is the same, thus $P$ lies on the radical axis of $\gamma$ and $\omega$. But $\sqrt{bc}$ inversion sends $\gamma$ to $CM_A$ and $\omega$ to itself, two orthogonal figures since $CM_A$ passes through the center of $\omega$, so $\gamma$ and $\omega$ are orthogonal and hence the polar of $M_A$ with respect to $\gamma$ is the radical axis of $\gamma$ and $\omega$, so $P$ is on the polar of $M_A$ with respect to $\gamma$. Then by La Hire, $M_A$ is on the polar of $P$ with respect to $\gamma$, so $AM_A\cap\gamma=D$ is the second point of tangency from $P$ onto $\gamma$. Hence, $P$ is the pole of $AD$ with respect to $\gamma$.

Thus, the tangents to $\omega$ at $X$ and $Y$ and the tangents to $\gamma$ at $A$ and $D$ concur on $\Gamma$ (specifically at $P$).