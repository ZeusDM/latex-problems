The answers are $\left(a,b,c,n\right)=\boxed{\left(1,1,0,3\right)}$, $\boxed{\left(2,0,0,3\right)}$, and $\boxed{\left(4,1,1,4\right)}$.

This can be verified: \[2^1+3^1+5^0=2+3+1=6=3!\]\[2^2+3^0+5^0=4+1+4=6=3!\]\[2^4+3^1+5^1=16+3+5=24=4!.\]

We will show that these are all the possibilities by casework on $n$.

Case 1: $n\leq2$

Then \[2\geq n!=2^a+3^b+5^c\geq1+1+1=3,\] contradiction.

Case 2: $n=3$

Then \[2^a+3^b+5^c=6.\]

Assume that $c\geq1$. Then \[6=2^a+3^b+5^c\geq1+1+5=7,\] contradiction. Thus, $c=0$, so $2^a+3^b=5$.

Assume that $a\geq3$. Then \[5=2^a+3^b\geq8+1=9,\] contradiction. Thus, $a\leq2$.

\begin{itemize}
	
	\item If $a=0$, then $3^b=4$, contradiction.
	
	\item If $a=1$, then $3^b=3$, so $b=1$. Thus, $\left(a,b,c,n\right)=\left(1,1,0,3\right)$. This is a solution.
	
	\item If $a=2$, then $3^b=1$, so $b=0$. Thus, $\left(a,b,c,n\right)=\left(2,0,0,3\right)$. This is a solution.
	
\end{itemize}

Case 3: $n\geq4$

Assume that $a=0$. Then \[2^a+3^b+5^c\equiv1\pmod2,\] but $n!$ is even, contradiction. Thus, $a\geq1$.

Now, we take the equation modulo $8$. Note that $8\mid4!\mid n!$.

\begin{itemize}
	
	\item If $a=1$, then \[3^b+\left(-3\right)^c\equiv6\pmod8.\] Because $3^{\text{odd}}\equiv3\pmod8$, $3^{\text{even}}\equiv1\pmod8$, $\left(-3\right)^{\text{odd}}\equiv-3\pmod8$, and $\left(-3\right)^{\text{even}}\equiv1\pmod8$, this implies that $b$ is even and $c$ is odd.\
	
	\item If $a=2$, then \[3^b+\left(-3\right)^c\equiv4\pmod8.\] This implies that $b$ is odd and $c$ is even.
	
	\item If $a\geq3$, then \[3^b+\left(-3\right)^c\equiv0\pmod8.\] This implies that both $b$ and $c$ are odd.
	
\end{itemize}

First, assume that $b=0$. Then $2^a+5^c=n!-1$. Taking this equation modulo $3$ (noting that $3\mid3!\mid n!$), we have that \[\left(-1\right)^a+\left(-1\right)^c\equiv2\pmod3.\] This implies that both $a$ and $c$ are even. If $a\neq2$, then $c$ is odd, contradiction, so $a=2$. Then $5+5^c=n!$. But note that $4$ divides $n!$ but \[5+5^c\equiv1+1\equiv2\pmod4,\] contradiction. Thus, $b\geq1$.

Taking the equation modulo $3$, we have that \[\left(-1\right)^a+\left(-1\right)^c\equiv0\pmod3.\] Thus, $a$ and $c$ have different parity. If $a=1$, then $c$ is odd, contradiction. If $a=2$, then $c$ is even, contradiction. Thus, $a\geq3$. Then $c$ is odd, so $a$ is even. In addition, $b$ is odd. Thus, $\frac{a}{2}$, $\frac{b-1}{2}$, and $c-1$ are all nonnegative integers.

Taking the equation modulo $5$, \[n!\equiv2^a+3^b+5^c\equiv4^{\frac{a}{2}}+3\cdot9^{\frac{b-1}{2}}+5\cdot5^{c-1}\equiv\left(-1\right)^{\frac{a}{2}}+3\left(-1\right)^{\frac{b-1}{2}}\pmod5.\] Assume that $n\geq5$. Then $5\mid n!$, so \[\left(-1\right)^{\frac{a}{2}}+3\left(-1\right)^{\frac{b-1}{2}}\equiv0\pmod5.\] Then \[\left(-1\right)^{\frac{a-b+1}{2}}\equiv2\pmod5,\] contradiction. Thus, $n=4$. Then \[2^a+3^b+5^c=24.\]

\begin{itemize}
	
	\item If $b\geq3$, then \[24=2^a+3^b+5^c\geq16+27+5=48,\] contradiction. Thus, $b=1$.
	
	\item If $c\geq3$, then \[24=2^a+3^b+5^c\geq16+3+125=144,\] contradiction. Thus, $c=1$.
	
\end{itemize}

Thus, $2^a=24-3-5=16$, so $a=4$. Thus, $\left(a,b,c,n\right)=\left(4,1,1,4\right)$. This is a solution.

In conclusion, the only nonnegative integer solutions to \[2^a+3^b+5^c=n!\] are $\left(a,b,c,n\right)=\left(1,1,0,3\right)$, $\left(2,0,0,3\right)$, and $\left(4,1,1,4\right)$.