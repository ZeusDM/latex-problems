A \emph{Nim-style game} is defined as follows. Two positive integers $k$ and $n$ are specified, along with a finite set $S$ of $k$-tuples of integers (not necessarily positive). At the start of the game, the $k$-tuple $(n, 0, 0, ..., 0)$ is written on the blackboard.
A legal move consists of erasing the tuple $(a_1,a_2,...,a_k)$ which is written on the blackboard and replacing it with $(a_1+b_1, a_2+b_2, ..., a_k+b_k)$, where $(b_1, b_2, ..., b_k)$ is an element of the set $S$. Two players take turns making legal moves, and the first to write a negative integer loses. In the event that neither player is ever forced to write a negative integer, the game is a draw.
Prove that there is a choice of $k$ and $S$ with the following property: the first player has a winning strategy if $n$ is a power of 2, and otherwise the second player has a winning strategy.