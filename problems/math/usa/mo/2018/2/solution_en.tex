The answer is $\boxed{C\left(\frac{1}{x+1}-\frac{1}{3}\right)+\frac{1}{3}}$ for some constant $C\in\left[-\frac{1}{2},1\right]$. To confirm that this works, note that the range is always a subset of $\left(0,\infty\right)$ and observe that
\begin{align*}
	\cycsum f\left(x+\frac{1}{y}\right)&=\cycsum\left(C\left(\frac{1}{x+\frac{1}{y}+1}-\frac{1}{3}\right)+\frac{1}{3}\right)\\
	&=C\left(\cycsum\frac{1}{x+\frac{1}{y}+1}\right)-C+1\\
	&=C\left(\frac{y}{xy+1+y}+\frac{1}{y+xy+1}+\frac{xy}{1+y+xy}\right)-C+1\\
	&=1.
\end{align*}

Let $g:\left(-\frac{1}{3},\frac{2}{3}\right)\to\left(-\frac{1}{3},\frac{2}{3}\right)$ such that $g\left(x\right)=f\left(\frac{1}{x+\frac{1}{3}}-1\right)-\frac{1}{3}$. If $a,b,c\in\left(-\frac{1}{3},\frac{2}{3}\right)$ and $a+b+c=0$, then
\begin{align*}
	\cycsum g\left(a\right)&=\cycsum\left(f\left(\frac{1}{a+\frac{1}{3}}-1\right)-\frac{1}{3}\right)\\
	&=\left(\cycsum f\left(\frac{c+\frac{1}{3}}{a+\frac{1}{3}}+\frac{1}{\frac{a+\frac{1}{3}}{b+\frac{1}{3}}}\right)\right)-1\\
	&=0.
\end{align*}
With $a=b=c=0$, we have $g\left(0\right)=0$. Now, for $x\in\left(-\frac{1}{3},\frac{1}{3}\right)$, with $a=x$, $b=-x$, and $c=0$, we have \[0=-g\left(-x\right)-g\left(x\right),\] so $g$ is odd on $\left(-\frac{1}{3},\frac{1}{3}\right)$. Now, set $a=-x$, $b=-y$, $c=x+y$ with $x,y\in\left(-\frac{1}{3},\frac{1}{3}\right)$ and $x+y\in\left(-\frac{1}{3},\frac{2}{3}\right)$. Then \[g\left(x+y\right)=-g\left(-x\right)-g\left(-y\right)=g\left(x\right)+g\left(y\right).\]

Now, define a function $h:\mathbb{R}\to\mathbb{R}$ such that \[h\left(x\right)=\begin{cases}g\left(x\right) & x\in\left[0,\frac{1}{3}\right)\\2^ng\left(\frac{x}{2^n}\right) & x\in\left[\frac{2^{n-1}}{3},\frac{2^n}{3}\right)\\-h\left(-x\right) & x<0.\end{cases}\] Observe that $h\left(x\right)=g\left(x\right)$ for $x\in\left(-\frac{1}{3},\frac{2}{3}\right)$. Indeed, this is obvious if $x\in\left(-\frac{1}{3},\frac{1}{3}\right)$ and if $x\in\left[\frac{1}{3},\frac{2}{3}\right)$, then we have that \[h\left(x\right)=2g\left(\frac{x}{2}\right)=g\left(x\right).\] Also observe that $h\left(x\right)=2^ng\left(\frac{x}{2^n}\right)$ when $0\leq x<\frac{2^n}{3}$ as we have the identity $g\left(x\right)=2g\left(\frac{x}{2}\right)$ when $0\leq x<\frac{1}{3}$ so we can repeat this once we have divided out enough powers of $2$ from $x$ to get below $\frac{1}{3}$.

I claim that $h\left(x+y\right)=h\left(x\right)+h\left(y\right)$ for all $x,y\in\mathbb{R}$. Clearly this is true if any of $x,y$ are $0$, so assume they are non-zero. If $x,y\in\left(0,\frac{1}{3}\right)$, then \[h\left(x+y\right)=g\left(x+y\right)=g\left(x\right)+g\left(y\right)=h\left(x\right)+h\left(y\right).\] If $x\in\left(0,\frac{1}{3}\right)$, $y\in\left[\frac{2^{n-1}}{3},\frac{2^n}{3}\right)$, then \[h\left(x+y\right)=2^{n+1}g\left(\frac{x+y}{2^{n+1}}\right)=2^{n+1}g\left(\frac{x}{2^{n+1}}\right)+2^{n+1}g\left(\frac{y}{2^{n+1}}\right)=h\left(x\right)+h\left(y\right).\] If $x\in\left[\frac{2^{m-1}}{3},\frac{2^m}{3}\right)$, $y\in\left[\frac{2^{n-1}}{3},\frac{2^n}{3}\right)$ with $m\leq n$, we have that \[h\left(x+y\right)=2^{n+1}g\left(\frac{x+y}{2^{n+1}}\right)=2^{n+1}g\left(\frac{x}{2^{n+1}}\right)+2^{n+1}g\left(\frac{y}{2^{n+1}}\right)=h\left(x\right)+h\left(y\right).\] Thus, if $x,y>0$ then $h\left(x+y\right)=h\left(x\right)+h\left(y\right)$. If $x,y<0$ then \[h\left(x+y\right)=-h\left(-x-y\right)=-h\left(-x\right)-h\left(-y\right)=h\left(x\right)+h\left(y\right).\] If $x<0$ and $y>0$, WLOG $\left|x\right|<\left|y\right|$. Then \[h\left(x+y\right)=-h\left(-x\right)+h\left(y\right)=h\left(x\right)+h\left(y\right).\] Thus, in all cases, $h\left(x+y\right)=h\left(x\right)+h\left(y\right)$.

Now by induction, for any $m,n\in\mathbb{Z}$, $h\left(\frac{m}{n}x\right)=mh\left(\frac{x}{n}\right)$ and $h\left(x\right)=nh\left(\frac{x}{n}\right)$, so $h\left(\frac{m}{n}x\right)=\frac{m}{n}h\left(x\right)$.

Now, set $C=4g\left(\frac{1}{4}\right)=4h\left(\frac{1}{4}\right)$ and suppose that $h\left(x\right)\neq Cx$ for some $x\in\mathbb{R}$. Set $\gamma=\max\left(\frac{1}{4}+\left|x\right|,\left|h\left(\frac{1}{4}\right)\right|+\left|h\left(x\right)\right|\right)$ and let $\alpha,\beta$ be rational numbers such that \[\left|\alpha+\frac{4x}{h\left(x\right)-Cx}\right|,\left|\beta-\frac{1}{h\left(x\right)-Cx}\right|<\frac{1}{3\gamma}\] (this is possible because $\mathbb{Q}$ is dense in $\mathbb{R}$). Then
\begin{align*}
	\left|\frac{\alpha}{4}+\beta x\right|&\leq\left|\frac{\alpha}{4}+\frac{x}{h\left(x\right)-Cx}\right|+\left|\beta x-\frac{x}{h\left(x\right)-Cx}\right|\\
	&<\frac{1}{3\gamma}\left(\frac{1}{4}+\left|x\right|\right)\\
	&\leq\frac{1}{3}
\end{align*}
and
\begin{align*}
	\left|\alpha h\left(\frac{1}{4}\right)+\beta h\left(x\right)-1\right|&\leq\left|\alpha h\left(\frac{1}{4}\right)+\frac{Cx}{h\left(x\right)-Cx}\right|+\left|\beta h\left(x\right)-\frac{h\left(x\right)}{h\left(x\right)-Cx}\right|\\
	&<\frac{1}{3\gamma}\left(\left|h\left(\frac{1}{4}\right)\right|+\left|h\left(x\right)\right|\right)\\
	&\leq\frac{1}{3}
\end{align*}
so we can say that \[h\left(\frac{\alpha}{4}+\beta x\right)=h\left(\frac{\alpha}{4}\right)+h\left(\beta x\right)=\alpha h\left(\frac{1}{4}\right)+\beta h\left(x\right)>\frac{2}{3}.\] But $\frac{\alpha}{4}+\beta x\in\left(-\frac{1}{3},\frac{1}{3}\right)$, so $h\left(\frac{\alpha}{4}+\beta x\right)<\frac{2}{3}$, contradiction. Thus, $h\left(x\right)=Cx$ for all $x\in\mathbb{R}$ and hence $g\left(x\right)=h\left(x\right)=Cx$ for all $x\in\left(-\frac{1}{3},\frac{2}{3}\right)$. Then $f\left(x\right)=C\left(\frac{1}{x+1}-\frac{1}{3}\right)+\frac{1}{3}$ for all $x\in\left(0,\infty\right)$. With the condition that $f\left(x\right)>0$ for all $x>0$, we have that $-\frac{1}{3}C+\frac{1}{3}\geq0$ and $\frac{2}{3}C+\frac{1}{3}\geq0$ so $-\frac{1}{2}\leq C\leq1$.