Let $f(n)$ be the number of ways to take an $n$-element set $S$, and, if $S$ has more than one element, to partition $S$ into two disjoint nonempty subsets $S_1$ and $S_2$ , then to take one of the sets $S_1$, $S_2$ with more than one element and partition it into two disjoint nonempty subsets $S_3$ and $S_4$, then to take one of the sets with more than one element not yet partitioned and partition it into two disjoint nonempty subsets, etc., always taking a set with more than one element that is not yet partitioned and partitioning it into two nonempty disjoint subsets, until only one-element subsets remain.
For example, we could start with $12345678$ (short for $\{1, 2, 3, 4, 5, 6, 7, 8\}$), then partition it into $126$ and $34578$, then partition $34578$ into $4$ and $3578$, then $126$ into $6$ and $12$, then $3578$ into $37$ and $58$, then $58$ into $5$ and $8$, then $12$ into $1$ and $2$, and finally $37$ into $3$ and $7$.
(The order we partition the sets is important; for instance, partitioning $1234$ into $12$ and $34$, then $12$ into $1$ and $2$, and then $34$ into $3$ and $4$, is different from partitioning $1234$ into $12$ and $34$, then $34$ into $3$ and $4$, and then $12$ into $1$ and $2$.
However, partitioning $1234$ into $12$ and $34$ is the same as partitioning it into $34$ and $12$.)
Find a simple formula for $f(n)$. For instance, $f(1) = 1$, $f(2) = 1$, $f(3) = 3$, and $f (4) = 18$.
