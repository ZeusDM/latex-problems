For a single variable polynomial $Q\left(x\right)$, define $\Delta^1Q\left(x\right)=\Delta Q\left(x\right)=Q\left(x+1\right)-Q\left(x\right)$ and $\Delta^kQ\left(x\right)=\Delta\Delta^{k-1}Q\left(x\right)$. Note that $\Delta^0Q=Q$. Also define $\binom{x}{i}=\frac{x\left(x-1\right)\left(x-2\right)\cdots\left(x-i+1\right)}{i!}$ when $i$ is a non-negative integer and $\binom{x}{i}=\binom{x}{x-i}$ when $i$ is a negative integer and $x$ is a non-negative integer. These are the standard definitions of these operators/functions (called \emph{Finite Differences} and \emph{Generalized Binomial Coefficients}, respectively).

\textbf{Lemma (Coefficients of Binomial Basis):} Let $Q\left(x\right)$ be a polynomial with degree $d$. Then \[Q\left(x+k\right)=\displaystyle\sum_{i=0}^d\left(\Delta^iQ\left(k\right)\right)\binom{x}{i}\] holds as a polynomial identity for any real number $k$.

\begin{lemmaproof}
Induct on $d$. If $Q$ is a constant $c$, then the right hand side is $P\left(k\right)=c$. For the inductive step, note that $\Delta Q$ is a polynomial of degree $d-1$, so we can write \[\Delta Q\left(x+k\right)=\displaystyle\sum_{i=0}^{d-1}\left(\Delta^{i+1}Q\left(k\right)\right)\binom{x}{i}.\] Then \begin{align*}Q\left(x+k\right)-Q\left(k\right)&=\displaystyle\sum_{y=0}^{x-1}\Delta Q\left(y+k\right)\\&=\displaystyle\sum_{y=0}^{x-1}\displaystyle\sum_{i=0}^{d-1}\left(\Delta^{i+1}Q\left(k\right)\right)\binom{y}{i}\\&=\displaystyle\sum_{i=0}^{d-1}\Delta^{i+1}Q\left(k\right)\displaystyle\sum_{y=0}^{x-1}\binom{y}{i}\\&=\displaystyle\sum_{i=0}^{d-1}\left(\Delta^{i+1}Q\left(k\right)\right)\binom{x}{i+1}\\&=\displaystyle\sum_{i=1}^d\left(\Delta^iQ\left(k\right)\right)\binom{x}{i}\end{align*} by the Hockey Stick Identity. But this rearranges to \[Q\left(x+k\right)=\displaystyle\sum_{i=0}^d\left(\Delta^iQ\left(k\right)\right)\binom{x}{i},\] so we are good.
\end{lemmaproof}

This lemma easily extends to a multi-variate version of which we will use the two-variate one: \[P\left(x+h,y+k\right)=\displaystyle\sum_{i=0}^{\deg_xP}\displaystyle\sum_{j=0}^{\deg_yP}\left(\Delta_x^i\Delta_y^jP\left(h,k\right)\right)\binom{x}{i}\binom{y}{j},\] where we have defined $\Delta_x$ to take the $\Delta$ operator on the $x$ component of the polynomial. If you are familiar with Multivariate Calculus, this is similar to the partial derivative (where the regular finite difference is similar to the regular derivative). If you wish to prove this extension, try applying the lemma to the $x$ component then the $y$ component and rearranging the sum a little.

Now, we can look at the problem.

Use $\left(h,k\right)=\left(0,0\right)$ in the above to get that \[P\left(x,y\right)=\displaystyle\sum_{i=0}^{2020}\displaystyle\sum_{j=0}^{2020}\left(\Delta_x^i\Delta_y^jP\left(0,0\right)\right)\binom{x}{i}\binom{y}{j}.\] Note that we can take the sums up to $2020$ because any terms where $i>\deg_xP$ will have $\Delta_x^i=0$.

We can prove by induction on $i$ that $\Delta_x^iP\left(x,y\right)=\binom{x+y}{x+i}$ for integers $0\leq x\leq2020-i$ (note that $\binom{x+y+1}{x+i+1}-\binom{x+y}{x+i}=\binom{x+y}{x+i+1}$ by Pascal's Identity). We can follow this by induction on $j$ that $\Delta_x^i\Delta_y^jP\left(x,y\right)=\binom{x+y}{x+i-j}$ for integers $0\leq x\leq2020-i$, $0\leq y\leq2020-j$. Thus, with $\left(x,y\right)=\left(0,0\right)$ (valid for all $0\leq i,j\leq2020$), we get that $\Delta_x^i\Delta_y^jP\left(0,0\right)=\binom{0}{i-j}$. In other words, this is $0$ when $i\neq j$ and $1$ if they are equal.

Then we can interpret our equation for $P\left(x,y\right)$ to be \[P\left(x,y\right)=\displaystyle\sum_{i=0}^{2020}\binom{x}{i}\binom{y}{i}.\] Thus, \[P\left(4040,4040\right)=\displaystyle\sum_{i=0}^{2020}\binom{4040}{i}^2=\displaystyle\sum_{i=0}^{2020}\binom{4040}{i}\binom{4040}{4040-i}=\frac{1}{2}\left(\binom{8080}{4040}+\binom{4040}{2020}^2\right)\] by Vandermonde's identity. This can be calculated to be $\frac{1}{2}\left(5544+40^2\right)\equiv\boxed{1555}\pmod{2017}$ by Lucas' Theorem.