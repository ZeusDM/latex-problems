Let $ABC$ be a triangle with $AB=7, AC=9, BC=10$, circumcenter $O$, circumradius $R$, and circumcircle $\omega$. Let the tangents to $\omega$ at $B,C$ meet at $X$. A variable line $\ell$ passes through $O$. Let $A_1$ be the projection of $X$ onto $\ell$ and $A_2$ be the reflection of $A_1$ over $O$. Suppose that there exist two points $Y,Z$ on $\ell$ such that $\angle YAB+\angle YBC+\angle YCA=\angle ZAB+\angle ZBC+\angle ZCA=90^{\circ}$, where all angles are directed, and furthermore that $O$ lies inside segment $YZ$ with $OY*OZ=R^2$. Then there are several possible values for the sine of the angle at which the angle bisector of $\angle AA_2O$ meets $BC$. If the product of these values can be expressed in the form $\frac{a\sqrt{b}}{c}$ for positive integers $a,b,c$ with $b$ squarefree and $a,c$ coprime, determine $a+b+c$.
