Let $ABC$ be a triangle with $AB=13,BC=15,AC=14$, circumcenter $O$, and orthocenter $H$, and let $M,N$ be the midpoints of minor and major arcs $BC$ on the circumcircle of $ABC$. Suppose $P\in AB, Q\in AC$ satisfy that $P,O,Q$ are collinear and $PQ||AN$, and point $I$ satisfies $IP\perp AB,IQ\perp AC$. Let $H'$ be the reflection of $H$ over line $PQ$, and suppose $H'I$ meets $PQ$ at a point $T$. If $\frac{MT}{NT}$ can be written in the form $\frac{\sqrt{m}}{n}$ for positive integers $m,n$ where $m$ is not divisible by the square of any prime, then find $100m+n$.
