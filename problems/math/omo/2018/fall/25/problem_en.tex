Given two positive integers $x,y$, we define $z=x\,\oplus\,y$ to be the bitwise XOR of $x$ and $y$; that is, $z$ has a $1$ in its binary representation at exactly the place values where $x,y$ have differing binary representations. It is known that $\oplus$ is both associative and commutative. For example, $20 \oplus 18 = 10100_2 \oplus 10010_2 = 110_2 = 6$. Given a set $S=\{a_1, a_2, \dots, a_n\}$ of positive integers, we let $f(S) = a_1 \oplus a_2 \oplus a_3\oplus \dots \oplus a_n$. We also let $g(S)$ be the number of divisors of $f(S)$ which are at most $2018$ but greater than or equal to the largest element in $S$ (if $S$ is empty then let $g(S)=2018$). Compute the number of $1$s in the binary representation of $\displaystyle\sum_{S\subseteq \{1,2,\dots, 2018\}} g(S)$.
