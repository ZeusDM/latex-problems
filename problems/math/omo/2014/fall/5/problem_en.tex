A crazy physicist has discovered a new particle called an omon. He has a machine, which takes two omons of mass $a$ and $b$ and entangles them; this process destroys the omon with mass $a$, preserves the one with mass $b$, and creates a new omon whose mass is $\frac 12 (a+b)$. The physicist can then repeat the process with the two resulting omons, choosing which omon to destroy at every step. The physicist initially has two omons whose masses are distinct positive integers less than $1000$. What is the maximum possible number of times he can use his machine without producing an omon whose mass is not an integer?
