For a positive integer $n$, an \textit{$n$-branch} $B$ is an ordered tuple $(S_1, S_2, \dots, S_m)$ of nonempty sets (where $m$ is any positive integer) satisfying $S_1 \subset S_2 \subset \dots \subset S_m \subseteq \{1,2,\dots,n\}$. An integer $x$ is said to \textit{appear} in $B$ if it is an element of the last set $S_m$.  Define an \textit{$n$-plant} to be an (unordered) set of $n$-branches $\{ B_1, B_2, \dots, B_k\}$, and call it \textit{perfect} if each of $1$, $2$, \dots, $n$ appears in exactly one of its branches.

Let $T_n$ be the number of distinct perfect $n$-plants (where $T_0=1$), and suppose that for some positive real number $x$ we have the convergence \[ \ln \left( \sum_{n \ge 0} T_n \cdot \frac{\left( \ln x \right)^n}{n!} \right) = \frac{6}{29}. \] If $x = \frac mn$ for relatively prime positive integers $m$ and $n$, compute $m+n$.
