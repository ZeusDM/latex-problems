Given the following triangular arrangement of circles:
\begin{center}
	\begin{asy}
		size(5cm);
		pair A=dir(90), B=dir(210), C=dir(330), A_1=(2*B+C)/3, A_2=(B+2*C)/3, B_1=(2*C+A)/3, B_2=(C+2*A)/3, C_1=(2*A+B)/3, C_2=(A+2*B)/3;
		real r=sqrt(3)/9, s=sqrt(3)/3-2*r, d;
		pair[] centers = {A,C_1,C_2,B,A_1,A_2,C,B_1,B_2,A};
		for (int i = 0; i < 9; ++i) {
			draw(circle(centers[i],r));
			if (i < 3) {
				d = 240;
			}
			else if (i < 6) {
				d = 0;
			}
			else if (i < 9) {
				d = 120;
			}
			draw(centers[i]+r*dir(d)--centers[i]+(r+s)*dir(d));
		}
	\end{asy}
\end{center}
Each of the numbers $1,2,\ldots,9$ is to be written into one of these circles, so that each circle contains exactly one of these numbers and
\begin{enumerate}[label=(\roman*)]
	\item the sums of the four numbers on each side of the triangle are equal;
	\item the sums of the squares of the four numbers on each side of the triangle are equal.
\end{enumerate}
Find all ways in which this can be done.