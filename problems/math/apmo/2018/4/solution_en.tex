When the ray hits a side, reflect the triangle and the resulting ray over that side. Do this until the ray hits a vertex. Then the ray becomes a line segment between the original vertex $A$ and one of the reflected versions of $A$. Here is an example of what the resulting diagram might look like:

\begin{center}
\begin{asy}
size(5cm);
draw(0*(1,0)-0*(1/2,sqrt(3)/2)--0*(1,0)-3*(1/2,sqrt(3)/2));
draw(1*(1,0)-1*(1/2,sqrt(3)/2)--1*(1,0)-6*(1/2,sqrt(3)/2));
draw(2*(1,0)-4*(1/2,sqrt(3)/2)--2*(1,0)-7*(1/2,sqrt(3)/2));
draw(0*(1,0)-0*(1/2,sqrt(3)/2)--1*(1,0)-1*(1/2,sqrt(3)/2));
draw(0*(1,0)-1*(1/2,sqrt(3)/2)--1*(1,0)-2*(1/2,sqrt(3)/2));
draw(0*(1,0)-2*(1/2,sqrt(3)/2)--2*(1,0)-4*(1/2,sqrt(3)/2));
draw(0*(1,0)-3*(1/2,sqrt(3)/2)--2*(1,0)-5*(1/2,sqrt(3)/2));
draw(1*(1,0)-5*(1/2,sqrt(3)/2)--2*(1,0)-6*(1/2,sqrt(3)/2));
draw(1*(1,0)-6*(1/2,sqrt(3)/2)--2*(1,0)-7*(1/2,sqrt(3)/2));
draw(0*(1,0)-1*(1/2,sqrt(3)/2)--1*(1,0)-1*(1/2,sqrt(3)/2));
draw(0*(1,0)-2*(1/2,sqrt(3)/2)--1*(1,0)-2*(1/2,sqrt(3)/2));
draw(0*(1,0)-3*(1/2,sqrt(3)/2)--1*(1,0)-3*(1/2,sqrt(3)/2));
draw(1*(1,0)-4*(1/2,sqrt(3)/2)--2*(1,0)-4*(1/2,sqrt(3)/2));
draw(1*(1,0)-5*(1/2,sqrt(3)/2)--2*(1,0)-5*(1/2,sqrt(3)/2));
draw(1*(1,0)-6*(1/2,sqrt(3)/2)--2*(1,0)-6*(1/2,sqrt(3)/2));

draw(0*(1,0)-0*(1/2,sqrt(3)/2)--2*(1,0)-7*(1/2,sqrt(3)/2));
label("$A$",(0,0),N);
\end{asy}
\end{center}

Thus, the ray takes some straight line path in a large equilateral triangle like this:

\begin{center}
\begin{asy}
size(5cm);
for (int i=0; i<=7; ++i)
{
    draw(0*(1,0)-i*(1/2,sqrt(3)/2)--i*(1,0)-i*(1/2,sqrt(3)/2));
    draw(i*(1,0)-i*(1/2,sqrt(3)/2)--i*(1,0)-7*(1/2,sqrt(3)/2));
    draw(0*(1,0)-i*(1/2,sqrt(3)/2)--(7-i)*(1,0)-7*(1/2,sqrt(3)/2));
}

draw(0*(1,0)-0*(1/2,sqrt(3)/2)--2*(1,0)-7*(1/2,sqrt(3)/2));
label("$A$",(0,0),N);
\end{asy}
\end{center}

Each bounce corresponds to each time the line segment hits one of the reflected sides. These are just when the line segment hits one of the horizontal lines or any of the diagonal lines. Call the diagonals that go from top right to bottom left the ``RL diagonals'', and the ones that go from top left to bottom right the ``LR diagonals''. Also call the horizontal lines ``rows''. Suppose that the end vertex is in the $y$th row down and on the $x$th RL diagonal down (equivalently the $\left(y+1-x\right)$th LR diagonal down). Then the line segment must hit the $2$nd through $\left(x-1\right)$th RL diagonals, $2$nd through $\left(y-x\right)$th LR diagonals, and $2$nd through $\left(y-1\right)$th rows for a total of $2y-5$ hits. This does not overcount because the line segment never hits any two rows or diagonals at the same time --- doing so would stop the line segment at that intersection, which never happens.

It is easy to see by induction that a coordinate $\left(x,y\right)$ above corresponds to a reflected version of $A$ precisely when $x+y\equiv2\pmod3$, a reflected version of $B$ when $x+y\equiv0\pmod3$, and a reflected version of $C$ when $x+y\equiv1\pmod3$. Furthermore, it is clear that $\left(x,y\right)$ is the first time that the ray reaches a vertex if and only if there are no $\left(x',y'\right)$ on the line between $\left(1,1\right)$ and $\left(x,y\right)$. By taking an affine transformation, we can turn this into the Cartesian plane with these coordinates, so we need $\gcd\left(x-1,y-1\right)=1$. Thus, we want to find the different values of $2y-5$ when $x-1,y-1$ are relatively prime nonnegative integers, $1\leq x\leq y$, and $x+y\equiv2\pmod3$. Let $a=x-1,b=y-1$, so we want to find the different values of $2b-3$ when $a,b$ are relatively prime nonnegative integers with $0\leq a\leq b$ and $a+b\equiv0\pmod3$.

If $b\equiv0\pmod3$, then $a\equiv0\pmod3$, so $a,b$ are not relatively prime, contradiction.

If $b\equiv2\pmod3$, then take $a=1$. Then all conditions are satisfied, so this works. Thus, we can achieve all values of $2b-3$ when $b\equiv2\pmod3$. This corresponds to all $1\pmod6$ positive integers.

Now, assume that $b\equiv1\pmod3$ and $b\neq1,4,10$.

Let $p_1<p_2<p_3<\ldots$ be the $2\pmod3$ primes. Let $p_k$ be the smallest $2\pmod3$ prime not dividing $b$, so $p_1p_2\ldots p_{k-1}$ divides $b$.

If $k=1$, then $b$ is odd, so $b\equiv1\pmod6$. Then $b\geq7>2=p_1$, so pick $a=p_1$.

If $k=2$, then $b\equiv1\pmod3,b\equiv0\pmod2,b\not\equiv0\pmod5$. Then $b\equiv4\pmod6$. Then $b\geq16>5=p_2$, so pick $a=p_2$.

If $k=3$, then $b\equiv1\pmod3,b\equiv0\pmod{10},b\not\equiv0\pmod{11}$. Then $b\equiv10\pmod{30}$. Then $b\geq40>11=p_3$, so pick $a=p_3$.

Otherwise, $k\geq4$. If $k$ is even, consider $p_1p_2\ldots p_{k-1}-3>0$. This quantity is $2\pmod3$ and $1\pmod2$. So there is some $p_j$ dividing it. But $j\geq k$ otherwise $p_j\mid3$, so \[p_k\leq p_j\leq p_1p_2\ldots p_{k-1}-3.\] If $k$ is odd, consider $p_2p_3\ldots p_{k-1}-6>0$. This quantity is $2\pmod3$ and $1\pmod2$. So there is some $p_j$ dividing it. But $j\geq k$ otherwise $p_j\mid6$ (but $p_j\neq2$), so \[p_k\leq p_j\leq p_2p_3\ldots p_{k-1}-6.\] Either way, we have that \[p_k<p_1p_2\ldots p_{k-1}\leq b,\] so pick $a=p_k$.

Thus, all $b\equiv1\pmod3$ work except $1,4,10$. This corresponds to all positive $5\pmod6$ values of $2b-3$ except  $5,17$.

Thus, we can hit all positive integers which are $\pm1\pmod6$ except $5$ and $17$.