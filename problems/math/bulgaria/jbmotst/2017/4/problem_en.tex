Given is a table nxn and in every square there is a checker. In a move a checker goes to adjacent (side, not vertex)  square. In square there can be many checkers. Find the minimum and the maximum covered cells for n=5,6,7?