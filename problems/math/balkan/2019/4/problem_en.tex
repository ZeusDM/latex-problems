A grid consists of all points of the form $(m, n)$ where $m$ and $n$ are integers with $|m|\le 2019,|n| \le 2019$ and $|m| +|n| < 4038$. We call the points $(m,n)$ of the grid with either $|m| = 2019$ or $|n| = 2019$ the boundary points. The four lines $x = \pm 2019$ and $y= \pm 2019$ are called boundary lines. Two points in the grid are called neighbours  if the distance between them is equal to $1$.

Anna and Bob play a game on this grid.

Anna starts with a token at the point $(0,0)$. They take turns, with Bob playing first.

1) On each of his turns. Bob deletes  at most two boundary points on each boundary line.

2) On each of her turns. Anna makes exactly three steps , where a step  consists of moving her token from its current point to any neighbouring point, which has not been deleted.

As soon as Anna places her token on some boundary point which has not been deleted, the game is over and Anna wins.

Does Anna have a winning strategy?