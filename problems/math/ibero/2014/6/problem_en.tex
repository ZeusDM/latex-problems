Given a set $X$ and a function $f: X \rightarrow X$,  for each $x \in X$ we define $f^1(x)=f(x)$ and, for each $j \ge 1$,  $f^{j+1}(x)=f(f^j(x))$. We say that $a \in X$ is a fixed point of $f$ if $f(a)=a$. For each $x \in \mathbb{R}$,  let $\pi (x)$ be the quantity of positive primes lesser or equal to $x$.

Given an positive integer $n$,  we say that $f: \{1,2, \dots, n\} \rightarrow \{1,2, \dots, n\}$ is catracha if $f^{f(k)}(k)=k$,  for every $k=1, 2, \dots n$. Prove that:

(a) If $f$ is catracha, $f$ has at least $\pi (n) -\pi (\sqrt{n}) +1$ fixed points.

(b) If $n \ge 36$,  there exists a catracha function $f$ with exactly $ \pi (n) -\pi (\sqrt{n}) + 1$ fixed points.