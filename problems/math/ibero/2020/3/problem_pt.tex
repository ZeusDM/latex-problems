Seja $n \ge 2$ um inteiro. Uma sequência $\alpha = (a_1, a_2, \dots, a_n)$ de $n$ números é chamada \emph{limenha} se
\[\mathrm{mdc}\left\{ a_i - a_j \text{\ tal que\ } a_i > a_j \text{\ e\ } 1 \le i, j \le n\right\} = 1,\]
isto é, se o máximo divisor comum de todas as diferenças $a_i - a_j$, com $a_i > a_j$, é $1$.

Uma \emph{operação} consiste em escolher dois elementos $a_k$ e $a_\ell$ da sequência, com $k \neq \ell$, e substituir $a_\ell$ por $a'_\ell = 2a_k - a_\ell$.

Demonstre que, dada uma coleção de $2^n - 1$ sequências limenhas, cada uma formada por $n$ números inteiros, existem duas destas sequências, digamos $\beta$ e $\gamma$, tais que é possível transformar $\beta$ em $\gamma$ efetuando um número finito de operações.
