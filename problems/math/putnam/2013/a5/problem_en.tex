For $m \geq 3$, a list of $\binom{m}{3}$ real numbers $a_{ijk}$ ($1 \leq i < < j < k \leq m$) is said to be \emph{area definite} for $\mathbb{R}^n$ if the inequality
\[
\sum_{1 \leq i < j < k \leq m} a_{ijk} \cdot \mathrm{Area}(\Delta A_i A_j A_k) \geq 0
\]
holds for every choice of $m$ points $A_1,\dots,A_m$ in $\mathbb{R}^n$.
For example, the list of four numbers $a_{123} = a_{124} = a_{134} = 1$, $a_{234} = -1$ is area definite for $\mathbb{R}^2$. Prove that if a list of $\binom{m}{3}$ numbers is area definite for $\mathbb{R}^2$,
then it is area definite for $\mathbb{R}^3$.
