Let $A(h)=\frac{f(h)g(h)-f(0)g(0)}{h}$ and $B(h)=\frac{\frac{f(h)}{g(h)}-\frac{f(0)}{g(0)}}{h}$. Let $\delta>0$ such that $|h|<\delta$ implies $|g(h)-g(0)|<\frac{g(0)}{2}$. Then $g(0)+g(h)\neq0$, so
\[
	\frac{A(h)+g(0)g(h)B(h)}{g(0)+g(h)}=\frac{f(h)-f(0)}{h}
\]
by some algebraic manipulation. Since $A(h),B(h),g(h)$ all have limits as $h\to0$ and $g(0)+g(h)\not\to0$, the limit of the left hand side exists as $h\to0$ so the limit of the right hand side does too, and thus $f$ is differentiable at $0$.