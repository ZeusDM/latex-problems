Evaluate
\begin{gather*}
\int_0^\infty \left(x-\frac{x^3}{2}+\frac{x^5}{2\cdot
4}-\frac{x^7}{2\cdot 4\cdot 6}+\cdots\right) \\
\left(1+\frac{x^2}{2^2}+
\frac{x^4}{2^2\cdot 4^2}+\frac{x^6}{2^2\cdot 4^2 \cdot 6^2}+\cdots\right)\,dx.
\end{gather*}
