Alina traça $2019$ cordas em uma circunferência. Os pontos extremos dessas cordas são todos distintos.
Um ponto é considerado \textit{marcado} se ele é de um dos seguintes tipos:

\begin{enumerate}[label = (\alpha*)]
	\item um dos 4038 pontos extremos das cordas; ou
	\item um ponto de interseção de pelo menos duas das cordas.
\end{enumerate}

Alina escreve um número em cada ponto marcado.
Dos 4038 pontos do tipo (i), ela escreve o número $0$ em $2019$ destes pontos, e escreve o número $1$ nos outros $2019$ pontos.
Em cada ponto do tipo (ii), Alina escreve um inteiro qualquer (não necessariamente positivo).

Em cada corda, Alina considera os segmentos que conectam $2$ pontos marcados consecutivos (uma corda com $k$ pontos marcados tem $k−1$ desses segmentos).
Em cada um desses segmentos ela escreve $2$ números.
Em amarelo, ela escreve a soma dos números escritos nos pontos extremos desse segmento.
Em azul, ela escreve o valor absoluto da diferença dos números escritos nos pontos extremos dessesegmento.

Alina percebe que os $N+1$ números escritos em amarelo são exatamente os números $0, 1, \dots, N$.
Mostre que pelo menos um dos números escritos em azul é um número múltiplo de 3.

(Uma corda é um segmento de reta que conecta $2$ pontos distintos de uma circunferência.)
