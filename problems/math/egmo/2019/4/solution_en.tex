Let $r,r_A,I_A$ be the inradius, $A$-exradius, and $A$-excenter of $\triangle{ABC}$.

By Power of a Point, $AP\cdot AB=AI^2$ and $AQ\cdot AC=AI^2$. Let $P'\in AC$ and $Q'\in AB$ be the reflections of $P$ and $Q$ over $AI$; clearly $PQ$ is tangent to the incircle of $\triangle{ABC}$ if and only if $P'Q'$ is.

Since $\frac{AQ'}{AP'}=\frac{AQ}{AP}=\frac{AB}{AC}$, $Q'P'\parallel BC$. Then $\triangle{AQ'P'}\sim\triangle{ABC}$ with ratio $\frac{AQ'}{AB}=\frac{AI^2}{AB\cdot AC}$. Then the $A$-exradius of $\triangle{AQ'P'}$ is \[\frac{AI^2}{AB\cdot AC}\cdot r_A=\frac{AI^2}{AB\cdot AC}\cdot\frac{AI_A}{AI}\cdot r=\frac{AI\cdot AI_A}{AB\cdot AC}\cdot r=r\] so the $A$-excircle of $\triangle{AQ'P'}$ is the incircle of $\triangle{ABC}$ (there is a unique circle with radius $r$ tangent to rays $\overrightarrow{AB}$ and $\overrightarrow{AC}$). It follows that $Q'P'$ is tangent to the incircle of $\triangle{ABC}$, so $PQ$ is tangent to the incircle of $\triangle{ABC}$.