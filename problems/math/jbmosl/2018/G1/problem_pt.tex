Seja $H$ o ortocentro do triângulo acutângulo $ABC$, com $BC > AC$, inscrito na circunferência $\Gamma$.
A circunferência com centro $C$ e raio $CB$ intersecta $\Gamma$ novamente no ponto $D$, que está no arco $AB$ que não contêm $C$.
A circunferência com centro $C$ e raio $CA$ intersecta o segmento $CD$ no ponto $K$.
A reta paralela a $BD$ por $K$ intersecta $AB$ em $L$.
Se $M$ é o ponto médio de $AB$ e $N$ é o pé da perpendicular de $H$ em $CL$, prove que a reta $MN$ bisecta o segmento $CH$.
