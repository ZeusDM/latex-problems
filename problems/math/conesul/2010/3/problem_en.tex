Let us define cutting a convex polygon with $n$ sides by choosing a pair of consecutive sides $AB$ and $BC$ and substituting them by three segments $AM, MN$,  and $NC$,  where $M$ is the midpoint of $AB$ and $N$ is the midpoint of $BC$. In other words, the triangle $MBN$ is removed and a convex polygon with $n+1$ sides is obtained.

Let $P_6$ be a regular hexagon with area $1$. $P_6$ is cut and the polygon $P_7$ is obtained. Then $P_7$ is cut in one of seven ways and polygon $P_8$ is obtained, and so on. Prove that, regardless of how the cuts are made, the area of $P_n$ is always greater than $2/3$.