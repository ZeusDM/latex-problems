Seja $p$ um primo ímpar. \playerA{Ana} e \playerB{Ben} alternam turnos ao jogar o seguinte jogo: em cada movimento, o jogador do turno escolhe um número não escolhido anteriormente do conjunto $\{1,2,...,2p-3,2p-2\}$. Esse processo continua até que não sobre nenhum número.
Depois do fim do processo, cada jogador cria um número multiplicando os números escolhidos e, então, somando 1.
Dizemos que um jogador ganha se o número criado é divisível por $p$, enquanto o número criado pelo oponente não é divisível por $p$. Caso contrário, o jogo termina em empate. \playerA{Ana} começa a jogar.
Algum dos jogadores possui estratégia vencedora?
