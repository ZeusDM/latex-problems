A train consist of $2010$ wagons containing gold coins, all of the same shape. Any two coins have equal weight provided that they are in the same wagon, and differ in weight if they are in different ones. The weight of a coin is one of the positive reals
\begin{align*} m_1 <m_2 <\ldots <m_{2010} \end{align*}Each wagon is marked by a label with one of the numbers $m_1,m_2, \ldots , m_{2010}$ (the numbers on different labels are different).

A controller has a pair of scales (allowing only to compare masses) at his disposal. During each measurement he can use an arbitrary number of coins from any of the wagons. The controller has the task to establish: if all labels show rightly the common weight of the coins in a wagon or if there exists at least one wrong label. What is the least number of measurement that the controller has to perform to accomplish his task?