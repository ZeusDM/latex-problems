Let $\omega$ be a circle. Let $E$ be on $\omega$ and $S$ outside $\omega$ such that line segment $SE$ is tangent to $\omega$. Let $R$ be on $\omega$. Let line $SR$ intersect $\omega$ at $B$ other than $R$,  such that $R$ is between $S$ and $B$. Let $I$ be the intersection of the bisector of $\angle ESR$ with the line tangent to $\omega$ at $R$; let $A$ be the intersection of the bisector of $\angle ESR$ with $ER$. If the radius of the circumcircle of $\triangle EIA$ is $10$,  the radius of the circumcircle of $\triangle SAB$ is $14$,  and $SA = 18$,  then $IA$ can be expressed in simplest form as $\frac{m}{n}$. Find $m + n$.