King Tin writes the first $n$ perfect squares on the royal chalkboard, but he omits the first (so for n = $3$, he writes $4$ and $9$). His son, Prince Tin, comes along and repeats the following process until only one number remains:
\textit{He erases the two greatest numbers still on the board, calls them a and b, and writes the value of  on the board.
}Let $S(n)$ be the last number that Prince Tin writes on the board. Let $\lim_{n\to \infty} S(n) = r$, meaning that $r$ is the unique number such that for every $\epsilon > 0$ there exists a positive integer $N$ so that $|S(n) - r| < \epsilon$ for all $n > N$. If $r$ can be written in simplest form as $\frac{m}{n}$, find $m + n$.