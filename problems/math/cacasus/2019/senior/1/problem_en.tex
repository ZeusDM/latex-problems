Pasha placed numbers from 1 to 100 in the cells of the square $10\times 10$, each number exactly once. After that, Dima considered all sorts of squares, with the sides going along the grid lines, consisting of more than one cell, and painted in green the largest number in each such square (one number could be colored many times). Is it possible that all two-digit numbers are painted green?