Olhando a equação módulo $2$, temos que:
$p^2 \equiv 1 \pmod{2}$,
isto é, $p \neq 2$

Olhando módulo 6, sabemos que $p = 3$ ou $p \equiv \pm 1 \pmod{6}$, ou seja:
$$p^2 \equiv 
\begin{cases}
	3, \text{se } p = 3\\
	1, \text{caso contrário}
\end{cases}
\pmod{6} \implies
p^2 + 2019 \equiv 
\begin{cases}
	0, \text{se } p = 3\\
	4, \text{caso contrário}
\end{cases}
\pmod{6}.
$$

\begin{itemize}
	\item[Se $p \neq 3$:] $26(q^2 + r^2 + s^2) \equiv 2(q^2 + r^2 + s^2) \equiv 4 \pmod{6}$. Logo, $$q^2 + $$

\end{itemize}
