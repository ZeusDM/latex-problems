Existiam 3 candidatos $A$, $B$ e $C$ nas eleições para governador.
No primeiro turno, $A$ ganhou $44\%$ dos votos que $B$ e $C$ receberam em conjunto; e $C$ recebeu menos votos dentre os três.
Nenhum candidato ganhou a maioria dos votos necessários pra ganhar no primeiro turno, então houve um segundo turno com os candidatos $A$ e $B$.
Os eleitores do segundo turno foram os mesmos eleitores da primeiro turno, exceto $p\%$ dos eleitores de $C$ que decidiram não votar no segundo turno; $p$ inteiro, $1 \le p \le 100$. Além disso, os eleitores de $B$ no primeiro turno decidiram se manter eleitores de $B$ no segundo turno. 

Um jornalista afirma que, sabendo das informações acima, é possível descobrir o vencedor do segundo turno com certeza. Para quais valores de $p$ o jornalista está certo?
