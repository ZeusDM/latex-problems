Temos que $\angle LQR = 90^\circ + \angle LCQ = 90^\circ + \angle KCP = \angle KPR$.

Logo, $[LQR] = [RPK] \iff \dfrac{LQ \cdot QR \cdot \sen(\angle LQR)}{2} = \dfrac{KP \cdot PR \cdot \sen(\angle KPR)}{2} \iff$
$$\iff \frac{LQ}{KP} = \frac{PR}{QR}.$$

Além disso, $CR = 2R \sen(A + \frac{C}{2}) = 2R \sen(B + \frac{C}{2}) = R (\sen (A + \frac{C}{2}) + \sen (B + \frac{C}{2})) = R \sen 90^\circ \sen(\frac{A-B}{2}) = R \sen(\frac{A-B}{2}).$

Mas, $\triangle CLQ$ é semelhante a $\triangle CKP \implies \dfrac{LQ}{KP} = \dfrac{CQ}{CP}$.

Sabemos que $CP = R \frac{\sen A}{\cos C}$ e $CQ = R \frac{\sen B}{\cos C}$, que implica que $CP + CQ = \frac{R}{\cos C} (\sen A + \sen B) = \frac{R \sen(\frac{A+B}{2}) \cos(\frac{A-B}{2})}{\cos C} = R \cos(\frac{A-B}{2}) = CR.$

Como $CP + CQ = CR$, temos $CQ = PR$ e $CP = QR$, que implica
$$\frac{LQ}{KP} = \frac{CQ}{CP} = \frac{PR}{QR},$$
como queríamos demonstrar.
