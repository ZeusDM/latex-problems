For a sequence 
$x_1,x_2,\ldots,x_n$
 of real numbers, we define its 
$\textit{price}$
 as 
\[\max_{1\le i\le n}|x_1+\cdots +x_i|.\]
 Given 
$n$
 real numbers, Dave and George want to arrange them into a sequence with a low price. Diligent Dave checks all possible ways and finds the minimum possible price 
$D$.
 Greedy George, on the other hand, chooses 
$x_1$
 such that 
$|x_1 |$
 is as small as possible; among the remaining numbers, he chooses 
$x_2$
 such that 
$|x_1 + x_2 |$
 is as small as possible, and so on. Thus, in the 
$i$-th step he chooses 
$x_i$
 among the remaining numbers so as to minimise the value of 
$|x_1 + x_2 + \cdots  x_i |$.
 In each step, if several numbers provide the same value, George chooses one at random. Finally he gets a sequence with price 
$G$.


Find the least possible constant 
$c$
 such that for every positive integer 
$n$, 
 for every collection of 
$n$
 real numbers, and for every possible sequence that George might obtain, the resulting values satisfy the inequality 
$G\le cD$.


