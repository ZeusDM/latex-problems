Checkerboard color $\mathcal{R}$ so that the corners are black. Clearly the small rectangles have sides along the grid lines of the checkerboard. For a rectangle $\mathcal{S}$ that is checkerboard colored, let $f\left(\mathcal{S}\right)$ be the number of black squares in $\mathcal{S}$ minus the number of white squares in $\mathcal{S}$. Clearly $f$ is additive. Let $B$ small rectangles have corners all black (with evaluation of $f$ being $1$), $W$ small rectangles have corners all white (with evaluation of $f$ being $-1$), and $E$ small rectangles have both black and white corners (with evaluation of $f$ being $0$). Then $1=f\left(\mathcal{R}\right)=B-W$, so $B\geq1$. But a rectangle with corners all black clearly has distances from the four sides of $\mathcal{R}$ being all odd or all even.