Let $\mathbb Z$ be the set of integers. We consider functions $f :\mathbb Z\to\mathbb Z$ satisfying
\[f\left(f(x+y)+y\right)=f\left(f(x)+y\right)\]for all integers $x$ and $y$. For such a function, we say that an integer $v$ is \textit{f-rare} if the set
\[X_v=\{x\in\mathbb Z:f(x)=v\}\]is finite and nonempty.
(a) Prove that there exists such a function $f$ for which there is an $f$-rare integer.
(b) Prove that no such function $f$ can have more than one $f$-rare integer.
