Let the perpendicular bisector of $BD$ hit $\Gamma$ at $F,F'$ and the perpendicular bisector of $CE$ hit $\Gamma$ at $G,G'$. Since $B$ is the reflection of $D$ over $FF'$, $D$ is the orthocenter of $\triangle{AFF'}$. Similarly, $E$ is the orthocenter of $\triangle{AGG'}$, so if $O$ is the center of $\Gamma$, then the distances from $O$ to $FF'$ and $GG'$ are the same (equal to $\frac{1}{2}AD=\frac{1}{2}AE$). It follows that $FF'=GG'$, so $FG\parallel F'G'$.

Work in the complex plane with $\Gamma$ as the unit circle, setting $a=x^2,b=y^2,c=z^2$ such that the midpoint of minor arc $BC$ (call it $M_A$) is at $-yz$. Then $ff'+ab=0$ and $gg'+ac=0$, so $fgf'g'=a^2bc=x^4y^2z^2$. Since $FG\parallel F'G'$, we have that $fg=f'g'$, so $fg=\pm x^2yz$. The two cases correspond to $FG\perp AM_A$ and $FG\parallel AM_A$. Since $F$ and $G$ are on minor arcs $AB$ and $AC$ while $M_A$ is on minor arc $BC$, $FG\parallel AM_A$ is impossible, so $FG\perp AM_A$. Since $AM_A\perp DE$, we have that $DE\parallel FG$.