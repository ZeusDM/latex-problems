Three circular arcs $\gamma_1, \gamma_2,$ and $\gamma_3$ connect the points $A$ and $C$.
These arcs lie in the same half-plane defined by line $AC$ in such a way that arc $\gamma_2$ lies between the arcs $\gamma_1$ and $\gamma_3.$
Point $B$ lies on the segment $AC.$
Let $h_1, h_2$, and $h_3$ be three rays starting at $B,$ lying in the same half-plane, $h_2$ being between $h_1$ and $h_3.$ For $i, j = 1, 2, 3,$ denote by $V_{ij}$ the point of intersection of $h_i$ and $\gamma_j$ (see the Figure below).
Denote by $\widehat{V_{ij}V_{kj}}\widehat{V_{kl}V_{il}}$ the curved quadrilateral, whose sides are the segments $V_{ij}V_{il},$ $V_{kj}V_{kl}$ and arcs $V_{ij}V_{kj}$ and $V_{il}V_{kl}$.
We say that this quadrilateral is $circumscribed$ if there exists a circle touching these two segments and two arcs.
Prove that if the curved quadrilaterals $\widehat{V_{11}V_{21}}\widehat{V_{22}V_{12}}, \widehat{V_{12}V_{22}}\widehat{V_{23}V_{13}},\widehat{V_{21}V_{31}}\widehat{V_{32}V_{22}}$ are circumscribed, then the curved quadrilateral $\widehat{V_{22}V_{32}}\widehat{V_{33}V_{23}}$ is circumscribed, too.
