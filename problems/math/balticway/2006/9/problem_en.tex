To every vertex of a regular pentagon a real number is assigned. We may perform the following operation repeatedly: we choose two adjacent vertices of the pentagon and replace each of the two numbers assigned to these vertices by their arithmetic mean. Is it always possible to obtain the position in which all five numbers are zeroes, given that in the initial position the sum of all five numbers is equal to zero?