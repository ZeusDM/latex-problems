Nas condições ambiente, ao inspirar, puxamos para nossos pulmões aproximadamente, $0,5$ L de ar, então aquecido na temperatura ambiente de $25^\circ$C até a temperatura do corpo de $36^\circ$C.
Fazemos isso cerca de $16 \times 10^3$ vezes em $24$ horas.
Se, nesse tempo, recebermos por meio da alimentação, $1,0 \times 10^7$ J de energia, a porcentagem aproximada desta energia que será gasta para aquecer o ar inspirado será de:

Ar atmosférico nas condições ambiente:
densidade = $1,2$ g/L, calor específico = $1,0$ J/(g$^\circ$C)

\begin{enumerate}[label=(\alph*)]
	\item $3,0$  \%			
	\item $2,0$  \%			
	\item $1,0$  \%		
	\item $10,0$ \%
	\item $15,0$ \%
\end{enumerate}
