Nas condições ambiente, ao inspirar, puxamos para nossos pulmões aproximadamente, \SI{0,5}{\liter} de ar, então aquecido na temperatura ambiente de \SI{25}{\celsius} até a temperatura do corpo de \SI{36}{\celsius}.
Fazemos isso cerca de \num{16e3} vezes em \num{24} horas.
Se, nesse tempo, recebermos por meio da alimentação, \SI{1,0e6}{\joule} de energia, a porcentagem aproximada desta energia que será gasta para aquecer o ar inspirado será de:

Ar atmosférico nas condições ambiente:
densidade = \SI{1,2}{\gram\per\liter}, calor específico = \SI{1,0}{\joule\per\gram\per\celsius}.

\begin{enumerate}[label = (\scalealph{\alph*})]
	\item \SI{3,0}{\percent}
	\item \SI{2,0}{\percent}
	\item \SI{1,0}{\percent}
	\item \SI{10,0}{\percent}
	\item \SI{15,0}{\percent}
\end{enumerate}
