Com relação às equações iônicas abaixo:

\begin{enumerate}[label = (\scaleroman{\roman*})]

	\item
		\schemestart
		\chemfig{Fe^{3+}} + \chemfig{Cu} \arrow{->} \chemfig{Fe^{2+}} + \chemfig{Cu^{+}}
		\schemestop

	\item
		\schemestart
		2\chemfig{I^{-}} + \chemfig{Br_2} \arrow{->} \chemfig{I_2} + 2\chemfig{Br^{-}}
		\schemestop

	\item
		\schemestart
		3\chemfig{I_2} + 6\chemfig{OH^{-}} \arrow{->} 5\chemfig{I^{-}} + \chemfig{IO_3^{-}} + \chemfig{H_2O}
		\schemestop
\end{enumerate}

Estão corretas:

\begin{enumerate}[label = (\scalealph{\alph*})]
	\item todas				
	\item apenas (\scaleroman{i}) e (\scaleroman{ii})			
	\item apenas (\scaleroman{i}) e (\scaleroman{iii})		
	\item apenas (\scaleroman{ii}) e (\scaleroman{iii})
	\item nenhuma
\end{enumerate}
