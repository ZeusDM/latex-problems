O crescimento e o desenvolvimento normal das plantas exigem a presença
de vários minerais entre os quais os chamados macronutrientes (nitrogênio,
fósforo e potássio) são particularmente importantes. Estes macronutrientes
podem ser fornecidos sob a forma de “um composto fertilizante” ou “NPK”, tipo
\chemfig{NH_4H_2PO_4} + \chemfig{{{(NH_4)}_2}PO_4} + \chemfig{KNO_3}.
De acordo com as normas da agroindústria, cada $1,0$ m$^2$ de solo recém-preparado deve conter $5,0$ g de nitrogênio, $5,0$ g de fósforo e $4,0$ g de potássio.

\begin{enumerate}[label=(\alph*)]
	\item Calcule a composição percentual em massa de uma mistura de nitrato de potássio e fosfato de amônio que seria ideal para atender os requisitos acima.
	\item Uma pequena fazenda não tem o fertilizante NPK, mas tem em estoque outros produtos químicos, incluindo \chemfig{KCl}, \chemfig{NaNO_3}, \chemfig{NH_4NO_3}, \chemfig{CaHPO_4 \cdot 2H_2O}, \chemfig{Ca{(H_2PO_4})_2 \cdot H_2O}.
		Quais destes compostos e em que medida devem ser combinados para preparar fertilizante NPK em quantidade suficiente para tratar $30$ ha?
		Suponha que cada um dos ingredientes listados contém 2\% de impureza em massa.
		Encontrar uma solução ótima, ou seja, a composição que minimiza a massa total da mistura e, portanto, reduz custos de transporte.
\end{enumerate}
