Um elemento X ocorre na forma moléculas diatômicas, \chemfig{X_2} , com massas 70, 72 e 74 e abundâncias relativas na razão de $9 : 6 : 1$, respectivamente.
Com base nessas informações analise as afirmações abaixo:

\begin{enumerate}[label = (\scaleroman{\roman*})]
\item o elemento X possui três isótopos.
\item a massa atômica média desse elemento é 36.
\item esse elemento possui um isótopo de massa 35 com abundância de \SI{75}{\percent}.
\item esse elemento é o cloro.
\end{enumerate}

Estão corretas:

\begin{enumerate}[label = (\scalealph{\alph*})]
	\item todas as afirmações
	\item apenas (\scaleroman{i}) e (\scaleroman{ii})
	\item apenas (\scaleroman{ii}) e (\scaleroman{iv})
	\item apenas (\scaleroman{iii}) e (\scaleroman{iv})
	\item apenas (\scaleroman{i})
\end{enumerate}
