Realizaram-se dois experimentos de combustão de uma amostra de \SI{1}{\gram} de magnésio para avaliar o rendimento do óxido de magnésio produzido:
o primeiro em oxigênio puro e o segundo ao ar.
No primeiro experimento observou-se um acréscimo de \SI{0,64}{\gram} no peso da amostra, enquanto que no segundo, aumentou menos que \SI{0,64}{\gram} no peso da amostra.

Essa diferença ocorreu por que:

\begin{enumerate}[label = (\scalealph{\alph*})]
	\item a combustão ao ar é incompleta.
	\item houve um erro na pesagem do produto do segundo experimento.
	\item a combustão ao ar leva à formação de sub-produtos.
	\item o magnésio reage com o \chemfig{CO_2} presente no ar.
	\item parte do óxido formado foi consumido na reação reversível.
\end{enumerate}

