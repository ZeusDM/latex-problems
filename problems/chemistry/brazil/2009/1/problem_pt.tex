O gás \chemfig{SO_2} é formado na queima de combustíveis fósseis.
Sua liberação na atmosfera é um grave problema ambiental, pois através de uma série de reações ele irá se transformar em \chemfig{H_2SO_4{(aq)}}, um ácido muito corrosivo, no fenômeno conhecido como chuva ácida.
A sua formação pode ser simplificadamente representada por:

\begin{center}

\schemestart
\chemfig{S{(s)}} + \chemfig{O_2{(g)}} \arrow{->} \chemfig{SO_2{(g)}}
\schemestop

\end{center}

Quantas toneladas de dióxido de enxofre serão formadas caso ocorra a queima
de uma tonelada de enxofre? (dados \chemfig{S} = 32 g/mol e \chemfig{O} = 16 g/mol)

\begin{enumerate}[label = (\alph*)]
	\item 1 tonelada		
	\item 2 toneladas
	\item 3 toneladas	
	\item 4 toneladas		
	\item 5 toneladas
\end{enumerate}
