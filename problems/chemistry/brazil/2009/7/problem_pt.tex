Os produtos da combustão do \chemfig{H_2S{(g)}} são \chemfig{H_2O{(g)}} e \chemfig{SO_2{(g)}}.
Usando as informações dadas nas equações termoquímicas abaixo:

\begin{center}

\begin{tabular}{cc}

\schemestart
\chemfig{H_2{(g)}} + \chemfig{S{(s)}} \arrow{->} \chemfig{H_2S{(g)}}
\schemestop
&  $\Delta H = \SI{-21}{\kilo\joule}$

\\

\schemestart
\chemfig{S{(s)}} + \chemfig{O_2{(g)}} \arrow{->} \chemfig{SO_2{(g)}}
\schemestop
& $\Delta H = \SI{-297}{\kilo\joule}$

\\

\schemestart
\chemfig{H_2{(g)}} + $\frac{1}{2}$\chemfig{O_2{(g)}} \arrow \chemfig{H_2O{(g)}}
\schemestop
& $\Delta H = \SI{-242}{\kilo\joule}$

\end{tabular}

\end{center}

Conclui-se que a energia desprendida na combustão de \SI{1}{\mol} de \chemfig{H_2S{(g)}} é:

\begin{enumerate}[label = (\scalealph{\alph*})]
	\item \SI{-67}{\kilo\joule}
	\item \SI{34}{\kilo\joule}	
	\item \SI{-560}{\kilo\joule}
	\item \SI{-34}{\kilo\joule}
	\item \SI{-518}{\kilo\joule}
\end{enumerate}
