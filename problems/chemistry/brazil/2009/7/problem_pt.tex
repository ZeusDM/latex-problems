Os produtos da combustão do \chemfig{H_2S{(g)}} são \chemfig{H_2O{(g)}} e \chemfig{SO_2{(g)}}.
Usando as informações dadas nas equações termoquímicas abaixo:

\begin{center}

\schemestart
\chemfig{H_2{(g)}} + \chemfig{S{(s)}} \arrow{->} \chemfig{H_2S{(g)}} \qquad $\Delta H = - 21$ kJ
\schemestop

\schemestart
\chemfig{S{(s)}} + \chemfig{O_2{(g)}} \arrow{->} \chemfig{SO_2{(g)}} \qquad $\Delta H = -297$ kJ
\schemestop

\schemestart
\chemfig{H_2{(g)}} + $\dfrac{1}{2}$\chemfig{O_2{(g)}} \arrow \chemfig{H_2O{(g)}} \qquad $\Delta H = -242$ kJ
\schemestop

\end{center}

Conclui-se que a energia desprendida na combustão de $1$ mol de \chemfig{H_2S{(g)}} é:

\begin{enumerate}[label = (\alph*)]
	\item $-67$ kJ
	\item $34$ kJ	
	\item $-560$ kJ
	\item $-34$ kJ
	\item $-518$ kJ
\end{enumerate}
