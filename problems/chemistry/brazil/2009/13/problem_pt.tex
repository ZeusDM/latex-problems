A reação de metanol a partir de hidrogênio e monóxido de carbono (equação seguinte) é exotérmica:

\begin{center}
\schemestart
\chemfig{CO{(g)}} + 2\chemfig{H_2{(g)}} \arrow{<=>} \chemfig{CH_3OH{(g)}}
\schemestop
\end{center}

Essa reação está em equilíbrio a $500$ K e $10$ bar. Assumindo que todos os gases são ideais, prediga as mudanças observadas nos valores de:

\begin{enumerate}[label = (\scalealph{\alph*})]
	\item Kp.
	\item pressão parcial de \chemfig{CH_3OH{(g)}}.
	\item número de mols de \chemfig{CH_3OH{(g)}}.
	\item fração molar de \chemfig{CH_3OH{(g)}}.
\end{enumerate}

Quando, acontece cada um dos seguintes eventos:
\begin{enumerate}[label = (\scaleroman{\roman*})]
	\item a temperatura é aumentada.
	\item a pressão é aumentada.
	\item um gás inerte é adicionado a volume constante.
	\item \chemfig{CO{(g)}} é adicionado a pressão constante.
\end{enumerate}
