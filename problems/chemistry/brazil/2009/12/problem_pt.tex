O uso do modelo da \textit{Repulsão dos Pares de Elétrons da Camada de Valência} é um bom caminho para predizer a geometria de pequenas moléculas, sem a necessidade de usar modernas teorias e computadores potentes.

\begin{enumerate}[label = (\scalealph{\alph*})]
	\item Usando este modelo prediga as estruturas dos seguintes compostos: difluoreto de xenônio, tetrafluoreto de xenônio, trióxido de xenônio, tetróxido de xenônio, trifluoreto de boro e tetrafluoreto de enxofre.
	\item Em cada caso, explique se a estrutura é ou não é distorcida em relação à geometria ideal.
	\item Represente, em cada caso, os pares de elétrons não ligantes sobre o átomo central se existirem.
	\item Sugira equações para as sínteses dos fluoretos de xenônio mencionados em (a) e para o trióxido de xenônio, este último a partir do hexafluoreto de xenônio.
	\item Explique porque os gases nobres hélio, neônio e argônio não formam tais compostos em similares condições.
\end{enumerate}
