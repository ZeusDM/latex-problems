O p\chemfig{H} do suco gástrico em um indivíduo normal é igual a $2,00$. Porém, devido a certos distúrbios esse valor pode chegar a $1,50$ e a sensação de desconforto causada recebe o nome de azia. Uma das maneiras de restaurar o $p$\chemfig{H} ao nível normal é através da ingestão de antiácidos, como o bicarbonato de sódio. Considerando que o volume de suco gástrico de um indivíduo é $400$ mL, assinale a alternativa que indica a massa de bicarbonato de sódio presente num comprimido de antiácido capaz de restaurar o $p$\chemfig{H} do suco gástrico no volume considerado. 

Dados: $\log 3 = 0,5$

\begin{enumerate}[label = (\alph*)]	
	\item 0,672 g 
	\item 0,267 g   
	\item 0,476 g      
	\item 0,785 g
	\item 1,145 g
\end{enumerate} 
