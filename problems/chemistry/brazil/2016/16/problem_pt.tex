Nitrito de sódio é empregado como aditivo em alimentos tais como bacon, salame, presunto, linguiça e embutidos, para evitar o desenvolvimento do Clostridium botulinum, (causador do botulismo) e para propiciar a cor rósea, característica desses alimentos, uma vez que participam da seguinte reação química:

\schemestart
Mioglobina + \chemfig{NaNO2}   \arrow{->}   mioglobina nitrosa
\schemestop 

(Obs.: a Mioglobina é uma proteína presente na carne, cor vermelho vivo; por sua vez, a mioglobina nitrosa está presente na carne processada, de cor rósea). 

A legislação prevê uma concentração máxima permitida de 0,015 g de \chemfig{NaNO_2}‚ por 100 g do alimento, uma vez que nitritos são considerados mutagênicos, pois no organismo humano reagem com bases nitrogenadas, formando nitrosaminas, que são carcinogênicas. Sendo a mioglobina uma proteína, ela possui átomos de carbono, entre outros. Entre esses átomos de carbono, uma pequena parte corresponde ao carbono-14, radioativo e emissor de partículas Beta ($\beta$). 

\begin{enumerate}[label = (\alph*)]
	\item Quando um desses nuclídeos emite radiação, a estrutura molecular da proteína sofre uma pequena mudança, devida à transmutação de um átomo do elemento carbono em um átomo de outro elemento. Descreva a equação nuclear correspondente: 
	\item Átomos de carbono-14 podem ser obtidos pelo bombardeamento de átomos de nitrogênio da atmosfera por raios cósmicos de alta energia (isto é, prótons, fótons, núcleos pesados, etc). Os raios cósmicos interagem com núcleos presentes na atmosfera, gerando partículas de energia mais baixa, como os nêutrons.  Esses são absorvidos por átomos de nitrogênio-14 e transformam-se em carbono-14. Equacione esse processo nuclear: 
	\item O tempo de meia-vida do carbono-14 é de 5730 anos.  A abundância do carbono-14 em um organismo vivo é de cerca de 10 ppb (partes por bilhão).  Assim, a descoberta de um alimento fossilizado que contenha cerca de 1,25 ppb de carbono 14 pode ter a sua ‘idade’ estimada em quantos anos? Justifique: 
	\item Considerando a meia-vida do item anterior, determine a vida média e a constante cinética do carbono-14:($\ln 2 = 0,693$)
	\item A matéria orgânica viva possui uma relação carbono-14/carbono-12 constante. Se o organismo morre, a razão é alterada com o tempo, de forma exponencial. Em um acidente ecológico, ocorreu uma mortandade de animais, devido a um possível vazamento de produtos químicos orgânicos de uma fábrica próxima àquele meio ambiente.  Como é possível, através das análises pertinentes da relação carbono-14/carbono-12, que a mortandade não ocorrera de causas naturais, mas deveu-se a produtos químicos daquela fábrica?

\end{enumerate}
