O que é matéria, o que é energia, como elas se relacionam? A reflexão humana sobre isso é bem antiga. Quando se delimita essa relação às aplicações tecnológicas utilizadas atualmente, mais especificamente ao tema combustíveis, muito há o que se discutir. Na tabela abaixo, são apresentadas algumas informações de combustíveis utilizados no cotidiano.

\begin{center}
Tabela 1: Entalpia de combustão padrão para alguns combustíveis.

\renewcommand{\arraystretch}{1.5}
\begin{tabular}{|c|c|}
	\hline
	Combustível & $\Delta H^\circ$(kJ/mol) \\
	\hline 
	Carbono (carvão) &- 393,5 \\ 
	\hline
	Metano (gás natural) & - 802 \\ 
	\hline
	Propano (componente do gás de cozinha) &  - 2.220 \\
	\hline
	Butano (componente do gás de cozinha) &  - 2.878 \\
	\hline
	Octano (componente da gasolina) & - 5.471 \\
	\hline
	Etino (acetileno, usado em maçarico) &  - 1.300 \\
	\hline
	Etanol (álcool) &  - 1.368 \\
	\hline
	Hidrogênio gasoso &  - 286 \\
	\hline 
\end{tabular}
\end{center}

Assinale a alternativa que representa a proposição verdadeira.

\begin{enumerate}[label = (\alph*)]
	\item  O \chemfig{C_8H_18} é um líquido nas condições padrão. A combustão completa de um mol desta substância produz mais dióxido de carbono do que a queima de um mol de qualquer outro combustível da tabela 1. 
	\item O etino apresenta menor calor de combustão do que o etanol devido à hibridização $sp^2$ dos átomos de carbono em sua molécula.
	\item Num ambiente fechado, em condições normais de temperatura e pressão o propano entra em equilíbrio produzindo gás hidrogênio e grafita. 
	\item  Dentre os combustíveis da tabela 1, apenas a queima do etanol produz água, devido ao mesmo apresentar hidroxila.
	\item A queima do gás hidrogênio produz gás carbônico e água, uma vez que a queima de qualquer combustível tem como produtos gás carbônico e água.
\end{enumerate}
