A fonte de oxigênio que aciona o motor de combustão interna de um automóvel é o ar. O ar é uma mistura de gases, principalmente, \chemfig{N_2} (~79 $\%$ e \chemfig{O_2} (~21 $\%$). No cilindro de um motor de automóvel, o nitrogênio pode reagir com o oxigênio para produzir o gás de óxido nítrico, \chemfig{NO}. Como o \chemfig{NO} é emitido a partir do tubo de escape do carro, ele pode reagir com mais oxigênio para produzir gás de dióxido de nitrogênio. 
\begin{enumerate}[label = (\alph*)]
	\item Apresente as estruturas de Lewis (representação por pontos) para o óxido de nitrogênio e dióxido de nitrogênio. Qual é a geometria e hibridação sobre o átomo N? Justifique a sua resposta. 
	\item Escreva as equações químicas balanceadas para ambas as reações. 
	\item Tanto o óxido de nitrogênio e dióxido de nitrogênio são poluentes que podem levar à chuva ácida e aquecimento global; coletivamente, eles são chamados de gases “NOx”. Em 2007, os Estados Unidos emitiram aproximadamente 22 milhões de toneladas de dióxido de nitrogênio na atmosfera. Considere que a reação do nitrogênio e oxigênio seja completa e estime quantos gramas de \chemfig{O_2} foram consumidos para isso. 
	\item Os termos chuva ácida e aquecimento global foram citados no item (c). Com base em seus conhecimentos defina com clareza esses respectivos termos. Além dos gases Nox, quais os outros gases que conjuntamente são responsáveis pela chuva ácida? Justifique sua resposta. 
	\item  A produção dos gases NOX é uma reação lateral indesejada do principal processo de combustão do motor que transforma octano (\chemfig{C_8H_18}), em \chemfig{CO_2} e água. Se 85 $\%$ do oxigênio em um motor são usados para fazer a combustão do octano e o restante usado para produzir o dióxido de nitrogênio, calcule quantos gramas de dióxido de nitrogênio seriam produzidos durante a combustão de 500 gramas de octano.
\end{enumerate}