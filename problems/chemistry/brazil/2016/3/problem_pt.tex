A eletrólise é um processo químico não espontâneo aplicado em diversas etapas de fabricação de produtos. Para realizar a eletrólise da água é necessário fornecer certa quantidade de energia através de uma fonte de energia elétrica. Porém, como a água pura é um mau condutor de corrente elétrica, faz-se necessário adicionar uma pequena quantidade de \chemfig{K_2SO_4} para tornar o meio condutor. Com base nas semirreações a seguir, assinale a alternativa que indica a quantidade de energia que a bateria deve fornecer para decompor $1,0$ mol de água?7

\begin{center}
\schemestart 
\chemfig{O_2}(g) + 4 \chemfig{H^{+}}(aq) + 4 \chemfig{e^{-}} \arrow{->} 2 \chemfig{H_2O}(l) \qquad $E^\circ = +1,23$ V 
\schemestop
\schemestart
2 \chemfig{H_2O}(l) + 2 \chemfig{e^{-}} \arrow{->} \chemfig{H_2}(g) + 2 \chemfig{OH^{-}}(aq) \qquad $E^\circ = -0,83$
\schemestop
\end{center}

Dados: Constante de Avogadro = $6,02 \times 10^{23}$ mol$^{-1}$
  Carga elementar $1,60 \times 10^{-19}$ C 
  Constate de Faraday = $9,65 \times 10^{4}$ C

\begin{enumerate}[label = (\alph*)]
	
	\item $\ge +397,6$ kJ/mol 
	\item $\le -397,6$ kJ/mol
	\item $\le -795,2$ kJ/mol 
	\item $\ge +795,2$ kJ/mol 
	\item $\ge +198,8$ kJ/mol
\end{enumerate}
