Um estudante formulou as proposições abaixo:

\begin{enumerate}[label = (\Roman*)]
	\item No estado sólido, as ligações de hidrogênio presentes na água sofrem um rearranjo, resultando em efeitos estruturais que conferem menor densidade ao estado sólido do que ao líquido. 
	\item Quanto maior for a eletronegatividade do átomo ligado ao hidrogênio na molécula, maior será a densidade de carga negativa no hidrogênio e, portanto, mais fraca será a interação com a extremidade positiva de outras moléculas.
	\item As temperaturas de ebulição do tetraclorometano (\chemfig{CCl_4}) e metano (\chemfig{CH_4}) são iguais a + 77 $^\circ$C  e - 164 $^\circ$C, respectivamente; logo, a energia necessária para quebrar as ligações C - Cl é maior que aquela necessária para quebrar as ligações C - H. 
	\item Pesquisando os dados referentes à temperatura de ebulição e à massa molar de algumas substâncias, o estudante construiu a seguinte tabela:
\end{enumerate}

\begin{center}
\renewcommand{\arraystretch}{1.5}
\begin{tabular}{|c|c|c|}
	\hline
	Substância & $T_e$($^\circ$C) & Massa molar (g/mol) \\
	\hline
	\chemfig{H_2O} & 100 & 18,0 \\
	\hline
	\chemfig{H_2S} & -50 & 34,0 \\
	\hline
	\chemfig{H_2Se} & -35 & 81,0 \\
	\hline
	\chemfig{H_2Te} & -20 & 129,6 \\
	\hline
\end{tabular}
\end{center}

O estudante, ao verificar que a água apresenta temperatura superior às demais substâncias, concluiu que essa observação pode ser explicada pelo aumento das massas molares e das interações intermoleculares, respectivamente. 

Assinale a alternativa que indica as proposições corretas:

\begin{enumerate}[label = (\alph*)]
	
	\item I e IV 
	\item I, II e III 
	\item II, III e IV
	\item I, II, III e IV 
	\item I, II e IV

\end{enumerate}
