A ebulioscopia é uma técnica utilizada para a determinação da massa molar de substâncias desconhecidas. As substâncias moleculares são dissolvidas em solventes como benzeno, hexano ou tetracloreto de carbono, e em função do efeito coligativo a massa molar é determinada. Num determinado ensaio de laboratório, um técnico dissolveu 2,0 g de uma substância desconhecida (não iônica) em 63 mL de \chemfig{CCl_4}. Considerando os dados abaixo e a temperatura de ebulição da solução de 77$^\circ$C, assinale a alternativa que indica a massa molar aproximada da substância dissolvida.  
Dados:\begin{enumerate}[label =]
	
	\item $T_f = 250$ K
	\item $T_e = 349,5$ K
	\item Densidade (\chemfig{CCl_4}) $= 1,59 kg/L a 20^\circ C$
	\item $K_c = 29,8$ K$\cdot$kg$\cdot$mol$^{-1}$
	\item $K_e = 5,00$ K$\cdot$kg$\cdot$mol$^{-1}$
\end{enumerate}

\begin{enumerate}[label = (\alph*)]
	
	\item 200 g/mol
	\item 250 g/mol
	\item 90 g/mol
	\item 100 g/mol
	\item 80 g/mol
\end{enumerate}
