O zinco (do alemão Zink; Zn) é um elemento químico essencial para o nosso organismo, pois é responsável por inúmeras funções, como a síntese de proteínas, o funcionamento de alguns hormônios, o bom funcionamento do sistema imunológico e também do reprodutor. O zinco metálico pode ser obtido a partir de óxido de zinco, \chemfig{ZnO}, pela reação a alta temperatura com o monóxido de carbono, \chemfig{CO}. 

\begin{center}
\schemestart
\chemfig{ZnO}(s) + \chemfig{CO}(g) \arrow{->} \chemfig{Zn}(s) + \chemfig{CO_2}(g)
\schemestop
\end{center}

O monóxido de carbono é obtido a partir de carbono. 

\begin{center}
\schemestart
2\chemfig{C}(s) + \chemfig{O_2}(g) \arrow{->} 2\chemfig{CO}(g)
\schemestop
\end{center}

Assinale a alternativa que indica a quantidade máxima de zinco (em gramas) que pode ser obtido a partir de uma amostra de 75,0 g de óxido de zinco com pureza de 87 \% e 10,0 g de carbono.

\begin{enumerate}[label = (\alph*)]
	
	\item 52,4 
	\item 35,3 
	\item 54,4 
	\item 36,6 
	\item 65,3

\end{enumerate} 
