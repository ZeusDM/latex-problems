Quando a amônia se dissolve em água, ela ioniza e estabelece o equilíbrio a seguir: 
\begin{center}

\schemestart
\chemfig{NH_3}(g) + \chemfig{H_2O}{l} \arrow{->} \chemfig{NH_4^+} + \chemfig{OH^{-}}(aq)
\schemestop
\end{center}

Uma solução de amoníaco foi preparada dissolvendo 0,04 mol de amônia em 200 mL de água sem que nenhuma variação de volume fosse observada e o pH da solução foi 11,3.
Se um sal com um íon comum (por exemplo, cloreto de amônio) for adicionado ao sistema, o equilíbrio irá se deslocar até que se restabeleça uma nova situação de equilíbrio.
Por apresentar um odor relativamente forte e irritante enquanto o equilíbrio estiver sendo restabelecido o odor da amônia ficará mais evidenciado.
Diante da situação apresentada, responda aos itens a seguir: 

\begin{enumerate}[label = (\alph*)]
	\item Qual o valor do grau de ionização e da constante de ionização da amônia? 
	\item Ao adicionar o cloreto de amônio o equilíbrio sofreu uma perturbação. Para qual lado o equilíbrio se deslocou, explique utilizando o Princípio de Le Chatelier. 
	\item Como se chama a solução resultante após a adição do sal? Expliqu
	\item Se a quantidade de sal adicionada foi 1,07 g, qual o novo pH da solução? 
	\item Quais são os pares conjugados e a geometria das espécies químicas nitrogenadas na equação inicial?
\end{enumerate}

Dados:
\begin{enumerate}[label = --]
	\item $\log 2 = 0,3$
	\item $\log 3 = 0,5$
	\item $\log 5 = 0,7$
	\item $K_b=2,00 \times 10^{-5}$
\end{enumerate}
