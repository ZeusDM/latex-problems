Atualmente, muitos suplementos alimentares contêm substâncias que beneficiam naturalmente a produção do óxido nítrico no organismo. Como fármaco, a produção de óxido nítrico se inicia com a reação entre dióxido de enxofre, ácido nítrico e água, originando, além desse gás, o ácido sulfúrico. Como produto final, o óxido nítrico é comercializado em cilindros de 32 litros, diluído em nitrogênio com uma concentração máxima de 0,08 $\%$ em massa e chega a fornecer cerca de 4.800 litros de gás a 25 $^\circ$C e 1 atmosfera. 

\begin{enumerate}[label = (\alph*)]
	\item Escreva a equação química da reação de produção do \chemfig{NO}.
	\item Qual é a massa aproximada de \chemfig{NO} contida no cilindro à qual se refere o enunciado da questão?
	\item Determine a densidade do óxido nítrico em relação ao ar e ao dióxido de enxofre. 
	\item A densidade de um gás X, em relação ao dióxido de enxofre, é 2. Nas mesmas condições de temperatura e pressão, determine a massa molecular de X. 
	\item  Em um recipiente fechado foram colocados 2 mols de \chemfig{NO}(g), 4 mols de \chemfig{SO_2}(g) e 4 mols de \chemfig{H_2}(g) sem que pudessem reagir entre si. Tendo conhecimento que o volume total ocupado foi de 22,0 L e que a temperatura foi mantida a 0 $^\circ$C, calcule as frações molares e a pressão total exercida pela mistura. 
\end{enumerate}

Dados: R = $0,082$ atm$\cdot$L$\cdot$K$^{-1}$$\cdot$mol$^{-1}$