A ingestão de alimentos gordurosos pode causar uma elevação no índice de colesterol no indivíduo e, como conseqüência, geram-se obstruções nas artérias. Um dos exames mais utilizados para verificar tais obstruções é a cintilografia do miocárdio. Para realizá-lo, o paciente recebe uma dose de contraste que contém tecnécio metaestável (Tc-99). Esse isótopo emite radiação gama, com uma constante de decaimento igual a $3,2 \times 10^{-5}$ s$^{-1}$. Considerando um paciente que recebeu uma quantidade de contraste às 14 horas de uma segunda feira, e sabendo que após 8 meias-vidas a radiação volta ao nível seguro, assinale a alternativa que indica em qual dia da semana e hora isto irá acontecer com o paciente. 
Dados: $\ln 2 = 0,693$

\begin{enumerate}[label = (\alph*)]
	
	\item 14 horas da quarta-feira 
	\item 08 horas da manhã da quarta-feira 
	\item 14 horas da quinta-feira 
	\item 12 horas da quarta-feira 
	\item 20 horas da quarta-feira
\end{enumerate}
