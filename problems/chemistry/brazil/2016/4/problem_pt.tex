Quando uma pequena quantidade de íons \chemfig{H^+} ou \chemfig{OH^{-}} é adicionada à água destilada a 25$^\circ$C, ocorrem variações no pH. Considere que um pequeno cristal de \chemfig{NaOH} de massa igual a 0,4 micrograma foi adicionado a 1,0 litro de água destilada. Essa quantidade é tão pequena que não ocorre variação de volume. Mesmo assim, é capaz de modificar o pH da água pura. Assinale a alternativa que indica o valor do novo pH: 
Dados: $K_w= 1 \times 10^{-14}$ a 25$^\circ$C 
$\log 1,1 = 0.04$

\begin{enumerate}[label = (\alph*)]	
	\item $7,04$
	\item $5,96$
	\item $6,00$
	\item $7,02$
	\item $8,04$
\end{enumerate}
