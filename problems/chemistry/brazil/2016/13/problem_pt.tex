Em uma atividade experimental de Química, um grupo de alunos estudou o comportamento ácido/base de diversas substâncias. Os resultados obtidos com os experimentos estão sumarizados no quadro abaixo.

\begin{center}
\renewcommand{\arraystretch}{1.5}
\begin{tabular}{|c|c|c|}
	\hline
	Experimento & Sistema/Solução &  Observação \\ 
	\hline
	1 & \chemfig{NaNO_2}(s) + \chemfig{H_2O}(l) & Formação de uma solução levemente básica \\
	\hline
	2 & \chemfig{K}(s) + \chemfig{H_2O}(l) & Formação de uma solução básica e liberação de um gás. \\
	\hline
	3 & \chemfig{NaH_2PO_4}(s) + \chemfig{H_2O}(l) & Formação de uma solução ácida. \\
	\hline
	4 & \chemfig{HCOOH}(aq) &  pH = 2,40 \\
	\hline
	5 & \chemfig{HCOOH}(aq) & pH = 4,50 \\
	\hline
\end{tabular}
\end{center}

\begin{enumerate}[label = (\alph*)]
	\item Escreva uma equação química que represente o processo ocorrido no experimento 2. 
	\item Justifique o resultado obtido no experimento 3. Complete a sua resposta escrevendo uma equação química que justifique a observação no experimento 3. 
	\item Leia as afirmações que são apresentadas abaixo. Marque (V) para aquelas que julgar verdadeiras e (F) para aquelas que julgar falsas. 
	\begin{enumerate}[label = (\ \ )]
		\item No experimento 1 utilizou-se uma substância que pode ser classificada como uma base de Bronsted-Lowry. 
		\item A diferença de pH observada nos experimentos 4 e 5 pode ser justificada pela força do ácido utilizado.
		\item Uma solução aquosa de \chemfig{H_2SO_4}, na mesma concentração da solução usada no experimento 4, apresenta um valor de pH maior que 2,40. 
		\item O \chemfig{HCOOH} do experimento 4 pode ser classificado como um ácido de Arrhenius.  
\end{enumerate}
	\item Considere a seguinte afirmativa: “nos experimentos 1 e 3, se utilizarmos a mesma massa dos dois sais e o mesmo volume de água, as soluções resultantes apresentarão a mesma temperatura de ebulição.” Indique se esta afirmativa é verdadeira ou falsa e justifique sua resposta.
	\item Com base nos valores de pH observados nos experimentos 4 e 5, determine a diferença de concentração do \chemfig{HCOOH}. 
\end{enumerate}
