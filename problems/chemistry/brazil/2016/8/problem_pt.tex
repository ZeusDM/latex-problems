O que é matéria, o que é energia, como elas se relacionam? A reflexão humana sobre isso é bem antiga. Quando se delimita essa relação às aplicações tecnológicas utilizadas atualmente, mais especificamente ao tema combustíveis, muito há o que se discutir. Na tabela abaixo, são apresentadas algumas informações de combustíveis utilizados no cotidiano.

\begin{center}

Tabela 1: Entalpia de combustão padrão para alguns combustíveis.

\renewcommand{\arraystretch}{1.5}
\begin{tabular}{|c|c|}
	\hline
	Combustível & $\Delta H^\circ$(kJ/mol) \\
	\hline 
	Carbono (carvão) &- 393,5 \\ 
	\hline
	Metano (gás natural) & - 802 \\ 
	\hline
	Propano (componente do gás de cozinha) &  - 2.220 \\
	\hline
	Butano (componente do gás de cozinha) &  - 2.878 \\
	\hline
	Octano (componente da gasolina) & - 5.471 \\
	\hline
	Etino (acetileno, usado em maçarico) &  - 1.300 \\
	\hline
	Etanol (álcool) &  - 1.368 \\
	\hline
	Hidrogênio gasoso &  - 286 \\
	\hline
\end{tabular}
\end{center}

Considere as proposições:

\begin{enumerate}[label = (\Roman*)]
	\item A combustão completa de oito gramas de propano gera mais calor do que a combustão completa de oito gramas de octano.
	\item Um mol de etino contém a mesma quantidade de átomos de hidrogênio do que um mol de gás hidrogênio.
	\item A substância em maior proporção na gasolina é o carbono na forma elementar. 
	\item  A ordem crescente dos pontos de ebulição de etanol, metano e propano é metano < etanol < propano.
\end{enumerate}

	Assinale a alternativa que indica as proposições corretas:

	\begin{enumerate}[label = (\alph*)]
	\item I e II 
	\item I e III 
	\item II e III 
	\item II e IV 
	\item I e IV
\end{enumerate}

