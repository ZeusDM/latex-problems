Um estudante, a pedido de seu professor, precisa preparar 400 mL de uma solução de amônia 5 mol/L. No
rótulo do frasco de amônia, lacrado, que utilizará para preparar sua solução, o estudante observou as
seguintes informações:

\begin{enumerate}[label = --]
	\item Concentração (m/m): $29,0\%$
	\item Densidade: $0,9$ g$\cdot$cm$^{-3}$
	\item Massa molar: $17,02$ g$\cdot$mol$^{-1}$
\end{enumerate}

A partir dessas informações, deduz-se que o volume de solução concentrada, medida pelo estudante, para
preparar a solução solicitada pelo professor foi de:

\begin{enumerate*}[label = (\alph*), itemjoin={\qquad}]
	\item $86,00$ ml
	\item $94,15$ ml
	\item $112,03$ ml
	\item $130,42$ ml
	\item $145,31$ ml
\end{enumerate*}

