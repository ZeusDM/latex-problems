Uma grande diferença entre os elementos do segundo período para os demais é a falta de capacidade de formar um grande número de ligações químicas.
São observados moléculas ou íons como o \chemfig{SiF_6^{2-}}, \chemfig{PF_6^{-}} e \chemfig{SF_6}, mas nenhum análogo é observado para carbono, nitrogênio ou oxigênio.

\begin{enumerate}[label = (\alph*)]
	\item Utilizando de conceitos da Teoria da Ligação de Valência, explique por que os elementos silício, fósforo e enxofre podem fazer um maior número de ligações que o máximo possível para carbono, nitrogênio ou oxigênio.
	Além do \chemfig{SF_6}, o enxofre forma uma vasta série de compostos com o flúor: \chemfig{S_2F_2}, \chemfig{SOF_2}, \chemfig{SF_4}, \chemfig{SOF_4} e o \chemfig{S_2F_{10}}.
	\item Existem dois compostos com fórmula química \chemfig{S_2F_2}, um dos exemplos de isomeria mais simples da química inorgânica.
	Escreva a estrutura de Lewis para os dois isômeros.
\end{enumerate}

As moléculas \chemfig{SF_4} e \chemfig{SOF_4} possuem igual número pares de elétrons ao redor do átomo central, para esses pares está prevista uma geometria de bipirâmide trigonal.

\begin{enumerate}[resume*]
	\item Represente espacialmente o arranjo bipirâmide de base trigonal e identifique as posições axiais (ax) e equatorais (eq) em sua representação.
		Defina os ângulos teóricos formados entre as posições equatoriais e entre uma axial e uma equatorial.
	\item O \chemfig{SOF_4} tem, obviamente, uma ligação diferente das demais.
		Represente essa molécula considerando o seu arranjo espacial e explique a sua escolha para a posição dessa ligação.
	\item Entre as moléculas \chemfig{SF_4} e \chemfig{XeF_4}, qual apresentará o menor ângulo entre as ligações? Justifique sua resposta.
\end{enumerate}
