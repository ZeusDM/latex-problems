Certo óxido foi dissolvido em água dando origem a uma solução incolor. Borbulhou-se gás carbônico através da solução sendo observada a formação de precipitado branco.
A mistura foi levada a uma centrífuga e separou-se o sólido do filtrado.
Ao sólido foi acrescentado $5,0$ mL de uma solução de ácido clorídrico $10\%$ (m/v).
Observou-se a liberação de gás e, ao final do processo, o sólido foi inteiramente consumido.

A fórmula química do gás liberado ao acrescentar ácido clorídrico ao sólido é:

\begin{enumerate}[label = (\alph*), itemjoin={\qquad}]
	\item \chemfig{H_2}
	\item \chemfig{Cl_2}
	\item \chemfig{O_2}
	\item \chemfig{CO_2}
	\item \chemfig{H_2O}
\end{enumerate}
