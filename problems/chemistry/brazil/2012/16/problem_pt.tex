Em uma estação padrão de tratamento de água para consumo humano, a água, após sua captação de um rio
ou represa, passa pela seguinte sequencia de processos:

\begin{enumerate}[label = (\roman*)]
	\item Adição de sulfato de alumínio para reagir com a alcalinidade da água e agregar as impurezas dissolvidas e em suspensão na água.
	\item Processo de agitação lenta (mistura lenta) da água para aumentar o tamanho das partículas formadas no processo anterior.
	\item Processo de separação por sedimentação das partículas formadas nos processos anteriores ficando a água superficial límpida.
	\item Processo destinado a remover partículas em suspensão em meio filtrante constituído de areia.
	\item Processo no qual é utilizado cloro para matar os microorganismos patogênicos.
	\item Processo em que é adicionado ácido fluorsilícico.
	\item Adição de uma suspensão de cal hidratada para eliminar a acidez da água.
\end{enumerate}

\begin{enumerate}[label = (\alph*)]
	\item Identifique cada um desses processos.
	\item Escreva a fórmula do ácido fluorsisícico (ácido hexafluorossilícico).
	\item Segundo norma do Ministério da Saúde, o valor máximo permitido de fluoreto em água para consumo
humano é de $1,5$ mg/L. Considerando que o ácido flurossilísico, adiconado à água, é utilizado na forma
de uma solução aquosa a $23\%$, com densidade igual a $1,19$ g/mL, e que todo o flúor presente
é disponibilizado na forma de fluoreto, calcule o volume máximo dessa solução que pode ser
adicionado a cada m$^3$ de água para consumo humano.
\end{enumerate}
