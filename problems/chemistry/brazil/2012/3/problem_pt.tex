Certo óxido foi dissolvido em água dando origem a uma solução incolor. Borbulhou-se gás carbônico através da solução sendo observada a formação de precipitado branco.
A mistura foi levada a uma centrífuga e separou-se o sólido do filtrado.
Ao sólido foi acrescentado $5,0$ mL de uma solução de ácido clorídrico $10\%$ (m/v).
Observou-se a liberação de gás e, ao final do processo, o sólido foi inteiramente consumido.

A quantidade de matéria, em mol, de ácido clorídrico adicionada foi de:

\begin{enumerate}[label = (\alph*), itemjoin={\qquad}]
	\item $1,4 \times 10^{-2}$
	\item $4,2 \times 10^{-2}$
	\item $1,8 \times 10^{-1}$
	\item $2,7 \times 10^{-1}$
	\item $5,0 \times 10^{-1}$ 
\end{enumerate}
