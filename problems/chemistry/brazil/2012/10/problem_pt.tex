A decomposição do \chemfig{N_2O_4} em \chemfig{NO_2} é dada pela seguinte reação:

\begin{center}
	\schemestart
		\chemfig{N_2O_4} \arrow{<=>} 2 \chemfig{NO_2}
	\schemestop
\end{center}

Coloca-se $n$ mols de \chemfig{N_2O_4} em um recipiente de pressão $p$ e temperatura $T$ e espera-se o equilíbrio ser atingido.
Sabendo que o grau de decomposição é $\alpha$, a constante de equilíbrio $Kc$ pode ser expressa como:

\begin{enumerate}[label = (\alph*)]
	\item $Kc = \dfrac{2\alpha}{pRT(n-\alpha)^2}$
	\item $Kc = \dfrac{4p\alpha^2}{RT(n^2 - \alpha^2)}$
	\item $Kc = \dfrac{\alpha}{4pRT(n^2 + \alpha^2)}$
	\item $Kc = \dfrac{4p\alpha}{((RT)(n+a))^2}$
	\item $Kc = \dfrac{4\alpha pRT}{n^2 - \alpha^2}$
\end{enumerate}
