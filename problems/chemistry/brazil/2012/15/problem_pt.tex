A amônia, nas condições ambientes, é um composto gasoso, usado como matéria-prima para diversas
substâncias, por exemplo, na fabricação de fertilizantes agrícolas, explosivos para fins militares, gás de
refrigeração, etc.
É preparada através de síntese direta com gás hidrogênio (processo Haber-Bosch).
Na fabricação de fertilizantes e de explosivos, usa-se um sal, obtido a partir da sua reação com ácido nítrico.
Com relação à amônia:

\begin{enumerate}[label = (\alph*)]
	\item Qual a sua geometria molecular?
	\item Escreva a equação de ionização que ocorre, quando é dissolvida em água, citando o nome comercial da solução obtida.
	\item Ao entrar em contato com gás clorídrico, produz um determinado sal. Qual a cor que a solução aquosa desse sal desenvolverá, na presença de fenolftaleína? Explique.
	\item Escreva as reações de síntese da amônia e da formação de seu sal, conforme texto acima.
\end{enumerate}
