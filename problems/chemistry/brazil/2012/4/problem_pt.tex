A queima de $1,6163$ g uma substância líquida formada apenas por \chemfig{C}, \chemfig{H} e \chemfig{O} em um laboratório de Química formou $1,895$ g de \chemfig{H_2O} e $3,089$ g de \chemfig{CO_2}.
Com base nas informações, podemos concluir que a fórmula empírica da substância queimada é:

\begin{enumerate}[label = (\alph*), itemjoin={\qquad}]
	\item \chemfig{CH_4O}
	\item \chemfig{C_3H_4O_2}
	\item \chemfig{C_2H_4O_2}
	\item \chemfig{C_2H_6O}
	\item \chemfig{C_2H_4O}
\end{enumerate}
