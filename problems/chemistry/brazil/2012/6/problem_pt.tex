Um  acidente	  em	  um	  laboratório	  provocou	
  a	
  intoxicação	
  de	
  um	
  grupo	
  de	
  pessoas	
  por	
  inalação	
  de	
  um	
  gás.	
Um	
   analista	
   coletou	
   uma	
   amostra	
   desse	
   gás	
   e	
   a	
   introduziu	
   em	
   um	
   recipiente	
   inelástico	
   de	
   $1$	
   dm$^3$ ,	
   à	
temperatura	
   de	
   $27 ^\circ$C.	
   A	
   amostra	
   de	
   gás	
   contida	
   no	
   recipiente	
   pesou	
   $1,14$ g
   e a	
   pressão	
   medida	
   no	
recipiente	
  foi	
  de	
  1	
  atm.	
  Assim,	
  pode-se	afirmar que este gás é:	


 Dados: $R  =  0,082$ atm$\cdot$l$\cdot$K$^{-1}\cdot$mol$^{-1}$.
 
 
 \begin{enumerate*}[label = (\alph*), itemjoin={\qquad}]
	\item \chemfig{CO}
	\item \chemfig{C_2H_2}
	\item \chemfig{H_2S}
	\item \chemfig{NO}
	\item \chemfig{NO_2}
\end{enumerate*}
