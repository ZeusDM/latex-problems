Uma fábrica que produz cal (\chemfig{CaO}) necessita reduzir o custo de produção para se manter no mercado com preço competitivo para o produto.
A direção da fábrica solicitou ao departamento técnico o estudo da viabilidade de reduzir a temperatura do forno de calcinação de Carbonato de Cálcio dos atuais 1500 K para 800 K.

\begin{enumerate}[label = (\alph*)]
	\item Considerando apenas o efeito termodinâmico, pergunta-se: O departamento técnico pode aceitar a nova temperatura de calcinação?
	\item Em caso afirmativo, o departamento técnico pode fornecer outra temperatura de operação que proporcioene maior economia?
	\item Em caso negativo, qual é a temperatura mais econômica para se operar o forno de calcinação?
\end{enumerate}

Dados a $25 ^\circ$C:

\renewcommand{\arraystretch}{1.5}
\begin{center}
\begin{tabular}{|c|c|c|}
	\hline
	Substância & $\Delta S$ / J$\cdot$mol$^{-1}\cdot$K$^{-1}$ & $\Delta H^0$ / kJ$\cdot$mol$^{-1}$\\
	\hline
	\chemfig{CaCO_3{(s)}} & $92,2$ & $-1206,9$\\
	\hline
	\chemfig{CaO{(s)}} & $39,8$ & $-635,1$\\
	\hline
	\chemfig{CO_2{(g)}} & $213,6$ & $-393,5$\\
	\hline
\end{tabular}

OBS: desconsidere a variação das propriedades com a temperatura.
\end{center}
