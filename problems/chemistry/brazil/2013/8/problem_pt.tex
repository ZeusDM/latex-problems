À temperatura ambiente e pressão atmosférica ao nível do mar, a água encontra-se na fase líquida.
Ela passa para a fase gasosa, numa temperatura que é $200 ^\circ$C acima daquela que se esperaria, teoricamente, na ausência de ligações de hidrogênio.
Pode-se concluir, portanto:


\begin{enumerate}[label = (\alph*)]
	\item Essas ligações são muito fortes entre átomos de moléculas diferentes; por isso, a água encontra-se na fase líquida nessas condições.
	\item A massa da molécula de água é grande em relação ao seu tamanho; por isso, a água é compactada e torna-se líquida.
	\item A densidade da água é maior que a soma da densidade do gás oxigênio e do gás hidrogênio; por isso, a massa é maior e faz com que as moléculas se aproximem, formando a fase líquida. 
	\item As ligações covalentes polares entre os átomos de hidrogênio e de oxigênio são mais fortes do que as covalentes apolares entre os átomos de hidrogênio; assim, a repulsão é maior do que a atração, formando a fase líquida.
	\item Essas ligações são chamadas de pseudo-iônicas; ao invés de formarem uma fase
sólida, formam uma fase líquida.
\end{enumerate}
