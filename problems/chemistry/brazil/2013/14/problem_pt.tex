O trióxido de boro é um composto utilizado como aditivo da fibra óptica, na produção de vidro de borossilicato, entre outros.
Esse composto é obtido pela desidratação do ácido bórico, porém também é possível consegui-lo a partir das seguintes etapas de reação.

\begin{enumerate}[label = (\roman*)]
	\item 
		\schemestart
		\chemfig{B_2O_3{(s)}} + 3 \chemfig{H_2O{(g)}} \arrow{->} 3 \chemfig{O_2{(g)}} + \chemfig{B_2H_6{(g)}}
		\schemestop

	\item[--] $\Delta H = 2035$ kJ

	\item
		\schemestart
		\chemfig{H_2O{(l)}} \arrow{->} \chemfig{H_2O{(s)}}
		\schemestop

	\item[--] $\Delta H = 44$ kJ

	\item
		\schemestart
		\chemfig{H_2{(g)}} + $\frac{1}{2}$ \chemfig{O_2{(g)}} \arrow{->} \chemfig{H_2O{(l)}}
		\schemestop

	\item[--] $\Delta H = -826$ kJ

	\item
		\schemestart
		2 \chemfig{B{(s)}} + 3 \chemfig{H_2{(g)}} \arrow{->} \chemfig{B_2H_6{(g)}}
		\schemestop

	\item[--] $\Delta H = 36$ kJ
\end{enumerate}

Tendo como base as reações acima e suas respectivas entalpias, calcule a entalpia
geral de formação do trióxido de boro e informe a sua equação global.
