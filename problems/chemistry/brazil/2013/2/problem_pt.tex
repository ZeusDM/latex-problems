Para verificar se o teor de álcool misturado na gasolina está de acordo com os
padrões um teste muito simples é realizado: Em um funil de separação adiciona-se uma quantidade conhecida de água à gasolina a ser testada, agita-se a mistura
até a formação de duas fases, em cima fica a fase rica em gasolina, embaixo fica
a fase rica na mistura água + álcool. Subtrai-se o volume da fase rica em gasolina
do volume inicial da mistura gasolina + álcool e determina-se o teor de álcool na
gasolina. A respeito do texto acima assinale a alternativa correta:

\begin{enumerate}[label = (\alph*)]
	\item A água é completamente miscível no álcool e na gasolina.
	\item O processo de separação de misturas descrito é a filtração.
	\item Quando a água é adicionada forma-se uma mistura homogênea dos três componentes.
	\item O álcool se desloca da gasolina para a água quando esta é adicionada devido a maior afinidade entre as moléculas de álcool e água por serem polares.
	\item As operações realizadas são insuficientes para determinar o teor de álcool na gasolina.
\end{enumerate}
