Uma das preocupações das autoridades policiais durante eventos esportivos é o aumento do consumo de bebidas alcoólicas, principalmente a cerveja.
A cerveja é, em geral, ingerida numa baixa temperatura para mascarar o seu gosto amargo.
Esse gosto é devido à presença do mirceno, que está presente nas folhas de lúpulo, um dos componentes da bebida.
Considerando a sua fórmula estrutural: \chemfig{C_{10}H_{16}}

\begin{center}
	\chemfig{-[:30](-[:90])=[:-30]-[:30]-[:-30]-[:30](=[:90])-[:-30]=[:30]}
\end{center}

Qual o total de ligações sigma ($\sigma$) + pi ($\pi$) presentes no mirceno?

\begin{enumerate}[label = (\alph*), itemjoin={\quad}]
	\item 12
	\item 17
	\item 18
	\item 25
	\item 28
\end{enumerate}

