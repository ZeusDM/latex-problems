Passaram-se muitos séculos, durante os quais o homem foi acumulando observações e experiências.
Diversos processos empíricos foram realizados por distintos cientistas, a fim de contribuir para a construção do modelo atômico atual.
De acordo com o exposto, considere as afirmações abaixo:

\begin{enumerate}[label = (\Roman*)]
	\item A Lei de Lavoisier (Conservação das Massas) e a Lei de Proust (Proporções Definidas) serviram de base de sustentação para a Teoria Atômica de Dalton.
	\item 	A massa atômica do elemento é calculada pela média ponderada do número de massa dos isótopos naturais do elemento multiplicado pela respectiva abundância de cada isótopo na natureza.
	\item O número máximo de elétrons em um subnível é dado pela seguinte expressão matemática:
		$4L + 2$, onde $L$ é o número quântico secundário.
		Enquanto a expressão $N^2$ , onde $N$ é o número quântico principal determina o número de orbitais em um nível.
	\item A distribuição eletrônica do átomo de cobre ($Z = 29$) não obedece ao diagrama de Linus Pauling.
		Logo, pode-se afirmar que o conjunto de números quânticos para o elétron de diferenciação do átomo de cobre é dado por:
		$n = 4$, $L= 2$, $m = 2$ e $s = +\frac{1}{2}$, tendo propriedades ferromagnéticas.
	\item A realização de experiências com descargas elétricas em tubo de vidro fechado contendo gás a baixa pressão produz os raios catódicos que, por sua vez, são constituídos por um feixe de elétrons cuja carga elétrica foi determinada por Milikan.
\end{enumerate}

Estão corretas somente as afirmações:

\begin{enumerate}[label = (\alph*), itemjoin={\qquad}]
	\item I e IV
	\item II, III e V
	\item II, IV e V
	\item III, IV e V
	\item I, II, III e V
\end{enumerate}

