Na natureza encontram-se certos elementos químicos que estabelecem um número de ligações covalente maior do que aquele previsto pela sua configuração eletrônica no estado fundamental.
É o caso do carbono, berílio, boro e iodo, por exemplo.
O carbono que, na maioria dos compostos, estabelece 4 ligações covalentes comuns mesmo tendo apenas 2 elétrons desemparelhados no estado fundamental.
Para explicar o que acontece, foi proposta a Teoria da Hibridização, que consiste na interpenetração de orbitais puros.
Considerando a teoria da repulsão dos pares eletrônicos, as polaridades das ligações, as estruturas moleculares e a hibridação, analise as afirmações abaixo:

\begin{enumerate}[label = (\Roman*)]
	\item Podemos afirmar que a molécula de \chemfig{NH_3} é: polar, piramidal e tem hibridação $sp^3$ enquanto a molécula de \chemfig{CO_2} é apolar, linear e possui hibridação $sp^2$.
	\item Os compostos \chemfig{BF_3}, \chemfig{PCl_5}, \chemfig{BeH_2} e \chemfig{IF_7} são moléculas que desobedecem a regra do octeto e apresentam hibridação, respectivamente: $sp^3$, $sp^3d$, $sp$ e $sp^3d^2$.
	\item No gás acetileno, \chemfig{C_2H_2} muito usado em soldas, ocorre uma ligação sigma (dspsp) entre os átomos de carbono.
	\item Quando colocamos um refrigerante no congelador, por um tempo prolongado, ocorre a expansão de seu conteúdo como consequência da reorganização das moléculas de água em uma estrutura cristalina hexagonal e da formação de ligações de hidrogênio.
\end{enumerate}

Após a análise, assinale a alternativa correta.

\begin{enumerate}[label = (\alph*), itemjoin={\quad}]
	\item III e IV
	\item I, II e IV
	\item II, III e IV
	\item I, III e IV
	\item I, II, III, IV
\end{enumerate}
