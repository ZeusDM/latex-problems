O ácido bórico é frequentemente utilizado como insecticida relativamente atóxico, para matar baratas, cupins, formigas, pulgas e muitos outros insetos.
Pode ser utilizado diretamente sob a forma de pó em pulgas, misturando-o com açúcar de confeiteiro como atrativo para as formigas e baratas.
Sobre esse ácido e suas propriedades, responda:

\begin{enumerate}[label = (\alph*)]
	\item O ácido bórico pode reagir com a água, gerando o ânion tetrahidroxiborato.
		Esse processo é descrito pela seguinte equação:
	
		\begin{center}
		\schemestart
		\chemfig{B{(OH)}_3{(aq)}} + 2 \chemfig{H_2O{(l)}} \arrow{->} \chemfig{H_3O{(aq)}} + \chemfig{B{(OH)}{(aq)}}
		\schemestop
		\end{center}

	Desenhe a estrutura de Lewis para todas as espécies apresentadas, incluindo em sua resposta, a geometria e a hibridização do átomo central em cada uma delas.

	\item Quando o ácido bórico, \chemfig{B{(OH)}_3}, é aquecido acima de 170 $^\circ$C, o mesmo desidrata gerando o ácido metabórico.
		Se o aquecimento continuar até cerca de 300 $^\circ$C, há uma nova desidratação e a consequente formação do ácido pirobórico.
	Caso haja aquecimento adicional, o ácido pirobórico transforma-se em trióxido de boro.
	Escreva as três reações descritas nesta questão e apresente a fórmula estrutural das espécies em negrito.

	\item Outro composto de boro relevante é a borazina de fórmula molecular \chemfig{B_3H_6N_3}.
		Este composto é também chamado de “benzeno inorgânico”, visto que é isoeletrônico e isoestrutural ao benzeno.
		É possível, ainda, observar na estrutura da borazina a presença de um ácido de Lewis e de uma base de Lewis.
		Identifique-os na molécula em questão, utilizando um exemplo como embasamento para sua resposta.
\end{enumerate}
