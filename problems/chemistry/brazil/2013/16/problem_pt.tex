A solubilidade de certos sais em solução pode ser alterada drasticamente pela
adição de agentes complexantes, o que pode gerar sérios problemas ambientais
devido a aumentos de solubilidade de alguns poluentes na presença de resíduos
químicos com propriedades complexantes.

\begin{enumerate}[label = (\alph*)]
	\item Calcule a solubilidade do cianeto de prata, considerando apenas seu equilíbrio de precipitação.
		
		($K_{ps} = 2,2 \times 10^{-16}$)
	
	\item Calcule a solubilidade deste mesmo sal, considerando agora também seu equilíbrio de complexação entre os íons prata e cianeto e a razão $\dfrac{[\chemfig{Ag^+}]}{[\chemfig{Ag{(CN)}^{2-}}]}$.
		
		($K_1 \cdot K_2 = 5,3 \times 10^{18}$)

	\item Calcule novamente a solubilidade deste mesmo sal, e a razão $\dfrac{[\chemfig{Ag^+}]}{[\chemfig{Ag{(CN)}^{2-}}]}$, sabendo que o pH de sua solução saturada é $7,15$, o que não é suficiente para que precipite óxido de prata.
		
		($K_a(\chemfig{HCN}) = 4,0 \times 10^{-10}$)

\end{enumerate}
