A prática de mudar a cor do cabelo é muito comum atualmente, mas ela já é conhecida há mais de 2000 anos.
Os saxões, por exemplo, eram povos que gostavam de tingir suas barbas de cores fortes e diferentes, tais como azul, verde e alaranjado.
Os tipos de coloração podem ser classificados de acordo com as formulações e as substâncias químicas presentes na tintura.
Como um dos principais produtos, destaca-se o hidróxido de amônio, que possui toxicidade aguda mesmo em baixas concentrações e a sua inalação pode causar dificuldades respiratórias.

{\footnotesize (Adaptação de http://www.brasilescola.com/quimica/quimica-no-tingimento-dos-cabelos.htm acessado em 08 de março de 2013)}

Adicionando-se $0,12$ mol de \chemfig{NH_4OH} a $0,09$ mol de \chemfig{NH_4^+} de modo a obter 500 mL de
uma solução tampão.

Dados: $\log 1,8 = 0,26$ e $\log 0,75 = -0,12$.

Se $K_b = 1,8 \times 10^{-5}$, o pH deste tampão é igual a:

\begin{enumerate}[label = (\alph*), itemjoin={\quad}]
	\item 9,38
	\item 4,62
	\item 3,98
	\item 5,44
	\item 6,75
\end{enumerate}
