Uma macromolécula biológica foi isolada de uma fonte natural em quantidade muito pequena e sua massa molar foi determinada como sendo $4,0 \times 10^5$ g$\cdot$mol$^{-1}$.
Para uma solução preparada pela dissolução de $0,8$ mg dessa macromolécula em $10,0$ g de água.

\begin{enumerate}[label = (\alph*)]
	\item Calcule
		\begin{enumerate}[label = (a.\roman*)]
			\item o ponto de congelamento.
			\item a pressão osmótica, a $25 ^\circ$C.
		\end{enumerate}
	\item Suponha que a massa molar dessa macromolécula não fosse conhecida e que se pretendesse calculá-la a partir da determinação da pressão osmótica da solução citada acima e que fosse cometido um erro de $0,1$ torr na medida dessa pressão osmótica, qual seria o valor encontrado para a massa molar da macromolécula?
\end{enumerate}
