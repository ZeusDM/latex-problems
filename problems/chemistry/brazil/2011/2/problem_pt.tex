Até $1982$, a pressão padrão era tomada como uma atmosfera ($1$ atm ou $101325$ Pa) e a temperatura como $0$ $^\circ$C ($273,15$ K) e, portanto, o volume molar de um gás nas CNTP era $22,4$ L/mol.
A partir de 1982, a União Internacional de Química Pura e Aplicada (IUPAC) alterou o valor da pressão padrão, de forma que as novas condições normais de temperatura e pressão passaram a ser:
$t = 0 ^\circ$C ou $T = 273,15$ K e $p = 100.000$ Pa ou $1$ bar.
Assim, o valor recomendado hoje pela IUPAC, para o volume molar é:

\begin{enumerate}[label = (\alph*)]
	\item $v_m = 0,021631$ m$^3$ mol$^{-1}$
	\item $v_m = 0,035845$ m$^3$ mol$^{-1}$
	\item $v_m = 0,022711$ m$^3$ mol$^{-1}$
	\item $v_m = 0,035745$ m$^3$ mol$^{-1}$
	\item $v_m = 0,027211$ m$^3$ mol$^{-1}$
\end{enumerate}
