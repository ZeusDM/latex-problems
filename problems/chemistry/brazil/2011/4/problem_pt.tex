A porcentagem de álcool adicionado à gasolina é regulamentada por Lei, e recentemente foi estabelecido um novo padrão que é de $18$ a $24\%$ (volume/volume).
Quando $50$ mL de água forem misturados a $50$ mL de gasolina comercializada nos postos de serviço com o máximo teor permitido de álcool,será observada a formação de:

\begin{enumerate}[label = (\alph*)]
	\item $100$ mL de uma mistura homogênea.
	\item Duas fases de $50$ mL cada.
	\item Duas fases, sendo a mais densa de $38$ mL.
	\item Duas fases, sendo a mais densa de $62$ mL.
	\item Duas fases, sendo a mais densa de $74$ mL.
\end{enumerate}
