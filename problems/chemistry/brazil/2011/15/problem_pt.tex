Para a reação:
\begin{center}
	\schemestart
	2\chemfig{NO{(g)}} + \chemfig{Br_2{(g)}} \arrow{->} 2 \chemfig{BrNO{(g)}}   
	\schemestop
\end{center}

tem-se um Kc $= 0,21$ L$\cdot$mol$^{-1}$ a $350 ^\circ$C

Sobre esta reação, responda, com justificativa, as questões abaixo.

\begin{enumerate}[label = (\alph*)]
	\item Se $2,0 \times 10^{-3}$ mols de \chemfig{NO}, $4,0 \times 10^{-3}$ mols de \chemfig{Br_2} e $4,0 \times 10^{-3}$ mols de \chemfig{BrNO} são introduzidos em um recipiente de volume igual a $50,0$ mL, $350 ^\circ$C, em que sentido ocorrerá a reação? Justifique sua resposta.
	\item Qual o valor Kp para essa reação a $350 ^\circ$C?
	\item Se o mesmo recipiente contém, no equilíbrio, $1,4 \times 10^{-3}$ mols de \chemfig{NO} e $1,4 \cdot 10^{-4}$ mols de \chemfig{BrNO} a $350 ^\circ$C, que quantidade de \chemfig{Br_2} estará presente?
	\item Se ao sistema descrito em (c) se adiciona um gás inerte, de modo que a pressão total dentro do recipiente passe a ser de $3$ atm, a $350 ^\circ$C:
		\begin{enumerate}[label = (d.\roman*)]
			\item Ocorrerá mudanças nas concentrações dos componentes da mistura?
			\item O equilíbrio será deslocado?
			\item Se ocorre deslocamento, em que sentido será?
		\end{enumerate}
\end{enumerate}
