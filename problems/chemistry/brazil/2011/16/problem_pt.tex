Um técnico de laboratório dispõe de uma solução de \chemfig{NaOH} que não era utilizada há muito tempo, e em cujo rótulo está escrito: \chemfig{NaOH} $0,25$ mol$\cdot$L$^{-1}$.
Como está solução é instável, antes de usá-la o técnico decidiu titular $25,0$ mL dessa solução com uma solução de \chemfig{HCl} $0,25$ mol$\cdot$L$^{-1}$ e gastou $22,5$ mL desta última solução.
Responda:

\begin{enumerate}[label = (\alph*)]
	\item Por que a solução de \chemfig{NaOH} é instável?
	\item A concentração indicada no rótulo está correta?
	\item Como você poderia preparar $250$ mL de uma solução exatamente $0,25$ mol$\cdot$L$^{-1}$ a partir da solução anterior? (considere que você dispõe de água destilada e de \chemfig{NaOH} sólido e que a adição de \chemfig{NaOH} sólido não altera o volume da solução.)
\end{enumerate}

Dados: 
\begin{enumerate}[label = \textbullet]
	\item $R= 0,082$ dm$^3\cdot$atm$\cdot$K$^{-1}\cdot$mol$^{-1}$
	\item Constante crioscópica da água (KC) $= 1,86$ K$\cdot$kg$\cdot$mol$^{-1}$
	\item Massas atomicas aproximadas:

		\begin{enumerate*}[label = , itemjoin={{;\ \ }}]
			\item H = 1,0
			\item C = 12,0
			\item O = 16,0
			\item Na = 23,0
			\item Si = 28,1
			\item Ba = 137,3.
		\end{enumerate*}
	\item Números atômicos:

		\begin{enumerate*}[label = , itemjoin={{;\ \ }}]
			\item H = 1
			\item B = 5
			\item C = 6
			\item N = 7
			\item O = 8
			\item F = 9
			\item P = 15
			\item Cl = 17.
		\end{enumerate*}
\end{enumerate}

