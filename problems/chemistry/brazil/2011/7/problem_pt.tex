O etanol anidro, ou seja, etanol isento de água, pode ser obtido a partir do etanol $96^\circ$ GL por tratamento com cal virgem – \chemfig{CaO}.
A cal virgem reage com a água conforme a equação abaixo, desidratando o etanol.

\begin{center}
	\schemestart
	\chemfig{CaO{(s)}} + \chemfig{H_2O{(\text{dissolvido em álcool})}} \arrow{->} \chemfig{Ca{(OH)}_2{(s)}}
	\schemestop
\end{center}

Sobre esse processo é \textit{correto} afirmar que:

\begin{enumerate}[label = (\alph*)]
	\item o hidróxido de cálcio formado reage com o etanol.
	\item o óxido de cálcio reage com etanol para retirar a água.
	\item o hidróxido de cálcio formado pode ser separado por filtração.
	\item a mistura obtida após a reação é uma mistura homogênea.
	\item o óxido de cálcio atua como um agente redutor.
\end{enumerate}

