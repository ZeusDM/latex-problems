O carbeto de silício (\chemfig{SiC}), também conhecido como carborundum, uma substância dura empregada como abrasivo, pode ser obtido a partir da reação de \chemfig{SiO_2} com carbono, a altas temperaturas, conforme a equação química (não balanceada) abaixo:

\begin{center}
	\schemestart
	\chemfig{SiO_2{(s)}} + \chemfig{C{(s)}} \arrow{->} \chemfig{SiC{(s)}} + \chemfig{CO{(g)}}     
	\schemestop
\end{center}

\begin{enumerate}[label = (\alph*)]
	\item Reescreva a equação química acima, devidamente balanceada.
\end{enumerate}

Em um experimento colocou-se para reagir 6,01 g de SiO 2 e 7,20 g de carbono.

\begin{enumerate}[resume*]
	\item Qual será a massa do reagente limitante?
	\item Que massa de carborundum poderá ser obtida, considerando o consumo completo do reagente (rendimento de 100\%)?
	\item Que massa restará do reagente em excesso?
	\item Se, no experimento acima, obtém-se $2,56$ g de \chemfig{SiC}, qual o rendimento da reação?
\end{enumerate}


