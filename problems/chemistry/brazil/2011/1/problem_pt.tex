É comum encontrarmos objetos que brilham no escuro, particularmente, brinquedos de crianças.
Tais objetos podem apresentar o sulfeto de zinco em sua constituição.
Este fenômeno ocorre em razão de que alguns elétrons destes átomos absorvem energia luminosa e com isso saltam para níveis de energia mais externos.
Esses elétrons retornam aos seus níveis de origem, liberando energia luminosa e, no escuro, é possível observar o objeto brilhar.
Essa característica pode ser explicada considerando o modelo atômico proposto por:

\begin{enumerate}[label = (\alph*)]
	\item Thomson				
	\item Rutherford
	\item Bohr					
	\item Marie Curie
	\item Planck
\end{enumerate}
