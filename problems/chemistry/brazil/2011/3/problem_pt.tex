Dentre os principais fatores que influenciam na produção agropecuária, podemos citar:
o clima, o material genético, o manejo de pragas, as doenças e plantas daninhas e o manejo do solo, com especial ênfase no manejo químico como base para a nutrição das plantas.
Em razão da produção de alimentos em escala cada vez maior, os nutrientes do solo que dão vida às plantas vão se esgotando.
Para supri-los, produtos químicos conhecidos como fertilizantes são incorporados à terra em quantidades crescentes.
Para correção da acidez do solo, o procedimento de rotina é a calagem através da incorporação de um óxido básico.
É correto afirmar que esse óxido básico pode ser:

\begin{enumerate}[label = (\alph*)]
	\item \chemfig{MgO_2}
	\item \chemfig{CaO}
	\item \chemfig{SO_2}
	\item \chemfig{NaO}
	\item \chemfig{CO}
\end{enumerate}
