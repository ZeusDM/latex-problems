Uma amostra de um ácido diprótico pesando 12,25 g foi dissolvida em água e o volume da solução completado para 500 mL.
Se 25,0 mL desta solução são neutralizados com 12,5 mL de uma solução de \chemfig{KOH} $1,00$ mol$\cdot$L$^{-1}$, a massa molar desse ácido, considerando que os dois prótons foram neutralizados, é igual a:

\begin{enumerate}[label = (\alph*)]
	\item 2,25
	\item 24,5
	\item 49,0
	\item 98,0
	\item 122,5
\end{enumerate}

