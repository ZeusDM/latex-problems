A apatita é o nome dado a um grupo de minerais de fórmula geral \chemfig{Ca_5{(PO_4)}_3X}.
Dependo se \chemfig{X} = \chemfig{F}, \chemfig{Cl} ou \chemfig{OH}, tem-se a fluoroapatita, a cloroapatita ou hidroxiapatita, respectivamente.
Todos esses minerais são insolúveis em água, mas, solúveis em ácido mineral.
A hidroxiapatita tem uma densidade de $3,156$ g$\cdot$ml$^-1$ e dissolve em ácido de acordo com a reação, representada pela equação (não balanceada) abaixo:


\begin{center}
\schemestart
	\chemfig{Ca_5{(PO_4)}_3OH} + \chemfig{H^+} \arrow{->} \chemfig{Ca^{2+}} + \chemfig{HPO_4^{2-}} + \chemfig{H_2O}
\schemestop
\end{center}

O esmalte dentário é composto principalmente de hiroxiapatita, e sua espessura na superfície de mastigação dos dentes molares é de cerca de $2,5$ mm.
A cárie é resulta da destruição do esmalte dos dentes, sob a ação de agentes ácidos.
Um dos principais destruidores do esmalte é o ácido láctico (\chemfig{CH_3CH{(OH)}COOH}, ácido 2-hidroxipropanóico), que é produzido após a degradação da sacarose, sob a ação de bactérias contidas na cavidade oral.

\begin{enumerate}[label = (\alph*)]
	\item Reescreva a equação acima, balanceada;
	\item  Quantos miligramas de ácido láctico irão causar uma cárie com área (da cavidade) de $1$ mm$^2$?
\end{enumerate}	

O lactato de cálcio é um sal de ácido lático, que tem ação anti-ácido, podendo neutralizar os ácidos do estômago devido à sua capacidade de converter ácido clorídrico do suco gástrico (ácido forte) no ácido mais fraco e menos irritante, ácido láctico. 

\begin{enumerate}[label = (\alph*)]
	\item[(c)]  Escreva as fórmulas estruturais do ácido lático e do lactato de cálcio;
	\item[(d)]  Qual a área da cavidade dentária que seria necessária para produzir lactato de cálcio em quantidade suficiente para a neutralização do ácido clorídrico contido em $47,5$ mL de suco gástrico, se a concentração de ácido clorídrico no suco gástrico é $0,12$ g / L ?
\end{enumerate}
