A primeira observação da transmutação de um núcleo foi observada por Ernest Rutherford, em 1919 e consistiu na transformação de núcleos de nitrogênio-14 em oxigênio-17.
Sabendo que as reações de transmutação nuclear são representadas por uma equação onde são mencionados, nesta ordem, o núcleo alvo, a partícula projétil, o núcleo remanescente e a partícula ejetada, escreva:

\begin{enumerate}[label = (\alph*)]
	\item A equação nuclear da transmutação de nitrogênio-14 em oxigênio-17.
	\item Escreva a equação da transmutação de alumínio-27 em magnésio-24.
\end{enumerate}

Complete e equilibre as seguintes equações nucleares:

\begin{enumerate}[resume*]
	\item
		\schemestart
		\chemfig{^{252}Cf_{98}} + \chemfig{^{10}B_5} \arrow{->} 3 \chemfig{^1n_0} + \_\_\_\_\_\_
		\schemestop
	\item
		\schemestart
		\chemfig{^{122}I_{53}} \arrow{->} \chemfig{^{122}Xe_{54}} + \_\_\_\_\_\_
		\schemestop
	\item
		\schemestart
		\_\_\_\_\_\_ \arrow{->} \chemfig{^{187}Os_{76}}  +  \chemfig{^1n_0}
		\schemestop
\end{enumerate}
