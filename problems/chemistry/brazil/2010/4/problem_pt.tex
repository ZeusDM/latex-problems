O carbonato de sódio, \chemfig{Na_2CO_3}, é um sal branco e translúcido, usado principalmente na produção de vidro, em sínteses químicas, em sabões e detergentes etc. Em 1791, o químico francês Nicolas Leblanc patenteou um método de produção que utilizava como matérias primas sal marinho (\chemfig{NaCl}), por meio das reações a seguir: 

\begin{center}

\schemestart
$x$\chemfig{NaCl} + \chemfig{H_2SO_4} \arrow{->} \chemfig{M} + 2\chemfig{HCl}
\schemestop

\schemestart
\chemfig{M} + \chemfig{CaCO_3} + 2\chemfig{C} \arrow{->} \chemfig{NaCO_3} +$y$\chemfig{CO_2} + \chemfig{CaS} 
\schemestop

\end{center}

As equações ficarão corretas se $x$, \chemfig{M} e $y$ forem substituídos respectivamente por: 

\begin{enumerate}[label = (\alph*)]
	\item 1, \chemfig{Na_2SO_4} e 1.
	\item 2, \chemfig{Na_2SO_4} e 2.
	\item 1, \chemfig{NaHSO_4} e 1.
	\item 2, \chemfig{NaHSO_4} e 1.
	\item 2, \chemfig{NaHSO_4} e 2.
\end{enumerate}
