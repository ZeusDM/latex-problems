A amônia é uma matéria prima importante para a produção de fertilizantes inorgânicos e pode ser obtida através da reação, representada pela equação abaixo:

\begin{center}
\schemestart
\chemfig{N_2{(g)}} + 3\chemfig{H_2{(g)}} \arrow{<=>} 2\chemfig{NH_3{(g)}}
\schemestop
\end{center}

Se essa reação é realizada em um recipiente fechado, e em determinado instante constata-se a existência de um mesmo número de mols de cada um dos reagentes e do produto, pode-se afirmar que a razão inicial entre os números de mols \chemfig{N_2{(g)}} e \chemfig{H_2{(g)}} era de:

\begin{enumerate}[label = (\alph*)]
	\item $1:1$
	\item $2:3$
	\item $1:3$
	\item $3:1$
	\item $3:5$
\end{enumerate}
