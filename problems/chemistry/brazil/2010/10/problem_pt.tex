Em um laboratório, há 5 frascos idênticos numerados de I a V, eles contêm amostras de gases à mesma temperatura.

\begin{enumerate*}[label =  \Roman*:]
	\item 0,10 mol de H2.
	\item 0,10 mol de N2.
	\item 0,10 mol de O2.
	\item 0,05 mol de NO2.
	\item 0,05 mol de CO2.
\end{enumerate*}

Considerando que todos são gases ideais, assinale a alternativa correta.

\begin{enumerate}[label = (\alph*)]
	
	\item Os frascos I, III e V contêm o mesmo número de átomos. 
	\item Os frascos que contêm as maiores densidades de gás são os frascos IV e V. 
	\item Os frascos II e IV contêm o mesmo número de moléculas. 
	\item A pressão exercida pelos gases dos frascos IV e V é menor do que a pressão exercida pelos outros gases. 
	\item  O frasco IV contém a maior massa de gases.
\end{enumerate}
