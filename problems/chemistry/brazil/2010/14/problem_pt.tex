Considere uma solução de um ácido hipotético \chemfig{H_2X} $0,010$ mol$\cdot$L$^{-1}$ e calcule:

\begin{enumerate}[label = (\alph*)]
	\item O pH dessa solução, admitindo a ionização de apenas 1 próton;
	\item O pH da mesma solução, admitindo que os dois prótons se ionizam com-
pletamente;
\end{enumerate}

Se em um experimento determina-se que o pH de uma solução $0,050$ mol$\cdot$L$^{-1}$ desse ácido é $1,26$:

\begin{enumerate}[resume*]
	\item Compare as forças dos ácidos \chemfig{H_2X} e \chemfig{HX^{-}};
	\item Uma solução do sal \chemfig{NaHX} seria ácida, básica ou neutra?
\end{enumerate}

Dados: $\log 2 = 0,30$; $\log 5 = 0,70$; $\log 6 = 0,77$.
