Uma das fases do processo de tratamento de água é a fluoretação, que tem como objetivo contribuir para a prevenção da cárie dentária.
Um reagente empregado nesse processo é o ácido hexafluorossilícico, também chamado, simplesmente, ácido fluorossilícico (\chemfig{H_2SiF_6}), Segundo norma do Ministério da Saúde, o valor máximo permitido de fluoreto em água para consumo humano é de $1,5$ mg/L.
Assim considerando que o ácido flurossilísico é utilizado na forma de uma solução aquosa de \chemfig{H_2SiF_6} a $23\%$, com densidade igual a $1,19$ g/mL, e que todo o flúor presente é disponibilizado na forma de fluoreto, o volume máximo dessa solução que pode ser adicionado a cada m$^3$ de água para consumo humano está entre: 

\begin{enumerate}[label = (\alph*)]
	\item $5$ e $9$ ml
	\item $9$ e $13$ ml
	\item $13$ e $17$ ml
	\item $17$ e $21$ ml
	\item $21$ e $26$ ml
\end{enumerate}
