O ácido fórmico (\chemfig{HCOOH}) recebe esse nome porque foi obtido pela primeira vez a partir da “destilação destrutiva” de formigas.
Trata-se de um ácido monoprótico, moderadamente fraco, cujo valor de Ka é igual a $1,8 \times 10^{-4}$. Em uma solução de ácido fórmico de concentração igual $1,0 \times 10^{-3}$ mol$\cdot$L$^{-1}$, a porcentagem de moléculas ionizadas está entre:

\begin{enumerate}[label = (\scalealph{\alph*})]
	\item $20$ e $30 \%$
	\item $30$ e $40 \%$
	\item $40$ e $50 \%$
	\item $50$ e $60 \%$
	\item $60$ e $70 \%$
\end{enumerate}
