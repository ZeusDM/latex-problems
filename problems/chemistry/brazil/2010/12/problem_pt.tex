O cromo ocorre na natureza em minérios, tais como as cromitas, constituídas por proporções variadas de óxidos de cromo, ferro, alumínio e magnésio, além de outros elementos em quantidades mínimas, da ordem de ppm, como vanádio, níquel, zinco, titânio, manganês e cobalto.
Em função da composição dos óxidos presentes, se distinguem as seguintes espécies minerais mais importantes:
a cromita propriamente dita, \chemfig{Cr_2O_3}$\cdot$\chemfig{FeO},
a magnesiocromita, \chemfig{{(Mg,Fe)}Cr_2O_4},
a aluminocromita, \chemfig{Fe{(Cr,Al)}_2O_4}
e a cromopicotita, \chemfig{{(Mg, Fe)}{(Cr, Al)}_2O_4}.
A partir da cromita, o cromo metálico pode ser obtido por aquecimento com carvão em forno elétrico, de acordo com a equação química (não balanceada) abaixo:

\begin{center}
\schemestart
\chemfig{Cr_2O_3}$\cdot$\chemfig{FeO{(s)}} + \chemfig{C{(s)}} \arrow{->} \chemfig{Fe{(s)}} + \chemfig{Cr{(s)}} + \chemfig{CO{(g)}}
\schemestop
\end{center}

Por ser um metal resistente aos agentes corrosivos comuns, o cromo é muito empregado no revestimento de peças de outros metais, através de um processo de eletrodeposição, pelo sistema de imersão.
Pode-se aplicar um revestimento de cromo em uma peça metálica por imersão dessa peça em um tanque que há uma solução de dicromato de potássio e aplicação de uma corrente elétrica.

\begin{enumerate}[label = (\alph*)]
	\item indique o estado de oxidação do Cr na cromita propriamente dita
	\item escreva a equação química acima balanceada
\end{enumerate}

Se uma peça de aço é imersa em um tanque que contém $1$ litro de uma solução $0,500$ mol$\cdot$L$^{-1}$ de dicromato de potássio e submetida a uma corrente elétrica de $0,500$ A, durante $20,00$ minutos.

\begin{enumerate}[label = (\alph*)]
	\item[(c)] Qual a massa de cromo que será depositada?
	\item[(d)] Qual será a concentração da solução de dicromato de potássio remanescente?
\end{enumerate}
