O escritor italiano Primo Levi, em seu livro de contos Sistema Periódico (O Sistema Periódico, Ed. Leya), narra fatos de sua vida e os associa a elementos químicos.
Em um dos trechos, ele narra: ``(este metal) é mole como a cera. Reage com a água onde flutua (um metal que flutua!), dançando freneticamente e produzindo hidrogênio.''. Sobre esse metal, assinale a alternativa que está correta:

\begin{enumerate}[label = (\alph*)]
	\item É um metal de transição, caracterizado por sua baixa densidade. A reação química com a água é devida ao baixo potencial de oxidação desse metal, frente ao potencial de oxidação do hidrogênio.
	\item É um metal que possui uma grande variação de número de oxidação (Nox), indo de +1 até +6. Esta grande variação de números de oxidação Nox confere ao elemento baixa densidade e alta reatividade, uma vez que a perda significativa de elétrons altera suas propriedades físicas. 
	\item É um metal radioativo (Z>91). A instabilidade nuclear, observada pela razão entre número de prótons e número de nêutrons, faz com que o elemento tenha uma massa atômica elevada e um volume atômico grande, originando uma densidade menor do que 1,0 g cm$^{-3}$ nas condições ambientes.
	\item É um metal que apresenta propriedades diamagnéticas, por isso sofre a repulsão elétrica dos polos das moléculas de água, que o fazem flutuar. A reação que ocorre é de oxirredução, uma vez que os polos elétricos da água reagem com os polos elétricos do metal
	\item É um metal representativo, que reage com a água formando um hidróxido correspondente, com a liberação de gás hidrogênio. O referido metal possui potencial padrão de redução negativo.
\end{enumerate}
