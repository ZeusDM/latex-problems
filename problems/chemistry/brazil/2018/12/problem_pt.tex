Henry G. J. Moseley (1887-1915), físico inglês, foi declarado por Rutherford como seu aluno mais talentoso. Ele estabeleceu o conceito de número atômico ao estudar os raios X emitidos pelos elementos químicos. Os raios X emitidos por alguns elementos têm os seguintes comprimentos de onda:

\begin{center}
\renewcommand{\arraystretch}{1.5}
\begin{tabular}{c c}
	\hline
	Elemento & Comprimento de Ondas (\AA) \\
	\hline
	Ne & 14,610 \\
	Ca & 3,358 \\
	Zn & 1,435\\
	Zr & 0,786 \\
	Sn & 0,491 \\
	\hline
\end{tabular}
\end{center}

\begin{enumerate}[label = (\alph*)]
	\item Calcule a frequência, $\nu$, dos raios X emitidos por cada um dos elementos, em Hz.

Dados: Constante da velocidade da luz, $c = 3 \times 10^{8}$ m$\cdot$s$^{-1}$.

	\item Desenhe e analise o gráfico da raiz quadrada de $v$ versus número atômico do elemento.
	\item Explique como a curva do item (b) permitiu a Moseley prever a existência de elementos ainda não descobertos.
	\item Use o resultado do item (b) para prever o comprimento de onda dos raios X emitidos pelo elemento ferro. 
	\item Se um determinado elemento emite raios X com um comprimento de onda de $0,980 Å$, qual seria esse elemento?
\end{enumerate}
