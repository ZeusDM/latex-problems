 As histórias em quadrinhos (HQs) e os filmes de série daí derivados cativam milhões de fãs pelo mundo. Muitas vezes, existem erros científicos grosseiros, que fogem da realidade. Por outras vezes, as HQs criam uma ponte entre o conhecimento científico e o dia a dia das pessoas. A franquia Marvel Comics ® tem lançado filmes que se tornam grandes eventos para os aficionados. Dois metais fictícios foram utilizados em seus enredos: o adamantium (usado principalmente no esqueleto e nas garras do Wolverine e nas espadas de Deadpool), e o vibranium, utilizado na armadura do Pantera Negra e no escudo do Capitão América, além de outros usos.

\begin{enumerate}[label = (\alph*)]
	\item Levando-se em conta a atual Tabela Periódica, justifique o fato não ser possível a inserção de uma nova família (ou grupo) na mesma, como, por exemplo, a família 19, situada logo após a família dos gases nobres.
	\item Explique como a temperatura influi na condutividade elétrica de um condutor metálico e de um semicondutor.
	\item Quais as propriedades físicas comumente apresentadas pelos metais? Quais são as características estruturais responsáveis por essas propriedades?
	\item O vibranium apresenta alta resistência mecânica, baixa densidade e é um isolante elétrico. Isso por si só é uma contradição. Explique essa contradição.
	\item O adamantium apresenta uma alta densidade e uma alta resistência mecânica. Conforme a HQ, sabe-se que ele uma vez resfriado, não pode ser mais fundido ou moldado. Justifique a esse fato.
\end{enumerate}