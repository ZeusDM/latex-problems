Segundo o cientista da NASA, James Hansen, a temperatura da Terra alcançou, nas últimas décadas, uma rápida ascensão de cerca de 0,2 $^\circ$C, fenômeno esse que não havia ocorrido desde a última era glacial, há 12.000 anos. Tal aquecimento se explica, conforme o cientista, pelo aumento da emissão de gases estufa. Com base nestes estudos, pode-se afirmar corretamente que são consequências do fenômeno de aquecimento global:

\begin{enumerate}[label = (\Roman*)]
	\item Devastação das florestas e savanas.
	\item Redução do volume das geleiras alpinas e das calotas glaciais.
	\item Maior possibilidade de formações de tempestades e ciclones tanto no Atlântico Norte quanto no Atlântico Sul.
	\item Redução da acidez das chuvas.
	\item Transgressão marinha sobre partes das faixas costeiras.
	\item Rebaixamento do nível dos oceanos e consequente expansão das áreas litorâneas.
	\item Aumento do risco de degradação dos ecossistemas coralíneos.
\end{enumerate}
 
 A alternativa que apresenta apenas as consequências desse fenômeno é:


\begin{enumerate}[label = (\alph*)]
	\item II, III, V e VII, apenas.
	\item I, II, III, IV, VI e VII.
	\item I, III, IV e VI, apenas.
	\item II, IV, VI e VII, apenas.
	\item II, III e VI, apenas.
\end{enumerate}
