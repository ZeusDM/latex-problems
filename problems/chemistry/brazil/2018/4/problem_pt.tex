A formação de um sólido a partir de líquidos e/ou de gases é uma das evidências de que ocorreu uma reação química. Para demostrar tal evidência, um recipiente de 2,5 L que contém amônia gasosa a 0,78 atm e 18,5 $^\circ$C foi conectado em outro recipiente de 1,4 L com cloreto de hidrogênio gasoso a 0,93 atm e 18,5 $^\circ$C, respectivamente. Sabe-se que a combinação desses gases leva a formação de cloreto de amônio sólido, logo, qual a quantidade aproximada de massa formada desse composto, o gás que sobrou nos recipientes conectados e a sua pressão?

\begin{enumerate}[label = (\alph*)]
	\item 1,44 g; amônia; 0,166 atm.
	\item 3,19 g; cloreto de hidrogênio; 0,332 atm.
	\item 1,44 g; cloreto de hidrogênio; 0,332 atm.
	\item 1,44 g; amônia; 0,332 atm.
	\item 2,91 g; amônia; 0,166 atm.
\end{enumerate}
