O grau de dissociação, $\alpha$, é definido como a fração de reagente que se decompõe, quando a quantidade inicial de reagente é $n$ e a quantidade em equilíbrio é $n_{eq}$, então $\alpha = {\frac{n-n_{eq}}{n}}$.
A energia de Gibbs padrão de reação para a decomposição
\schemestart
\chemfig{H_2O}(g) \arrow{->} \chemfig{H_2}(g) + $\frac{1}{2}$\chemfig{O}(g)
\schemestop
é  +118,08 kJ$\cdot$mol$^{–1}$ a 2300 K.


\begin{enumerate}[label = (\alph*)]
	\item Calcule a constante de equilíbrio a partir da energia de Gibbs padrão de reação no equilíbrio pela equação $\Delta G_m^{\circ} = -RT\ln K$
	Dado da constante dos gases: $R = 8,314$ J$\cdot$K$^{-1}\cdot$mol$^{-1}$.
	\item Mostre que $\alpha = \left(\frac{2^{\frac{1}{2}}K}{p^{\frac{1}{2}}}\right)^{\frac{2}{3}}$.
	\item Calcule o grau de dissociação de \chemfig{H_2O} a 2300 K e 1,00 bar.
	\item Calcule as frações molares das substâcias no equilíbrio.
	\item O grau de dissociação aumentará ou diminuirá, se o valor da pressão for duplicado na mesma temperatura? Além do cálculo, qual o efeito que justifica que mantem o equilíbrio?
\end{enumerate}
