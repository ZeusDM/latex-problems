Com base em medidas experimentais e cálculos de mudanças de entalpia, o químico suíço G. H. Hess sugeriu em 1840, que para uma dada reação, a variação de entalpia é sempre a mesma, esteja essa reação ocorrendo em uma ou em várias etapas (Lei de Hess). Para exemplificar, em um experimento para estimar a entalpia molar padrão de formação de butano (a partir de seus elementos), os seguintes valores foram determinados por calorimetria:


\begin{enumerate}[label = (\arabic*)]
	\item
	
	\schemestart
	\chemfig{C_4H_10}(g) + $\frac{13}{2}$\chemfig{O_2}(g) \arrow{->} 4 \chemfig{CO_2}(g) + 5 \chemfig{H_2O}(g) \qquad $\Delta _c H_m^{\circ} = -2657,4 $kJ$\cdot$mol$^{-1}$
	\schemestop

	\item
	\schemestart
	\chemfig{C}(s) + \chemfig{O_2}(g) \arrow{->} \chemfig{CO_2}(g) \qquad $\Delta _f H_m^{\circ} = -393,5 $kJ$\cdot$mol$^{-1}$
	\schemestop
	
	\item
	\schemestart	
	2 \chemfig{H_2}(g) + \chemfig{O_2} \arrow{->} 2 \chemfig{H_2O}(g) \qquad $\Delta _f H_m^{\circ} = -483,6 $kJ$\cdot$mol$^{-1}$
	\schemestop

\end{enumerate}

\begin{enumerate}[label = (\alph*)]
	\item Qual é a variação de entalpia molar padrão de formação do butano?
	\item A partir dos dados das reações de combustão e de formação faça um diagrama de entalpia representando a formação do butano.
	\item A partir dos dados de entalpias de ligação e mudança de estado físico estime a entalpia de formação para 1 mol de butano?
	
	\begin{center}
	\renewcommand{\arraystretch}{1.5}
	\begin{tabular}{ c | c }
	\hline
	Tipo de Ligação & $\Delta_f H_m^{\circ}$ (kJ$\cdot$mol$^{-1}$) \\
	\hline
	\chemfig{C-C} & 346,8 \\
	\hline
	\chemfig{C-H} & 413,4 \\
	\hline
	\chemfig{H-H} & 436,0 \\
	\hline
	Mudança de estado físico & \\
	\hline
	\schemestart \chemfig{C}(grafita,s) \arrow{->} \chemfig{C}(g) \schemestop & 716,7 \\
	\hline
	\end{tabular}
	\end{center}

	\item Qual o erro percentual observado para a entalpia de formação de 1 mol de \chemfig{C_4H_10}(g) estimada no item (a) e (c), respectivamente? A que você atribuiria tal erro?
	\item O butano é o principal componente do GLP (gás liquefeito de petróleo), conhecimento como gás de cozinha. Atualmente, o valor comercial médio de um botijão de gás de 13 kg é de R\$ 70,00, e ocupa o volume de 31 litros. Considere que uma doméstica comprou um botijão de gás para seu consumo em um fogão a gás, após certo período não se conseguia produzir chama no fogão a 25 $^\circ$C. Com base nas informações acima, calcule a massa de GLP retido no botijão e qual a perda financeira na substituição do mesmo (considere que esse GLP seja composto apenas de butano).
\end{enumerate}
