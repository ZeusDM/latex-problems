Visto que o petróleo é um combustível não renovável e que contribui para a poluição do meio ambiente, várias indústrias e centros de pesquisas têm se mobilizado na busca por novas fontes de energia combustível. É nesse contexto que surge o hidrogênio combustível, considerado como o combustível do futuro, por ser renovável, inesgotável e principalmente por não liberar gases tóxicos para a atmosfera. Abaixo têm-se algumas vantagens do combustível hidrogênio:

\begin{enumerate}[label = \textbullet]
	\item Utilização de motores elétricos no lugar de motores a combustão, minimizando a poluição do meio ambiente;
	\item Seu processo de geração de energia é descentralizado, não sendo necessária a construção de hidrelétricas;
	\item A geração de energia por meio de pilhas à combustível é mais eficiente do que a obtida pelos processos tradicionais.
\end{enumerate}

A reação abaixo representada indica a possibilidade de obtenção de hidrogênio a partir do monóxido de carbono

\schemestart
\chemfig{CO}(g) + \chemfig{H_2O}(g) \arrow{<=>} \chemfig{CO_2}(g) + \chemfig{H_2}(g) \qquad $\Delta H_m^\circ = -41,5$ kJ$\cdot$mol$^{-1}$ e $K_c = 0,625$ a $1240$ K.
\schemestop

Analisando os dados da reação aciman afima-se:

\begin{enumerate}[label = (\Roman*)]
	\item Um aumento da pressão total sobre o sistema não altera o estado de quilíbrio;
	\item  Uma diminuição da temperatura favorece o aumento na produção de gás hidrogênio;
	\item O valor de $K_p \> K_c$ nas condições dadas;
	\item A concentração final de cada componente do sistema, em equilíbrio, quando se misturam um mol de cada um dos reagentes com dois mols de cada um dos produtos temperatura da experiência, considerando um balão volumétrico de 1 L é:
	\schemestart
	[\chemfig{CO}] = [\chemfig{H_2O}] = 0,46 mol$\cdot$L$^{-1}$ e [\chemfig{CO_2}] = [\chemfig{H_2}] = 2,54 mol$\cdot$L$^{-1}$
	\schemestop
\end{enumerate}

Estão corretas as afirmações:

\begin{enumerate}[label = (\alph*)]
	\item  I, II e IV apenas.
	\item  II e IV apenas.
	\item  I e II apenas.
	\item  III e IV apenas.
	\item  I e III apenas.
\end{enumerate}
