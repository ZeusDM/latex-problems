O tetrafluoreto de enxofre (\chemfig{SF_4}) reage vagarosamente com oxigênio (\chemfig{O_2}) para formar monóxido de tetrafluoreto de enxofre (\chemfig{SF_4O}), de acordo com a reação representada a seguir: 

\begin{center}
\schemestart
2 \chemfig{SF_4}(g) + \chemfig{O_2}(g) \arrow{->} 2 \chemfig{SF_4O}(g)
\schemestop
\end{center}

\begin{enumerate}[label = (\alph*)]
	\item Escreva a estrutura de Lewis para \chemfig{SF_4O}, na qual as cargas formais de todos os átomos sejam iguais a zero. 
	\item Determine o arranjo estrutural de SF4O e defina qual é a geometria molecular mais provável baseada nesse arranjo. 
	\item Use as entalpias médias de ligação fornecidas na tabela abaixo para calcular a entalpia de reação e indique se a reação é endotérmica ou exotérmica.

		\begin{center}
	\renewcommand{\arraystretch}{1,5}
	\begin{tabular}{ c | c }
		\hline
		Ligação & Entalpia (kJ/mol) \\
		\hline
		\chemfig{S=O} & 523 \\
		\hline
		\chemfig{O=O} & 495 \\
		\hline
		\chemfig{S-F} & 327 \\
		\hline
		\chemfig{O-F} & 190 \\
		\hline
		\chemfig{F-F} & 155 \\
		\hline
		\chemfig{O-O} & 146 \\
		\hline
	\end{tabular}
		\end{center}

	\item A reação do enxofre e flúor forma vários compostos diferentes, inclusive o tetrafluoreto de enxofre e o hexafluoreto de enxofre que podem ser precursores para o monóxido de tetrafluoreto de enxofre. A decomposição de uma amostra de tetrafluoreto de enxofre produz 4,43 g de flúor e 1,87 g de enxofre, enquanto que a decomposição de uma amostra de hexafluoreto de enxofre produz 4,45 g de flúor e 1,25 g de enxofre. Mostre que os dados são consistentes com a lei de proporções múltiplas
	\item O decafluoreto de dissulfeto é intermediário em reatividade entre \chemfig{SF_4} e \chemfig{SF_6}. Ele se decompõe a 150 $^\circ C$ resultando nestes fluoretos monossulfurados. Escreva uma equação balanceada para esta reação e identifique o estado de oxidação de enxofre em cada composto.
\end{enumerate}
