O conhecimento e o estudo da velocidade das reações são de grande interesse industrial, pois permitem reduzir custos e aumentar a produtividade dos processos fabris. Sabe-se que as reações químicas ocorrem com velocidades diferentes e estas podem ser alteradas. Para exemplificar, considere a reação abaixo representada:

A representação matemática de velocidade desta reação é:

Assim sendo, para esse caso, qual afirmação está correta?

\begin{enumerate}[label = (\alph*)]
	\item A ordem geral é 12.
	\item Dobrando a concentração de  e  e reduzindo a metade da concentração de  a velocidade de reação não se altera.
	\item A unidade da constante de velocidade, , é mol$\cdot$dm$^{-3}$ s$^{-1}$.
	\item Uma alteração na concentração de  ou  não afeta a velocidade de reação.
	\item Dobrando a concentração de  ou  e reduzindo a metade da concentração de  a velocidade de reação não se altera.
\end{enumerate}
