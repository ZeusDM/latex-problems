Um acumulador elétrico (bateria) é um conjunto composto por eletrodos de carga oposta e uma solução carregadora de íons, o eletrólito, que, a partir de uma reação química, produz trabalho elétrico.
As baterias de automóveis, as industriais, os telefones celulares e outras contêm metais-traço, como chumbo (Pb), em concentrações elevadas e, por isso, o descarte deve ser feito de acordo com as normas estabelecidas para proteção do meio ambiente e da saúde.
Mas, o atendimento a essas normas ainda é incipiente e pouco efetivo devido às falhas de fiscalização.
Vários estudos têm registrado elevados níveis de contaminação por Pb em águas, com concentrações acima de 0,05 ppm.
Para confirmar se uma dada amostra de água residual estava contaminada, uma análise de íons \chemfig{Pb^{2+}} foi realizada. Assim sendo, uma alíquota de 25,0 mL
dessa amostra foi evaporada até secagem e foi, novamente, redissolvida em 2,0 mL de \chemfig{H_2O} destilada.
Ao recipiente contendo a amostra adicionou-se ainda: 2,0 mL de uma solução mista de tampão e 2,0 mL de uma solução de ditizona.
Após, a solução foi diluída até volume de 10,0 mL.
A absorbância do complexo colorido ditizona-\chemfig{Pb^{2+}} formado pode ser comparada com o gráfico Beer-Lambert do complexo padrão, abaixo.

%imagem

Sabendo que a absorbância de uma porção da solução final é de 0,13.
Qual é a concentração de íons \chemfig{Pb^{2+}} na água residual (inicial), em ppm?

\begin{enumerate}[label = (\alph*)]
	\item 7,2
	\item 2,7
	\item 1,8
	\item 3,6
	\item 4,2
\end{enumerate}

