Das folhas de eucalipto pode-se extrair um óleo que contém um composto orgânico volátil, incolor, insolúvel em água, chamado eucaliptol, cuja estrutura
é representada abaixo. Devido ao seu gosto picante e cheiro agradável é usado
como aromas, fragrâncias e cosméticos.

\begin{center}
	\chemfig{-[:270]-[:330]-[:270]-[:210](-[:270](-[:330])(<[:105.1,1.951]O>[:65.5,1.227])-[:210])-[:150]-[:90](-[:30])}
\end{center}

Considerando que esse composto volátil se comporta como gás ideal à
temperatura de 189 $^\circ$C e pressão de 78 mmHg, suas densidades, absoluta e em
relação ao \chemfig{SO_2}, igualmente considerado como gás ideal, são, respectivamente:

\begin{enumerate}[label = (\alph*)]
	\item 1,03 e 3,25
	\item 0,55 e 1,48
	\item 0,42 e 2,40
	\item 0,68 e 1,16
	\item 0,42 e 2,40
\end{enumerate}
