A combustão, ou seja, reação com \chemfig{O_2}, é muitas vezes utilizada para a eliminação de resíduos químicos. A combustão da metilamina, \chemfig{CH_3NH_2}, produz monóxido de nitrogênio, dióxido de carbono e água.

\begin{enumerate}[label = (\alph*)]
	\item Escreva as fórmulas eletrônicas (fórmula de Lewis) da metilamina e seus produtos da combustão.
	\item Escreva a equação balanceada para a oxidação de metilamina.
	\item Calcule a massa de água produzida a partir da combustão de 62 g de metilamina.
	\item Calcule o número de moléculas de \chemfig{O_2} necessário para esta combustão.
	\item Um engenheiro manuseia uma instalação de eliminação com 5000 L de solução aquosa de metilamina. A fim de determinar aconcentração de metilamina, o engenheiro recolheu uma alíquota de 25,00 mL da solução e titulou com 0,100 mol$\cdot$L$^{-1}$ de ácido clorídrico. O volume da solução padrão de HCl no ponto final foi 35,00 mL. Calcule a concentração molar de metilamina nos 5000 L de amostra. A reação estequiométrica é:
	\schemestart
	\chemfig{CH_3NH_2}(aq) + \chemfig{H^+}(aq) \arrow{->} \chemfig{CH_3NH_3^+}(aq)
	\schemestop
\end{enumerate}