A Demanda Bioquímica de Oxigênio -- DBO -- e a Demanda Química de Oxigênio -- DQO -- são parâmetros físico-químicos para a análise de águas residuais.
No Brasil, a Resolução do Conselho Nacional do Meio Ambiente -- CONAMA -- n. 357/2005 classifica os corpos d'água e determina o valor máximo para DBO a 5 mg$\cdot$L$^{-1}$ de oxigênio, para os de Classe 2, ou seja, as águas que podem ser destinadas ao abastecimento para consumo humano, após tratamento convencional.
Um determinado frigorífico produz um efluente com vazão contínua de 100 m$^3$h$^{-1}$ e possui uma DBO de $2,47$ g$\cdot$L$^{-1}$.
Assinale a alternativa que apresenta o valor máximo de DBO na saída da estação de tratamento de efluentes o frigorífico e a eficiência mínima do tratamento, para que o mesmo possa ser lançado em um riacho de Classe 2 com vazão de referência igual a 1000 L$\cdot$s$^{-1}$ e DBO = zero.

\begin{enumerate}[label = (\alph*)]
	\item 195 mg$\cdot$L$^{-1}$ e 87,5 \%
	\item 185 mg$\cdot$L$^{-1}$ e 92,5 \%
	\item 180 mg$\cdot$L$^{-1}$ e 90,5 \%
	\item 175 mg$\cdot$L$^{-1}$ e 95,5 \%
	\item 190 mg$\cdot$L$^{-1}$ e 85,5 \%
\end{enumerate}

