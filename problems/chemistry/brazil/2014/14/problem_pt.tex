O metanol (\chemfig{CH_3OH}), o mais simples álcool existente, foi primeiramente isolado por Robert Boyle, em 1661.
É um líquido volátil, incolor, altamente polar, inflamável e tóxico.
É amplamente empregado nas indústrias como solvente, nos EUA como combustível e como principal agente no processo de transesterificação de ácidos graxos, na produção do biodiesel.
Pode ser obtido da destilação da madeira ou sintetizado a partir do gás natural (fóssil).
Experimentalmente, a sua entalpia padrão de formação, $\Delta H_f^\circ$, é -239 kJ$\cdot$mol$^{-1}$.
Com base nos valores das variações de entalpia de reações intermediárias, estime a variação de entalpia da reação global final.

\schemestart
\chemfig{C_{(s,grafita)}} + 2 \chemfig{H_2}(g) + $\frac{1}{2}$ \chemfig{O_2}(g) \arrow{->} \chemfig{CH_3OH}(l)
\schemestop

em que o metanol líquido é formado a partir de seus elementos, a 25 $^\circ$C. Para isso, proceda da seguinte forma:

\begin{enumerate}[label = (\alph*)]
	\item Determine a equação global e calcule a entalpia global de atomização dos elementos.
	\item Determine a equação global e calcule a entalpia global de formação do metanol gasoso a partir das energias de ligações dos seus elementos.
	\item Com os valores de entalpia de atomização (item (a)), de ligações (item (b)) e condensação do metanol (tabela de dados) estime a entalpia (calor) de formação do metanol.
	\item Compare o valor estimado com o valor experimental de entalpia de formação do metanol líquido.n
	
	\renewcommand{\arraystretch}{1,5}
	\begin{tabular}{c c}
		\hline
		Reação de Alomização & $\Delta H^\circ$/kJ$\cdot$mol${^{-1}}$, é -239 kJ$\cdot$mol$^{-1}$. \\
		\hline
		\schemestart \chemfig{C_{(s,grafita)}} \arrow{->} \chemfig{C}(g) \schemestop & 716,7 \\
		\hline
		Reação de Dissociação & \\
		\hline
		\schemestart \chemfig{H_2}(g) \arrow{->} 2\chemfig{H}(g) \schemestop & 435,9 \\

		\schemestart \chemfig{O_2}(g) \arrow{->} 2\chemfig{O}(g) \schemestop & 498,4 \\
		\hline
		Entalpia da ligação{média} &  \\
		\hline
		\chemfig{C-H} & 412 \\

		\chemfig{C-O} & 360 \\

		\chemfig{O-H} & 463 \\
		\hline
		Entalpia de vaporização & \\
		\hline
		\schemestart \chemfig{CH_3OH}(l) \arrow{->} \chemfig{CH_3OH}(g) \schemestop & 38 \\
		\hline
	\end{tabular}
\end{enumerate}
