Na tabela periódica, os elementos são agrupados em blocos: $s$, $p$, $d$ e $f$.
Sendo assim, considere:

\begin{enumerate}[label = (\alph*)]
	\item conjunto $H$ sendo o dos elementos com elétron no subnível s do último nível preenchido no estado fundamental;
	\item conjunto $I$ sendo o dos elementos com elétron no subnível p do último nível preenchido no estado fundamental;
	\item conjunto $J$ sendo o dos elementos com elétron no subnível d do penúltimo
nível preenchido no estado fundamental;
\end{enumerate}

Sabendo disso, assinale o diagrama correto, levando em consideração que não devem existir conjuntos vazios:

\begin{enumerate*}[label = (\alph*)]
	\item \def\svgwidth{0.15\textwidth} \input{a.pdf_tex}
	\item \def\svgwidth{0.15\textwidth} \input{b.pdf_tex}
	\item \def\svgwidth{0.15\textwidth} \input{c.pdf_tex}
\end{enumerate*}

\begin{enumerate*}[resume*]
	\item \def\svgwidth{0.15\textwidth} \input{d.pdf_tex}
	\item \def\svgwidth{0.15\textwidth} \input{e.pdf_tex}
\end{enumerate*}

