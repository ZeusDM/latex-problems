Em condições normais, uma reação redox ocorre quando há contato entre o agente oxidante e o agente redutor. Contudo, a reação espontânea pode ocorrer sem contato direto entre oxidante e redutor, cada qual numa interface eletrodo/solução (eletrólito), sendo o par conectado através de um condutor eletrônico (fio) e um condutor iônico (ponte salina). Esse dispositivo é denominado cela galvânica ou eletroquímica, e permite a conversão de energia química em energia elétrica. Uma típica reação redox é a da oxidação do cátion de manganês (II) a óxido de manganês (IV) sólido pelo peróxido de hidrogênio, em água acidificada. Para melhor entender o fenômeno redox e termodinâmico deste sistema, um aluno de química construiu uma cela eletroquímica com eletrodos de Pt e ponte salina de nitrato de potássio, e colocou a cela em operação (fechamento do circuito) a 25 $^\circ$C. Com base nas informações acima:

\begin{enumerate}[label = (\alph*)]
	\item Faça um esquema (desenho ou figura) que represente a cela eletroquímica adequada para o estudo. Identifique todos os constituintes, por exemplo, das espécies químicas no eletrólito e componentes. Coloque o ânodo no lado esquerdo e o cátodo no lado direito.
	\item Escreva as equações iônicas parciais e total balanceadas
	\item Calcule o potencial da cela padrão e $\Delta G$ para a reação.
	\item Proponha uma equação do potencial reversível da cela que você esquematizou no item (a), incluindo as concentrações das espécies que podem afetar o seu potencial.
\end{enumerate}

		Dados de Potendial Padrão de Redução, a 25 $^\circ$C:

\begin{center}
\begin{tabular}{cl}
	\schemestart
	\chemfig{MnO_2}(s) + 4 \chemfig{H^+}(aq) + 2 \chemfig{e^{-}} \arrow{->} \chemfig{Mn^{2+}}(aq) + 2 \chemfig{H_2O}(l)
	\schemestop
	& E$^\circ = +1,23$V \\
	\schemestart
	 \chemfig{H_2O_2}(aq) + 2 \chemfig{H^+}(aq) + 2 \chemfig{e^{-}} \arrow{->} 2 \chemfig{H_2O}(l)
	 \schemestop
	 & E$^\circ = +1,78$V
 \end{tabular}
\end{center}

	Constante de Faraday: $F = 965000$ C$\cdot$mol$^-1$
