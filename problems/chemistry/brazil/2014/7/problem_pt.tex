A reação nuclear é a modificação da composição do núcleo atômico de um elemento, transformando-se em outro(s) elemento(s) e emitindo grande quantidade de energia.
Devido a esse enorme potencial energético, a tecnologia nuclear tem, como uma de suas principais finalidades, gerar eletricidade.
No entanto, a reação nuclear pode ocorrer, controladamente, em um reator de usina nuclear ou, descontroladamente, em uma bomba atômica.
Nesse contexto, qual(is), dentre as equação(ões) abaixo, representa(m) uma reação de fusão?

\begin{enumerate}[label = Reação \arabic*:]
	\item
		\schemestart
		2 \chemfig{H_2O} \arrow{->} 2 \chemfig{_1^2H_2} + \chemfig{_{8}^{14}O_2}
		\schemestop

	\item
		\schemestart
			\chemfig{_1^3H} + \chemfig{_1^2H} \arrow{->} \chemfig{_2^4He} + 2 \chemfig{_0^1n} 
		\schemestop

	\item
		\schemestart
		\chemfig{^{235}_{92}U} + \chemfig{_0^1} \arrow{->} \chemfig{^{144}_{54}Xe} + \chemfig{^{90}_{38}Sr} + 2 \chemfig{^1_0n}
		\schemestop
\end{enumerate}

\begin{enumerate}[label = (\alph*)]
	\item Somente a reação 3
	\item Reações 1 e 2
	\item Reações 1 e 3
	\item Somente reação 1
	\item Somente reação 2
\end{enumerate}
