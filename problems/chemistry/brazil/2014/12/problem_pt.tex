No modelo cinético dos gases ideais, a pressão sobre as paredes do recipiente pode ser quantitativamente atribuída às colisões aleatórias das partículas, essas com energia média, a qual depende da temperatura do gás. A pressão do gás pode, por conseguinte, estar diretamente relacionada à temperatura e à densidade. As partículas são consideradas como pontos infinitesimalmente pequenos. Do ponto de vista da teoria cinética dos gases, explique ou resolva:

\begin{enumerate}[label = (\alph*)]
	\item Por que a pressão de um gás é diretamente proporcional à temperatura?
	\item A lei de Dalton das pressões parciais em termos do modelo cinético dos gases.
	\item O comportamento da efusão de gases, para o seguinte caso: considere que um recipiente de vidro é preenchido, a temperatura ambiente, com um número igual de mols de \chemfig{H_2}(g), \chemfig{O_2}(g), e \chemfig{NO_2}(g). Os gases escoam, lentamente, através de um pequeno furo, para fora do recipiente. Após certo tempo, qual é a relação remanescente das pressões parciais dos gases no recipiente?
	\item Considere um cilindro (com pistão móvel e de atrito desprezível) de 10 L que contém uma mistura gasosa 0,20 mol de dióxido de enxofre, 0,30 mol de nitrogênio e 0,50 mol de dióxido de carbono, a 27 $^\circ$C. Admitindo comportamento de gás ideal, determine as pressões parciais dos gases na mistura quando o volume do cilindro for reduzido para 5 L.
\end{enumerate}

Dados: $R = 0,082$ atm$\cdot$L$\cdot$K$^{-1}\cdot$mol$^{-1}$
