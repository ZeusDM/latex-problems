O ozônio (\chemfig{O_3}) é um gás de cor azul claro, instável e altamente reativo, utilizado para a purificação de água.
O processo de ozonização da água é uma forma de tratamento oxidativo que serve para degradar moléculas orgânicas que estejam na água como contaminante.
É um processo muito utilizado na indústria, mas ultimamente acoplados em filtros caseiros de água, a fim de melhorar a qualidade da água consumida pelas pessoas.
O ozônio também se forma fotoquimicamente na troposfera da Terra e se decompõe de acordo com a equação:

\schemestart
2 \chemfig{O_2}(g) \arrow{->} 3 \chemfig{O_2}(g)
\schemestop

Esta reação ocorre via proposta de mecanismo em duas etapas:

\begin{enumerate}[label = Etapa \arabic*:]
	\item 
		\schemestart
		\chemfig{O_3}(g) \arrow{->} \chemfig{O_2}(g), \qquad \qquad \qquad rápida e reversível.
		\schemestop

	\item
		\schemestart
		\chemfig{O_3}(g) + \chemfig{O}(g) \arrow{->} 2 \chemfig{O_2}(g), \qquad lenta.
		\schemestop
\end{enumerate}

Qual lei de velocidade é consistente com o mecanismo proposto?
