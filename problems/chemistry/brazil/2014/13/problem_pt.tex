Conceitos iniciais sobre ácidos e bases foram propostos pelo químico sueco Svante Arrhenius, em 1887. Em seguida, e de forma independente, novos conceitos foram formulados pelo dinamarquês Johannes Brönsted e pelo inglês Thomas Lowry, coincidentemente no mesmo ano, em 1923. E, também em 1923, o americano Gilbert Lewis propôs conceitos e definições para as reações ácido-base. Diante disso, responda:

\begin{enumerate}[label = (\alph*)]
	\item Qual a principal limitação do conceito de Arrhenius?
	\item Segundo Brönsted-Lowry, a água pode apresentar tanto um caráter ácido, quanto básico. Justifique a afirmação com exemplos.
	\item Sabendo-se que o íon hidrogenossulfito é anfótero, escreva a equação química que descreva sua reação com água (hidrólise), na qual o íon atue como um ácido.
	\item Sabendo-se que o íon hidrogenossulfito é anfótero, escreva equação química que descreva sua reação com água, na qual o íon atue como uma base.
\end{enumerate}