Desde o Império, a seca tem causado grandes transtornos à população do Nordeste brasileiro e é uma das mais graves causas de seus problemas sociais.
Caminhões pipa, que levam água às populações carentes, ajudam a amenizar os problemas.
Suponha que um caminhão pipa contenha $30000$ litros de água de açude, a ser clorada na dosagem de $54,0$ mg/L de “cloro ativo” para eliminar microrganismos nocivos à saúde.
Dispondo-se de solução de hipoclorito de sódio comercial a $12 \%$ m/V (pureza), qual o volume a ser adicionado à pipa?

Dado: considere que, em meio de ácido clorídrico, cada íon de hipoclorito reage formando uma molécula de cloro, nos cálculos de “cloro ativo”.

\begin{enumerate}[label = (\alph*)]
	\item 1,31 L
	\item 0,90 L
	\item 18 L
	\item 0,72 L
	\item 1,25 L
\end{enumerate}
