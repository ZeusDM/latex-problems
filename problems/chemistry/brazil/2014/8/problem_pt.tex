A prática da Química, seja a nível profissional ou de aprendizado, exige que
Normas de Segurança sejam rigorosamente seguidas para evitar acidentes e
prejuízos de ordem humana ou material. Os acidentes podem, se tomadas as
devidas precauções, ser evitados ou, ao menos, ter suas consequências mini-
mizadas. Neste sentido, assinale Falso (F) ou Verdadeiro (V) para as “Normas
de Segurança em Laboratório Químico” observadas.

\begin{enumerate}[label = (\Roman*)]
	\item Dedicar especial atenção a qualquer operação que necessite aquecimento prolongado ou que desenvolva grande quantidade de energia.
	\item Evitar armazenar reagentes em lugares altos e de difícil acesso.
	\item Não deixar vidro quente em lugar que alguém possa pegá-lo inadvertidamente. “Vidro quente se difere de vidro frio em seu aspecto”.
	\item Provar ou ingerir drogas ou reagentes de laboratório, se necessário.
	\item Rotular todas as soluções e guardar.
	\item Sempre que fizer a diluição de um ácido concentrado, adicione a água lentamente sobre ele, sob agitação constante e nunca ao contrário.
\end{enumerate}

\begin{enumerate}[label = (\alph*)]
	\item VVVFFF
	\item VFFFVV
	\item VVFFVF
	\item VFVFVF
	\item FVVVFF
\end{enumerate}

