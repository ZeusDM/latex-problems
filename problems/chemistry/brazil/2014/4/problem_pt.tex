Os estudos de velocidade de reações químicas são de grande interesse para as indústrias químicas, nos seus diversos segmentos.
Dentre os fatores que alteram a velocidade de uma reação tem-se: pressão, superfície de contato, temperatura e o uso de catalisadores.
A fim de ratificar esta informação, alunos efetuaram em laboratório um experimento no qual usaram três frascos, cada um com 500 mL de ácido clorídrico 6 mol$\cdot$L$^{-1}$, e amostras de zinco conforme abaixo:

\begin{enumerate}[label = Frasco \Roman*:]
	\item um cubo de zinco com 1 g de massa;
	\item mil cubos de zinco com 1 mg de massa cada um;
	\item mil esferas de zinco com 1 mg de massa cada uma;
\end{enumerate}

Chamando de $v_1$, $v_2$ e $v_3$ as velocidades de dissolução nos casos dos frascos I, II e III, qual a ordem correta de velocidade de reação?


Dados:
\begin{enumerate*}[label = , itemjoin={\qquad}]
	\item Área da superfície esférica, $A = 4 \pi r^2$
	\item $\sqrt{36\pi} \approx 4,8$
\end{enumerate*}

\begin{enumerate}[label = (\alph*)]
	\item $v_2 >> v_1 > v_3$
	\item $v_1 > v_3 > v_2$
	\item $v_1 > v_2 >> v_3$
	\item $v_2 > v_3 >> v_3$
	\item $v_2 > v_1 >> v_3$
\end{enumerate}
