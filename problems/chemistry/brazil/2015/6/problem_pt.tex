O calor específico pode ser definido como a quantidade de calor que um grama de determinado material deve ganhar ou perder para que sua temperatura varie em um grau Celsius.
O calor específico dos metais é baixo quando comparado a materiais como argila ou pedra, todos materiais usados na fabricação de panelas.
Isso significa que, considerando panelas de mesma massa, é necessário fornecer menos calor para o metal do que para a argila para fazer com que ele atinja a temperatura de cozimento (Mortimer, E. F.; Amaral, L. O. F. Calor e temperatura no ensino da termoquímica. Química Nova na Escola, n. 7, 1998).
Nesse contexto, e para efeito de comparação, são dados abaixo os calores específicos de três metais.

\renewcommand{\arraystretch}{1.5}
\begin{tabular}{|c|c|c|c|}
\hline
Metal & Fe & Pb & Zn \\
\hline
Calor específico, J$\cdot$g$^{-1}\cdot$$^\circ$C$^{-1}$ & 0,470 & 0,130 & 0,388 \\
\hline
\end{tabular}

Se 1,00 g de cada metal é aquecido a 100 $^\circ$C e se adicionar 10,00 g de \chemfig{H_2O} a 25,0 $^\circ$C, qual será a ordem das temperaturas das misturas finais a partir do menor para o maior? 

\begin{enumerate}[label = (\alph*)]
	\item Fe $<$ Zn $<$ Pb 
	\item Pb $<$ Zn $<$ Fe 
	\item Zn $<$ Fe $<$ Pb 
	\item Fe $<$ Pb $<$ Zn 
	\item Pb $<$ Fe $<$ Zn
\end{enumerate}
