As lâmpadas incandescentes ou de filamento transformam energia elétrica em energia luminosa e térmica, mas, progressivamente, estão sendo substituídas por outras de menor consumo, pois perdem em calor a maior parte da energia que consomem, e transformam em iluminação apenas 5\% desta.
Essas lâmpadas utilizam um filamento de tungstênio que, quando percorrido por uma corrente elétrica, torna-se incandescente, produzindo luz.
Uma lâmpada de 60 W, submetida a uma diferença de potencial de 220 V, é ligada quatro horas diariamente durante um mês em um cômodo onde há uma pequena planta.
Essa planta consegue aproveitar cerca de 10\% da energia luminosa que a atinge para a realização da fotossíntese.
A partir do exposto, responda:

Dados:
\begin{enumerate}[label = --]
	\item Entalpia de formação do dióxido de carbono: $-94$ kcal$\cdot$mol$^{-1}$ 
	\item Entalpia de formação da água: $-58$ kcal$\cdot$mol$^{-1}$ 
	\item  Entalpia de formação da glicose: $-242$ kcal$\cdot$mol$^{-1}$
	\item $1$ cal = $4$ J
	\item $e = 1,6 \times 10^{-19}$ 
\end{enumerate}

\begin{enumerate}[label = (\alph*)]
	\item Qual a energia absorvida pela planta nesse período? 
	\item Qual o número de mols de gás oxigênio gerado? 
	\item  Quantos litros, aproximadamente, de oxigênio são gerados, sendo que a sala tem uma temperatura média de 104 $^\circ$F e pressão de 1 atm? 
	\item Sabendo que um adulto consome em média 3 L de oxigênio por minuto, quantas plantas, iguais a essa e recebendo energia nas mesmas proporções, seriam necessárias para suprir esse consumo no período de 30 dias? 
	\item Quantos elétrons atravessaram a lâmpada nesse período? 
\end{enumerate}
