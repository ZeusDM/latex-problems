 O sangue humano é um líquido ligeiramente básico, tamponado por processos metabólicos que mantêm o pH entre 7,35 e 7,45. Para controlar o pH do sangue, o corpo usa inicialmente o sistema ácido carbônico/bicarbonato, conforme mostrado abaixo: 

\schemestart
\chemfig{CO_2} + \chemfig{H_2O} \arrow{<=>} \chemfig{H_2CO_3} \arrow{<=>} \chemfig{H^+} + \chemfig{HCO_3^{-}}
\schemestop

Se o pH sobe acima da faixa normal, a condição é chamada de alcalose, cujo valor limite de sobrevivência por tempo reduzido é 7,8. Quando o pH do sangue está abaixo da faixa normal, a condição é chamada de acidose e o valor limite de sobrevivência por tempo reduzido é 7,0. 

 Sobre esse sistema-tampão são feitas as seguintes afirmações: 

 \begin{enumerate}[label = (\Roman*)]
	\item Respirando mais rápido e profundamente aumentamos a quantidade de \chemfig{CO_2} exalado e, assim, a concentração de ácido carbônico no sangue decresce, favorecendo a alcalose. 
	\item A inalação excessiva de fumaça aumenta a concentração de \chemfig{CO_2} no sangue, favorecendo a acidose.
	\item O aumento da concentração dos íons bicarbonato no sangue provoca um aumento de pH, favorecendo a alcalose. 
	\item A liberação excessiva de ácido láctico durante a realização de exercícios físicos pesados, provoca um aumento da concentração hidrogeniônica no sangue, favorecendo a acidose. 
\end{enumerate}

Estão corretas: 

\begin{enumerate}[label = (\alph*)]
	\item Todas as alternativas 
	\item Somente I, II e III
	\item Somente II, III e IV 
	\item Somente II e III
	\item Somente I e IV

\end{enumerate}
