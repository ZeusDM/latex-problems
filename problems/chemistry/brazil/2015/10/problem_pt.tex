A química nuclear no pouco tempo de história da humanidade é polêmica e controversa, mas é inegável a sua importância no nosso cotidiano. Como a meia-vida de qualquer nuclídeo é constante, a meia-vida pode servir como um relógio nuclear para determinar as idades de diferentes materiais. O \chemfig{^{14}C}, por exemplo, tem sido usado para determinar a idade de materiais orgânicos (Figura 1). O procedimento é baseado na formação de \chemfig{^{14}C} por captura de nêutrons na atmosfera superior:

\begin{center}
\schemestart
\chemfig{^{14}_7N} + \chemfig{^1_0n} \arrow{->} \chemfig{^{14}_6C} + \chemfig{^1_1p}
\schemestop
\end{center}

Essa reação fornece uma fonte de pequena, mas razoavelmente constante. O \chemfig{^{14}_6C} é radioativo, sofrendo decaimento beta com meia-vida de 5.715 anos: 

\begin{center}
\schemestart
\chemfig{^{14}_6C} \arrow{->} \chemfig{^{14}_7N} +
\chemfig{^0_{-1}\beta}
\schemestop
\end{center}

Todas as seguintes sentenças abaixo são verdadeiras para o método de datação por \chemfig{^{14}C}, exceto: 

\begin{enumerate}[label = (\alph*)]
	\item A proporção de \chemfig{^{14}C}/\chemfig{^{12}C} é a mesma em organismos vivos terrestres como na atmosfera. 
	\item \chemfig{^{14}C} sofre $\beta$-decaimento para produzir \chemfig{^{14}N}
	\item O teor de \chemfig{^{14}C} de um organismo é mantido constante durante sua vida e inicia decréscimo depois de sua morte. 
	\item  A datação por carbono é igualmente útil para as amostras que tem milhões de anos de idade, como para as amostras que tem cerca de 10.000 anos de idade. 
	\item A proporção de \chemfig{^{14}C}/\chemfig{^{12}C} pode ser usada para datar uma amostra de um organismo morto
\end{enumerate} 
