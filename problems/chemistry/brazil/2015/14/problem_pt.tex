O ácido propanoico, \chemfig{CH_3CH_2COOH}, é um ácido carboxílico que reage com a água de acordo com a equação abaixo.

\begin{center}
\schemestart
\chemfig{H_3CCH_2COOH}(aq) + \chemfig{H_2O}(l) \arrow{<=>} \chemfig{H_3CCH_2COO^{-}} + \chemfig{H_3O^+}(aq)
\schemestop
\end{center}

A 25 $^\circ$C o pH de uma amostra de 50,0 mL de \chemfig{CH_3CH_2COOH} 0,20 mol$\cdot$L$^{-1}$ é 2,79.

\begin{enumerate}[label = (\alph*)]
	\item Identifique o par ácido-base conjugado de Bronsted-Lowry na reação.
		Rotule claramente qual é o ácido e o qual é a base.
	\item Determine o valor de $K_a$ para o ácido a 25 $^\circ$C. 
	\item  Para cada uma das seguintes afirmações, determinar se a afirmação é verdadeira ou falsa.
		Em cada caso, explicar o raciocínio que suporta a sua resposta. 
		\begin{enumerate}[label = (\Roman*)]
			\item O pH de uma solução preparada pela mistura de 50,0 mL da amostra de \chemfig{CH_3CH_2COOH} 0,20 mol$\cdot$L$^{-1}$ com uma amostra de 50,0 mL de \chemfig{NaOH} 0,20 mol$\cdot$L$^{-1}$ é 7,00. 
			\item  Se o pH de uma solução de ácido clorídrico é o mesmo que o pH de uma solução de ácido propanoico, em seguida, a concentração molar da solução de ácido clorídrico deve ser menor do que a concentração molar da solução do ácido propanoico. 
		\end{enumerate}
	\item Um estudante recebe a tarefa de determinar a concentração de uma solução de ácido propanoico.
		Uma solução \chemfig{NaOH} 0,173 mol$\cdot$L$^{-1}$ está disponível para usar como titulante.
		O estudante utiliza uma pipeta volumétrica de 25,00 mL para transferir a solução de ácido propanoico a um erlenmeyer limpo e seco.
		Após a adição de um indicador apropriado para o erlenmeyer, o estudante titula a solução com \chemfig{NaOH} 0,173 mol$\cdot$L$^{-1}$, atingindo o ponto final após a adição de 20,52 mL de solução de base.
		Calcule a concentração molar da solução de ácido propanoico.
	\item  O estudante é solicitado para redesenhar a experiência para determinar a concentração de uma solução de ácido butanoico \chemfig{CH_3-CH_2-CH_2-COOH} em vez de uma solução de ácido propanoico. Para o ácido butanoico o valor de p$K_a$ é 4,83. O estudante reivindica que um indicador diferente será necessário para determinar o ponto de equivalência da titulação com precisão. Com base na sua resposta ao item (b), você concorda com a afirmação do estudante? Justifique sua resposta.
\end{enumerate}
