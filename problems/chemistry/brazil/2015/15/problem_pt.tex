Para a reação genérica abaixo, a 298 K: 

\begin{center}
\schemestart
3 \chemfig{X_2Y} + \chemfig{WZ_3} \arrow{->} produtos
\schemestop
\end{center}

Foram obtidos os seguintes dados cinéticos:

\begin{center}
\renewcommand{\arraystretch}{1.4}
\begin{tabular}{|c|c|c|}
	\hline
	Experimento & Concentração inicial (mol$\cdot$L$^{-1}$) & Velocidade Inicial (mol$\cdot$L$^{-1}\cdot$s$^{-1}$) \\
	\hline
	& [\chemfig{X_2Y}]$_0$ \qquad [\chemfig{WZ_3}]$_0$ & \\
	\hline
	I   & 1,72 \qquad \quad 2,44 & 0,62 \\
	II  & 3,44 \qquad \quad 2,44 & 5,44 \\
	III & 1,72 \qquad \quad 0,10 & $2,8 \times 10^{-2}$ \\
	IV  & 2,91 \qquad \quad 1,33 & ? \\
	\hline
\end{tabular}
\end{center}

\begin{enumerate}[label = (\alph*)]
	\item Em relação a cada reagente, determine a ordem da reação. Determine também a ordem global da reação
	\item A partir das informações da tabela, determine a Lei da Velocidade para a reação:  
	\item A partir dos dados, determine o valor da Constante de Velocidade para a reação genérica acima. 
	\item Utilizando os dados fornecidos, calcule a velocidade de reação para o Experimento IV. 
	\item A velocidade de reação aumenta por um fator de 100 na presença de um catalisador, a 298 K. A energia de ativação aumentará, diminuirá ou permanecerá a mesma? Justifique.
\end{enumerate}
