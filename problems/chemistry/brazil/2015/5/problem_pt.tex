As titulações estão entre os procedimentos analíticos mais exatos.
Em uma titulação, o analito reage com um reagente padronizado (o titulante) em uma reação de estequiometria conhecida, em que a quantidade de titulante é variada até que a equivalência química seja atingida, sendo esta equivalência verificada pela mudança de cor de um indicador ou pela mudança na resposta de um instrumento.
A quantidade do reagente padronizado necessária para atingir a equivalência química é relacionada com a quantidade de analito (Skoog, D. A. Fundamentos de Química Analítica, 8ª e., Thomson, 2010, 1026 p).
Para exemplificar, foram titulados $25,0$ mL de uma solução que contém íons de \chemfig{Fe^{2+}} e de \chemfig{Fe^{3+}} com $23,0$ mL de \chemfig{KMnO_4} $0,0200$ mol L$^{-1}$ (em ácido sulfúrico diluído).
Como resultado, todos os íons de \chemfig{Fe^{2+}} foram oxidados para íons \chemfig{Fe^{3+}}.
Em seguida, uma nova alíquota de $25,0$ mL da solução foi tratada com \chemfig{Zn} metálico para converter todos os íons de \chemfig{Fe^{3+}} em íons de \chemfig{Fe^{2+}}.
Finalmente, a solução que contém apenas os íons de \chemfig{Fe^{2+}} consumiu $40,0$ mL do mesmo titulante (solução de \chemfig{KMnO4}) para a oxidação para \chemfig{Fe^{3+}}.
Calcule as concentrações molares de \chemfig{Fe^{2+}} e de \chemfig{Fe^{3+}} na solução original.
A equação iônica simplificada é:

\schemestart
\chemfig{MnO_4^{-}} + 5 \chemfig{Fe^{2+}} + 8 \chemfig{H^+} \arrow{->} \chemfig{Mn^{2+}} + 5 \chemfig{Fe^{3+}} + 4 \chemfig{H_2O}
\schemestop

\begin{enumerate}[label = (\alph*)]
	\item[] [\chemfig{Fe^{2+}}]/mol$\cdot$L$^{-1}$ \qquad [\chemfig{Fe^{3+}}]/mol$\cdot$L$^{-1}$
	\item \qquad  0,0680 \qquad \qquad \qquad 0,0920
	\item \qquad  0,0920 \qquad \qquad \qquad 0,0680
	\item \qquad  0,0680 \qquad \qquad \qquad 0,0460 
	\item \qquad  0,0920 \qquad \qquad \qquad 0,0340 
	\item \qquad  0,0460 \qquad \qquad \qquad 0,0340 
\end{enumerate}
