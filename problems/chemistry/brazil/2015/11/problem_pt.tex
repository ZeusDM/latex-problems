 A água de um reservatório de 500 m$^3$ foi analisada quanto ao teor de cloreto, considerando-se o padrão de potabilidade estabelecido pela Portaria Nº 2.914/2011 da ANVISA/MS, de limite máximo de 250 mg$\cdot$L$^{-1}$ para este parâmetro. Dessa forma, durante a titulação, uma amostra de 100,0 mL de água consumiu 11,5 mL de solução de \chemfig{AgNO_3} 0,1 mol$\cdot$L$^{-1}$, somente para reagir com os íons cloretos. Sabendo que esse reservatório deverá receber mais 400 m3 de água, com teor de cloretos de 105 mg$\cdot$L$^{-1}$, responda: 

 \begin{enumerate}[label = (\alph*)]
	\item Qual a massa de cloreto na amostra titulada?
	\item Qual a concentração inicial de cloretos no tanque?
	\item Ao final da adição de água no reservatório, o teor de cloretos irá atender a legislação vigente? 
	\item A fim de manter o teor de cloretos dentro do valor máximo permitido pela portaria, qual o volume de água livre de cloretos que poderia ser adicionada ao reservatório? 
	\item Qual a massa de sal de cozinha, em quilogramas, que poderia ser produzida com a massa de cloreto presente no reservatório após o recebimento dos 400 m$^3$ de água contendo 105 mg$\cdot$L$^{-1}$ de cloretos?
\end{enumerate}
