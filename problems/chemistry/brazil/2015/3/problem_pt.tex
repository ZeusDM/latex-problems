As substâncias de valência mista são aquelas que contêm íons em mais de um estado de oxidação formal, em uma mesma unidade molecular, que lhes atribuem propriedades supramoleculares originais com aplicabilidade em diversas áreas: conversão de energia, novos materiais, catálise e eletrônica molecular, entre outros.
O caráter de valência mista é, na verdade, responsável pela coloração de vários minerais bem conhecidos (Rocha, R. C.; Toma, H. E. Transferência de elétrons em sistemas inorgânicos de valência mista. Química Nova, v. 25, n. 4, p. 624-638, 2002).
O elemento ferro forma um sulfeto com a fórmula aproximada \chemfig{Fe_7S_8} (mineral pirrotita).
Suponha que o estado de oxidação do enxofre é -2 e que os átomos de ferro, ambos existem, se encontram nos estados de oxidação +2 e +3. Qual é a proporção de Fe$^{2+}$ para Fe$^{3+}$ na substância? 

\begin{enumerate}[label = (\alph*)]
	\item $1,00$
	\item $1,33$
	\item $0,75$
	\item $0,40$
	\item $2,50$
\end{enumerate}
