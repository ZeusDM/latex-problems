Medicina Nuclear é a especialidade que utiliza pequenas quantidades de substâncias radioativas ou “traçadores” para diagnosticar ou tratar certas doenças.
Traçadores são substâncias que são atraídas para órgãos específicos (os ossos por exemplo).
Quando introduzidos no corpo eles marcam as moléculas participantes nesses processos fisiológicos com isótopos radioativos.
Estes denunciam sua localização por emitirem radiação nuclear (onda eletromagnética de comprimento de 0,01 a 1 nm do espectro dos raios gama).
A detecção localizada de muitos fótons gama com uma câmara gama permite formar imagens ou filmes que informem acerca do estado funcional dos órgãos.
Entre os radioisótopos mais utilizados está o Tecnécio-99 meta estável, usado em exames de cintilografia do miocárdio e os isótopos de Iodo 123 e 131, usados nos diagnósticos da tireoide.
Sobre as informações do texto responda os itens a seguir: 

\begin{enumerate}[label = (\alph*)]
	\item Entre os isótopos mencionados o Iodo-131 emite partícula beta, os demais emitem apenas radiação gama. Escreva as suas equações de decaimento utilizando a simbologia química apropriada. 
	\item A atividade de uma amostra radioativa ou taxa de decaimento é a velocidade com que uma amostra se desintegra por unidade de tempo. No S.I. sua unidade é o becquerel (Bq) e equivale a uma desintegração por segundo. Qual a atividade de uma amostra com $2,0 \times 10^{20}$ átomos de 99Tc, se sua constante de decaimento for $3,2 \times 10^{-5}$ s$^{-1}$? 
	\item  As meias vidas dos radioisótopos do iodo apresentados são, respectivamente, 13 horas para o 123, e 8 dias para o 131.
		Identifique o mais instável e explique através de suas velocidades de decaimento (atividades), considerando que ambos apresentam amostras com o mesmo número de átomos.
		
		Dado: $\ln 2 =  0,693$ 
	\item Uma amostra a ser usada em um exame de cintilografia miocárdica é rotulada com \chemfig{^{99}Tc}, radioisótopo que tem uma constante de decaimento igual a 0,1155 h$^{-1}$. Caso tenha sido injetado 0,5 mg desse radioisótopo no corpo de um indivíduo, quanto ele ainda apresentará em seu organismo de tecnécio-99 após dois dias e meio? 
	\item Calcule a energia gerada por 0,5 mol de fótons mais energéticos, em MeV, ou seja, em milhões de elétron-volts, que são detectados pela câmara gama 
\end{enumerate}

Dados:
\begin{enumerate}[label = --]
	\item Constante de Planck,  $K = 6,6 \times 10^{-34}$ J$\cdot$s;
	\item Velocidade da onda eletromagnética,  $c = 3,0 \times 108$ m$\cdot$s$^{-1}$;
	\item $1$ eV $= 1,6 \times 10^{-19}$
\end{enumerate}
 
