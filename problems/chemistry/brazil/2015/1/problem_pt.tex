 Quando pequenas quantidades de certas substâncias são aquecidas, através de uma técnica chamada ‘teste da chama’, elas emitem luz, visível ou não. Por exemplo, o cloreto de sódio, quando aquecido, emite uma luz amarela, característica do sódio. Outros sais apresentam as mesmas características quando aquecidos, porém, com cores distintas. Por exemplo, o cloreto de cálcio apresenta coloração vermelha; o cloreto de potássio, violeta; o cloreto de bário, verde.

Devido ao fato de cada um dos átomos de metais citados emitir radiação em comprimento de onda característico (luz de cor específica), o teste de chama pode, então, ser utilizado para a identificação destes elementos, teste este baseado na Teoria Atômica.
Para o teste de chama acima descrito é correto afirmar que:

 \begin{enumerate}[label = (\alph*)]
	\item As cores são explicadas porque existe diferença de energia entre níveis eletrônicos e, ao aquecer as substâncias, ocorre excitação eletrônica. O elétron, ao retornar à sua orbita original, emite energia na forma de luz visível. 
	\item As cores que surgem no aquecimento são devidas a transições eletrônicas. Quando os elétrons são excitados, eles saltam de suas orbitas originais, liberando energia, na forma de luz visível. 
	\item As cores observadas estão de acordo com a Teoria Atômica de Rutherford, cientista que estudou as órbitas eletrônicas. 
	\item As diferentes cores observadas devem-se ao número de nêutrons no núcleo de cada átomo, conforme estudos de Niels Bohr. 
	\item Somente os elementos dos subgrupos 1 e 2 apresentam essas propriedades, que foram previstas pelo químico russo D. Mendeleiev.  
\end{enumerate}
