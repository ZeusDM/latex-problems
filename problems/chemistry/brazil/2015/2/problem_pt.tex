As reações químicas são o coração da química. Compreender a ocorrência e os mecanismos das reações químicas permite ainda o entendimento de muitos processos que ocorrem em nossas vidas, como o metabolismo, a ação de medicamentos, o cozimento de alimentos, entre tantos outros exemplos (Rosa, M. I. F. P. S.; Schnetzler, R. P. O Conceito de Transformação Química. Química Nova na Escola, n. 8, 1998). Ao aplicar as reações químicas para quatro metais distintos (A, B, C e D) foram obtidos os seguintes resultados. 

\begin{enumerate}[label = (\alph*)]
	\item Apenas B e C reagem com \chemfig{HCl} 0,5 mol L$^{-1}$ para produzir \chemfig{H_2} no estado gasoso.
	\item Quando o metal B é adicionado a soluções que contêm os íons dos outros metais, são formados A, C e D metálicos. 
	\item A reage com \chemfig{HNO_3} 6 mol L$^{-1}$, mas D não reage.
\end{enumerate}

Com base nas informações acima,disponha os metais em ordem crescente como agentes redutores.

\begin{enumerate}[label = (\alph*)]
	\item D $<$ A $<$ C $<$ B
	\item D $<$ C $<$ A $<$ B 
	\item B $<$ A $<$ D $<$ C 
	\item A $<$ D $<$ B $<$ C
	\item B $<$ A $<$ C $<$ D
\end{enumerate}
