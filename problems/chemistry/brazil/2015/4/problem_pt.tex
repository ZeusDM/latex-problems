As transformações químicas são representadas por equações químicas em que as substâncias que sofrem transformação – os reagentes – são escritas no lado esquerdo e as substâncias formadas – os produtos – aparecem no lado direito. As equações químicas devem ser balanceadas de acordo com as leis ponderais, principalmente na lei da conservação das massas e na lei das proporções fixas (ou definidas), nas quais o número de átomos de cada tipo de elemento tem de ser igual nos reagentes e nos produtos, bem como as quantidades de cargas. Quando a equação

\schemestart
\_\_\_\chemfig{NO_3^{-}}(aq) + \_\_\_\chemfig{Al}(s) + \_\_\_\chemfig{H_2O}(l) + \_\_\_\chemfig{OH^{-}} 
\arrow{->} \chemfig{{(Al{(OH)}_6)}^{3-}} + \_\_\_\chemfig{NH_3}(g)
\schemestop

é equilibrada corretamente com os menores coeficientes de números inteiros, qual é a soma dos coeficientes dos reagentes e dos produtos, respectivamente.

\begin{enumerate}[label = (\alph*)]
	\item 48 e 11
	\item 48 e 10
	\item 50 e 10
	\item 50 e 11
	\item 49 e 11
\end{enumerate}
