O equilíbrio entre uma substância sólida e seus íons hidratados em solução fornece um exemplo de equilíbrio heterogêneo.
A extensão do equilíbrio na qual a reação de dissolução ocorre é expressa pela ordem de grandeza de sua constante de equilíbrio, conhecida como constante do produto de solubilidade, $K_{ps}$.
Considerando-se o conhecimento de algumas regras gerais de precipitação e os equilíbrios de solubilidade, em contraste, podemos fazer suposições quantitativas sobre quanto de certa substância se dissolverá ou formará precipitado.
Assim, ao misturar $15,0$ mL de $0,0040$ mol$\cdot$L$^{-1}$ de nitrato de chumbo (II) com $15,0$ mL de cloreto de sódio $0,0040$ mol$\cdot$L$^{-1}$, resultará: 

\schemestart
\chemfig{PbCl_2}(s) \arrow{<=>} \chemfig{Pb^{2+}}(aq) + 2 \chemfig{Cl^{-}}(aq) \qquad $K_{ps} = 1,7 \times 10^{-5}$
\schemestop

Dado: $4,25^{1/3} = 1,62$

\begin{enumerate}[label = (\alph*)]
	\item Um sólido \chemfig{PbCl_2} irá precipitar e ìons \chemfig{Pb^{2+}} em excesso irão permanecer em solução.
	\item Um sólido \chemfig{PbCl_2} irá precipitar e ìons \chemfig{Cl^{-}} em excesso irão permanecer em solução.
	\item Um sólido \chemfig{PbCl_2} irá precipitar em meio aquoso.
	\item Um sólido \chemfig{PbCl_2} irá precipitar e ìons \chemfig{Na^{+}} e \chemfig{NO_3^{-}} em excesso irão permanecer em solução.
	\item Uma solução límpida sem precipitado.
\end{enumerate}
