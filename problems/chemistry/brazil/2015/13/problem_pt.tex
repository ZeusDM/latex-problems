O ácido láctico, CH$_3$–CH(OH)–COOH, recebeu esse nome porque está presente no leite azedo de gosto desagradável como um produto de ação bacteriana. É também responsável pela irritabilidade nos músculos depois de exercício vigoroso. 

Dados: $10^{-3,85} = 1,4 \times 10^{-4}$; $10^{7,1 \times 10^{-10,15}} = 7,1 \times 10^{-11}$; $\sqrt{7} = 2,65$ e $10^{-2,7} = 1,6 \times 10^{-3}$

\begin{enumerate}[label = (\alph*)]
	\item A adição de hidróxido de sódio para reduzir a acidez causada pelo ácido láctico formado pela ação de microrganismos no leite comercial para consumo humano é crime de adulteração de produtos alimentícios (art. 272 do Código Penal). Considere uma concentração de 1,8 g$\cdot$L$^{-1}$ de ácido láctico em um lote de 500 L de leite. Qual o volume necessário para neutralizar completamente todo o ácido contido nesse lote, sabendo que a concentração do hidróxido de sódio é 0,5 mol$\cdot$L$^{-1}$
	\item O p$K_a$ do ácido láctico é 3,85. Compare esse valor com o valor para o ácido propiônico (\chemfig{CH_3CH_2COO}, p$K_a$ = 4,89) e explique a diferença.
	\item Calcule a concentração de íon lactato em uma solução de 0,050 mol$\cdot$L$^{-1}$ de ácido láctico.
	\item Quando o lactato de sódio, \chemfig{{(CH_3CH(OH)COO)}Na}, é misturado com uma solução de cobre (II), é possível obter um sal sólido de lactato de cobre (II)
	como um hidrato de azul-esverdeado, cuja fórmula molecular é  \chemfig{{(CH_3CH{(OH)}COO)}_2Cu}$\cdot x$\chemfig{H_2O}.
		A análise elementar do sólido nos diz que ele contém 22,9 \% de Cu e 26,0 \% de C em massa.
		Qual é o valor de $x$ para o hidrato?
	\item A constante de dissociação ácida para o íon \chemfig{Cu^{2+}}(aq) é $1,0 \times 10^{-8}$.
		Com base nesse valor, determine se uma solução de lactato de cobre (II) será ácida, básica ou neutra.
		Justifique sua resposta.
\end{enumerate}
