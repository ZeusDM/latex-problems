Durante algum tempo os químicos acreditavam que bastava conhecer a entalpia da reação para conhecer sua espontaneidade. Porém, há vários exemplos de transformações químicas que contradizem essa ideia. Para corrigir essa distorção, introduziu-se um outro parâmetro para avaliar a espontaneidade de uma reação, a entropia. 
O enunciado de Kelvin é:
“É impossível remover energia cinética de um sistema a uma certa temperatura e converter essa energia integralmente em trabalho mecânico sem que haja uma modificação no sistema ou em suas vizinhanças”. 
Sobre a entropia é correto afirmar que:

\begin{enumerate}[label = (\alph*)]
	\item A segunda Lei da Termodinâmica afirma que a entropia do universo diminui numa transformação espontânea. 
	\item Se a entropia de um sistema diminui, a transformação será necessariamente não espontânea.
	\item A entropia padrão de uma substância pura é zero nas condições padrão.
	\item Uma reação endotérmica e com diminuição de entropia do sistema é espontânea. 
	\item Numa transformação espontânea, a entropia do universo irá aumentar.
\end{enumerate}
