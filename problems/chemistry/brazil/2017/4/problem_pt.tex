As soluçôes têm uma importante presença no cotidiano, pois podem apresentar diversas aplicações como na composição da água mineral e do ar atmosférico, além de outras aplicações em diversas áreas, como a farmacêutiva e a biológica. Dessa forma, a respeito das soluções, suas propriedades e concentrações, analisar as proposições a seguir:

\begin{enumerate}[label = (\Roman*)]
	
	\item Soluções verdadeiras e dispersões coloidais podem ser exemplificadas, respectivamente, pelo sangue e pela salmoura.
	\item Misturando-se $25,0$ mL de uma solução aquosa de ácido sulfúrico decimolar com $25,0\%$ ml de solução aquosa de hidróxido de sódio $0,40$ mol L$^{-1}$, após o término da reação, o p\chemfig{H} da solução final, após a adição de reagente em excesso, é $13,0$ e sua concentração em quantidade de matéria é $0,10$ mol L$^{-1}$
	\item Emulsão é uma dispersão coloidal que se dá entre dois líquidos miscíveis, que formam micelas polares.
	\item Considerando a solubilidade do \chemfig{K_2CO_3} em água igual a $1,12$ g$\cdot$mL$^{-1}$ a $20 ^\circ$C, uma solução que apresenta $25,0$ g dessa substância em $25,0$ g dessa substância dissolvida em $25,0$ mL de água, nessa temperatura, é classificada como concentrada e insaturada.
	\item Depois de certo tempo, numa salada de alface temperada com vinagre e sal, as folhas murcham em função dos efeitos coligativos de crioscopia e osmose. 
\end{enumerate}

\begin{enumerate}[label = (\alph*)]
	
	\item Somente II e IV. 
	\item Somente III e IV. 
	\item Somente I, IV e V. 
	\item Somente I, II e III
	\item Somente III, IV e V.
\end{enumerate}
