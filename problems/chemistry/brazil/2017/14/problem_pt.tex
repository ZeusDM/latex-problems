Atualmente, as questões ambientais passam a limitar a competitividade das indústrias químicas ou das empresas que tenham algum processo químico, com uma legislação ambiental cada vez mais restritiva, esgotamento dos recursos naturais e com uso consciente e responsável. Sendo assim, os profissionais da química possuem importante papel na operação e otimização de processos relacionados ao controle de resíduos e ao tratamento dos efluentes industriais. Diante disso, responda os seguintes questionamentos sobre os processos de tratamento de efluentes: 

\begin{enumerate}[label = (\alph*)]
	\item A cloração é considerada um processo de desinfecção aplicável à todas as águas em função de razões econômicas e de praticabilidade operacional. Dessa forma, quais os compostos de cloro mais comumente usados na desinfecção? Indique o teor aproximado de cloro ativo em cada um deles. 
	\item O processo de coagulação ou floculação possui como objetivo aumentar o tamanho das partículas dispersas na água, formando flocos que favoreçam a sedimentação mais acelerada. As substâncias que realizam esse trabalho são denominadas coagulantes e seu papel é neutralizar as cargas superficiais presentes nas partículas contaminantes, permitindo que ocorra a atração entre essas partículas. Uma das principais substâncias coagulantes, utilizada em larga escala no setor industrial, é o sulfato de alumínio. Escreva a reação balanceada que representa a forma como o sulfato de alumínio se comporta quando é adicionado à água.
	\item A adsorção é um processo empregado para a remoção de partículas dissolvidas no efluente, que não podem ser removidas por processos biológicos e não foram precipitadas nos processos de coagulação e floculação. Quais os principais contaminantes que podem ser removidos por esse tipo de processo? Indique os processos de adsorção e descreva o princípio de cada um.
	\item Lodo é o material formado nos processos de tratamento primário, nos sedimentadores do tratamento secundário (biológico) e nos sedimentadores dos floculadores. Porém, para minimizar custos com seu transporte aos aterros sanitários ou mesmo condicioná-lo a outros fins, deve-se aumentar sua concentração, em termos de matéria seca. Indique quais os principais processos utilizados para a concentração de lodo e descreva o princípio de cada um. 
	\item O que são os processos oxidativos avançados? Descreva.
\end{enumerate}
