 A água pura é um mau condutor de corrente elétrica e dessa forma fica muito difícil realizar sua eletrólise. Por isso, para realizar este fenômeno é necessário adicionar uma pequena quantidade de sal, como o \chemfig{Na_2SO_4} por exemplo, que torna o meio condutor. Considerando o movimento das espécies químicas para os eletrodos, é correto afirmar que:

\begin{enumerate}[label = (\alph*)]	
	\item no cátodo ocorre a semirreação:
		\schemestart
		2\chemfig{H^+{(aq)}} + 2\chemfig{e^{-}} \arrow{->} \chemfig{H_2{(g)}}
		\schemestop
	\item no ânodo ocorre a semirreação:
		\schemestart
		2\chemfig{H_2O{(l)}} \arrow{->} \chemfig{O_2{(g)}} + 4\chemfig{H^+} + 4\chemfig{e^{-}}
		\schemestop
	\item no ânodo ocorre a semirreação:
		\schemestart
		4\chemfig{OH^{-}{(aq)}} \arrow{->} \chemfig{O_2{(g)}} + 2\chemfig{H_2O{(l)}} + 4\chemfig{e^{-}}
		\schemestop
	\item no ânodo ocorre a semirreação:
		\schemestart
		2\chemfig{H_2O{(l)}} + \chemfig{e^{-}} \arrow{->} \chemfig{H_2{(g)}} + 2\chemfig{OH^{-}{(aq)}}
		\schemestop
	\item no cátodo ocorre a semirreação:
		\schemestart
		2\chemfig{Na^+{(aq)}} + 2\chemfig{e^{-}} \arrow{->} 2\chemfig{Na{(s)}}
		\schemestop
\end{enumerate}
