Aulas em laboratórios são utilizadas, entre outros tantos motivos, para se verificar, experimentalmente, um conhecimento adquirido em sala de aula. Assim, um professor apresenta um béquer a um aluno, com uma solução incolor desconhecida que pode ser de ácido clorídrico, ou ácido nítrico ou ácido sulfúrico. Um aluno identificou corretamente a solução de ácido sulfúrico.  Para tal, ele: 

\begin{enumerate}[label = (\alph*)]	
	\item adicionou gotas de um indicador para determinar a presença de íons hidrônio. 
	\item adicionou uma gota de solução de nitrato de prata e verificou a formação de precipitado. 
	\item adicionou uma solução de acetato de sódio, para formar ácido acético aquoso.
	\item adicionou solução de amônia para obter uma solução de amônio.
	\item adicionou gotas de uma solução de nitrato de bário, para verificar a formação de precipitado
\end{enumerate}
