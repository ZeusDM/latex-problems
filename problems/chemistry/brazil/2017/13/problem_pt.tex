Segundo o dicionário Houaiss, anfótero é um adjetivo usado para uma substância (ou íon) que pode se comportar como ácido ou como base.
Considerando o Golden Book da União Internacional de Química Pura e Aplicada (IUPAC – Internacional Union of Pure and Applied Chemistry) esta propriedade depende do meio no qual a substância (ou íon) é investigada (adaptado de http://goldbook.iupac.org/A00306.html).
Considere os quatro sistemas representados abaixo, que envolvem as seguintes espécies químicas: \chemfig{HCO_3^{-}}, \chemfig{Al{(OH)}_3}, \chemfig{HBr} e \chemfig{ZnO}.

\begin{enumerate}[label = (\Roman*)]
	
	\item
		\schemestart
		\chemfig{HCO_3^{-}{(aq)}} + \chemfig{H_2O{(l)}} \arrow{} \chemfig{H_3O^+{(aq)}} + \chemfig{CO_3^{2-}{(aq)}}
		\schemestop

		\schemestart
		\chemfig{HCO_3^{-}{(aq)}} + \chemfig{H_2O{(l)}} \arrow{} \chemfig{H_2CO_3{(aq)}} + \chemfig{OH^{-}{(aq)}}
		\schemestop

	\item
		\schemestart
		\chemfig{Al{(OH)}_3{(s)}} + 3 \chemfig{H_3O^+{(aq)}} \arrow{->} \chemfig{Al^{3+}{(aq)}} + 6 \chemfig{H_2O{(l)}}
		\schemestop

		\schemestart
		\chemfig{Al{(OH)}_3{(s)}} + \chemfig{OH^{-}{(aq)}} \arrow{->} \chemfig{Al{(OH)}_4^{-}{(aq)}}
		\schemestop

	\item
		\schemestart
		\chemfig{HBr{(g)}} + \chemfig{H_2O{(l)}} \arrow{->} 3 \chemfig{H_3O^+{(aq)}} + \chemfig{Br^{-}}
		\schemestop
		
		\schemestart
		\chemfig{HBr{(g)}} + \chemfig{OH^{-}{(aq)}}  \arrow{->} \chemfig{H_2O{(l)}} + \chemfig{Br^{-}{(aq)}}
		\schemestop
		
	\item
		\schemestart
		\chemfig{ZnO{(s)}} + 2 \chemfig{HCl{(aq)}} \arrow{->} \chemfig{ZnCl_2{(aq)}} + \chemfig{H_2O{(l)}} 
		\schemestop
		
		\schemestart
		\chemfig{ZnO{(s)}} + \chemfig{NaOH{(aq)}} \arrow{->} \chemfig{Na_2ZnO_2} + \chemfig{H_2O{(l)}}
		\schemestop
\end{enumerate}

\begin{enumerate}[label = (\alph*)]
	\item Qual(is) do(s) sistema(s) acima apresenta(m) espécie(s) anfótera(s)?
		Justifique.

	\item Dê o nome de todas as espécies envolvidas nos sistemas do item anterior:
\end{enumerate}

	O \chemfig{Al{(OH)}_3} tem sido utilizado, comumente, na forma de antiácido, para combater a azia estomacal.
	Por sua vez, o \chemfig{ZnO} é usado como pomada antisséptica, secativa e anti-inflamatória, que facilita a cicatrização da pele.
	O \chemfig{HCO^{-}} tem uma vasta utilização, tais como, combate à azia, uso como fermento químico, combate ao mau cheiro oriundo da sudorese, etc.

\begin{enumerate}[resume*]
	\item Tanto o \chemfig{NaHCO_3} como o \chemfig{NH_4HCO_3} são utilizados como fermento químico para uso alimentício, uma vez que sofrem decomposição durante seu aquecimento, formando gases que fazem crescer, por exemplo, um bolo.
		Partindo da mesma quantidade de ambos, uma colher de chá (equivalente a $5$ g), justifique a escolha de um deles como o melhor fermento para fazer um bolo.
\end{enumerate}

	O \chemfig{Al{(OH)}_3} tem sido contraindicado para o uso como antiácido, porque estudos indicam que lesões cerebrais encontradas na doença de Alzheimer contém alumínio.
	Assim, o consumo excessivo de compostos de alumínio pode causar ou contribuir para o desenvolvimento dessa doença.
	Com base nesta informação, o \chemfig{Mg{(OH)}_2} tem sido mais indicado para problemas de azia estomacal, uma vez que a sua eliminação pelo corpo humano é mais rápida.

\begin{enumerate}[resume*]
	\item Considerando uma mesma dose medicinal de \chemfig{Al{(OH)}_3} e \chemfig{Mg{(OH)}_2}, qual das duas substâncias teria uma ação mais eficaz no combate à azia estomacal?
		Justifique em termos químicos:
	\item Um aluno decide dissolver o \chemfig{ZnO} em água deionizada para verificar o p\chemfig{H} da solução resultante.
		Qual deverá o resultado desse experimento, tanto em relação à dissolução e ao p\chemfig{H}?
\end{enumerate}
