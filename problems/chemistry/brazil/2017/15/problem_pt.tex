O dióxido de enxofre é um gás emitido juntamente com óxidos de carbono na queima de combustíveis fósseis em veículos e indústrias.
Esse gás, produzido naturalmente nos vulcões, é usado em alguns processos industriais, como por exemplo, na produção de ácido sulfúrico.
O dióxido de enxofre é obtido a partir da combustão de enxofre ou de pirita.
Outro óxido de destaque, é o óxido nítrico, um gás produzido em algumas células, que regulam o funcionamento de outras células, configurando-se como um princípio sinalizador em sistemas biológicos.
Essa descoberta não só conferiu o Prêmio Nobel de Medicina em 1998 para Ignaro, Furchgott e Murad, como também abriu as portas para o desenvolvimento de tecnologias, inclusive na produção do Viagra$^\text{®}$.
Como fármaco, a produção do óxido nítrico começa com a reação entre \chemfig{SO_2}, ácido nítrico e água, originando, além desse gás, o ácido sulfúrico.
Como outro exemplo, tem-se o sulfeto de hidrogênio, um gás com odor de ovos podres e carne em decomposição.

Baseado nos conhecimentos sobre os gases, considere um recipiente de volume igual a $100$ L a $127$ $^\circ$C, no qual foram adicionados $6,80$ g de gás sulfídrico, $9,60$ g de anidrido sulfuroso, $6,00$ g de óxido nítrico e $6,60$ g de anidrido carbônico. Responda: 

\begin{enumerate}[label = (\alph*)]
	\item Escreva a equação química da reação de produção do óxido nítrico. 
	\item Qual a pressão total do sistema, em equilíbrio? 
	\item Calcule as frações molares das substâncias.
	\item Qual é a pressão parcial do gás de maior fração molar? 
	\item Com base na Lei de Graham, determine a velocidade de efusão do anidrido carbônico em relação ao gás sulfídrico.
\end{enumerate}
