Em um laboratório, dispõem-se de alguns frascos de soluções, recém preparados nas condições ambientes, que apresentam os seguintes solutos: 

\begin{enumerate}[label = (\roman*)]
	\item Cloro.
	\item Sulfeto de sódio.
	\item Iodeto de potássio.
	\item Iodeto de potássio.
\end{enumerate}

Em relação às propriedades dessas soluções, assinalar a alternativa que contém a proposição incorreta:

\begin{enumerate}[label = (\alph*)] 	
	\item Ao verificar o p\chemfig{H} da solução (ii), ela terá um valor maior do que $7,0$ e a solução (iii) terá caráter neutro, à $25 ^\circ$C. 
	\item Misturando-se (ii) e (iv), será formado um precipitado de cor azul, o sulfeto de cobre (II).
	\item A solução (iv) é colorida e a solução (iii) é incolor. 
	\item A solução (i) é a que apresenta a menor condutibilidade elétrica.
	\item A mistura resultante entre as soluções (ii) e (iii) não formará precipitado.
\end{enumerate}
