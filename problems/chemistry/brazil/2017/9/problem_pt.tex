Para se entender o comportamento de moléculas, é importante conhecer as suas geometrias. A geometria molecular pode definir a polaridade das moléculas e, por conseguinte, suas propriedades físicas. Sobre esse assunto, assinalar a alternativa correta: 

\begin{enumerate}[label = (\alph*)]	
	\item O gás carbônico possui ligações polares, mas a molécula é apolar. A sua geometria é definida pela hibridação $sp$. 
	\item As moléculas simétricas são sempre apolares, independentemente de sua hibridação.
	\item A amônia possui geometria piramidal, é polar e possui $sp^3$, enquanto íon amônio possui geometria tetraédrica, sendo apolar e com hibridação $sp^3d$.
	\item As moléculas apolares apresentam forças intermoleculares do tipo ligações de hidrogênio e apresentam geometria linear, devido à hibridação $sp$.
	\item  O \chemfig{BF_3} tem geometria trigonal planar, por conta de sua hibridação $sp$. É uma molécula altamente polar devido à presença de flúor, que é o elemento mais eletronegativo da tabela periódica. 
\end{enumerate}
