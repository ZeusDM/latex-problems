O ácido acetilsalicílico, mais conhecido pelo nome comercial de aspirina, é o princípio ativo em muitos medicamentos de propriedades analgésica, antipirética e anti-inflamatória.
Ele também é usado como anti-agregante plaquetário.
É um monoácido fraco (fórmula molecular \chemfig{C_9H_8O_4}) cuja base conjugada é o ânion acetilsalicilato (\chemfig{C_9H_7O_4^{-}}).
É facilmente sintetizado a partir do ácido salicílico.
Considere que um grama de ácido acetilsalicílico dissolve-se em 450 mL de água para obter uma solução saturada com pH $\approx 2,73$.
Com base nas informações determine o que pede:

\begin{enumerate}[label = (\alph*)]
	\item Escrever a equação química e a expressão da constante de equilíbrio do ácido na solução? 
	\item Qual é o $K_a$ do ácido acetisalicílico?
	\item Qual é o pH final se $50,0$ mL de acetilsalicilato de sódio $0,10$ mol$\cdot$L$^{-1}$ são adicionados a solução saturada de ácido acetilsalicílico.
	\item Qual é o pH final se $50,0$ mL de \chemfig{HCl} $0,10$ mol$\cdot$L$^{-1}$ são adicionados a solução saturada de ácido acetilsalicílico.
	\item Qual é o pH final se $50,0$ mL de NaOH $0,10$ mol$\cdot$L$^{-1}$ são adicionados a solução saturada de ácido acetilsalicílico.
\end{enumerate}
