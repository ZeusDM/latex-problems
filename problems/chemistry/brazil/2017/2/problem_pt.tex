No final do século XIX,  Mendeleiev apresentou à Sociedade Russa de Química sua proposta de Tabela Periódica, com muitos espaços em branco, reservados para elementos ainda não descobertos. A proposta de Mendeleiev era baseada na convicção da existência de relações periódicas entre as propriedades físico-químicas dos elementos. Um outro químico russo,  Berlikov, fez duras críticas, concluindo com uma pergunta: “Pode a natureza ter espaços em branco?” Mendeleiev manteve sua proposta, que se mostrou coerente, sendo que os espaços em branco foram preenchidos gradativamente. A tabela abaixo apresenta duas propriedades periódicas, com unidades de medida adequadas: 

\begin{center}
\renewcommand{\arraystretch}{1.5}
\begin{tabular}{|c|c|c|}
	\hline
	\textbf{Elemento} & \textbf{Propriedade 1} & \textbf{Propriedade 2}\\\hline
	Berílio & 1,12 & 215\\\hline
	Cálcio & 1,97 & 141\\\hline
	Selênio & 1,40 & 255 \\\hline
\end{tabular}
\end{center}

Assinalar a alternativa que indica as propriedades 1 e 2, respectivamente: 

\begin{enumerate}[label = (\alph*)]
	
	\item Raio Atômico e Densidade Absoluta.
	\item Eletropositividade e Potencial de Ionização.
	\item Raio Atômico e Potencial de Ionização.
	\item Eletronegatividade e Raio Atômico. 
	\item Densidade absoluta e Eletroafinidade.
\end{enumerate}
