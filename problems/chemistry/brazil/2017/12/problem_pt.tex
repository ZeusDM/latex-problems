O nitrogênio é o sétimo elemento da tabela periódica, ocorrendo naturalmente
na forma gasosa \chemfig{N_2} à temperatura ambiente e é muito abundante na atmosfera. Trata-se de um elemento muito versátil, que se combina facilmente com o oxigênio, resultando em diversos óxidos de nitrogênio. Sobre o nitrogênio e seus óxidos responda:

\begin{enumerate}[label = (\alph*)]
	\item Desenhe as estruturas de Lewis de \chemfig{NO}, \chemfig{NO^+} e \chemfig{NO^{-}}.
	\item \chemfig{NO} é um gás incolor, que se torna marrom quando exposto ao ar devido a formação de \chemfig{NO_2}, conforme a reação \schemestart 2\chemfig{NO_2{(g)}} + \chemfig{O_2{(g)}} \arrow{->} 2\chemfig{NO_2} \schemestop.
		O dióxido de nitrogênio pode reagir com a água, regenerando o NO.
		Escreva está reação química balanceada, indicando os estados físicos das espécies e usando a seta de reação adequada.
	\item O dióxido de nitrogênio existe em equilíbrio com seu dímero, o \chemfig{N_2O_4}.
		Qual o estado de oxidação do átomo de \chemfig{N} no monóxido de nitrogênio, no dióxido de nitrogênio e no tetróxido de dinitrogênio?
		Ocorre mudança no estado de oxidação do átomo de nitrogênio do dióxido de nitrogênio quando este composto sofre dimerização?
		Explique.
	\item A constante de equilíbrio em termos das concentrações ($K_c$) de \chemfig{NO_2} e \chemfig{N_2O_4} a $298$ K é $1,70$ mol$^{-1}\cdot$L, enquanto a constante de equilíbrio em termos de pressões parciais ($K_p$) é $6,7 \times 10^{-5}$ para unidade em Pascal (Pa) ou $6,8$ em atmosfera (atm).
		Escreva as expressões para as constantes de equilíbrio $K_c$ e $K_p$ para esta reação e, considerando a equação de Clapeyron ($pV = nRT$), deduza a relação entre $K_c$ e $K_p$.
	\item O $\Delta_{r}H(298 \mathrm{K})$ da reação de dimerização do dióxido de nitrogênio é $-57$ kJ mol$^{-1}$.
		Esta reação e endotérmica ou exotérmica? Justifique.
		Usando os dados contidos nessa questão sobre a dimerização do \chemfig{NO_2}, calcule a variação da entropia da reação de dimerização do dióxido de nitrogênio e relate o valor como $T\Delta S$.
		Se esta reação dependesse somente do termo entrópico, ela ocorreria espontaneamente?
		Justifique, considerando as condições padrão.
\end{enumerate}

Dados:
\begin{enumerate*}[label = , itemjoin={\qquad}]
	\item 	$\Delta G = \Delta H - T\Delta S$
	\item	$\Delta G=-RT\ln(K)$
	\item	$R=8,314$ J$\cdot$K$^{-1}\cdot$mol$^{-1}$
\end{enumerate*}

