Corrosão é um termo para a deterioração dos metais através da reação química com o ambiente.
Um problema particularmente difícil para o químico arqueológico é a formação de \chemfig{CuCl}, uma substância instável, que é formada pela corrosão do cobre e suas ligas.
Embora objetos de cobre e bronze possam sobreviver a soterramento por séculos sem deterioração significativa, a exposição ao ar pode fazer com que o cloreto cuproso reaja com o oxigênio atmosférico para formar óxido cuproso e cloreto cúprico.
O cloreto cúprico reage então com o metal livre para produzir cloreto cuproso.
A reação contínua do oxigênio e da água com cloreto cuproso causa ``doença de bronze'',
que consiste em manchas de um depósito verde pálido e em pó de
[\chemfig{CuCl_2}$\cdot${$3$}\chemfig{Cu{(OH)}_2}$\cdot$\chemfig{H_2O}]
na superfície do objeto que continuam a crescer.

\begin{enumerate}[label = (\alph*)]
	
	\item Usando esta série de reações descritas, complete e equilibre as seguintes equações, que
juntas resultam em doença de bronze:
	
	Equação 1:
	\schemestart
		\_\_\_\_\_\_ + \chemfig{O_2} \arrow{->} \_\_\_\_\_\_ + \_\_\_\_\_\_
	\schemestop
	
	Equação 2:
	\schemestart
		\_\_\_\_\_\_ + \chemfig{Cu} \arrow{->} \_\_\_\_\_\_
	\schemestop
	
	Equação 3:
	\schemestart
	\_\_\_\_\_\_ + \chemfig{O_2} + \chemfig{H_2O} \arrow{->} \chemfig{CuCl_2}$\cdot${$3$}\chemfig{Cu{(OH)}_2}$\cdot$\chemfig{H_2O} (doença de bronze) + \chemfig{CuCl_2}
	\schemestop

	\item Quais espécies são os oxidantes e os redutores em cada equação?
	\item Se $8,0\%$ em massa de uma estátua de cobre de $350,0$ kg consistia de \item{CuCl}, e a estátua sucumbisse à doença de bronze, quantas libras de hidrato verde em pó seriam formadas?

		Dado: 1 libra = $0,4536$ kg.
	\item  Quais fatores podem afetar a taxa de deterioração de um artefato de bronze recentemente escavado?
	\item A chuva ácida, que tem pH $< 5,6$, se dá em locais com elevadas concentrações na atmosfera de óxidos de enxofre, nitrogênio e carbono. Esses óxidos, carreados pela água precipitada da chuva, formam basicamente os ácidos sulfúrico, nítrico e carbônico, que são os causadores de danos ambientais e em monumentos históricos. Considerando o objeto citado no item (c), quando exposto a eventos intempéricos, nesse caso, chuva ácida, escreva a(s) provável(is) reação(ões) química(s) balanceada(s) do cobre com cada ácido. 
\end{enumerate}
