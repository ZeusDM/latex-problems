Os ácidos, quanto à sua composição, são agrupados em hidrácidos e oxiácidos. No caso de oxiácidos, em que o elemento central é o enxofre, podem-se derivar vários ácidos devido à mudança do seu estado de oxidação. Nesse contexto, considerar os seguintes oxiácidos: \chemfig{H_2SO_4}, \chemfig{H_2SO_3}, \chemfig{H_2SO_2}, \chemfig{H_2S_2O_7}, \chemfig{H_2S_2O_6}, \chemfig{H_2S_2O_5} e \chemfig{H_2S_2O_4},. Indicar a alternativa CORRETA que mostra os óxiácidos em ordem crescente de estado de oxidação do enxofre:

\begin{center}
\renewcommand{\arraystretch}{1.5}
\begin{tabular}{c|p{1.7cm}p{1.7cm}p{1.7cm}p{1.7cm}p{1.7cm}}
	NOx & +2 & +3 & +4 & +5 & +6\\\hline
	(a) & \chemfig{H_2SO_2} & \chemfig{H_2S_2O_4} & \chemfig{H_2SO_3} \newline \chemfig{H_2S_2O_5} & \chemfig{H_2S_2O_6} & \chemfig{H_2SO_4} \newline \chemfig{H_2S_2O_7}\\
	(b) & \chemfig{H_2SO_2} \newline \chemfig{H_2S_2O_5} & \chemfig{H_2S_2O_4} \newline \chemfig{H_2SO_4} & \chemfig{H_2SO_3} & \chemfig{H_2S_2O_6} & \chemfig{H_2S_2O_7}\\
	(c) & \chemfig{H_2SO_2} & \chemfig{H_2S_2O_4} & \chemfig{H_2S_2O_5} \newline \chemfig{H_2S_2O_7} & \chemfig{H_2S_2O_6} \newline \chemfig{H_2SO_3} & \chemfig{H_2SO_4}\\
	(d) & \chemfig{H_2SO_2} & \chemfig{H_2S_2O_7} & \chemfig{H_2SO_3} \newline \chemfig{H_2S_2O_5} & \chemfig{H_2S_2O_6} \newline \chemfig{H_2S_2O_6} & \chemfig{H_2SO_4}\\
	(e) & \chemfig{H_2S_2O_4} & \chemfig{H_2S_2O_6} & \chemfig{H_2SO_3} \newline \chemfig{H_2S_2O_5} & \chemfig{H_2SO_2} & \chemfig{H_2SO_4} \newline \chemfig{H_2S_2O_7}	
\end{tabular}
\end{center}
