\item[\textbf{C1.}]
In the plane we consider rectangles whose sides are parallel to the coordinate axes and have positive length. Such a rectangle will be called a 
box. Two boxes 
intersect
 if they have a common point in their interior or on their boundary. Find the largest 
$ n$
 for which there exist 
$ n$
 boxes 
$ B_1$, 
$ \ldots$, 
$ B_n$
 such that 
$ B_i$
 and 
$ B_j$
 intersect if and only if 
$ i\not\equiv j\pm 1\pmod n$.

\item[\textbf{C2.}]
Let 
$n \in \mathbb N$
 and 
$A_n$
 set of all permutations 
$(a_1, \ldots, a_n)$
 of the set 
$\{1, 2, \ldots , n\}$
 for which
\[k|2(a_1 + \cdots+ a_k), \text{ for all } 1 \leq k \leq n.\]

Find the number of elements of the set 
$A_n$.

\item[\textbf{C3.}]
In the coordinate plane consider the set 
$ S$
 of all points with integer coordinates. For a positive integer 
$ k$, 
 two distinct points 
$ a$,
$ B\in S$
 will be called 
$ k$-friends
 if there is a point 
$ C\in S$
 such that the area of the triangle 
$ ABC$
 is equal to 
$ k$.
 A set 
$ T\subset S$
 will be called 
$ k$-clique
 if every two points in 
$ T$
 are 
$ k$-friends. Find the least positive integer 
$ k$
 for which there exits a 
$ k$-clique with more than 200 elements.

\item[\textbf{C4.}]
Let 
$ n$
 and 
$ k$
 be positive integers with 
$ k \geq n$
 and 
$ k - n$
 an even number. Let 
$ 2n$
 lamps labelled 
$ 1$, 
$ 2$, 
$\dots$, 
$ 2n$
 be given, each of which can be either 
on
 or 
off
. Initially all the lamps are off. We consider sequences of steps: at each step one of the lamps is switched (from on to off or from off to on).


Let 
$ N$
 be the number of such sequences consisting of 
$ k$
 steps and resulting in the state where lamps 
$ 1$
 through 
$ n$
 are all on, and lamps 
$ n + 1$
 through 
$ 2n$
 are all off.


Let 
$ M$
 be number of such sequences consisting of 
$ k$
 steps, resulting in the state where lamps 
$ 1$
 through 
$ n$
 are all on, and lamps 
$ n + 1$
 through 
$ 2n$
 are all off, but where none of the lamps 
$ n + 1$
 through 
$ 2n$
 is ever switched on.


Determine 
$ \frac{N}{M}$.

\item[\textbf{C5.}]
Let 
$ S = \{x_1, x_2, \ldots, x_{k + l}\}$
 be a 
$ (k + l)$-element set of real numbers contained in the interval 
$ [0, 1]$; $k$
 and 
$ l$
 are positive integers. A 
$ k$-element subset 
$ A\subset S$
 is called 
nice
 if
\[ \left |\frac {1}{k}\sum_{x_i\in A} x_i - \frac {1}{l}\sum_{x_j\in S\setminus A} x_j\right |\le \frac {k + l}{2kl}\]


Prove that the number of nice subsets is at least 
$ \dfrac{2}{k + l}\dbinom{k + l}{k}$.

\item[\textbf{C6.}]
For 
$ n\ge 2$, 
 let 
$ S_1$, 
$ S_2$, 
$ \ldots$, 
$ S_{2^n}$
 be 
$ 2^n$
 subsets of 
$ A = \{1, 2, 3, \ldots, 2^{n + 1}\}$
 that satisfy the following property: There do not exist indices 
$ a$
 and 
$ b$
 with 
$ a < b$
 and elements 
$ x$, 
$ y$, 
$ z\in A$
 with 
$ x < y < z$
 and 
$ y$, 
$ z\in S_a$, 
 and 
$ x$, 
$ z\in S_b$.
 Prove that at least one of the sets 
$ S_1$, 
$ S_2$, 
$ \ldots$, 
$ S_{2^n}$
 contains no more than 
$ 4n$
 elements.