
\item[\textbf{N1.}]Determine all pairs $(x,y)$ of positive integers such that $x^{2}y+x+y$ is divisible by $xy^{2}+y+7$.

\item[\textbf{N2.}]Determine all pairs $(a,b)$ of real numbers such that $a \lfloor bn \rfloor =b \lfloor an \rfloor $ for all positive integers $n$. (Note that $\lfloor x\rfloor $ denotes the greatest integer less than or equal to $x$.)

\item[\textbf{N3.}]Determine the smallest integer $n\geq 4$ for which one can choose four different numbers $a,b,c$ and $d$ from any $n$ distinct integers such that $a+b-c-d$ is divisible by $20$.

\item[\textbf{N4.}]A sequence of integers $ a_{1},a_{2},a_{3},\ldots$ is defined as follows: $ a_{1} = 1$ and for $ n\geq 1$,  $ a_{n + 1}$ is the smallest integer greater than $ a_{n}$ such that $ a_{i} + a_{j}\neq 3a_{k}$ for any $ i,j$ and $ k$ in $ \{1,2,3,\ldots ,n + 1\}$,  not necessarily distinct. Determine $ a_{1998}$.

\item[\textbf{N5.}]Determine all positive integers $n$ for which there exists an integer $m$ such that ${2^{n}-1}$ is a divisor of ${m^{2}+9}$.

\item[\textbf{N6.}]For any positive integer $n$,  let $\tau (n)$ denote the number of its positive divisors (including 1 and itself). Determine all positive integers $m$ for which there exists a positive integer $n$ such that $\frac{\tau (n^{2})}{\tau (n)}=m$.

\item[\textbf{N7.}]Prove that for each positive integer $n$,  there exists a positive integer with the following properties: It has exactly $n$ digits. None of the digits is 0. It is divisible by the sum of its digits.

\item[\textbf{N8.}]Let $a_{0},a_{1},a_{2},\ldots $ be an increasing sequence of nonnegative integers such that every nonnegative integer can be expressed uniquely in the form $a_{i}+2a_{j}+4a_{k}$,  where $i,j$ and $k$ are not necessarily distinct. Determine $a_{1998}$.

