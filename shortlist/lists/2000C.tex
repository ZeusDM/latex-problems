\item[\textbf{C1.}]A magician has one hundred cards numbered 1 to 100. He puts them into three boxes, a red one, a white one and a blue one, so that each box contains at least one card. A member of the audience draws two cards from two different boxes and announces the sum of numbers on those cards. Given this information, the magician locates the box from which no card has been drawn.

How many ways are there to put the cards in the three boxes so that the trick works?

\item[\textbf{C2.}]A staircase-brick with 3 steps of width 2 is made of 12 unit cubes. Determine all integers $ n$ for which it is possible to build a cube of side $ n$ using such bricks.

\item[\textbf{C3.}]Let $ n \geq 4$ be a fixed positive integer. Given a set $ S = \{P_1, P_2, \ldots, P_n\}$ of $ n$ points in the plane such that no three are collinear and no four concyclic, let $ a_t,$ $ 1 \leq t \leq n,$ be the number of circles $ P_iP_jP_k$ that contain $ P_t$ in their interior, and let \[m(S)=a_1+a_2+\cdots + a_n.\]Prove that there exists a positive integer $ f(n),$ depending only on $ n,$ such that the points of $ S$ are the vertices of a convex polygon if and only if $ m(S) = f(n).$

\item[\textbf{C4.}]Let $ n$ and $ k$ be positive integers such that $ \frac{1}{2} n < k \leq \frac{2}{3} n.$ Find the least number $ m$ for which it is possible to place $ m$ pawns on $ m$ squares of an $ n \times n$ chessboard so that no column or row contains a block of $ k$ adjacent unoccupied squares.

\item[\textbf{C5.}]In the plane we have $n$ rectangles with parallel sides.  The sides of distinct rectangles lie on distinct lines.  The boundaries of the rectangles cut the plane into connected regions.  A region is nice if it has at least one of the vertices of the $n$ rectangles on the boundary.  Prove that the sum of the numbers of the vertices of all nice regions is less than $40n$.  (There can be nonconvex regions as well as regions with more than one boundary curve.)

\item[\textbf{C6.}]Let $ p$ and $ q$ be relatively prime positive integers. A subset $ S$ of $ \{0, 1, 2, \ldots \}$ is called ideal if $ 0 \in S$ and for each element $ n \in S,$ the integers $ n + p$ and $ n + q$ belong to $ S.$ Determine the number of ideal subsets of $ \{0, 1, 2, \ldots \}.$