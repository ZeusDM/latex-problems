\item[\textbf{C1.}]
Let 
$n$
 be an positive integer. Find the smallest integer 
$k$
 with the following property; Given any real numbers 
$a_1 , \cdots , a_d $
 such that 
$a_1 + a_2 + \cdots + a_d = n$
 and 
$0 \le a_i \le 1$
 for 
$i=1,2,\cdots ,d$, 
 it is possible to partition these numbers into 
$k$
 groups (some of which may be empty) such that the sum of the numbers in each group is at most 
$1$.

\item[\textbf{C2.}]
A configuration of 
$4027$
 points in the plane is called Colombian if it consists of 
$2013$
 red points and 
$2014$
 blue points, and no three of the points of the configuration are collinear. By drawing some lines, the plane is divided into several regions. An arrangement of lines is good for a Colombian configuration if the following two conditions are satisfied:


(i) No line passes through any point of the configuration.


(ii) No region contains points of both colors.


Find the least value of 
$k$
 such that for any Colombian configuration of 
$4027$
 points, there is a good arrangement of 
$k$
 lines.

\item[\textbf{C3.}]
A crazy physicist discovered a new kind of particle wich he called an imon, after some of them mysteriously appeared in his lab. Some pairs of imons in the lab can be entangled, and each imon can participate in many entanglement relations. The physicist has found a way to perform the following two kinds of operations with these particles, one operation at a time.

\begin{itemize}

\item[(i)] If some imon is entangled with an odd number of other imons in the lab, then the physicist can destroy it.


\item[(ii)] At any moment, he may double the whole family of imons in the lab by creating a copy 
$I'$
 of each imon 
$I$.
 During this procedure, the two copies 
$I'$
 and 
$J'$
 become entangled if and only if the original imons 
$I$
 and 
$J$
 are entangled, and each copy 
$I'$
 becomes entangled with its original imon 
$I$;
 no other entanglements occur or disappear at this moment.

\end{itemize}

Prove that the physicist may apply a sequence of much operations resulting in a family of imons, no two of which are entangled.

\item[\textbf{C4.}]
Let 
$n$
 be a positive integer, and let 
$A$
 be a subset of 
$\{ 1,\cdots ,n\}$.
 An 
$A$-partition of 
$n$
 into 
$k$
 parts is a representation of n as a sum 
$n = a_1 + \cdots + a_k$, 
 where the parts 
$a_1 , \cdots , a_k $
 belong to 
$A$
 and are not necessarily distinct. The number of different parts in such a partition is the number of (distinct) elements in the set 
$\{ a_1 , a_2 , \cdots , a_k \} $.


We say that an 
$A$-partition of 
$n$
 into 
$k$
 parts is optimal if there is no 
$A$-partition of 
$n$
 into 
$r$
 parts with 
$r<k$.
 Prove that any optimal 
$A$-partition of 
$n$
 contains at most 
$\sqrt[3]{6n}$
 different parts.

\item[\textbf{C5.}]
Let 
$r$
 be a positive integer, and let 
$a_0 , a_1 , \cdots $
 be an infinite sequence of real numbers. Assume that for all nonnegative integers 
$m$
 and 
$s$
 there exists a positive integer 
$n \in [m+1, m+r]$
 such that
\[ a_m + a_{m+1} +\cdots +a_{m+s} = a_n + a_{n+1} +\cdots +a_{n+s} \]


Prove that the sequence is periodic, i.e. there exists some 
$p \ge 1 $
 such that 
$a_{n+p} =a_n $
 for all 
$n \ge 0$.

\item[\textbf{C6.}]
In some country several pairs of cities are connected by direct two-way flights. It is possible to go from any city to any other by a sequence of flights. The distance between two cities is defined to be the least possible numbers of flights required to go from one of them to the other. It is known that for any city there are at most 
$100$
 cities at distance exactly three from it. Prove that  there is no city such that more than 
$2550$
 other cities have distance exactly four from it.

\item[\textbf{C7.}]
Let 
$n \ge 3$
 be an integer, and consider a circle with 
$n + 1$
 equally spaced points marked on it. Consider all labellings of these points with the numbers 
$0, 1, ... , n$
 such that each label is used exactly once; two such labellings are considered to be the same if one can be obtained from the other by a rotation of the circle. A labelling is called 
beautiful
 if, for any four labels 
$a < b < c < d$
 with 
$a + d = b + c$, 
 the chord joining the points labelled 
$a$
 and 
$d$
 does not intersect the chord joining the points labelled 
$b$
 and 
$c$.


Let 
$M$
 be the number of beautiful labelings, and let N be the number of ordered pairs 
$(x, y)$
 of positive integers such that 
$x + y \le n$
 and 
$\gcd(x, y) = 1$.
 Prove that 
$$M = N + 1.$$

\item[\textbf{C8.}]
Players 
$A$
 and 
$B$
 play a "paintful" game on the real line. Player 
$A$
 has a pot of paint with four units of black ink. A quantity 
$p$
 of this ink suffices to blacken a (closed) real interval of length 
$p$.
 In every round, player 
$A$
 picks some positive integer 
$m$
 and provides 
$1/2^m $
 units of ink from the pot. Player 
$B$
 then picks an integer 
$k$
 and blackens the interval from 
$k/2^m$
 to 
$(k+1)/2^m$
 (some parts of this interval may have been blackened before). The goal of player 
$A$
 is to reach a situation where the pot is empty and the interval 
$[0,1]$
 is not completely blackened.


Decide whether there exists a strategy for player 
$A$
 to win in a finite number of moves.

