\item[\textbf{N1.}]Let $\tau(n)$ denote the number of positive divisors of the positive integer $n$. Prove that there exist infinitely many positive integers $a$ such that the equation $ \tau(an)=n $ does not have a positive integer solution $n$.

\item[\textbf{N2.}]The function $f$ from the set $\mathbb{N}$ of positive integers into itself is defined by the equality \[f(n)=\sum_{k=1}^{n} \gcd(k,n),\qquad n\in \mathbb{N}.\]

a) Prove that $f(mn)=f(m)f(n)$ for every two relatively prime ${m,n\in\mathbb{N}}$.

b) Prove that for each $a\in\mathbb{N}$ the equation $f(x)=ax$ has a solution.

c) Find all ${a\in\mathbb{N}}$ such that the equation $f(x)=ax$ has a unique solution.

\item[\textbf{N3.}]Find all functions $ f: \mathbb{N^{*}}\to \mathbb{N^{*}}$ satisfying\[ \left(f^{2}\left(m\right)+f\left(n\right)\right) \mid \left(m^{2}+n\right)^{2}\]

for any two positive integers $ m$ and $ n$.Remark. The abbreviation $ \mathbb{N^{*}}$ stands for the set of all positive integers:$ \mathbb{N^{*}}=\left\{1,2,3,...\right\}$.

By $ f^{2}\left(m\right)$,  we mean $ \left(f\left(m\right)\right)^{2}$ (and not $ f\left(f\left(m\right)\right)$).

\item[\textbf{N4.}]Let $k$ be a fixed integer greater than 1, and let ${m=4k^2-5}$. Show that there exist positive integers $a$ and $b$ such that the sequence $(x_n)$ defined by \[x_0=a,\quad x_1=b,\quad x_{n+2}=x_{n+1}+x_n\quad\text{for}\quad n=0,1,2,\dots,\] has all of its terms relatively prime to $m$.

\item[\textbf{N5.}]We call a positive integer alternating if every two consecutive digits in its decimal representation are of different parity.

Find all positive integers $n$ such that $n$ has a multiple which is alternating.

\item[\textbf{N6.}]Given an integer ${n>1}$,  denote by $P_{n}$ the product of all positive integers $x$ less than $n$ and such that $n$ divides ${x^2-1}$. For each ${n>1}$,  find the remainder of $P_{n}$ on division by $n$.

\item[\textbf{N7.}]Let $p$ be an odd prime and $n$ a positive integer. In the coordinate plane, eight distinct points with integer coordinates lie on a circle with diameter of length $p^{n}$. Prove that there exists a triangle with vertices at three of the given points such that the squares of its side lengths are integers divisible by $p^{n+1}$.

