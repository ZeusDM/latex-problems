
\item[\textbf{C1.}]
Several positive integers are written in a row. Iteratively, Alice chooses two adjacent numbers 
$x$
 and 
$y$
 such that 
$x>y$
 and 
$x$
 is to the left of 
$y$, 
 and replaces the pair 
$(x,y)$
 by either 
$(y+1,x)$
 or 
$(x-1,x)$.
 Prove that she can perform only finitely many such iterations.

\item[\textbf{C2.}]
Let 
$n \geq 1$
 be an integer. What is the maximum number of disjoint pairs of elements of the set 
$\{ 1,2,\ldots , n \}$
 such that the sums of the different pairs are different integers not exceeding 
$n$
?

\item[\textbf{C3.}]
In a 
$999 \times 999$
 square table some cells are white and the remaining ones are red. Let 
$T$
 be the number of triples 
$(C_1,C_2,C_3)$
 of cells, the first two in the same row and the last two in the same column, with 
$C_1,C_3$
 white and 
$C_2$
 red. Find the maximum value 
$T$
 can attain.

\item[\textbf{C4.}]
Players 
$A$
 and 
$B$
 play a game with 
$N \geq 2012$
 coins and 
$2012$
 boxes arranged around a circle. Initially 
$A$
 distributes the coins among the boxes so that there is at least 
$1$
 coin in each box. Then the two of them make moves in the order 
$B,A,B,A,\ldots $
 by the following rules:
 
(a)
 On every move of his 
$B$
 passes 
$1$
 coin from every box to an adjacent box.
 
(b)
 On every move of hers 
$A$
 chooses several coins that were 
not
 involved in 
$B$'s previous move and are in different boxes. She passes every coin to and adjacent box.


Player 
$A$'s goal is to ensure at least 
$1$
 coin in each box after every move of hers, regardless of how 
$B$
 plays and how many moves are made. Find the least 
$N$
 that enables her to succeed.

\item[\textbf{C5.}]
The columns and the row of a 
$3n \times 3n$
 square board are numbered 
$1,2,\ldots ,3n$.
 Every square 
$(x,y)$
 with 
$1 \leq x,y \leq 3n$
 is colored asparagus, byzantium or citrine according as the modulo 
$3$
 remainder of 
$x+y$
 is 
$0,1$
 or 
$2$
 respectively. One token colored asparagus, byzantium or citrine is placed on each square, so that there are 
$3n^2$
 tokens of each color.


Suppose that one can permute the tokens so that each token is moved to a distance of at most 
$d$
 from its original position, each asparagus token replaces a byzantium token, each byzantium token replaces a citrine token, and each citrine token replaces an asparagus token. Prove that it is possible to permute the tokens so that each token is moved to a distance of at most 
$d+2$
 from its original position, and each square contains a token with the same color as the square.

\item[\textbf{C6.}]
The 
liar's guessing game
 is a game played between two players 
$A$
 and 
$B$.
 The rules of the game depend on two positive integers 
$k$
 and 
$n$
 which are known to both players.


At the start of the game 
$A$
 chooses integers 
$x$
 and 
$N$
 with 
$1 \le x \le N.$
 Player 
$A$
 keeps 
$x$
 secret, and truthfully tells 
$N$
 to player 
$B$.
 Player 
$B$
 now tries to obtain information about 
$x$
 by asking player 
$A$
 questions as follows: each question consists of 
$B$
 specifying an arbitrary set 
$S$
 of positive integers (possibly one specified in some previous question), and asking 
$A$
 whether 
$x$
 belongs to 
$S$.
 Player 
$B$
 may ask as many questions as he wishes. After each question, player 
$A$
 must immediately answer it with 
yes
 or 
no
, but is allowed to lie as many times as she wants; the only restriction is that, among any 
$k+1$
 consecutive answers, at least one answer must be truthful.


After 
$B$
 has asked as many questions as he wants, he must specify a set 
$X$
 of at most 
$n$
 positive integers. If 
$x$
 belongs to 
$X$, 
 then 
$B$
 wins; otherwise, he loses. Prove that:


1. If 
$n \ge 2^k,$
 then 
$B$
 can guarantee a win.


2. For all sufficiently large 
$k$, 
 there exists an integer 
$n \ge (1.99)^k$
 such that 
$B$
 cannot guarantee a win.

\item[\textbf{C7.}]
There are given 
$2^{500}$
 points on a circle labeled 
$1,2,\ldots ,2^{500}$
 in some order. Prove that one can choose 
$100$
 pairwise disjoint chords joining some of theses points so that the 
$100$
 sums of the pairs of numbers at the endpoints of the chosen chord are equal.

