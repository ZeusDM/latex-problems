\item[\textbf{G1.}]
Given a triangle 
$ABC$
 satisfying 
$AC+BC=3\cdot AB$.
 The incircle of triangle 
$ABC$
 has center 
$I$
 and touches the sides 
$BC$
 and 
$CA$
 at the points 
$D$
 and 
$E$, 
 respectively. Let 
$K$
 and 
$L$
 be the reflections of the points 
$D$
 and 
$E$
 with respect to 
$I$.
 Prove that the points 
$A$, 
$B$, 
$K$, 
$L$
 lie on one circle.

\item[\textbf{G2.}]
Six points are  chosen on the sides of an equilateral triangle 
$ABC$:
$A_1$,
$A_2$ on 
$BC$,
$B_1$, 
$B_2$
 on 
$CA$
 and 
$C_1$, 
$C_2$
 on 
$AB$, 
 such that they are the vertices of a convex hexagon 
$A_1A_2B_1B_2C_1C_2$
 with equal side lengths.


Prove that the lines 
$A_1B_2$, 
$B_1C_2$
 and 
$C_1A_2$
 are concurrent.

\item[\textbf{G3.}]
Let 
$ABCD$
 be a parallelogram. A variable line 
$g$
 through the vertex 
$A$
 intersects the rays 
$BC$
 and 
$DC$
 at the points 
$X$
 and 
$Y$, 
 respectively. Let 
$K$
 and 
$L$
 be the 
$A$-excenters of the triangles 
$ABX$
 and 
$ADY$.
 Show that the angle 
$\measuredangle KCL$
 is independent of the line 
$g$.

\item[\textbf{G4.}]
Let 
$ABCD$
 be a fixed convex quadrilateral with 
$BC=DA$
 and 
$BC$
 not parallel with 
$DA$.
 Let two variable points 
$E$
 and 
$F$
 lie of the sides 
$BC$
 and 
$DA$, 
 respectively and satisfy 
$BE=DF$.
 The lines 
$AC$
 and 
$BD$
 meet at 
$P$, 
 the lines 
$BD$
 and 
$EF$
 meet at 
$Q$, 
 the lines 
$EF$
 and 
$AC$
 meet at 
$R$.


Prove that the circumcircles of the triangles 
$PQR$, 
 as 
$E$
 and 
$F$
 vary, have a common point other than 
$P$.

\item[\textbf{G5.}]
Let 
$\triangle ABC$
 be an acute-angled triangle with 
$AB \not= AC$.
 Let 
$H$
 be the orthocenter of triangle 
$ABC$, 
 and let 
$M$
 be the midpoint of the side 
$BC$.
 Let 
$D$
 be a point on the side 
$AB$
 and 
$E$
 a point on the side 
$AC$
 such that 
$AE=AD$
 and the points 
$D$, 
$H$, 
$E$
 are on the same line. Prove that the line 
$HM$
 is perpendicular to the common chord of the circumscribed circles of triangle 
$\triangle ABC$
 and triangle 
$\triangle ADE$.

\item[\textbf{G6.}]
Let 
$ABC$
 be a triangle, and 
$M$
 the midpoint of its side 
$BC$.
 Let 
$\gamma$
 be the incircle of triangle 
$ABC$.
 The median 
$AM$
 of triangle 
$ABC$
 intersects the incircle 
$\gamma$
 at two points 
$K$
 and 
$L$.
 Let the lines passing through 
$K$
 and 
$L$, 
 parallel to 
$BC$, 
 intersect the incircle 
$\gamma$
 again in two points 
$X$
 and 
$Y$.
 Let the lines 
$AX$
 and 
$AY$
 intersect 
$BC$
 again at the points 
$P$
 and 
$Q$.
 Prove that 
$BP = CQ$.

\item[\textbf{G7.}]
In an acute triangle 
$ABC$, 
 let 
$D$, 
$E$, 
$F$
 be the feet of the perpendiculars from the points 
$A$, 
$B$, 
$C$
 to the lines 
$BC$,
$CA$, 
$AB$, 
 respectively, and let 
$P$, 
$Q$, 
$R$
 be the feet of the perpendiculars from the points 
$A$, 
$B$, 
$C$
 to the lines 
$EF$, 
$FD$, 
$DE$, 
 respectively.


Prove that 
$p\left(ABC\right)p\left(PQR\right) \ge \left(p\left(DEF\right)\right)^{2}$, 
 where 
$p\left(T\right)$
 denotes the perimeter of triangle 
$T$.