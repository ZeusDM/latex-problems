
\item[\textbf{G1.}]Let $ABCD$ be a cyclic quadrilateral. Let $P$,  $Q$,  $R$ be the feet of the perpendiculars from $D$ to the lines $BC$,  $CA$,  $AB$,  respectively. Show that $PQ=QR$ if and only if the bisectors of $\angle ABC$ and $\angle ADC$ are concurrent with $AC$.
\item[\textbf{G2.}]Three distinct points $A$,  $B$,  and $C$ are fixed on a line in this order.  Let $\Gamma$ be a circle passing through $A$ and $C$ whose center does not lie on the line $AC$.  Denote by $P$ the intersection of the tangents to $\Gamma$ at $A$ and $C$.  Suppose $\Gamma$ meets the segment $PB$ at $Q$.  Prove that the intersection of the bisector of $\angle AQC$ and the line $AC$ does not depend on the choice of $\Gamma$.
\item[\textbf{G3.}]Let $ABC$ be a triangle and let $P$ be a point in its interior.  Denote by $D$,  $E$,  $F$ the feet of the perpendiculars from $P$ to the lines $BC$,  $CA$,  $AB$,  respectively.  Suppose that \[AP^2 + PD^2 = BP^2 + PE^2 = CP^2 + PF^2.\]Denote by $I_A$,  $I_B$,  $I_C$ the excenters of the triangle $ABC$.  Prove that $P$ is the circumcenter of the triangle $I_AI_BI_C$.
\item[\textbf{G4.}]Let $\Gamma_1$,  $\Gamma_2$,  $\Gamma_3$,  $\Gamma_4$ be distinct circles such that $\Gamma_1$,  $\Gamma_3$ are externally tangent at $P$,  and $\Gamma_2$,  $\Gamma_4$ are externally tangent at the same point $P$. Suppose that $\Gamma_1$ and $\Gamma_2$; $\Gamma_2$ and $\Gamma_3$; $\Gamma_3$ and $\Gamma_4$; $\Gamma_4$ and $\Gamma_1$ meet at $A$,  $B$,  $C$,  $D$,  respectively, and that all these points are different from $P$. Prove that\[
  \frac{AB\cdot BC}{AD\cdot DC}=\frac{PB^2}{PD^2}.
 \]
\item[\textbf{G5.}]Let $ABC$ be an isosceles triangle with $AC=BC$,  whose incentre is $I$. Let $P$ be a point on the circumcircle of the triangle $AIB$ lying inside the triangle $ABC$. The lines through $P$ parallel to $CA$ and $CB$ meet $AB$ at $D$ and $E$,  respectively. The line through $P$ parallel to $AB$ meets $CA$ and $CB$ at $F$ and $G$,  respectively. Prove that the lines $DF$ and $EG$ intersect on the circumcircle of the triangle $ABC$.

\item[\textbf{G6.}]Each pair of opposite sides of a convex hexagon has the following property: the distance between their midpoints is equal to  $\dfrac{\sqrt{3}}{2}$ times the sum of their lengths. Prove that all the angles of the hexagon are equal.

\item[\textbf{G7.}]Let $ABC$ be a triangle with semiperimeter $s$ and inradius $r$. The semicircles with diameters $BC$,  $CA$,  $AB$ are drawn on the outside of the triangle $ABC$. The circle tangent to all of these three semicircles has radius $t$. Prove that\[\frac{s}{2}<t\le\frac{s}{2}+\left(1-\frac{\sqrt{3}}{2}\right)r. \]Alternative formulation. In a triangle $ABC$,  construct circles with diameters $BC$,  $CA$,  and $AB$,  respectively. Construct a circle $w$ externally tangent to these three circles. Let the radius of this circle $w$ be $t$.

Prove: $\frac{s}{2}<t\le\frac{s}{2}+\frac12\left(2-\sqrt3\right)r$,  where $r$ is the inradius and $s$ is the semiperimeter of triangle $ABC$.
