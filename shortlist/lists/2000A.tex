\item[\textbf{A1.}]Let $ a, b, c$ be positive real numbers so that $ abc = 1$. Prove that\[ \left( a - 1 + \frac 1b \right) \left( b - 1 + \frac 1c \right) \left( c - 1 + \frac 1a \right) \leq 1.\]

\item[\textbf{A2.}]Let $ a, b, c$ be positive integers satisfying the conditions $ b > 2a$ and $ c > 2b.$ Show that there exists a real number $ \lambda$ with the property that all the three numbers $ \lambda a, \lambda b, \lambda c$ have their fractional parts lying in the interval $ \left(\frac {1}{3}, \frac {2}{3} \right].$

\item[\textbf{A3.}]Find all pairs of functions $ f : \mathbb \RR \to \mathbb \RR$,  $g : \mathbb \RR \to \mathbb \RR$ such that \[f \left( x + g(y) \right) = xf(y) - y  f(x) + g(x) \quad\text{for all } x, y\in\mathbb{R}.\]

\item[\textbf{A4.}]The function $ F$ is defined on the set of nonnegative integers and takes nonnegative integer values satisfying the following conditions: for every $ n \geq 0,$

\begin{itemize}
\item[(i)] $ F(4n) = F(2n) + F(n),$

\item[(ii)] $ F(4n + 2) = F(4n) + 1,$

\item[(iii)] $ F(2n + 1) = F(2n) + 1.$
\end{itemize}

Prove that for each positive integer $ m,$ the number of integers $ n$ with $ 0 \leq n < 2^m$ and $ F(4n) = F(3n)$ is $ F(2^{m + 1}).$

\item[\textbf{A5.}]Let $ n \geq 2$ be a positive integer and $ \lambda$ a positive real number. Initially there are $ n$ fleas on a horizontal line, not all at the same point. We define a move as choosing two fleas at some points $ A$ and $ B$,  with $ A$ to the left of $ B$,  and letting the flea from $ A$ jump over the flea from $ B$ to the point $ C$ so that $ \frac {BC}{AB} = \lambda$.

Determine all values of $ \lambda$ such that, for any point $ M$ on the line and for any initial position of the $ n$ fleas, there exists a sequence of moves that will take them all to the position right of $ M$.

\item[\textbf{A6.}]A nonempty set $ A$ of real numbers is called a $ B_3$-set if the conditions $ a_1, a_2, a_3, a_4, a_5, a_6 \in A$ and $ a_1 + a_2 + a_3 = a_4 + a_5 + a_6$ imply that the sequences $ (a_1, a_2, a_3)$ and $ (a_4, a_5, a_6)$ are identical up to a permutation. Let $A = \{a_0 = 0 < a_1 < a_2 < \cdots \}$,  $B = \{b_0 = 0 < b_1 < b_2 < \cdots \}$ be infinite sequences of real numbers with $ D(A) = D(B),$ where, for a set $ X$ of real numbers, $ D(X)$ denotes the difference set $ \{|x-y|\mid x, y \in X \}.$ Prove that if $ A$ is a $ B_3$-set, then $ A = B.$

\item[\textbf{A7.}]For a polynomial $ P$ of degree 2000 with distinct real coefficients let $ M(P)$ be the set of all polynomials that can be produced from $ P$ by permutation of its coefficients. A polynomial $ P$ will be called $ n$-independent if $ P(n) = 0$ and we can get from any $ Q \in M(P)$ a polynomial $ Q_1$ such that $ Q_1(n) = 0$ by interchanging at most one pair of coefficients of $ Q.$ Find all integers $ n$ for which $ n$-independent polynomials exist.

