\item[\textbf{G1.}]Let $A_1$ be the center of the square inscribed in acute triangle $ABC$ with two vertices of the square on side $BC$.  Thus one of the two remaining vertices of the square is on side $AB$ and the other is on $AC$. Points $B_1,\ C_1$ are defined in a similar way for inscribed squares with two vertices on sides $AC$ and $AB$,  respectively. Prove that lines $AA_1,\ BB_1,\ CC_1$ are concurrent.

\item[\textbf{G2.}]Consider an acute-angled triangle $ABC$. Let $P$ be the foot of the altitude of triangle $ABC$ issuing from the vertex $A$,  and let $O$ be the circumcenter of triangle $ABC$. Assume that $\angle C \geq \angle B+30^{\circ}$. Prove that $\angle A+\angle COP < 90^{\circ}$.

\item[\textbf{G3.}]Let $ABC$ be a triangle with centroid $G$. Determine, with proof, the position of the point $P$ in the plane of $ABC$ such that $AP{\cdot}AG + BP{\cdot}BG + CP{\cdot}CG$ is a minimum, and express this minimum value in terms of the side lengths of $ABC$.

\item[\textbf{G4.}]Let $M$ be a point in the interior of triangle $ABC$. Let $A'$ lie on $BC$ with $MA'$ perpendicular to $BC$. Define $B'$ on $CA$ and $C'$ on $AB$ similarly.  Define\[
p(M) = \frac{MA' \cdot MB' \cdot MC'}{MA \cdot MB \cdot MC}.
\]

Determine, with proof, the location of $M$ such that $p(M)$ is maximal.  Let $\mu(ABC)$ denote this maximum value.  For which triangles $ABC$ is the value of $\mu(ABC)$ maximal?

\item[\textbf{G5.}]Let $ABC$ be an acute triangle.  Let $DAC,EAB$,  and $FBC$ be isosceles triangles exterior to $ABC$,  with $DA=DC, EA=EB$,  and $FB=FC$,  such that\[
\angle ADC = 2\angle BAC, \quad \angle BEA= 2 \angle ABC, \quad
\angle CFB = 2 \angle ACB.
\]

Let $D'$  be the intersection of lines $DB$ and $EF$,  let $E'$ be the intersection of $EC$ and $DF$,  and let $F'$   be the intersection of $FA$ and $DE$. Find, with proof, the value of the sum\[
\frac{DB}{DD'}+\frac{EC}{EE'}+\frac{FA}{FF'}.
\]

\item[\textbf{G6.}]Let $ABC$ be a triangle and $P$ an exterior point in the plane of the triangle. Suppose the lines $AP$,  $BP$,  $CP$ meet the sides $BC$,  $CA$,  $AB$ (or extensions thereof) in $D$,  $E$,  $F$,  respectively. Suppose further that the areas of triangles $PBD$,  $PCE$,  $PAF$ are all equal. Prove that each of these areas is equal to the area of triangle $ABC$ itself.

\item[\textbf{G7.}]Let $O$ be an interior point of acute triangle $ABC$. Let $A_1$ lie on $BC$ with $OA_1$ perpendicular to $BC$. Define $B_1$ on $CA$ and $C_1$ on $AB$ similarly. Prove that $O$ is the circumcenter of $ABC$ if and only if the perimeter of $A_1B_1C_1$ is not less than any one of the perimeters of $AB_1C_1, BC_1A_1$,  and $CA_1B_1$.

\item[\textbf{G8.}]Let $ABC$ be a triangle with $\angle BAC = 60^{\circ}$. Let $AP$ bisect $\angle BAC$ and let $BQ$ bisect  $\angle ABC$,  with $P$ on $BC$ and $Q$ on $AC$. If $AB + BP = AQ + QB$,  what are the angles of the triangle?