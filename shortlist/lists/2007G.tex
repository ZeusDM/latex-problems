\item[\textbf{G1.}]
In triangle 
$ ABC$
 the bisector of angle 
$ BCA$
 intersects the circumcircle again at 
$ R$, 
 the perpendicular bisector of 
$ BC$
 at 
$ P$, 
 and the perpendicular bisector of 
$ AC$
 at 
$ Q$.
 The midpoint of 
$ BC$
 is 
$ K$
 and the midpoint of 
$ AC$
 is 
$ L$.
 Prove that the triangles 
$ RPK$
 and 
$ RQL$
 have the same area.

\item[\textbf{G2.}]
Denote by 
$ M$
 midpoint of side 
$ BC$
 in an isosceles triangle 
$ \triangle ABC$
 with 
$ AC = AB$.
 Take a point 
$ X$
 on a smaller arc 
$MA$
 of circumcircle of triangle 
$ \triangle ABM$.
 Denote by 
$ T$
 point inside of angle 
$ BMA$
 such that 
$ \angle TMX = 90$
 and 
$ TX = BX$.


Prove that 
$ \angle MTB - \angle CTM$
 does not depend on choice of 
$ X$.

\item[\textbf{G3.}]
The diagonals of a trapezoid 
$ ABCD$
 intersect at point 
$ P$.
 Point 
$ Q$
 lies between the parallel lines 
$ BC$
 and 
$ AD$
 such that 
$ \angle AQD = \angle CQB$, 
 and line 
$ CD$
 separates points 
$ P$
 and 
$ Q$.
 Prove that 
$ \angle BQP = \angle DAQ$.

\item[\textbf{G4.}]
Consider five points 
$ A$, 
$ B$, 
$ C$, 
$ D$
 and 
$ E$
 such that 
$ ABCD$
 is a parallelogram and 
$ BCED$
 is a cyclic quadrilateral. Let 
$ \ell$
 be a line passing through 
$ A$.
 Suppose that 
$ \ell$
 intersects the interior of the segment 
$ DC$
 at 
$ F$
 and intersects line 
$ BC$
 at 
$ G$.
 Suppose also that 
$ EF = EG = EC$.
 Prove that 
$ \ell$
 is the bisector of angle 
$ DAB$.

\item[\textbf{G5.}]
Let 
$ ABC$
 be a fixed triangle, and let 
$ A_1$, 
$ B_1$, 
$ C_1$ be the midpoints of sides 
$ BC$,
$ CA$, 
$ AB$, 
 respectively. Let 
$ P$
 be a variable point on the circumcircle. Let lines 
$ PA_1$, 
$ PB_1$, 
$ PC_1$
 meet the circumcircle again at 
$ A'$, 
$ B'$, 
$ C'$, 
 respectively. Assume that the points 
$ A$, 
$ B$, 
$ C$, 
$ A'$, 
$ B'$, 
$ C'$
 are distinct, and lines 
$ AA'$,
$ BB'$, 
$ CC'$
 form a triangle. Prove that the area of this triangle does not depend on 
$ P$.

\item[\textbf{G6.}]
Determine the smallest positive real number 
$ k$
 with the following property. Let 
$ ABCD$
 be a convex quadrilateral, and let points 
$ A_1$, 
$ B_1$, 
$ C_1$, 
 and 
$ D_1$
 lie on sides 
$ AB$, 
$ BC$, 
$ CD$, 
 and 
$ DA$, 
 respectively. Consider the areas of triangles 
$ AA_1D_1$, 
$ BB_1A_1$, 
$ CC_1B_1$ and 
$ DD_1C_1$;
 let 
$ S$
 be the sum of the two smallest ones, and let 
$ S_1$
 be the area of quadrilateral 
$ A_1B_1C_1D_1$.
 Then we always have 
$ kS_1\ge S$.

\item[\textbf{G7.}]
Given an acute triangle 
$ ABC$
 with 
$ \angle B > \angle C$.
 Point 
$ I$
 is the incenter, and 
$ R$
 the circumradius. Point 
$ D$
 is the foot of the altitude from vertex 
$ A$.
 Point 
$ K$
 lies on line 
$ AD$
 such that 
$ AK = 2R$, 
 and 
$ D$
 separates 
$ A$
 and 
$ K$.
 Lines 
$ DI$
 and 
$ KI$
 meet sides 
$ AC$
 and 
$ BC$
 at 
$ E,F$
 respectively. Let 
$ IE = IF$.


Prove that 
$ \angle B\leq 3\angle C$.

\item[\textbf{G8.}]
Point 
$ P$
 lies on side 
$ AB$
 of a convex quadrilateral 
$ ABCD$.
 Let 
$ \omega$
 be the incircle of triangle 
$ CPD$, 
 and let 
$ I$
 be its incenter. Suppose that 
$ \omega$
 is tangent to the incircles of triangles 
$ APD$
 and 
$ BPC$
 at points 
$ K$
 and 
$ L$, 
 respectively. Let lines 
$ AC$
 and 
$ BD$
 meet at 
$ E$, 
 and let lines 
$ AK$
 and 
$ BL$
 meet at 
$ F$.
 Prove that points 
$ E$, 
$ I$, 
 and 
$ F$
 are collinear.

