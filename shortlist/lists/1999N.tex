\item[\textbf{N1.}]Find all the pairs of positive integers $(x,p)$ such that p is a prime, $x \leq 2p$ and $x^{p-1}$ is a divisor of $ (p-1)^{x}+1$.

\item[\textbf{N2.}]Prove that every positive rational number can be represented in the form $\dfrac{a^{3}+b^{3}}{c^{3}+d^{3}}$ where a,b,c,d are positive integers.

\item[\textbf{N3.}]Prove that there exists two strictly increasing sequences $(a_{n})$ and $(b_{n})$ such that $a_{n}(a_{n}+1)$ divides $b^{2}_{n}+1$ for every natural n.

\item[\textbf{N4.}]Denote by S the set of all primes such the decimal representation of $\frac{1}{p}$ has the fundamental period divisible by 3. For every $p \in S$ such that $\frac{1}{p}$ has the fundamental period $3r$ one may write\[\frac{1}{p}=0,a_{1}a_{2}\ldots a_{3r}a_{1}a_{2} \ldots a_{3r} \ldots , \]

where $r=r(p)$; for every $p \in S$ and every integer $k \geq 1$ define $f(k,p)$ by \[ f(k,p)= a_{k}+a_{k+r(p)}+a_{k+2.r(p)}\]

\begin{itemize}
\item[(a)] Prove that $S$ is infinite.

\item[(b)] Find the highest value of $f(k,p)$ for $k \geq 1$ and $p \in S$
\end{itemize}

\item[\textbf{N5.}]Let $n,k$ be positive integers such that n is not divisible by 3 and $k \geq n$. Prove that there exists a positive integer $m$ which is divisible by $n$ and the sum of its digits in decimal representation is $k$.

\item[\textbf{N6.}]Prove that for every real number $M$ there exists an infinite arithmetic progression such that:

\begin{itemize}
\item each term is a positive integer and the common difference is not divisible by 10.
\item the sum of the digits of each term (in decimal representation) exceeds $M$.
\end{itemize}
