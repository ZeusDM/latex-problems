\item[\textbf{C1.}]
Let 
$n > 0$
 be an integer. We are given a balance and 
$n$
 weights of weight 
$2^0, 2^1, \cdots, 2^{n-1}$.
 We are to place each of the 
$n$
 weights on the balance, one after another, in such a way that the right pan is never heavier than the left pan. At each step we choose one of the weights that has not yet been placed on the balance, and place it on either the left pan or the right pan, until all of the weights have been placed.


Determine the number of ways in which this can be done.

\item[\textbf{C2.}]
Suppose that 
$1000$
 students are standing in a circle. Prove that there exists an integer 
$k$
 with 
$100 \leq k \leq 300$
 such that in this circle there exists a contiguous group of 
$2k$
 students, for which the first half contains the same number of girls as the second half.

\item[\textbf{C3.}]
Let 
$\mathcal{S}$
 be a finite set of at least two points in the plane. Assume that no three points of 
$\mathcal S$
 are collinear. A 
windmill
 is a process that starts with a line 
$\ell$
 going through a single point 
$P \in \mathcal S$.
 The line rotates clockwise about the 
pivot
$P$
 until the first time that the line meets some other point belonging to 
$\mathcal S$.
 This point, 
$Q$, 
 takes over as the new pivot, and the line now rotates clockwise about 
$Q$, 
 until it next meets a point of 
$\mathcal S$.
 This process continues indefinitely.


Show that we can choose a point 
$P$
 in 
$\mathcal S$
 and a line 
$\ell$
 going through 
$P$
 such that the resulting windmill uses each point of 
$\mathcal S$
 as a pivot infinitely many times.

\item[\textbf{C4.}]
Determine the greatest positive integer 
$k$
 that satisfies the following property: The set of positive integers can be partitioned into 
$k$
 subsets 
$A_1, A_2, \ldots, A_k$
 such that for all integers 
$n \geq 15$
 and all 
$i \in \{1, 2, \ldots, k\}$
 there exist two distinct elements of 
$A_i$
 whose sum is 
$n.$

\item[\textbf{C5.}]
Let 
$m$
 be a positive integer, and consider a 
$m\times m$
 checkerboard consisting of unit squares. At the centre of some of these unit squares there is an ant. At time 
$0$, 
 each ant starts moving with speed 
$1$
 parallel to some edge of the checkerboard. When two ants moving in the opposite directions meet, they both turn 
$90^{\circ}$
 clockwise and continue moving with speed 
$1$.
 When more than 
$2$
 ants meet, or when two ants moving in perpendicular directions meet, the ants continue moving in the same direction as before they met. When an ant reaches one of the edges of the checkerboard, it falls off and will not re-appear.


Considering all possible starting positions, determine the latest possible moment at which the last ant falls off the checkerboard, or prove that such a moment does not necessarily exist.

\item[\textbf{C6.}]
Let 
$n$
 be a positive integer, and let 
$W = \ldots x_{-1}x_0x_1x_2 \ldots$
 be an infinite periodic word, consisting of just letters 
$a$
 and/or 
$b$.
 Suppose that the minimal period 
$N$
 of 
$W$
 is greater than 
$2^n$.


A finite nonempty word 
$U$
 is said to 
appear
 in 
$W$
 if there exist indices 
$k \leq \ell$
 such that 
$U=x_k x_{k+1} \ldots x_{\ell}$.
 A finite word 
$U$
 is called 
ubiquitous
 if the four words 
$Ua$, 
$Ub$, 
$aU$, 
 and 
$bU$
 all appear in 
$W$.
 Prove that there are at least 
$n$
 ubiquitous finite nonempty words.

\item[\textbf{C7.}]
On a square table of 
$2011$
 by 
$2011$
 cells we place a finite number of napkins that each cover a square of 
$52$
 by 
$52$
 cells. In each cell we write the number of napkins covering it, and we record the maximal number 
$k$
 of cells that all contain the same nonzero number. Considering all possible napkin configurations, what is the largest value of 
$k$?