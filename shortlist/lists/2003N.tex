\item[\textbf{N1.}]Let $m$ be a fixed integer greater than $1$. The sequence $x_0$,  $x_1$,  $x_2$,  $\ldots$ is defined as follows:\[x_i = \begin{cases}2^i&\text{if }0\leq i \leq m - 1;\\\sum_{j=1}^mx_{i-j}&\text{if }i\geq m.\end{cases}\]Find the greatest $k$ for which the sequence contains $k$ consecutive terms divisible by $m$.

\item[\textbf{N2.}]Each positive integer $a$ undergoes the following procedure in order to obtain the number $d = d\left(a\right)$:

\begin{itemize}

\item[(i)] move the last digit of $a$ to the first position to obtain the numb er $b$;

\item[(ii)] square $b$ to obtain the number $c$;

\item[(iii)] move the first digit of $c$ to the end to obtain the number $d$.

\end{itemize}


(All the numbers in the problem are considered to be represented in base $10$.) 

For example, for $a=2003$,  we get $b=3200$,  $c=10240000$,  and $d = 02400001 = 2400001 = d(2003)$.)

Find all numbers $a$ for which $d\left( a\right) =a^2$.

\item[\textbf{N3.}]Determine all pairs of positive integers $(a,b)$ such that  \[ \dfrac{a^2}{2ab^2-b^3+1}  \]  is a positive integer.

\item[\textbf{N4.}]Let $ b$ be an integer greater than $ 5$. For each positive integer $ n$,  consider the number 
\[ x_n = \underbrace{11\cdots1}_{n - 1}\underbrace{22\cdots2}_{n}5, \]
written in base $b$.

Prove that the following condition holds if and only if $ b = 10$: there exists a positive integer $ M$ such that for any integer $ n$ greater than $ M$,   the number $ x_n$ is a perfect square.
\item[\textbf{N5.}]An integer $n$ is said to be good if $|n|$ is not the square of an integer. Determine all integers $m$ with the following property: $m$ can be represented, in infinitely many ways, as a sum of three distinct good integers whose product is the square of an odd integer.
\item[\textbf{N6.}]Let $p$ be a prime number. Prove that there exists a prime number $q$ such that for every integer $n$,  the number $n^p-p$ is not divisible by $q$.
\item[\textbf{N7.}]The sequence $a_0$,  $a_1$,  $a_2,$ $\ldots$ is defined as follows: \[a_0=2, \qquad a_{k+1}=2a_k^2-1 \quad\text{for }k \geq 0.\]Prove that if an odd prime $p$ divides $a_n$,  then $2^{n+3}$ divides $p^2-1$.

\item[\textbf{N8.}]Let $p$ be a prime number and let $A$ be a set of positive integers that satisfies the following conditions:

(i) the set of prime divisors of the elements in $A$ consists of $p-1$ elements;

(ii) for any nonempty subset of $A$,  the product of its elements is not a perfect $p$-th power.

What is the largest possible number of elements in $A$?