\item[\textbf{C1.}]
Consider 
$2009$
 cards, each having one gold side and one black side, lying on parallel on a long table. Initially all cards show their gold sides. Two player, standing by the same long side of the table, play a game with alternating moves. Each move consists of choosing a block of 
$50$
 consecutive cards, the leftmost of which is showing gold, and turning them all over, so those which showed gold now show black and vice versa. The last player who can make a legal move wins.

(a) Does the game necessarily end?

(b) Does there exist a winning strategy for the starting player?

\item[\textbf{C2.}]
For any integer 
$n\geq 2$, 
 let 
$N(n)$
 be the maxima number of triples 
$(a_i, b_i, c_i)$, 
$i=1, \ldots, N(n)$, 
 consisting of nonnegative integers 
$a_i$, 
$b_i$
 and 
$c_i$
 such  that the following two conditions are satisfied:

\begin{itemize}
\item $a_i+b_i+c_i=n$
 for all 
$i=1, \ldots, N(n)$, 

\item If 
$i\neq j$
 then 
$a_i\neq a_j$, 
$b_i\neq b_j$
 and 
$c_i\neq c_j$
\end{itemize}

Determine 
$N(n)$
 for all 
$n\geq 2$.

\item[\textbf{C3.}]
Let 
$n$
 be a positive integer.  Given a sequence 
$\varepsilon_1$, 
$\dots$, 
$\varepsilon_{n - 1}$
 with 
$\varepsilon_i = 0$
 or 
$\varepsilon_i = 1$
 for each 
$i = 1$, 
$\dots$, 
$n - 1$, 
 the sequences 
$a_0$, 
$\dots$, 
$a_n$
 and 
$b_0$, 
$\dots$, 
$b_n$
 are constructed by the following rules: 
\[a_0 = b_0 = 1, \quad a_1 = b_1 = 7,\]
 
\[\begin{array}{lll}
	a_{i+1} = 
	\begin{cases}
		2a_{i-1} + 3a_i, \\
		3a_{i-1} + a_i, 
	\end{cases} & 
        \begin{array}{l} 
                \text{if } \varepsilon_i = 0, \\  
                \text{if } \varepsilon_i = 1, \end{array}
         & \text{for each } i = 1, \dots, n - 1, \\[15pt]
        b_{i+1}= 
        \begin{cases}
		2b_{i-1} + 3b_i, \\
		3b_{i-1} + b_i, 
	\end{cases} & 
        \begin{array}{l} 
                \text{if } \varepsilon_{n-i} = 0, \\  
                \text{if } \varepsilon_{n-i} = 1, \end{array}
         & \text{for each } i = 1, \dots, n - 1.
	\end{array}\]
  Prove that
$a_n = b_n$.

\item[\textbf{C4.}]
For an integer 
$m\geq 1$, 
 we consider partitions of a 
$2^m\times 2^m$
 chessboard into rectangles consisting of cells of chessboard, in which each of the 
$2^m$
 cells along one diagonal forms a separate rectangle of side length 
$1$.
 Determine the smallest possible sum of rectangle perimeters in such a partition.

\item[\textbf{C5.}]
Five identical empty buckets of 
$2$-liter capacity stand at the vertices of a regular pentagon. Cinderella and her wicked Stepmother go through a sequence of rounds: At the beginning of every round, the Stepmother takes one liter of water from the nearby river and distributes it arbitrarily over the five buckets. Then Cinderella chooses a pair of neighbouring buckets, empties them to the river and puts them back. Then the next round begins. The Stepmother goal's is to make one of these buckets overflow. Cinderella's goal is to prevent this. Can the wicked Stepmother enforce a bucket overflow?

\item[\textbf{C6.}]
On a 
$999\times 999$
 board a 
limp rook
 can move in the following way: From any square it can move to any of its adjacent squares, i.e. a square having a common side with it, and every move must be a turn, i.e. the directions of any two consecutive moves must be perpendicular. A 
non-intersecting route
 of the limp rook consists of a sequence of pairwise different squares that the limp rook can visit in that order by an admissible sequence of moves. Such a non-intersecting route is called 
cyclic
, if the limp rook can, after reaching the last square of the route, move directly to the first square of the route and start over.


How many squares does the longest possible cyclic, non-intersecting route of a limp rook visit?

\item[\textbf{C7.}]
Let 
$ a_1, a_2, \ldots , a_n$
 be distinct positive integers and let 
$ M$
 be a set of 
$ n - 1$
 positive integers not containing 
$ s = a_1 + a_2 + \ldots + a_n.$
 A grasshopper is to jump along the real axis, starting at the point 
$ 0$
 and making 
$ n$
 jumps to the right with lengths 
$ a_1, a_2, \ldots , a_n$
 in some order. Prove that the order can be chosen in such a way that the grasshopper never lands on any point in 
$ M.$

\item[\textbf{C8.}]
For any integer 
$n\geq 2$, 
 we compute the integer 
$h(n)$
 by applying the following procedure to its decimal representation. Let 
$r$
 be the rightmost digit of 
$n$.
If 
$r=0$, 
 then the decimal representation of 
$h(n)$
 results from the decimal representation of 
$n$
 by removing this rightmost digit 
$0$.
If 
$1\leq r \leq 9$
 we split the decimal representation of 
$n$
 into a maximal right part 
$R$
 that solely consists of digits not less than 
$r$
 and into a left part 
$L$
 that either is empty or ends with a digit strictly smaller than 
$r$.
 Then the decimal representation of 
$h(n)$
 consists of the decimal representation of 
$L$, 
 followed by two copies of the decimal representation of 
$R-1$.
 For instance, for the number 
$17,151,345,543$, 
 we will have 
$L=17,151$, 
$R=345,543$
 and 
$h(n)=17,151,345,542,345,542$.

Prove that, starting with an arbitrary integer 
$n\geq 2$, 
 iterated application of 
$h$
 produces the integer 
$1$
 after finitely many steps.