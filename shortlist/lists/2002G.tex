\item[\textbf{G1.}]Let $B$ be a point on a circle $S_1$,  and let $A$ be a point distinct from $B$ on the tangent at $B$ to $S_1$. Let $C$ be a point not on $S_1$ such that the line segment $AC$ meets $S_1$ at two distinct points. Let $S_2$ be the circle touching $AC$ at $C$ and touching $S_1$ at a point $D$ on the opposite side of $AC$ from $B$.  Prove that the circumcentre of triangle $BCD$ lies on the circumcircle of triangle $ABC$.

\item[\textbf{G2.}]Let $ABC$ be a triangle for which there exists an interior point $F$ such that $\angle AFB=\angle BFC=\angle CFA$. Let the lines $BF$ and $CF$ meet the sides $AC$ and $AB$ at $D$ and $E$ respectively. Prove that \[ AB+AC\geq4DE. \]

\item[\textbf{G3.}]The circle $S$ has centre $O$,  and $BC$ is a diameter of $S$. Let $A$ be a point of $S$ such that $\angle AOB<120{{}^\circ}$.  Let $D$ be the midpoint of the arc $AB$ which does not contain $C$. The line through $O$ parallel to $DA$ meets the line $AC$ at $I$. The perpendicular bisector of $OA$ meets $S$ at $E$ and at $F$. Prove that $I$ is the incentre of the triangle $CEF.$

\item[\textbf{G4.}]Circles $S_1$ and $S_2$ intersect at points $P$ and $Q$. Distinct points $A_1$ and $B_1$ (not at $P$ or $Q$) are selected on $S_1$. The lines $A_1P$ and $B_1P$ meet $S_2$ again at $A_2$ and $B_2$ respectively, and the lines $A_1B_1$ and $A_2B_2$ meet at $C$.  Prove that, as $A_1$ and $B_1$ vary, the circumcentres of triangles $A_1A_2C$ all lie on one fixed circle.

\item[\textbf{G5.}]For any set $S$ of five points in the plane, no three of which are collinear, let $M(S)$ and $m(S)$ denote the greatest and smallest areas, respectively, of triangles determined by three points from $S$.  What is the minimum possible value of $M(S)/m(S)$?

\item[\textbf{G6.}]Let $n\geq3$ be a positive integer. Let $C_1$, $C_2$, $C_3$, $\ldots$, $C_n$ be unit circles in the plane, with centres $O_1$, $O_2$, $O_3$, $\ldots$, $O_n$ respectively. If no line meets more than two of the circles, prove that \[ \sum\limits^{}_{1\leq i<j\leq n}{\frac{1}{O_iO_j}}\leq{\frac{(n-1)\pi}{4}}.  \]

\item[\textbf{G7.}]The incircle $ \Omega$ of the acute-angled triangle $ ABC$ is tangent to its side $ BC$ at a point $ K$. Let $ AD$ be an altitude of triangle $ ABC$,  and let $ M$ be the midpoint of the segment $ AD$. If $ N$ is the common point of the circle $ \Omega$ and the line $ KM$ (distinct from $ K$), then prove that the incircle $ \Omega$ and the circumcircle of triangle $ BCN$ are tangent to each other at the point $ N$.

\item[\textbf{G8.}]Let two circles $S_{1}$ and $S_{2}$ meet at the points $A$ and $B$. A line through $A$ meets $S_{1}$ again at $C$ and $S_{2}$ again at $D$. Let $M$,  $N$,  $K$ be three points on the line segments $CD$,  $BC$,  $BD$ respectively, with $MN$ parallel to $BD$ and $MK$ parallel to $BC$. Let $E$ and $F$ be points on those arcs $BC$ of $S_{1}$ and $BD$ of $S_{2}$ respectively that do not contain $A$. Given that $EN$ is perpendicular to $BC$ and $FK$ is perpendicular to $BD$ prove that $\angle EMF=90^{\circ}$.