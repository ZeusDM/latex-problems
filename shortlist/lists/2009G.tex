\item[\textbf{G1.}]
Let 
$ ABC$
 be a triangle with 
$ AB = AC$
 . The angle bisectors of 
$ \angle C AB$
 and 
$ \angle AB C$
 meet the sides 
$ B C$
 and 
$ C A$
 at 
$ D$
 and 
$ E$
 , respectively. Let 
$ K$
 be the incentre of triangle 
$ ADC$.
 Suppose that 
$ \angle B E K = 45^\circ$
 . Find all possible values of 
$ \angle C AB$.

\item[\textbf{G2.}]
Let 
$ ABC$
 be a triangle with circumcentre 
$ O$.
 The points 
$ P$
 and 
$ Q$
 are interior points of the sides 
$ CA$
 and 
$ AB$
 respectively. Let 
$ K,L$
 and 
$ M$
 be the midpoints of the segments 
$ BP,CQ$
 and 
$ PQ$.
 respectively, and let 
$ \Gamma$
 be the circle passing through 
$ K,L$
 and 
$ M$.
 Suppose that the line 
$ PQ$
 is tangent to the circle 
$ \Gamma$.
 Prove that 
$ OP = OQ.$

\item[\textbf{G3.}]
Let 
$ABC$
 be a triangle. The incircle of 
$ABC$
 touches the sides 
$AB$
 and 
$AC$
 at the points 
$Z$
 and 
$Y$, 
 respectively. Let 
$G$
 be the point where the lines 
$BY$
 and 
$CZ$
 meet, and let 
$R$
 and 
$S$
 be points such that the two quadrilaterals 
$BCYR$
 and 
$BCSZ$
 are parallelogram.


Prove that 
$GR=GS$.

\item[\textbf{G4.}]
Given a cyclic quadrilateral 
$ABCD$, 
 let the diagonals 
$AC$
 and 
$BD$
 meet at 
$E$
 and the lines 
$AD$
 and 
$BC$
 meet at 
$F$.
 The midpoints of 
$AB$
 and 
$CD$
 are 
$G$
 and 
$H$, 
 respectively. Show that 
$EF$
 is tangent at 
$E$
 to the circle through the points 
$E$, 
$G$
 and 
$H$.

\item[\textbf{G5.}]
Let 
$P$
 be a polygon that is convex and symmetric to some point 
$O$.
 Prove that for some parallelogram 
$R$
 satisfying 
$P\subset R$
 we have 
\[\frac{|R|}{|P|}\leq \sqrt 2\]


where 
$|R|$
 and 
$|P|$
 denote the area of the sets 
$R$
 and 
$P$, 
 respectively.

\item[\textbf{G6.}]
Let the sides 
$AD$
 and 
$BC$
 of the quadrilateral 
$ABCD$
 (such that 
$AB$
 is not parallel to 
$CD$)
 intersect at point 
$P$.
 Points 
$O_1$
 and 
$O_2$
 are circumcenters and points 
$H_1$
 and 
$H_2$
 are orthocenters of triangles 
$ABP$
 and 
$CDP$, 
 respectively. Denote the midpoints of segments 
$O_1H_1$
 and 
$O_2H_2$
 by 
$E_1$
 and 
$E_2$, 
 respectively. Prove that the perpendicular from 
$E_1$
 on 
$CD$, 
 the perpendicular from 
$E_2$
 on 
$AB$
 and the lines 
$H_1H_2$
 are concurrent.

\item[\textbf{G7.}]
Let 
$ABC$
 be a triangle with incenter 
$I$
 and let 
$X$, 
$Y$
 and 
$Z$
 be the incenters of the triangles 
$BIC$, 
$CIA$
 and 
$AIB$, 
 respectively. Let the triangle 
$XYZ$
 be equilateral. Prove that 
$ABC$
 is equilateral too.

\item[\textbf{G8.}]
Let 
$ABCD$
 be a circumscribed quadrilateral. Let 
$g$
 be a line through 
$A$
 which meets the segment 
$BC$
 in 
$M$
 and the line 
$CD$
 in 
$N$.
 Denote by 
$I_1$, 
$I_2$
 and 
$I_3$
 the incenters of 
$\triangle ABM$, 
$\triangle MNC$
 and 
$\triangle NDA$, 
 respectively. Prove that the orthocenter of 
$\triangle I_1I_2I_3$
 lies on 
$g$.