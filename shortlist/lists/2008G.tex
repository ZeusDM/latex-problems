\item[\textbf{G1.}]
Let 
$ H$
 be the orthocenter of an acute-angled triangle 
$ ABC$.
 The circle 
$ \Gamma_{A}$
 centered at the midpoint of 
$ BC$
 and passing through 
$ H$
 intersects the sideline 
$ BC$
 at points  
$ A_{1}$
 and 
$ A_{2}$.
 Similarly, define the points 
$ B_{1}$, 
$ B_{2}$, 
$ C_{1}$
 and 
$ C_{2}$.


Prove that the six points 
$ A_{1}$, 
$ A_{2}$, 
$ B_{1}$, 
$ B_{2}$, 
$ C_{1}$
 and 
$ C_{2}$
 are concyclic.

\item[\textbf{G2.}]
Given trapezoid 
$ ABCD$
 with parallel sides 
$ AB$
 and 
$ CD$, 
 assume that there exist points 
$ E$
 on line 
$ BC$
 outside segment 
$ BC$, 
 and 
$ F$
 inside segment 
$ AD$
 such that 
$ \angle DAE = \angle CBF$.
 Denote by 
$ I$
 the point of intersection of 
$ CD$
 and 
$ EF$, 
 and by 
$ J$
 the point of intersection of 
$ AB$
 and 
$ EF$.
 Let 
$ K$
 be the midpoint of segment 
$ EF$, 
 assume it does not lie on line 
$ AB$.
 Prove that 
$ I$
 belongs to the circumcircle of 
$ ABK$
 if and only if 
$ K$
 belongs to the circumcircle of 
$ CDJ$.

\item[\textbf{G3.}]
Let 
$ ABCD$
 be a convex quadrilateral and let 
$ P$
 and 
$ Q$
 be points in 
$ ABCD$
 such that 
$ PQDA$
 and 
$ QPBC$
 are cyclic quadrilaterals. Suppose that there exists a point 
$ E$
 on the line segment 
$ PQ$
 such that 
$ \angle PAE = \angle QDE$
 and 
$ \angle PBE = \angle QCE$.
 Show that the quadrilateral 
$ ABCD$
 is cyclic.

\item[\textbf{G4.}]
In an acute triangle 
$ ABC$
 segments 
$ BE$
 and 
$ CF$
 are altitudes. Two circles passing through the point 
$ A$
 anf 
$ F$
 and tangent to the line 
$ BC$
 at the points 
$ P$
 and 
$ Q$
 so that 
$ B$
 lies between 
$ C$
 and 
$ Q$.
 Prove that lines 
$ PE$
 and 
$ QF$
 intersect on the circumcircle of triangle 
$ AEF$.

\item[\textbf{G5.}]
Let 
$ k$
 and 
$ n$
 be integers with 
$ 0\le k\le n - 2$.
 Consider a set 
$ L$
 of 
$ n$
 lines in the plane such that no two of them are parallel and no three have a common point. Denote by 
$ I$
 the set of intersections of lines in 
$ L$.
 Let 
$ O$
 be a point in the plane not lying on any line of 
$ L$.
 A point 
$ X\in I$
 is colored red if the open line segment 
$ OX$
 intersects at most 
$ k$
 lines in 
$ L$.
 Prove that 
$ I$
 contains at least 
$ \dfrac{1}{2}(k + 1)(k + 2)$
 red points.

\item[\textbf{G6.}]
There is given a convex quadrilateral 
$ ABCD$.
 Prove that there exists a point 
$ P$
 inside the quadrilateral such that
$\quad \angle PAB + \angle PDC = \angle PBC + \angle PAD = \angle PCD + \angle PBA = \angle PDA + \angle PCB = 90^{\circ}$
if and only if the diagonals 
$ AC$
 and 
$ BD$
 are perpendicular.

\item[\textbf{G7.}]
Let 
$ ABCD$
 be a convex quadrilateral with 
$ BA\neq BC$.
 Denote the incircles of triangles 
$ ABC$
 and 
$ ADC$
 by 
$ \omega_{1}$
 and 
$ \omega_{2}$
 respectively. Suppose that there exists a circle 
$ \omega$
 tangent to ray 
$ BA$
 beyond 
$ A$
 and to the ray 
$ BC$
 beyond 
$ C$, 
 which is also tangent to the lines 
$ AD$
 and 
$ CD$.
 Prove that the common external tangents to 
$ \omega_{1}$
 and 
$\omega_{2}$
 intersect on 
$ \omega$.