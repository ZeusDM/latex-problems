\item[\textbf{C1.}]Let $A = (a_1, a_2, \ldots, a_{2001})$ be a sequence of positive integers. Let $m$ be the number of 3-element subsequences $(a_i,a_j,a_k)$ with $1 \leq i < j < k \leq 2001$,  such that $a_j = a_i + 1$ and $a_k = a_j + 1$.  Considering all such sequences $A$,  find the greatest value of $m$.

\item[\textbf{C2.}]Let $n$ be an odd integer greater than 1 and let $c_1, c_2, \ldots, c_n$ be integers. For each permutation $a = (a_1, a_2, \ldots, a_n)$ of $\{1,2,\ldots,n\}$,  define $S(a) = \sum_{i=1}^n c_i a_i$. Prove that there exist permutations $a \neq b$ of $\{1,2,\ldots,n\}$ such that $n!$ is a divisor of $S(a)-S(b)$.

\item[\textbf{C3.}]Define a $ k$-clique to be a set of $ k$ people such that every pair of them are acquainted with each other. At a certain party, every pair of 3-cliques has at least one person in common, and there are no 5-cliques. Prove that there are two or fewer people at the party whose departure leaves no 3-clique remaining.

\item[\textbf{C4.}]A set of three nonnegative integers $\{x,y,z\}$ with $x < y < z$ is called historic if $\{z-y,y-x\} = \{1776,2001\}$.  Show that the set of all nonnegative integers can be written as the union of pairwise disjoint historic sets.

\item[\textbf{C5.}]Find all finite sequences $(x_0, x_1, \ldots,x_n)$ such that for every $j$,  $0 \leq j \leq n$,  $x_j$ equals the number of times $j$ appears in the sequence.

\item[\textbf{C6.}]For a positive integer $n$ define a sequence of zeros and ones to be balanced if it contains $n$ zeros and $n$ ones. Two balanced sequences $a$ and $b$ are neighbors if you can move one of the $2n$ symbols of $a$ to another position to form $b$. For instance, when $n = 4$,  the balanced sequences $01101001$ and $00110101$ are neighbors because the third (or fourth) zero in the first sequence can be moved to the first or second position to form the second sequence. Prove that there is a set $S$ of at most $\frac{1}{n+1} \binom{2n}{n}$ balanced sequences such that every balanced sequence is equal to or is a neighbor of at least one sequence in $S$.

\item[\textbf{C7.}]A pile of $n$ pebbles is placed in a vertical column. This configuration is modified according to the following rules. A pebble can be moved if it is at the top of a column which contains at least two more pebbles than the column immediately to its right. (If there are no pebbles to the right, think of this as a column with 0 pebbles.) At each stage, choose a pebble from among those that can be moved (if there are any) and place it at the top of the column to its right. If no pebbles can be moved, the configuration is called a final configuration. For each $n$,  show that, no matter what choices are made at each stage, the final configuration obtained is unique. Describe that configuration in terms of $n$.IMO ShortList 2001, combinatorics problem 7, alternative

\item[\textbf{C8.}]Twenty-one girls and twenty-one boys took part in a mathematical competition. It turned out that each contestant solved at most six problems, and for each pair of a girl and a boy, there was at least one problem that was solved by both the girl and the boy. Show that there is a problem that was solved by at least three girls and at least three boys.