\item[\textbf{C1.}]A rectangular array of numbers is given. In each row and each column, the sum of all numbers is an integer. Prove that each nonintegral number $x$ in the array can be changed into either $\lceil x\rceil $ or $\lfloor x\rfloor $ so that the row-sums and column-sums remain unchanged. (Note that $\lceil x\rceil $ is the least integer greater than or equal to $x$,  while $\lfloor x\rfloor $ is the greatest integer less than or equal to $x$.)

\item[\textbf{C2.}]Let $n$ be an integer greater than 2. A positive integer is said to be attainable if it is 1 or can be obtained from 1 by a sequence of operations with the following properties:

\begin{itemize}
\item  The first operation is either addition or multiplication.
\item Thereafter, additions and multiplications are used alternately.
\item  In each addition, one can choose independently whether to add 2 or $n$
\item  In each multiplication, one can choose independently whether to multiply by 2 or by $n$.
\end{itemize}
A positive integer which cannot be so obtained is said to be unattainable.a.)  Prove that if $n\geq 9$,  there are infinitely many unattainable positive integers.b.)  Prove that if $n=3$,  all positive integers except 7 are attainable.

\item[\textbf{C3.}]Cards numbered 1 to 9 are arranged at random in a row. In a move, one may choose any block of consecutive cards whose numbers are in ascending or descending order, and switch the block around. For example, $9\ 1$ $\underline{6\ 5\ 3}$ $2\ 7\ 4\ 8$ may be changed to $9\ 1$ $\underline{3\ 5\ 6}$ $2\ 7\ 4\ 8$. Prove that in at most 12 moves, one can arrange the 9 cards so that their numbers are in ascending or descending order.

\item[\textbf{C4.}]Let $U=\{1,2,\ldots ,n\}$,  where $n\geq 3$. A subset $S$ of $U$ is said to be split by an arrangement of the elements of $U$ if an element not in $S$ occurs in the arrangement somewhere between two elements of $S$. For example, 13542 splits $\{1,2,3\}$ but not $\{3,4,5\}$. Prove that for any $n-2$ subsets of $U$,  each containing at least 2 and at most $n-1$ elements, there is an arrangement of the elements of $U$ which splits all of them.

\item[\textbf{C5.}]In a contest, there are $m$ candidates and $n$ judges, where $n\geq 3$ is an odd integer. Each candidate is evaluated by each judge as either pass or fail. Suppose that each pair of judges agrees on at most $k$ candidates. Prove that \[{\frac{k}{m}} \geq {\frac{n-1}{2n}}. \]

\item[\textbf{C6.}]Ten points are marked in the plane so that no three of them lie on a line. Each pair of points is connected with a segment. Each of these segments is painted with one of $k$ colors, in such a way that for any $k$ of the ten points, there are $k$ segments each joining two of them and no two being painted with the same color. Determine all integers $k$,  $1\leq k\leq 10$,  for which this is possible.

\item[\textbf{C7.}]A solitaire game is played on an $m\times n$ rectangular board, using $mn$ markers which are white on one side and black on the other. Initially, each  square of the board contains a marker with its white side up, except for one corner square, which contains a marker with its black side up. In each move, one may take away one marker with its black side up, but must then turn over  all markers which are in squares having an edge in common with the square of the removed marker. Determine all pairs $(m,n)$ of positive integers such that all markers can be removed from the board.