\item[\textbf{C1.}]
We have 
$ n \geq 2$
 lamps 
$ L_{1}, . . . ,L_{n}$
 in a row, each of them being either on or off. Every second we simultaneously modify the state of each lamp as follows: if the lamp 
$ L_{i}$
 and its neighbours (only one neighbour for 
$ i = 1$
 or 
$ i = n$, 
 two neighbours for other 
$ i$)
 are in the same state, then 
$ L_{i}$
 is switched off; - otherwise, 
$ L_{i}$
 is switched on.


Initially all the lamps are off except the leftmost one which is on.
$ (a)$
 Prove that there are infinitely many integers 
$ n$
 for which all the lamps will eventually be off.
$ (b)$
 Prove that there are infinitely many integers 
$ n$
 for which the lamps will never be all off.

\item[\textbf{C2.}]
Let 
$P$
 be a regular 
$2006$-gon. A diagonal is called 
good
 if its endpoints divide the boundary of 
$P$
 into two parts, each composed of an odd number of sides of 
$P$.
 The sides of 
$P$
 are also called 
good.


Suppose 
$P$
 has been dissected into triangles by 
$2003$
 diagonals, no two of which have a common point in the interior of 
$P$.
 Find the maximum number of isosceles triangles having two good sides that could appear in such a configuration.

\item[\textbf{C3.}]
Let 
$ S$
 be a finite set of points in the plane such that no three of them are on a line. For each convex polygon 
$ P$
 whose vertices are in 
$ S$, 
 let 
$ a(P)$
 be the number of vertices of 
$ P$, 
 and let 
$ b(P)$
 be the number of points of 
$ S$
 which are outside 
$ P$.
 A line segment, a point, and the empty set are considered as convex polygons of 
$ 2$, 
$ 1$, 
 and 
$ 0$
 vertices respectively. Prove that for every real number 
$ x$
\[\sum_{P}{x^{a(P)}(1 - x)^{b(P)}} = 1,\]
 where the sum is taken over all convex polygons with vertices in 
$ S$.

\item[\textbf{C3'.}]

Let 
$ M$
 be a finite point set in the plane and no three points are collinear. A subset 
$ A$
 of 
$ M$
 will be called round if its elements is the set of vertices of a convex 
$ A -$
g
on 
$ V(A).$
 For each round subset let 
$ r(A)$
 be the number of points from 
$ M$
 which are exterior from the convex 
$ A -$
g
on 
$ V(A).$
 Subsets with 
$ 0,1$
 and 2 elements are always round, its corresponding polygons are the empty set, a point or a segment, respectively (for which all other points that are not vertices of the polygon are exterior). For each round subset 
$ A$
 of 
$ M$
 construct the polynomial
\[ P_A(x) = x^{|A|}(1 - x)^{r(A)}.
\]

Show that the sum of polynomials for all round subsets is exactly the polynomial 
$ P(x) = 1.$

\item[\textbf{C4.}]
A cake has the form of an 
$ n \times n$
 square composed of 
$ n^{2}$
 unit squares. Strawberries lie on some of the unit squares so that each row or column contains exactly one strawberry; call this arrangement 
$\mathcal{A}$.

Let 
$\mathcal{B}$
 be another such arrangement. Suppose that every grid rectangle with one vertex at the top left corner of the cake contains no fewer strawberries of arrangement 
$\mathcal{B}$
 than of arrangement 
$\mathcal{A}$.
 Prove that arrangement 
$\mathcal{B}$
 can be obtained from 
$ \mathcal{A}$
 by performing a number of switches, defined as follows:


A switch consists in selecting a grid rectangle with only two strawberries, situated at its top right corner and bottom left corner, and moving these two strawberries to the other two corners of that rectangle.

\item[\textbf{C5.}]
An 
$ (n, k) -$
 tournament is a contest with 
$ n$
 players held in 
$ k$
 rounds such that:
$ (i)$
 Each player plays in each round, and every two players meet at most once.
$ (ii)$
 If player 
$ A$
 meets player 
$ B$
 in round 
$ i$, 
 player 
$ C$
 meets player 
$ D$
 in round 
$ i$, 
 and player 
$ A$
 meets player 
$ C$
 in round 
$ j$, 
 then player 
$ B$
 meets player 
$ D$
 in round 
$ j$.


Determine all pairs 
$ (n, k)$
 for which there exists an 
$ (n, k)$-
 tournament.

\item[\textbf{C6.}]
A holey triangle is an upward equilateral triangle of side length 
$n$
 with 
$n$
 upward unit triangular holes cut out. A diamond is a 
$60^\circ$ - $120^\circ$
 unit rhombus.

Prove that a holey triangle 
$T$
can be tiled with diamonds if and only if the following condition holds: Every upward equilateral triangle of side length 
$k$
 in 
$T$
 contains at most 
$k$
 holes, for 
$1\leq k\leq n$.

\item[\textbf{C7.}]
Consider a convex polyhedron without parallel edges and without an edge parallel to any face other than the two faces adjacent to it. Call a pair of points of the polyhedron 
antipodal
 if there exist two parallel planes passing through these points and such that the polyhedron is contained between these planes. Let 
$A$
 be the number of antipodal pairs of vertices, and let 
$B$
 be the number of antipodal pairs of midpoint edges. Determine the difference 
$A-B$
 in terms of the numbers of vertices, edges, and faces.