\item[\textbf{G1.}]Let ABC be a triangle and $M$ be an interior point. Prove that\[ \min\{MA,MB,MC\}+MA+MB+MC<AB+AC+BC.\]

\item[\textbf{G2.}]A circle is called a separator for a set of five points in a plane if it passes through three of these points, it contains a fourth point inside and the fifth point is outside the circle. Prove that every set of five points such that no three are collinear and no four are concyclic has exactly four separators.

\item[\textbf{G3.}]A set $ S$ of points from the space will be called completely symmetric if it has at least three elements and fulfills the condition that for every two distinct points $ A$ and $ B$ from $ S$,  the perpendicular bisector plane of the segment $ AB$ is a plane of symmetry for $ S$. Prove that if a completely symmetric set is finite, then it consists of the vertices of either a regular polygon, or a regular tetrahedron or a regular octahedron.

\item[\textbf{G4.}]For a triangle $T = ABC$ we take the point $X$ on the side $AB$ such that $AX/AB=4/5$,  the point $Y$ on the segment $CX$ such that $CY = 2YX$ and, if possible, the point $Z$ on the ray $CA$ such that $\angle{CXZ} = 180^\circ - \angle{ABC}$. We denote by $\Sigma$ the set of all triangles $T$ for which $\angle{XYZ} = 45^\circ$. Prove that all triangles from $\Sigma$ are similar and find the measure of their smallest angle.

\item[\textbf{G5.}]Let $ABC$ be a triangle, $\Omega$ its incircle and $\Omega_{a}, \Omega_{b}, \Omega_{c}$ three circles orthogonal to $\Omega$ passing through $(B,C)$, $(A,C)$ and $(A,B)$ respectively. The circles $\Omega_{a}$ and $\Omega_{b}$ meet again in $C'$; in the same way we obtain the points $B'$ and $A'$. Prove that the radius of the circumcircle  of $A'B'C'$ is half the radius of $\Omega$.

\item[\textbf{G6.}]Two circles $\Omega_{1}$ and $\Omega_{2}$ touch internally the circle $\Omega$ in M and N and the center of $\Omega_{2}$ is on $\Omega_{1}$. The common chord of the circles $\Omega_{1}$ and $\Omega_{2}$ intersects $\Omega$ in $A$ and $B$. $MA$ and $MB$ intersects $\Omega_{1}$ in $C$ and $D$. Prove that $\Omega_{2}$ is tangent to $CD$.

\item[\textbf{G7.}]The point $M$ is inside the convex quadrilateral $ABCD$,  such that\[ MA = MC, \hspace{0,2cm} \angle{AMB} = \angle{MAD} + \angle{MCD} \quad \textnormal{and} \quad \angle{CMD} = \angle{MCB} + \angle{MAB}. \]

Prove that $AB \cdot CM = BC \cdot MD$ and $BM \cdot AD = MA \cdot CD.$

\item[\textbf{G8.}]Given a triangle $ABC$. The points $A$,  $B$,  $C$ divide the circumcircle $\Omega$ of the triangle $ABC$ into three arcs $BC$,  $CA$,  $AB$. Let $X$ be a variable point on the arc $AB$,  and let $O_{1}$ and $O_{2}$ be the incenters of the triangles $CAX$ and $CBX$. Prove that the circumcircle of the triangle $XO_{1}O_{2}$ intersects the circle $\Omega$ in a fixed point.

