\item[\textbf{G1.}]
Let 
$ABC$
 be an acute triangle with 
$D, E, F$
 the feet of the altitudes lying on 
$BC, CA, AB$
 respectively. One of the intersection points of the line 
$EF$
 and the circumcircle is 
$P.$
 The lines 
$BP$
 and 
$DF$
 meet at point 
$Q.$
 Prove that 
$AP = AQ.$

\item[\textbf{G2.}]
Let 
$P$
 be a point interior to triangle 
$ABC$
 (with 
$CA \neq CB$)
. The lines 
$AP$, 
$BP$
 and 
$CP$
 meet again its circumcircle 
$\Gamma$
 at 
$K$, 
$L$, 
 respectively 
$M$.
 The tangent line at 
$C$
 to 
$\Gamma$
 meets the line 
$AB$
 at 
$S$.
 Show that from 
$SC = SP$
 follows 
$MK = ML$.

\item[\textbf{G3.}]
Let 
$A_1A_2 \ldots A_n$
 be a convex polygon. Point 
$P$
 inside this polygon is chosen so that its projections 
$P_1, \ldots , P_n$
 onto lines 
$A_1A_2, \ldots , A_nA_1$
 respectively lie on the sides of the polygon. Prove that for arbitrary points 
$X_1, \ldots , X_n$
 on sides 
$A_1A_2, \ldots , A_nA_1$
 respectively,
\[\max \left\{ \frac{X_1X_2}{P_1P_2}, \ldots, \frac{X_nX_1}{P_nP_1} \right\} \geq 1.\]

\item[\textbf{G4.}]
Given a triangle 
$ABC$, 
 with 
$I$
 as its incenter and 
$\Gamma$
 as its circumcircle, 
$AI$
 intersects 
$\Gamma$
 again at 
$D$.
 Let 
$E$
 be a point on the arc 
$BDC$, 
 and 
$F$
 a point on the segment 
$BC$, 
 such that 
$\angle BAF=\angle CAE < \dfrac12\angle BAC$.
 If 
$G$
 is the midpoint of 
$IF$, 
 prove that the meeting point of the lines 
$EI$
 and 
$DG$
 lies on 
$\Gamma$.

\item[\textbf{G5.}]
Let 
$ABCDE$
 be a convex pentagon such that 
$BC \parallel AE,$
$AB = BC +  AE,$
 and 
$\angle ABC = \angle CDE.$
 Let 
$M$
 be the midpoint of 
$CE,$
 and let 
$O$
 be the circumcenter of triangle 
$BCD.$
 Given that 
$\angle DMO = 90^{\circ},$
 prove that 
$2 \angle BDA = \angle CDE.$

\item[\textbf{G6.}]
The vertices 
$X, Y , Z$
 of an equilateral triangle 
$XYZ$
 lie respectively on the sides 
$BC, CA, AB$
 of an acute-angled triangle 
$ABC.$
 Prove that the incenter of triangle 
$ABC$
 lies inside triangle 
$XYZ.$

\item[\textbf{G7.}]
Three circular arcs 
$\gamma_1, \gamma_2,$
 and 
$\gamma_3$
 connect the points 
$A$
 and 
$C.$
 These arcs lie in the same half-plane defined by line 
$AC$
 in such a way that arc 
$\gamma_2$
 lies between the arcs 
$\gamma_1$
 and 
$\gamma_3.$
 Point 
$B$
 lies on the segment 
$AC.$
 Let 
$h_1, h_2$, 
 and 
$h_3$
 be three rays starting at 
$B,$
 lying in the same half-plane, 
$h_2$
 being between 
$h_1$
 and 
$h_3.$
 For 
$i, j = 1, 2, 3,$
 denote by 
$V_{ij}$
 the point of intersection of 
$h_i$
 and 
$\gamma_j$
 (see the Figure below). Denote by 
$\widehat{V_{ij}V_{kj}}\widehat{V_{kl}V_{il}}$
 the curved quadrilateral, whose sides are the segments 
$V_{ij}V_{il},$
 
$V_{kj}V_{kl}$
 and arcs 
$V_{ij}V_{kj}$
 and 
$V_{il}V_{kl}.$
 We say that this quadrilateral is 
$circumscribed$
 if there exists a circle touching these two segments and two arcs. Prove that if the curved quadrilaterals 
$\widehat{V_{11}V_{21}}\widehat{V_{22}V_{12}}$, $ \widehat{V_{12}V_{22}}\widehat{V_{23}V_{13}}$, $\widehat{V_{21}V_{31}}\widehat{V_{32}V_{22}}$
 are circumscribed, then the curved quadrilateral 
$\widehat{V_{22}V_{32}}\widehat{V_{33}V_{23}}$
 is circumscribed, too.


