\item[\textbf{G1.}]
Let 
$ABC$
 be an acute triangle. Let 
$\omega$
 be a circle whose centre 
$L$
 lies on the side 
$BC$.
 Suppose that 
$\omega$
 is tangent to 
$AB$
 at 
$B'$
 and 
$AC$
 at 
$C'$.
 Suppose also that the circumcentre 
$O$
 of triangle 
$ABC$
 lies on the shorter arc 
$B'C'$
 of 
$\omega$.
 Prove that the circumcircle of 
$ABC$
 and 
$\omega$
 meet at two points.

\item[\textbf{G2.}]
Let 
$A_1A_2A_3A_4$
 be a non-cyclic quadrilateral. Let 
$O_1$
 and 
$r_1$
 be the circumcentre and the circumradius of the triangle 
$A_2A_3A_4$.
 Define 
$O_2,O_3,O_4$
 and 
$r_2,r_3,r_4$
 in a similar way. Prove that
\[\frac{1}{O_1A_1^2-r_1^2}+\frac{1}{O_2A_2^2-r_2^2}+\frac{1}{O_3A_3^2-r_3^2}+\frac{1}{O_4A_4^2-r_4^2}=0.\]

\item[\textbf{G3.}]
Let 
$ABCD$
 be a convex quadrilateral whose sides 
$AD$
 and 
$BC$
 are not parallel. Suppose that the circles with diameters 
$AB$
 and 
$CD$
 meet at points 
$E$
 and 
$F$
 inside the quadrilateral. Let 
$\omega_E$
 be the circle through the feet of the perpendiculars from 
$E$
 to the lines 
$AB,BC$
 and 
$CD$.
 Let 
$\omega_F$
 be the circle through  the feet of the perpendiculars from 
$F$
 to the lines 
$CD,DA$
 and 
$AB$.
 Prove that the midpoint of the segment 
$EF$
 lies on the line through the two intersections of 
$\omega_E$
 and 
$\omega_F$.

\item[\textbf{G4.}]
Let 
$ABC$
 be an acute triangle with circumcircle 
$\Omega$.
 Let 
$B_0$
 be the midpoint of 
$AC$
 and let 
$C_0$
 be the midpoint of 
$AB$.
 Let 
$D$
 be the foot of the altitude from 
$A$
 and let 
$G$
 be the centroid of the triangle 
$ABC$.
 Let 
$\omega$
 be a circle through 
$B_0$
 and 
$C_0$
 that is tangent to the circle 
$\Omega$
 at a point 
$X\not= A$.
 Prove that the points 
$D,G$
 and 
$X$
 are collinear.

\item[\textbf{G5.}]
Let 
$ABC$
 be a triangle with incentre 
$I$
 and circumcircle 
$\omega$.
 Let 
$D$
 and 
$E$
 be the second intersection points of 
$\omega$
 with 
$AI$
 and 
$BI$, 
 respectively. The chord 
$DE$
 meets 
$AC$
 at a point 
$F$, 
 and 
$BC$
 at a point 
$G$.
 Let 
$P$
 be the intersection point of the line through 
$F$
 parallel to 
$AD$
 and the line through 
$G$
 parallel to 
$BE$.
 Suppose that the tangents to 
$\omega$
 at 
$A$
 and 
$B$
 meet at a point 
$K$.
 Prove that the three lines 
$AE,BD$
 and 
$KP$
 are either parallel or concurrent.

\item[\textbf{G6.}]
Let 
$ABC$
 be a triangle with 
$AB=AC$
 and let 
$D$
 be the midpoint of 
$AC$.
 The angle bisector of 
$\angle BAC$
 intersects the circle through 
$D,B$
  and 
$C$
 at the point 
$E$
  inside the triangle 
$ABC$.
 The line 
$BD$
 intersects the circle through 
$A,E$
 and 
$B$
 in two points 
$B$
 and 
$F$.
 The lines 
$AF$
 and 
$BE$
 meet at a point 
$I$, 
 and the lines 
$CI$
 and 
$BD$
 meet at a point 
$K$.
 Show that 
$I$
 is the incentre of triangle 
$KAB$.

\item[\textbf{G7.}]
Let 
$ABCDEF$
 be a convex hexagon all of whose sides are tangent to a circle 
$\omega$
 with centre 
$O$.
 Suppose that the circumcircle of triangle 
$ACE$
 is concentric with 
$\omega$.
 Let 
$J$
 be the foot of the perpendicular from 
$B$
 to 
$CD$.
 Suppose that the perpendicular from 
$B$
 to 
$DF$
 intersects the line 
$EO$
 at a point 
$K$.
 Let 
$L$
 be the foot of the perpendicular from 
$K$
 to 
$DE$.
 Prove that 
$DJ=DL$.

\item[\textbf{G8.}]
Let 
$ABC$
 be an acute triangle with circumcircle 
$\Gamma$.
 Let 
$\ell$
 be a tangent line to 
$\Gamma$, 
 and let 
$\ell_a, \ell_b$
 and 
$\ell_c$
 be the lines obtained by reflecting 
$\ell$
 in the lines 
$BC$, 
 
$CA$
 and 
$AB$, 
 respectively. Show that the circumcircle of the triangle determined by the lines 
$\ell_a, \ell_b$
 and 
$\ell_c$
 is tangent to the circle 
$\Gamma$.

