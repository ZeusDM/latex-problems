\item[\textbf{G1.}]
Let 
$ABC$
 be an acute triangle with orthocenter 
$H$, 
 and let 
$W$
 be a point on the side 
$BC$, 
 lying strictly between 
$B$
 and 
$C$.
 The points 
$M$
 and 
$N$
 are the feet of the altitudes from 
$B$
 and 
$C$, 
 respectively. Denote by 
$\omega_1$
 is the circumcircle of 
$BWN$, 
 and let 
$X$
 be the point on 
$\omega_1$
 such that 
$WX$
 is a diameter of 
$\omega_1$.
 Analogously, denote by 
$\omega_2$
 the circumcircle of triangle 
$CWM$, 
 and let 
$Y$
 be the point such that 
$WY$
 is a diameter of 
$\omega_2$.
 Prove that 
$X,Y$
 and 
$H$
 are collinear.

\item[\textbf{G2.}]
Let 
$\omega$
 be the circumcircle of a triangle 
$ABC$.
 Denote by 
$M$
 and 
$N$
 the midpoints of the sides 
$AB$
 and 
$AC$, 
 respectively, and denote by 
$T$
 the midpoint of the arc 
$BC$
 of 
$\omega$
 not containing 
$A$.
 The circumcircles of the triangles 
$AMT$
 and 
$ANT$
 intersect the perpendicular bisectors of 
$AC$
 and 
$AB$
 at points 
$X$
 and 
$Y$, 
 respectively; assume that 
$X$
 and 
$Y$
 lie inside the triangle 
$ABC$.
 The lines 
$MN$
 and 
$XY$
 intersect at 
$K$.
 Prove that 
$KA=KT$.

\item[\textbf{G3.}]
In a triangle 
$ABC$, 
 let 
$D$
 and 
$E$
 be the feet of the angle bisectors of angles 
$A$
 and 
$B$, 
 respectively. A rhombus is inscribed into the quadrilateral 
$AEDB$
 (all vertices of the rhombus lie on different sides of 
$AEDB$)
. Let 
$\varphi$
 be the non-obtuse angle of the rhombus. Prove that 
$\varphi \le \max \{  \angle BAC, \angle ABC  \}$.

\item[\textbf{G4.}]
Let 
$ABC$
 be a triangle with 
$\angle B > \angle C$.
 Let 
$P$
 and 
$Q$
 be two different points on line 
$AC$
 such that 
$\angle PBA = \angle QBA = \angle ACB $
 and 
$A$
 is located between 
$P$
 and 
$C$.
 Suppose that there exists an interior point 
$D$
 of segment 
$BQ$
 for which 
$PD=PB$.
 Let the ray 
$AD$
 intersect the circle 
$ABC$
 at 
$R \neq A$.
 Prove that 
$QB = QR$.

\item[\textbf{G5.}]
Let 
$ABCDEF$
 be a convex hexagon with 
$AB=DE$, 
$BC=EF$,
$CD=FA$, 
 and 
$\angle A-\angle D = \angle C -\angle F = \angle E -\angle B$.
 Prove that the diagonals 
$AD$, 
 
$BE$, 
 and 
$CF$
 are concurrent.

\item[\textbf{G6.}]
Let the excircle of triangle 
$ABC$
 opposite the vertex 
$A$
 be tangent to the side 
$BC$
 at the point 
$A_1$.
 Define the points 
$B_1$
 on 
$CA$
 and 
$C_1$
 on 
$AB$
 analogously, using the excircles opposite 
$B$
 and 
$C$, 
 respectively. Suppose that the circumcentre of triangle 
$A_1B_1C_1$
 lies on the circumcircle of triangle 
$ABC$.
 Prove that triangle 
$ABC$
 is right-angled.

