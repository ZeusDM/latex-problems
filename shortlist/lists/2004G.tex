\item[\textbf{G1.}]1. Let $ABC$ be an acute-angled triangle with $AB\neq AC$. The circle with diameter $BC$ intersects the sides $AB$ and $AC$ at $M$ and $N$ respectively. Denote by $O$ the midpoint of the side $BC$. The bisectors of the angles $\angle BAC$ and $\angle MON$ intersect at $R$. Prove that the circumcircles of the triangles $BMR$ and $CNR$ have a common point lying on the side $BC$.

\item[\textbf{G2.}]Let $\Gamma$ be a circle and let $d$ be a line such that $\Gamma$ and $d$ have no common points. Further, let $AB$ be a diameter of the circle $\Gamma$; assume that this diameter $AB$ is perpendicular to the line $d$,  and the point $B$ is nearer to the line $d$ than the point $A$. Let $C$ be an arbitrary point on the circle $\Gamma$,  different from the points $A$ and $B$. Let $D$ be the point of intersection of the lines $AC$ and $d$. One of the two tangents from the point $D$ to the circle $\Gamma$ touches this circle $\Gamma$ at a point $E$; hereby, we assume that the points $B$ and $E$ lie in the same halfplane with respect to the line $AC$. Denote by $F$ the point of intersection of the lines $BE$ and $d$. Let the line $AF$ intersect the circle $\Gamma$ at a point $G$,  different from $A$.

Prove that the reflection of the point $G$ in the line $AB$ lies on the line $CF$.

\item[\textbf{G3.}]Let $O$ be the circumcenter of an acute-angled triangle $ABC$ with ${\angle B<\angle C}$. The line $AO$ meets the side $BC$ at $D$. The circumcenters of the triangles $ABD$ and $ACD$ are $E$ and $F$,  respectively. Extend the sides $BA$ and $CA$ beyond $A$,  and choose on the respective extensions points $G$ and $H$ such that ${AG=AC}$ and ${AH=AB}$. Prove that the quadrilateral $EFGH$ is a rectangle if and only if ${\angle ACB-\angle ABC=60^{\circ }}$.

\item[\textbf{G4.}]In a convex quadrilateral $ABCD$,  the diagonal $BD$ bisects neither the angle $ABC$ nor the angle $CDA$. The point $P$ lies inside $ABCD$ and satisfies \[\angle PBC=\angle DBA\quad\text{and}\quad \angle PDC=\angle BDA.\] Prove that $ABCD$ is a cyclic quadrilateral if and only if $AP=CP$.

\item[\textbf{G5.}]Let $A_1A_2A_3\ldots A_n$ be a regular $n$-gon. Let $B_1$ and $B_n$ be the midpoints of its sides $A_1A_2$ and $A_{n-1}A_n$. Also, for every $i\in\left\{2,3,4,\ldots ,n-1\right\}$. Let $S$ be the point of intersection of the lines $A_1A_{i+1}$ and $A_nA_i$,  and let $B_i$ be the point of intersection of the angle bisector bisector of the angle $\measuredangle A_iSA_{i+1}$ with the segment $A_iA_{i+1}$.

Prove that $\sum_{i=1}^{n-1} \measuredangle A_1B_iA_n=180^{\circ}$.

\item[\textbf{G6.}]Let $P$ be a convex polygon. Prove that there exists a convex hexagon that is contained in $P$ and whose area is at least $\frac34$ of the area of the polygon $P$.Alternative version. Let $P$ be a convex polygon with $n\geq 6$ vertices. Prove that there exists a convex hexagon witha) vertices on the sides of the polygon (or)b) vertices among the vertices of the polygon

such that the area of the hexagon is at least $\frac{3}{4}$ of the area of the polygon.

\item[\textbf{G7.}]For a given triangle $ ABC$,  let $ X$ be a variable point on the line $ BC$ such that $ C$ lies between $ B$ and $ X$ and the incircles of the triangles $ ABX$ and $ ACX$ intersect at two distinct points $ P$ and $ Q.$ Prove that the line $ PQ$  passes through a point independent of $ X$.

\item[\textbf{G8.}]Given a cyclic quadrilateral $ABCD$,  let $M$ be the midpoint of the side $CD$,  and let $N$ be a point on the circumcircle of triangle $ABM$. Assume that the point $N$ is different from the point $M$ and satisfies $\frac{AN}{BN}=\frac{AM}{BM}$. Prove that the points $E$,  $F$,  $N$ are collinear, where $E=AC\cap BD$ and $F=BC\cap DA$.