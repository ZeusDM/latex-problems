
\item[\textbf{N1.}]
Let 
$n$
 be a positive integer and let 
$p$
 be a prime number. Prove that if 
$a$, 
$b$, 
$c$
 are integers (not necessarily positive) satisfying the equations 
\[ a^n + pb = b^n + pc = c^n + pa\]
 then 
$a = b = c$.

\item[\textbf{N2.}]
Let 
$ a_1$, 
$ a_2$, 
$ \ldots$, 
$ a_n$
 be distinct positive integers, 
$ n\ge 3$.
 Prove that there exist distinct indices 
$ i$
 and 
$ j$
 such that 
$ a_i + a_j$
 does not divide any of the numbers 
$ 3a_1$, 
$ 3a_2$, 
$ \ldots$, 
$ 3a_n$.

\item[\textbf{N3.}]
Let 
$ a_0$, 
$ a_1$, 
$ a_2$, 
$ \ldots$
 be a sequence of positive integers such that the greatest common divisor of any two consecutive terms is greater than the preceding term; in symbols, 
$ \gcd (a_i, a_{i + 1}) > a_{i - 1}$.
 Prove that 
$ a_n\ge 2^n$
 for all 
$ n\ge 0$.

\item[\textbf{N4.}]
Let 
$ n$
 be a positive integer. Show that the numbers
\[ \binom{2^n - 1}{0},\; \binom{2^n - 1}{1},\; \binom{2^n - 1}{2},\; \ldots,\; \binom{2^n - 1}{2^{n - 1} - 1}\]


are congruent modulo 
$ 2^n$
 to 
$ 1$, 
$ 3$, 
$ 5$, 
$ \ldots$, 
$ 2^n - 1$
 in some order.

\item[\textbf{N5.}]
For every 
$ n\in\mathbb{N}$
 let 
$ d(n)$
 denote the number of (positive) divisors of 
$ n$.
 Find all functions 
$ f: \mathbb{N}\to\mathbb{N}$
 with the following properties:

$ d\left(f(x)\right) = x$
 for all 
$ x\in\mathbb{N}$.
$ f(xy)$
 divides 
$ (x - 1)y^{xy - 1}f(x)$
 for all 
$ x$, 
$ y\in\mathbb{N}$.

\item[\textbf{N6.}]
Prove that there are infinitely many positive integers 
$ n$
 such that 
$ n^{2} + 1$
 has a prime divisor greater than 
$ 2n + \sqrt {2n}$.