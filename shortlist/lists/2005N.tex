\item[\textbf{N1.}]
Determine all positive integers relatively prime to all the terms of the infinite sequence 
\[ a_n=2^n+3^n+6^n -1,\ n\geq 1. \]

\item[\textbf{N2.}]
Let 
$a_1,a_2,\ldots$
 be a sequence of integers with infinitely many positive and negative terms. Suppose that for every positive integer 
$n$
 the numbers 
$a_1,a_2,\ldots,a_n$
 leave 
$n$
 different remainders upon division by 
$n$.

Prove that every integer occurs exactly once in the sequence 
$a_1,a_2,\ldots$.

\item[\textbf{N3.}]
Let 
$ a$, 
$ b$, 
$ c$, 
$ d$, 
$ e$, 
$ f$
 be positive integers and let 
$ S = a+b+c+d+e+f$.


Suppose that the number 
$ S$
 divides 
$ abc+def$
 and  
$ ab+bc+ca-de-ef-df$.
 Prove that 
$ S$
 is composite.

\item[\textbf{N4.}]
Find all positive integers 
$ n$
 such that there exists a unique integer 
$ a$
 such that 
$ 0\leq a < n!$
 with the following property:
\[ n!\mid a^n + 1
\]

\item[\textbf{N5.}]
Denote by 
$d(n)$
 the number of divisors of the positive integer 
$n$.
 A positive integer 
$n$
 is called highly divisible if 
$d(n) > d(m)$
 for all positive integers 
$m < n$.


Two highly divisible integers 
$m$
 and 
$n$
 with 
$m < n$
 are called consecutive if there exists no highly divisible integer 
$s$
 satisfying 
$m < s < n$.


(a) Show that there are only finitely many pairs of consecutive highly divisible
integers of the form 
$(a, b)$
 with 
$a\mid b$.


(b) Show that for every prime number 
$p$
 there exist infinitely many positive highly divisible integers 
$r$
 such that 
$pr$
 is also highly divisible.

\item[\textbf{N6.}]
Let 
$a$, 
$b$
 be positive integers such that 
$b^n+n$
 is a multiple of 
$a^n+n$
 for all positive integers 
$n$.
 Prove that 
$a=b$.

\item[\textbf{N7.}]
Let 
$P(x)=a_{n}x^{n}+a_{n-1}x^{n-1}+\ldots+a_{0}$, 
 where 
$a_{0},\ldots,a_{n}$
 are integers, 
$a_{n}>0$, 
$n\geq 2$.
 Prove that there exists a positive integer 
$m$
 such that 
$P(m!)$
 is a composite number.