\item[\textbf{G1.}]
Given triangle 
$ABC$
 the point 
$J$
 is the centre of the excircle opposite the vertex 
$A.$
 This excircle is tangent to the side 
$BC$
 at 
$M$, 
 and to the lines 
$AB$
 and 
$AC$
 at 
$K$
 and 
$L$, 
 respectively. The lines 
$LM$
 and 
$BJ$
 meet at 
$F$, 
 and the lines 
$KM$
 and 
$CJ$
 meet at 
$G.$
 Let 
$S$
 be the point of intersection of the lines 
$AF$
 and 
$BC$, 
 and let 
$T$
 be the point of intersection of the lines 
$AG$
 and 
$BC.$
 Prove that 
$M$
 is the midpoint of 
$ST.$


(The 
excircle
 of 
$ABC$
 opposite the vertex 
$A$
 is the circle that is tangent to the line segment 
$BC$, 
 to the ray 
$AB$
 beyond 
$B$, 
 and to the ray 
$AC$
 beyond 
$C$.)

\item[\textbf{G2.}]
Let 
$ABCD$
 be a cyclic quadrilateral whose diagonals 
$AC$
 and 
$BD$
 meet at 
$E$.
 The extensions of the sides 
$AD$
 and 
$BC$
 beyond 
$A$
 and 
$B$
 meet at 
$F$.
 Let 
$G$
 be the point such that 
$ECGD$
 is a parallelogram, and let 
$H$
 be the image of 
$E$
 under reflection in 
$AD$.
 Prove that 
$D,H,F,G$
 are concyclic.

\item[\textbf{G3.}]
In an acute triangle 
$ABC$
 the points 
$D,E$
 and 
$F$
 are the feet of the altitudes through 
$A,B$
 and 
$C$
 respectively. The incenters of the triangles 
$AEF$
 and 
$BDF$
 are 
$I_1$
 and 
$I_2$
 respectively; the circumcenters of the triangles 
$ACI_1$
 and 
$BCI_2$
 are 
$O_1$
 and 
$O_2$
 respectively. Prove that 
$I_1I_2$
 and 
$O_1O_2$
 are parallel.

\item[\textbf{G4.}]
Let 
$ABC$
 be a triangle with 
$AB \neq AC$
 and circumcenter 
$O$.
 The bisector of 
$\angle BAC$
 intersects 
$BC$
 at 
$D$.
 Let 
$E$
 be the reflection of 
$D$
 with respect to the midpoint of 
$BC$.
 The lines through 
$D$
 and 
$E$
 perpendicular to 
$BC$
 intersect the lines 
$AO$
 and 
$AD$
 at 
$X$
 and 
$Y$
 respectively. Prove that the quadrilateral 
$BXCY$
 is cyclic.

\item[\textbf{G5.}]
Let 
$ABC$
 be a triangle with 
$\angle BCA=90^{\circ}$, 
 and let 
$D$
 be the foot of the altitude from 
$C$.
 Let 
$X$
 be a point in the interior of the segment 
$CD$.
 Let 
$K$
 be the point on the segment 
$AX$
 such that 
$BK=BC$.
 Similarly, let 
$L$
 be the point on the segment 
$BX$
 such that 
$AL=AC$.
 Let 
$M$
 be the point of intersection of 
$AL$
 and 
$BK$.


Show that 
$MK=ML$.

\item[\textbf{G6.}]
Let 
$ABC$
 be a triangle with circumcenter 
$O$
 and incenter 
$I$.
 The points 
$D,E$
 and 
$F$
 on the sides 
$BC,CA$
 and 
$AB$
 respectively are such that 
$BD+BF=CA$
 and 
$CD+CE=AB$.
 The circumcircles of the triangles 
$BFD$
 and 
$CDE$
 intersect at 
$P \neq D$.
 Prove that 
$OP=OI$.

\item[\textbf{G7.}]
Let 
$ABCD$
 be a convex quadrilateral with non-parallel sides 
$BC$
 and 
$AD$.
 Assume that there is a point 
$E$
 on the side 
$BC$
 such that the quadrilaterals 
$ABED$
 and 
$AECD$
 are circumscribed. Prove that there is a point 
$F$
 on the side 
$AD$
 such that the quadrilaterals 
$ABCF$
 and 
$BCDF$
 are circumscribed if and only if 
$AB$
 is parallel to 
$CD$.

\item[\textbf{G8.}]
Let 
$ABC$
 be a triangle with circumcircle 
$\omega$
 and 
$\ell$
 a line without common points with 
$\omega$.
 Denote by 
$P$
 the foot of the perpendicular from the center of 
$\omega$
 to 
$\ell$.
 The side-lines 
$BC,CA,AB$
 intersect 
$\ell$
 at the points 
$X,Y,Z$
 different from 
$P$.
 Prove that the circumcircles of the triangles 
$AXP$, 
$BYP$
 and 
$CZP$
 have a common point different from 
$P$
 or are mutually tangent at 
$P$.

