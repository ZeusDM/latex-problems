\documentclass[10pt,a4paper]{article}
\usepackage[utf8]{inputenc}
\usepackage[brazil]{babel}

\usepackage{../../../commands/problems}
\renewcommand{\mypath}{../../../}
%\usepackage{../../../commands/BraunChem}

\usepackage[inline]{enumitem}
\usepackage{lmodern}
\usepackage[left=1.5cm, right=1.5cm, top=2.5cm, bottom=1.5cm]{geometry}
\usepackage{siunitx}
\usepackage{fancyhdr}
\usepackage{chemfig}
\usepackage{icomma}
\usepackage{calc}

\newcommand{\ano}{2009}

\pagestyle{fancy}
\fancyhf{}
\rhead{\bfseries \ano}
\chead{\bfseries Olimpíada Brasileira de Química}
\lhead{\bfseries Modalidade A}
\rfoot{}
\cfoot{}
\lfoot{}

\renewcommand{\headrulewidth}{0.3pt}
\renewcommand{\footrulewidth}{0pt}

\addtolength{\textwidth}{-0.5cm}

\newcommand{\obq}[1]{
	\def\ano{#1}
	\begin{center}
	\textbf{Parte A - #1}
	\end{center}

	\setcounter{prob}{0}

		\problem*{chemistry/brazil/#1/1}
		\problem*{chemistry/brazil/#1/2}
		\problem*{chemistry/brazil/#1/3}
		\problem*{chemistry/brazil/#1/4}
		\problem*{chemistry/brazil/#1/5}
		\problem*{chemistry/brazil/#1/6}
		\problem*{chemistry/brazil/#1/7}
		\problem*{chemistry/brazil/#1/8}
		\problem*{chemistry/brazil/#1/9}
		\problem*{chemistry/brazil/#1/10}
	
	
	\newpage

	\begin{center}
		\textbf{Parte B - #1}
	\end{center}

	
		\problem*{chemistry/brazil/#1/11}
 		\problem*{chemistry/brazil/#1/12}
		\problem*{chemistry/brazil/#1/13}
		\problem*{chemistry/brazil/#1/14}
		\problem*{chemistry/brazil/#1/15}
		\problem*{chemistry/brazil/#1/16}
	

	\newpage
}

\begin{document}

\thispagestyle{empty}
\begin{center}
    \fontsize{.6cm}{.8cm}\selectfont
    
    \hrulefill\\\vspace{-1.25em}
    \hrulefill\\
    
	\textbf{IMC (1994 -- 2021)} \\\vspace{-.75em}
    
    \hrulefill\\\vspace{-1.25em}
    \hrulefill

    \vspace{.4cm}
    
    \fontsize{.4cm}{.7cm}\selectfont
    
    \begin{tabular*}{\textwidth}{c|@{\extracolsep{\fill}}c|c|c|c|c|c|c|c|c|c|c|c}
        & 1 & 2 & 3 & 4 & 5 & 6 & 7 & 8 & 9 & 10 & 11 & 12\\\hline
        21 &&&&&&&&&&&&\\\hline
        20 &&&&&&&&&&&&\\\hline
        19 &&&&&&&&&&&&\\\hline
        18 &&&&&&&&&&&&\\\hline
        17 &&&&&&&&&&&&\\\hline
        16 &&&&&&&&&&&&\\\hline
        15 &&&&&&&&&&&&\\\hline
        14 &&&&&&&&&&&&\\\hline
        13 &&&&&&&&&&&&\\\hline
        12 &&&&&&&&&&&&\\\hline
        11 &&&&&&&&&&&&\\\hline
        10 &&&&&&&&&&&&\\\hline
        09 &&&&&&&&&&&&\\\hline
        08 &&&&&&&&&&&&\\\hline
        07 &&&&&&&&&&&&\\\hline
        06 &&&&&&&&&&&&\\\hline
        05 &&&&&&&&&&&&\\\hline
        04 &&&&&&&&&&&&\\\hline
        03 &&&&&&&&&&&&\\\hline
        02 &&&&&&&&&&&&\\\hline
        01 &&&&&&&&&&&&\\\hline
        00 &&&&&&&&&&&&\\\hline
        99 &&&&&&&&&&&&\\\hline
        98 &&&&&&&&&&&&\\\hline
        97 &&&&&&&&&&&&\\\hline
        96 &&&&&&&&&&&&\\\hline
        95 &&&&&&&&&&&&\\\hline
        94 &&&&&&&&&&&&\\\hline
    \end{tabular*}

\end{center}

\vfill

	{

		\noindent \footnotesize Todo e qualquer feedback, especialmente sobre erros neste livreto (mesmo erros tipográficos pequenos), é apreciado. Você também pode contribuir enviando suas soluções (de preferência, formatadas em \TeX).
		
		\noindent Você pode enviar comentários e soluções para \href{mailto:zeusdanmou+tex@gmail.com}{\texttt{zeusdanmou+tex@gmail.com}}.
		
		\noindent Última atualização: \today.

}

\newpage

%-------OBQ's--------%

\obq{2009} %ok
\obq{2010} %ok
\obq{2011} %ok
\obq{2012} %ok
\obq{2013} %ok
\obq{2014} %ok 
\obq{2015} %ok
\obq{2016} %ok 
\obq{2017} %ok
%\obq{2018} %ok

\end{document}
