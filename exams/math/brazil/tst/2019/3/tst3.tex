\documentclass[10pt,a4paper]{article}
\usepackage[utf8]{inputenc}
\usepackage[brazil]{babel}
\usepackage{amsmath}
\usepackage{amsfonts}
\usepackage{amssymb}
\usepackage{graphicx}
\usepackage{enumitem}
\usepackage{lmodern}
\usepackage{fullpage}

\usepackage{../../../../../../commands/problems}
\renewcommand{\mypath}{../../../../../../}

\title{Terceiro Teste de Seleção}
\author{\small LX Olímpíada Internacional de Matemática e  XXXIV Olimpíada Iberoamericana}
\nomail
\titler{29 de março de 2019}

\begin{document}

\zeustitle

\problem*{math/imosl/2018/A1}
\begin{prob}
	Dado um triângulo $ABC$, sejam $A_1$, $B_1$ e $C_1$ pontos sobre os lados $BC$, $CA$ e $AB$, respectivamente, tais que o trângulo $A_1B_1C_1$ é equilátero.
	Sejam $I_1$ e $\omega_1$ o incentro e o incírculo de $AB_1C_1$.
	Defina $I_2, \omega_2$ e $I_3, \omega_3$ similarmente com respeito aos triângulos $BA_1C_1$ e $CA_1B_1$, respectivamente.
	Seja $\ell_1$ a reta tangente externamente a $\omega_2$ e $\omega_3$ diferente da reta $BC$.
	Defina $\ell_2$ e $\ell_3$ similarmente com respeito aos pares $\omega_1, \omega_3$ e $\omega_1, \omega_2$.
	Sabendo que $A_1I_2 = A_1I_3$, mostre que $\ell_1$, $\ell_2$ e $\ell_3$ são concorrentes.
\end{prob}
\problem*{math/imosl/2018/C3}
\problem*{math/imosl/2018/N6}

\end{document}
