\documentclass[10pt,a4paper]{article}
\usepackage[utf8]{inputenc}
\usepackage[brazil]{babel}
\usepackage{amsmath}
\usepackage{amsfonts}
\usepackage{amssymb}
\usepackage{graphicx}
\usepackage{enumitem}
\usepackage{lmodern}
\usepackage{fullpage}

\usepackage{../../../../../../commands/problems}
\renewcommand{\mypath}{../../../../../../}

\title{Quarto Teste de Seleção - Primeiro Dia}
\author{\small LX Olímpíada Internacional de Matemática e  XXXIV Olimpíada Iberoamericana}
\nomail
\titler{16 de abril de 2019}

\begin{document}

\zeustitle

\problem*{math/imosl/2018/G2}
\problem*{math/imosl/2018/A3}
\problem*{math/imosl/2018/C7}

\newpage

\title{Quarto Teste de Seleção - Segundo Dia}
\titler{17 de abril de 2019}

\zeustitle

\begin{prob}
	Considere um tabuleiro $2m \times 2n$, onde $m$, $n$ são inteiros positivos. Uma pedra é colocada em uma das casinhas unitárias do tabuleiro, casinha dessa diferente da casinha inferior esquerda e fa casinha superior direita. Um caracol parte da casa inferior esquerda e deseja chegar à casa superior direita, caminhando de uma casinha para outra adjacente, uma casinha por vez. Determine todas as casinhas em que a pedra pode estar, de modo que o caracol possa fazer esse percurso visitando casa casinha exatamente uma vez, com exceção da casinha onde está a pedra, que o caracol não visita nenhuma vez.
\end{prob}
\problem*{math/imosl/2018/N5}
\problem*{math/imosl/2018/G5}

\end{document}
