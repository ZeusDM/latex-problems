\begin{center}
    \fontsize{.6cm}{.8cm}\selectfont
    
    \hrulefill\\\vspace{-1.25em}
    \hrulefill\\
    
    \textbf{Olimpíada de Matemática do Estado do Rio de Janeiro \\ 2012 - 2019} \\\vspace{-.75em}
    
    \hrulefill\\\vspace{-1.25em}
    \hrulefill

	\vspace{1cm}
    
    \fontsize{.5cm}{.65cm}\selectfont
    
    \begin{tabular*}{\textwidth}{cc|@{\extracolsep{\fill}}c|c|c|c|c|c}
		&& 1 & 2 & 3 & 4 & 5 & 6\\\hline
		N3 & 12 &&&&&&\\\hline
		N4 & 12 &&&&&&\\\hline
		N3 & 13 &&&&&&\\\hline
		N4 & 13 &&&&&&\\\hline
		N3 & 14 &&&&&&\\\hline
		N4 & 14 &&&&&&\\\hline
		N3 & 15 &&&&&&\\\hline
		N4 & 15 &&&&&&\\\hline
		N3 & 16 &&&&&&\\\hline
		N4 & 16 &&&&&&\\\hline
		N3 & 17 &&&&&&\\\hline
		N4 & 17 &&&&&&\\\hline
		N3 & 18 &&&&&&\\\hline
		N4 & 18 &&&&&&\\\hline
		N3 & 19 &&&&&&\\\hline
		N4 & 19 &&&&&&
    \end{tabular*}
\end{center}


\vfill

	{

		\noindent \footnotesize Todo e qualquer feedback, especialmente sobre erros neste livreto (mesmo erros tipográficos pequenos), é apreciado. Você também pode contribuir enviando suas soluções (de preferência, formatadas em \TeX).
		
		\noindent Você pode enviar comentários e soluções para \href{mailto:zeusdanmou+tex@gmail.com}{\texttt{zeusdanmou+tex@gmail.com}}.
		
		\noindent A versão mais atualizada desse arquivo (provavelmente) está disponível \textcolor{red}{\href{https://github.com/ZeusDM/latex-problems/blob/master/exams/math/brazil/rio/livreto/livreto_omerj.pdf?raw=true}{clicando aqui}}. Última atualização: \today.

}

