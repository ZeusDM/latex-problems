\documentclass[10pt,a4paper]{article}
\usepackage[utf8]{inputenc}
\usepackage[brazilian]{babel}
\usepackage{lmodern}
\usepackage[left=1.5cm, right=1.5cm, top=2.5cm, bottom=1.5cm]{geometry}

\usepackage[booklet]{zeus}

\usepackage{fancyhdr}
 
\newcommand{\ano}{2018}
\newcommand{\niv}{3}

\pagestyle{fancy}
\fancyhf{}
\rhead{\bfseries \ano}
\chead{\bfseries Olimpíada de Matemática do Estado do Rio de Janeiro}
\lhead{\bfseries Nível \niv}
\rfoot{}
\cfoot{}
\lfoot{}
 
\renewcommand{\headrulewidth}{0.3pt}
\renewcommand{\footrulewidth}{0pt}

\addtolength{\textwidth}{-0.5cm}

\newcommand{\rio}[1]{
	\def\ano{#1}
	\def\niv{3}
	\setcounter{prob}{0}
	%\begin{center}
	%OLIMPÍADA DE MATEMÁTICA DO ESTADO DO RIO DE JANEIRO - #1\\
	%Nível 3 - 1$^\mathrm{a}$ e 2$^{a}$ séries
	%\end{center}

	\problem*{math/brazil/rio/#1/N3/1}
	\problem*{math/brazil/rio/#1/N3/2}
	\problem*{math/brazil/rio/#1/N3/3}
	\problem*{math/brazil/rio/#1/N3/4}
	\problem*{math/brazil/rio/#1/N3/5}
	\problem*{math/brazil/rio/#1/N3/6}
		
	\newpage

	\def\niv{4}
	\setcounter{prob}{0}
	%\begin{center}
	%OLIMPÍADA DE MATEMÁTICA DO ESTADO DO RIO DE JANEIRO - #1\\
	%Nível 4 - 3$^\mathrm{a}$ série
	%\end{center}

	\problem*{math/brazil/rio/#1/N4/1}
	\problem*{math/brazil/rio/#1/N4/2}
	\problem*{math/brazil/rio/#1/N4/3}
	\problem*{math/brazil/rio/#1/N4/4}
	\problem*{math/brazil/rio/#1/N4/5}
	\problem*{math/brazil/rio/#1/N4/6}
		
	\newpage

}


%\renewcommand{\playerA}[1]{Guilherme}
%\renewcommand{\playerB}[1]{Zeus}

\begin{document}

\thispagestyle{empty}
\begin{center}
    \fontsize{.6cm}{.8cm}\selectfont
    
    \hrulefill\\\vspace{-1.25em}
    \hrulefill\\
    
	\textbf{IMC (1994 -- 2021)} \\\vspace{-.75em}
    
    \hrulefill\\\vspace{-1.25em}
    \hrulefill

    \vspace{.4cm}
    
    \fontsize{.4cm}{.7cm}\selectfont
    
    \begin{tabular*}{\textwidth}{c|@{\extracolsep{\fill}}c|c|c|c|c|c|c|c|c|c|c|c}
        & 1 & 2 & 3 & 4 & 5 & 6 & 7 & 8 & 9 & 10 & 11 & 12\\\hline
        21 &&&&&&&&&&&&\\\hline
        20 &&&&&&&&&&&&\\\hline
        19 &&&&&&&&&&&&\\\hline
        18 &&&&&&&&&&&&\\\hline
        17 &&&&&&&&&&&&\\\hline
        16 &&&&&&&&&&&&\\\hline
        15 &&&&&&&&&&&&\\\hline
        14 &&&&&&&&&&&&\\\hline
        13 &&&&&&&&&&&&\\\hline
        12 &&&&&&&&&&&&\\\hline
        11 &&&&&&&&&&&&\\\hline
        10 &&&&&&&&&&&&\\\hline
        09 &&&&&&&&&&&&\\\hline
        08 &&&&&&&&&&&&\\\hline
        07 &&&&&&&&&&&&\\\hline
        06 &&&&&&&&&&&&\\\hline
        05 &&&&&&&&&&&&\\\hline
        04 &&&&&&&&&&&&\\\hline
        03 &&&&&&&&&&&&\\\hline
        02 &&&&&&&&&&&&\\\hline
        01 &&&&&&&&&&&&\\\hline
        00 &&&&&&&&&&&&\\\hline
        99 &&&&&&&&&&&&\\\hline
        98 &&&&&&&&&&&&\\\hline
        97 &&&&&&&&&&&&\\\hline
        96 &&&&&&&&&&&&\\\hline
        95 &&&&&&&&&&&&\\\hline
        94 &&&&&&&&&&&&\\\hline
    \end{tabular*}

\end{center}

\vfill

	{

		\noindent \footnotesize Todo e qualquer feedback, especialmente sobre erros neste livreto (mesmo erros tipográficos pequenos), é apreciado. Você também pode contribuir enviando suas soluções (de preferência, formatadas em \TeX).
		
		\noindent Você pode enviar comentários e soluções para \href{mailto:zeusdanmou+tex@gmail.com}{\texttt{zeusdanmou+tex@gmail.com}}.
		
		\noindent Última atualização: \today.

}

\newpage

%------OMERJ's-------%

\rio{2018}
\rio{2017}
\rio{2016}
\rio{2015}
\rio{2014}
\rio{2013}
\rio{2012}

\end{document}
