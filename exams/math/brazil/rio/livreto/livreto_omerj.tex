\documentclass[10pt,a4paper]{article}
\usepackage[utf8]{inputenc}
\usepackage[brazilian]{babel}
\usepackage{lmodern}
\usepackage[left=1.5cm, right=1.5cm, top=2.5cm, bottom=1.5cm]{geometry}
\usepackage{../../../../../commands/problems}
\renewcommand{\mypath}{../../../../../}

\usepackage{fancyhdr}
 
\newcommand{\ano}{2018}
\newcommand{\niv}{3}

\pagestyle{fancy}
\fancyhf{}
\rhead{\bfseries \ano}
\chead{\bfseries Olimpíada de Matemática do Estado do Rio de Janeiro}
\lhead{\bfseries Nível \niv}
\rfoot{}
\cfoot{}
\lfoot{}
 
\renewcommand{\headrulewidth}{0.3pt}
\renewcommand{\footrulewidth}{0pt}

\addtolength{\textwidth}{-0.5cm}

\newcommand{\rio}[1]{
	\def\ano{#1}
	\def\niv{3}
	\setcounter{prob}{0}
	%\begin{center}
	%OLIMPÍADA DE MATEMÁTICA DO ESTADO DO RIO DE JANEIRO - #1\\
	%Nível 3 - 1$^\mathrm{a}$ e 2$^{a}$ séries
	%\end{center}

	\problem*{math/brazil/rio/#1/N3/1}
	\problem*{math/brazil/rio/#1/N3/2}
	\problem*{math/brazil/rio/#1/N3/3}
	\problem*{math/brazil/rio/#1/N3/4}
	\problem*{math/brazil/rio/#1/N3/5}
	\problem*{math/brazil/rio/#1/N3/6}
		
	\newpage

	\def\niv{4}
	\setcounter{prob}{0}
	%\begin{center}
	%OLIMPÍADA DE MATEMÁTICA DO ESTADO DO RIO DE JANEIRO - #1\\
	%Nível 4 - 3$^\mathrm{a}$ série
	%\end{center}

	\problem*{math/brazil/rio/#1/N4/1}
	\problem*{math/brazil/rio/#1/N4/2}
	\problem*{math/brazil/rio/#1/N4/3}
	\problem*{math/brazil/rio/#1/N4/4}
	\problem*{math/brazil/rio/#1/N4/5}
	\problem*{math/brazil/rio/#1/N4/6}
		
	\newpage

}


%\renewcommand{\playerA}[1]{Guilherme}
%\renewcommand{\playerB}[1]{Zeus}

\begin{document}

\thispagestyle{empty}
\begin{center}
    \fontsize{.8cm}{1cm}\selectfont
    
    \hrulefill\\\vspace{-1.25em}
    \hrulefill\\
    
    \textbf{Olimpíada Búlgara de Matemática} \\\vspace{-.75em}
    
    \hrulefill\\\vspace{-1.25em}
    \hrulefill

    \vspace{1cm}
    
    %\fontsize{.5cm}{.65cm}\selectfont
    
    %\begin{tabular*}{\textwidth}{c|@{\extracolsep{\fill}}c|c|c|c|c|c}
    %    & 1 & 2 & 3 & 4 & 5 & 6\\\hline
    %    98 &&&&&&\\\hline
    %    99 &&&&&&\\\hline
    %    00 &&&&&&\\\hline
    %    01 &&&&&&\\\hline
    %    02 &&&&&&\\\hline
    %    03 &&&&&&\\\hline
    %    04 &&&&&&\\\hline
    %    05 &&&&&&\\\hline
    %    06 &&&&&&\\\hline
    %    07 &&&&&&\\\hline
    %    08 &&&&&&\\\hline
    %    09 &&&&&&\\\hline
    %    10 &&&&&&\\\hline
    %    11 &&&&&&\\\hline
    %    12 &&&&&&\\\hline
    %    13 &&&&&&\\\hline
    %    14 &&&&&&\\\hline
    %    15 &&&&&&\\\hline
    %    16 &&&&&&\\\hline
    %    17 &&&&&&\\\hline
    %    18 &&&&&&\\\hline
    %    19 &&&&&&\\\hline
	%	20 &&&&&&\\
    %    
    %\end{tabular*}

\end{center}

\vfill

	{

		\noindent \footnotesize Todo e qualquer feedback, especialmente sobre erros neste livreto (mesmo erros tipográficos pequenos), é apreciado. Você também pode contribuir enviando suas soluções (de preferência, formatadas em \TeX).
		
		\noindent Você pode enviar comentários e soluções para \href{mailto:zeusdanmou+tex@gmail.com}{\texttt{zeusdanmou+tex@gmail.com}}.
		
		\noindent A versão mais atualizada desse arquivo (provavelmente) está disponível \textcolor{red}{\href{https://github.com/ZeusDM/latex-problems/blob/master/exams/math/brazil/mo/livreto/livreto_obm.pdf?raw=true}{clicando aqui}}. Última atualização: \today.

}

\newpage

%------OMERJ's-------%

\rio{2018}
\rio{2017}
\rio{2016}
\rio{2015}
\rio{2014}
\rio{2013}
\rio{2012}

\end{document}
